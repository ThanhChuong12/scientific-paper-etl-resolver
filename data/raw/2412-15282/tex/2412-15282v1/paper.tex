\documentclass[]{fairmeta}
% Option "twocolumn" available, but please prioritize single-column
\usepackage{graphicx}
\usepackage{tabularx}

\usepackage{colortbl}
\usepackage{tcolorbox}
\usepackage{inconsolata}
\usepackage{amsmath}
\usepackage{arydshln}
\usepackage{enumitem}
\usepackage{listings}
\usepackage{fancyvrb}
\usepackage{framed}
\usepackage{float}

\definecolor{verylightgray}{rgb}{0.9,0.9,0.9}

\newcolumntype{P}[1]{>{\centering\arraybackslash}p{#1}}
\DeclareMathOperator*{\argmax}{\arg\!\max}

\title{A Systematic Examination of Preference Learning through the Lens of Instruction-Following}

\author[1,2]{Joongwon Kim}
\author[1]{Anirudh Goyal}
\author[1]{Aston Zhang}
\author[1]{Bo Xiong}
\author[1]{Rui Hou}
\author[1]{Melanie Kambadur}
\author[1]{Dhruv Mahajan}
\author[2]{Hannaneh Hajishirzi}
\author[1]{Liang Tan}

\affiliation[1]{Llama Team, AI @ Meta}
\affiliation[2]{University of Washington}

% \contribution[*]{Work done during internship at Meta}
% \contribution[\dagger]{Joint last author}

\abstract{
Preference learning is a widely adopted post-training technique that aligns large language models (LLMs) to human preferences and improves specific downstream task capabilities.
In this work we systematically investigate how specific attributes of preference datasets affect the alignment and downstream performance of LLMs in instruction-following tasks.
We use a novel synthetic data generation pipeline to generate 48,000 unique instruction-following prompts with combinations of 23 verifiable constraints that enable fine-grained and automated quality assessments of model responses.
With our synthetic prompts, we use two preference dataset curation methods -- rejection sampling (RS) and Monte Carlo Tree Search (MCTS) -- to obtain pairs of (chosen, rejected) responses.
Then, we perform experiments investigating the effects of (1) the presence of shared prefixes between the chosen and rejected responses, (2) the contrast and quality of the chosen, rejected responses and (3) the complexity of the training prompts.
Our experiments reveal that shared prefixes in preference pairs, as generated by MCTS, provide marginal but consistent improvements and greater stability across challenging training configurations.
High-contrast preference pairs generally outperform low-contrast pairs; however, combining both often yields the best performance by balancing diversity and learning efficiency.
Additionally, training on prompts of moderate difficulty leads to better generalization across tasks, even for more complex evaluation scenarios, compared to overly challenging prompts.
Our findings provide actionable insights into optimizing preference data curation for instruction-following tasks, offering a scalable and effective framework for enhancing LLM training and alignment.
}

\date{\today}
\correspondence{Joongwon Kim at \email{jwonkim@meta.com}}

% You can add additional metadata fields as follows 
% \metadata[Code]{TBD}
% \metadata[Blogpost]{\url{https://ai.meta.com/blog/?page=1}}

\begin{document}

\maketitle

\section{Introduction}
\label{section:intro}
 % !TEX root = ../main.tex

\section{Introduction}
\label{sec:intro}

\begin{figure}[t]
\centering
    \begin{subfigure}[m]{1\linewidth}
    	\includegraphics[width=\linewidth]{figures/worst_tradeoff_new3.png}\vspace{-0mm}
%	\subcaption{Worst-group accuracy}
	\label{fig:observation_worst}
%	    \vspace{3mm}
    \end{subfigure} 
%    \hspace{5mm}
%        \begin{subfigure}[m]{0.9\linewidth}
%    	\includegraphics[width=\linewidth]{figures/unbias_tradeoff_new3.png}\vspace{-0mm}
%	\subcaption{Unbiased accuracy}
%	\label{fig:observation_unbias}
%	\end{subfigure} 
    \vspace{-2mm}
    \caption{The scatter plots illustrate trade-offs between robust and average accuracies of existing algorithms with ResNet-18 on CelebA.
    We visualize the results from multiple runs of each algorithm and present the relationship between the two accuracies.
    The lines denote the linear regression results of individual algorithms and $r$ in the legend indicates the Pearson coefficient correlation.
    %which validates the strong negative correlation between both accuracies.
%    Compared to ERM baseline, CR, GR, and Group DRO achieve high robust accuracy at the expense of average accuracy.
    }
%    \vspace{-2mm}
    \label{fig:observation_tradeoff}
\end{figure}
% 


%
\begin{figure*}[t]
\centering
    \begin{subfigure}[m]{0.45\linewidth}
    	\includegraphics[width=\linewidth]{figures/teaser_worst_new.pdf} \vspace{-5mm}
	\subcaption{Worst-group accuracy}
	\label{fig:teaser_worst}
%	\vspace{3mm}
    \end{subfigure}     
    	\hspace{10mm}
        \begin{subfigure}[m]{0.45\linewidth}
    	\includegraphics[width=\linewidth]{figures/teaser_unbias_new.pdf} \vspace{-5mm}
	\subcaption{Unbiased accuracy}
	\label{fig:teaser_unbias}
	\end{subfigure} 
%    \vspace{-2mm}
    \caption{Comparison between the baseline ERM and existing debiasing approaches with ResNet-50 on CelebA.
%    $\ast$ indicates that the method exploits group supervision during training.
    Existing works have improved robust accuracy substantially compared to ERM, but our robust scaling strategies such as RS and IRS enable ERM to catch up with or even outperform them without further training.
%    Group DRO requires group supervision during training while all the other methods do not utilize it.
%    For some methods, we use their reported numbers because the source codes are unavailable.
    }
%    \vspace{-2mm}
    \label{fig:teaser}
\end{figure*}
%

Machine learning models achieve remarkable performance across various tasks via empirical risk minimization (ERM).
However, they are often vulnerable to spurious correlations and dataset biases, resulting in poor classification performance for minority groups despite high average accuracy.
For example, in the Colored MNIST dataset~\cite{IRM, ReBias}, a strong correlation exists between digit labels and foreground colors. 
Consequently, trained models tend to rely on these unintended patterns, resulting in significant performance degradation when classifying digits with rare color associations that are underrepresented in the training data.
%For instance, because most of the blonde people are women, when trained to classify \textit{hair color}, a model learns an unintended correlation with \textit{gender} and performs poorly in minority groups, such as men with blonde hair. 
%This can be problematic especially when the test distributions shift apart from the training ones.

%To mitigate spurious correlation, ...
Since spurious correlations are well-known to degrade generalization performance in minority groups, group distributionally robust optimization~\cite{GroupDRO} has been widely adopted to address algorithmic bias. 
%Although numerous approaches~\cite{huang2016learning, GroupDRO, seo2021unsupervised, LfF, sohoni2020no, levy2020large, JTT} have been proposed and have presented high worst-group accuracy in a variety of tasks and datasets, they clearly sacrifice the average accuracy and often ignore the trade-off in performance evaluation by focusing only on the robust accuracy such as worst-group and unbiased accuracies.
Numerous approaches~\cite{huang2016learning, GroupDRO, seo2021unsupervised, LfF, sohoni2020no, levy2020large, JTT} have achieved high robust accuracies, such as worst-group or unbiased accuracies, across various tasks and datasets. 
However, despite these improvements, they often come at the expense of average accuracy, and little effort has been made to comprehensively evaluate the robust and average accuracies together.
Figure~\ref{fig:observation_tradeoff} demonstrates the trade-offs of existing algorithms.
%often ignore this trade-off in performance evaluation by focusing only on the robust accuracy.

This paper addresses the limitations of current research trends by introducing a simple post-processing technique, \textit{robust scaling}, which efficiently performs class-specific scaling on prediction scores and conveniently controls the trade-off between robust and average accuracies at test time.
It allows us to identify any desired performance points under various metrics such as average accuracy, unbiased accuracy, worst-group accuracy, or balanced accuracy, along the accuracy trade-off curve derived from a pretrained model with negligible extra computational overhead.
%The proposed robust-scaling method can be easily plugged into various existing debiasing algorithms to achieve improved accuracies at various target objectives within the trade-off.
The proposed robust-scaling method can be seamlessly plugged into various existing debiasing algorithms to improve target objectives within the trade-off.

An interesting observation is that, by adopting the proposed robust scaling, even the ERM baseline accomplishes competitive performance without extra training compared to the recent group distributionally robust optimization approaches~\cite{JTT, LfF, GroupDRO, kim2022learning, seo2021unsupervised, creager2021environment, levy2020large, kirichenko2022last, zhang2022correct}, as illustrated in Figure~\ref{fig:teaser}.
%We will present the results from other debiasing algorithms in the experiment section.
%\footnote{Group DRO achieves better results than ERM+Scaling, mainly because it utilizes the supervision of group information during training. When we exploit group information for robust scaling, ERM+Scaling also achieves competitive performance to Group DRO, which will be discussed later.}.
%Furthermore, we propose a more advanced robust scaling algorithm, which selects a model for each example based on its cluster membership at test time to maximize performance.
Furthermore, we propose an advanced robust scaling algorithm that adaptively applies scaling to individual examples adaptively based on their cluster membership at test time to maximize performance.
This instance-wise adaptive scaling strategy effectively mitigates the trade-off and delivers performance improvements in both robust and average accuracies.

By taking advantage of the robust scaling technique, we develop a novel comprehensive evaluation metric that consolidates insights into the trade-off of group robustness algorithms, providing a unique perspective on group distributionally robust optimization. 
We argue that assessing robust accuracy in isolation, without accounting for average accuracy, provides an incomplete picture and a unified evaluation of debiasing algorithms is required.
For a comprehensive performance evaluation, we introduce \textit{robust coverage}, a new measure that effectively captures the trade-off between average and robust accuracies from a Pareto optimal perspective, summarizing each algorithm's performance with a single scalar value.


\iffalse
The evaluation metric is realized by a simple post-processing technique, \textit{robust scaling}, which efficiently performs class-specific scaling on prediction scores and conveniently controls the trade-off between robust and average accuracies.
%Motivated by the observation, we propose \textit{robust scaling}, a novel class-specific scaling strategy that controls the trade-off by considering both accuracies, which allow us to identify any desired performance points on the trade-off curve with a single model.
It allows us to identify any desired performance points, \eg, average accuracy, unbiased accuracy, worst-group accuracy, or balanced accuracy, on the accuracy trade-off curve using a single model with marginal computational overhead.
The proposed robust-scaling method can be easily plugged into various existing debiasing algorithms to achieve improved accuracies at various target objectives within the trade-off.
One interesting observation is that, by adopting the proposed robust scaling, even the ERM baseline accomplishes competitive performance compared to the recent group distributionally robust optimization approaches~\cite{JTT, LfF, GroupDRO, kim2022learning, seo2021unsupervised, creager2021environment, levy2020large, kirichenko2022last, zhang2022correct} without extra training, as illustrated in Figure~\ref{fig:teaser}\footnote{Group DRO achieves better results than ERM+Scaling, mainly because it utilizes the supervision of group information during training. When we use group information for robust scaling, ERM+Scaling achieves competitive performance to Group DRO, which will be discussed later.}.
We will present the results from other debiasing algorithms in the experiment section.
%In other words, the performance improvement in robust accuracy may be exaggerated.
\fi

%Figure~\ref{fig:teaser} shows the observation, where even the ERM baseline applied with class-specific scaling can achieve comparable performance to most of existing works.
%Motivated by the observation, we propose \textit{robust scaling}, a novel non-uniform scaling strategy that controls the trade-off by considering both robust and average accuracies and identifies the desired performance point.



% Moreover, it is possible to find optimal scaling factor.
% This enables that even the baseline model can achieve comparable performance to the existing group robustness approaches without any additional training.
%As presented in Figure~\ref{fig:teaser}, ERM with the robust scaling achieves competitive performance compared to the recent group distributionally robust optimization approaches with no additional training.
%From this point on, we argue that comparing only the robust accuracy without considering the average accuracy should be regarded as incomplete.


%Furthermore, we propose a more advanced robust scaling algorithm, which apply a class-specific scaling strategy to each example based on its cluster membership, to overcome the trade-off and achieve performance gains in both the accuracies.
%Those methods are referred to as attribute- and cluster-specific robust scaling, respectively, where the former requires group supervision while the latter is an unsupervised method.
%The two extensions provide more flexible trade-offs and match or even outperform the state-of-the-art models.
%Our approach is effective to capture the full landscape of the Pareto optimal points by adjusting non-uniform scaling factors and identify an appropriate model for robust prediction compared to the selection by simple hyper-parameter tuning.  

\vspace{-2mm}
\paragraph{Contribution} 
%We present a simple but effective approach for bias control by the analysis of trade-off between robust and average accuracies.
We propose a simple yet effective approach for group robustness by analyzing the trade-off between robust and average accuracies.
Our framework captures the complete landscape of robust-average accuracy trade-offs, facilitates understanding the behavior of existing debiasing techniques, and enables optimization of arbitrary objectives along the trade-off curve without additional training.
We emphasize that our framework does not solely focus on performance improvement in robust accuracy; more importantly, \textbf{our method not only highlights the inherent trade-offs in existing debiasing approaches but also facilitates the identification of desired performance points based on target objectives, paving the way for accurate, fair, and comprehensive evaluations of group robustness.}
%We believe our framework will help guide future research in the right direction.
%is effective to understand the exact behavior of existing debiasing approaches by effectively capturing the overall landscape of trade-off, and find the desired performance points with a single model without further training.
%
Our main contributions are summarized as follows.%     
%\vspace{-1mm}

 \begin{itemize} 
     \item[$\bullet$] We propose a training-free class-specific scaling strategy to capture and control the trade-off between robust and average accuracy with negligible computational cost. 
     %This approach plays a crucial role for comprehensive evaluation and advance of algorithm performance.
     This approach allows us to optimize a debiasing algorithm towards arbitrary objectives within the trade-off, building on top of any existing models. %\vspace{-1mm}

    \item[$\bullet$] We develop an instance-wise robust scaling algorithm by extending the original class-specific scaling with joint consideration of feature clusters. This technique is effective to alleviate the trade-off and improve both robust and average accuracy. %\vspace{-1mm}
    
    \item[$\bullet$] We introduce a novel comprehensive and unified performance evaluation metric based on the robust scaling method, which summarizes the trade-off as a scalar value from the Pareto optimal perspective. %\vspace{-1mm}
%    , which provides an accurate and straightforward performance evaluation of group robust optimization methods. \vspace{0.1cm}
%    \item[$\bullet$] We empirically show the trade-off between robust and average accuracies, which is often ignored in the existing robust optimization techniques, and argue that two accuracies should be considered simultaneously.
%    \item[$\bullet$] Unlike previous works, we argue that focusing only on the robust accuracy without considering the average accuracy is not sufficient.
%    \item[$\bullet$] We empirically show that existing robust optimization techniques achieve high robust accuracy at the expense of average accuracy, but such trade-off is often ignored.
%    , and argue that focusing only on the robust accuracy without considering the average accuracy should be regarded as incomplete.
%    \item[$\bullet$] We present two extensions of the robust scaling---with and without group supervision---which consider underlying attributes and clusters, respectively; they provide better flexibility to control the trade-offs and have the potential to further improve performance.
%    \item[$\bullet$] We propose a more sophisticated robust scaling method that can overcome trade-offs to improve both robust and average accuracies further.

    \item[$\bullet$] The extensive experiments analyze the characteristics of existing methods and validate the effectiveness of our frameworks on the multiple standard benchmarks.
\end{itemize}

%The rest of the paper is organized as follows.
%We review the prior works in Section~\ref{sec:related} and preliminaries in Section~\ref{sec:prelim}.
%Section~\ref{sec:robust_scaling} presents our proposed post-processing methods for group robustness and introduces a new measurement for its comprehensive evaluation.
%We validate their effectiveness in Section~\ref{sec:exp}.
%We conclude our paper in Section~\ref{sec:conclusion}.



 





\begin{figure}
    \centering
    \includegraphics[scale=0.71, clip, trim=0.6cm 6.2cm 1.5cm 1.0cm]{img/rs_mcts.pdf}
    \caption{
    Automatically curating preference pairs via rejection sampling (RS, left) and Monte Carlo Tree Search (MCTS, right).
    \textbf{RS}: We independently sample $N$ different outputs from the policy, score each output with a verifier and take (high, low) scoring responses as the (chosen, rejected) pairs.
    \textbf{MCTS}: We perform tree search with the policy while generating multiple actions per each search iteration.
    Then, we use the rollouts from sibling nodes with (high, low) reward scores as the (chosen, rejected) pairs to obtain preference pairs with common prefixes up to the parent nodes.
    }
    \label{fig:rs-mcts}
\end{figure}

\section{Background}
\label{section:background}
\textbf{Preference Learning.}
Preference learning is a technique that is used to align LLMs during their post-training phase, involving pairs of (chosen, rejected) responses in the training dataset.
It aligns LLMs by steering them towards generating the chosen responses and away from generating the rejected responses.
The Bradley-Terry model~\citep{DBLP:journals/corr/bradleyterry1952} provides the probabilistic framework for preference learning by modeling the pairwise comparison between two responses ($y_1$, $y_2$) provided by the LLM to a given prompt $x$:
$$p(y_1 \succ y_2 | x) = \frac{\text{exp}(r^*(x, y_1))}{\text{exp}(r^*(x, y_1)) + \text{exp}(r^*(x, y_2))}$$
Common methods for preference learning such as Direct Preference Optimization (DPO,~\cite{DBLP:conf/nips/RafailovSMMEF23}) directly optimize the model to update its parameters to increase the likelihood of generating the chosen response over the rejected response.
Meanwhile, other methods such as Proximal Policy Optimization (PPO,~\cite{DBLP:journals/corr/SchulmanWDRK17}) indirectly perform this optimization by first training a reward model to assign scores corresponding to the preferences in the training data, and then optimizing a policy model with the guidance of the reward model.
Both approaches have been instrumental in aligning LLMs with human preferences and enhancing their capabilities for a wide array of downstream tasks.

\textbf{Data Curation for Preference Learning.} 
The success of preference learning for LLM post-training has naturally led researchers to propose methods for automatically curating preference pairs designed to further boost model capabilities~\citep{DBLP:conf/icml/YuanPCLSXW24, DBLP:conf/naacl/KhakiLMYR24, DBLP:journals/corr/abs-2405-00451}.
Such methods assign scores to LLM-generated outputs using verifiers to determine which outputs should be preferred during training. 
% -- this explains why many preference data curation methods specifically address mathematics, as it contains ground-truth labels that are easy and quick to verify.
While various other methods have been proposed for curating preference data~\citep{DBLP:journals/corr/abs-2402-11411, DBLP:journals/corr/abs-2404-02078, DBLP:journals/corr/abs-2406-18629}, in this paper we focus on two popular methods: rejection sampling (RS) and Monte Carlo Tree Search (MCTS).
% A common method to obtain preference pairs is to use rejection sampling (RS).
During rejection sampling, the policy model generates $N$ independent responses to the given prompt and a verifier scores each response according to some evaluation metric or a reward model.
Responses with (high, low) scores are selected as the (chosen, rejected) responses, respectively.
% Another method for preference pair curation is Monte Carlo Tree Search (MCTS).
Meanwhile, in the MCTS framework, the policy model performs tree search for the given prompt by generating a fixed number of tokens at each iteration and builds the tree with nodes that represent each subsequence of generated tokens.
During MCTS, the policy model performs rollouts by generating full responses and backpropagates the reward scores assigned to the rollouts.
This results in a tree of possible responses generated for a given prompt, with each node being assigned a Q-value which measures the quality of the response generated so far.
Here, sibling nodes with sufficient differences in Q-values or reward scores are selected such that the preference pairs contain common prefixes and the suffixes account for the quality difference between the two responses.

While such methods return preference pairs that are effective for alignment and capability improvements, there is a lack of studies into exactly \textit{how} the preference pairs should be curated based on these methods.
In this work we perform a systematic investigation of how different characteristics of preference datasets affect downstream performance of LLMs, using instruction following accompanied with verifiable constraints as our task of interest.
We choose instruction-following as our task in order to incorporate multiple constraints into the prompt and score the response on a fine-grained level based on the ratio of constraints that are satisfied. 
We use verifiable constraints to assign quality scores to our responses in a reliable and efficient manner.

\section{Prompt Synthesis}
\label{section:prompt_synthesis}
\begin{table}
    \small
    \centering
    \begin{tabularx}{\textwidth}{llX}
        \toprule
        constraint & keyword args & description \\ \midrule
        \texttt{alliteration} & \texttt{num\_alliteration\_words} & the response should contain an alliteration, i.e. a sequence of X words starting with the same letter, where X $=$ \texttt{num\_alliteration\_words}.\\
        \texttt{ascending\_num\_words} & \texttt{n/a} & the response should contain sentences such that the number of words in each sentence is in ascending order.\\
        \texttt{max\_word\_length} & \texttt{max\_word\_length} & the maximum length of all words in the response should be at most X characters, where X $=$ \texttt{max\_word\_length}.\\
        \texttt{num\_words\_per\_sentence} & \texttt{relation}, \texttt{num\_words} & each sentence in the response should contain R $\in \{\text{at least, at most}\}$ X words, where R $=$ \texttt{relation} and X $=$ \texttt{num\_words}.\\
        \texttt{required\_sentence} & \texttt{sentence} & the response should contain a sentence S, where S $=$ \texttt{sentence}. \\
        \texttt{tldr\_summary} & \texttt{n/a} & the response should end with a “TL;DR” on a new line summarizing the response.\\
        \bottomrule
    \end{tabularx}
    \caption{
    Examples of verifiable constraints used for our synthetic prompts.
    Some constraints require one or more keyword arguments that materialize the constraint for its associated prompt, while others do not require any argument.
    }
    \label{tab:constraint_examples}
\end{table}

We perform all our experiments with a new set of synthetic prompts that incorporate mixtures of verifiable constraints.
These prompts allow us to evaluate the qualities of the generated responses in both a consistent and fine-grained manner, providing a suitable environment for us to control the attributes of the preference dataset and investigate their impact on downstream performance.

\subsection{Instruction-Following with Verifiable Constraints}
\label{subsection:constraint-ontology}
Our verifiable constraints resemble but are distinct from the set of constraints provided in \texttt{IFEval}~\citep{DBLP:journals/corr/abs-2311-07911}.
We define 23 constraints which can be deterministically verified using code, spanning ones that check for adherence to specific structural, stylistic, or formatting requirements.
The complete ontology of our constraints is provided in Table~\ref{tab:constraint_all} in the appendix.

We design our verifiable constraints such that they follow the formatting of the verifiable constraints provided in \texttt{IFEval} for unified evaluation -- each constraint is accompanied by a set of keyword arguments that are needed to actualize the constraint for the given prompt.
Refer to Table~\ref{tab:constraint_examples} for examples of such constraints and their associated keyword arguments.
For example, the \texttt{alliteration} constraint is paired with a keyword argument \texttt{num\_alliteration\_words}, which indicates the number of words that must display the alliteration.
Some constraints such as \texttt{tl;dr\_summary} do not contain any keyword arguments as they are self-explanatory and do not need any further specifications.

\begin{figure}
    \centering
    \includegraphics[scale=0.67, clip, trim=0.8cm 0.6cm 1.2cm 3.0cm]{img/prompt_synthesis.pdf}
    \caption{
    Overview of our pipeline for generating synthetic prompts with verifiable constraints.
    We first take a set of seed prompts from an existing dataset where the prompts contain constraints, and remove all constraints with an LLM (\texttt{llama-3.1-70b-instruct}) to obtain base prompts corresponding to the original dataset.
    Next, we randomly sample a small subset of the base prompts and use them as few-shot examples to generate new prompts \textit{without any constraints}.
    We remove duplicates among the newly-generated prompts and the existing base prompts using a sentence transformer.
    Then, we randomly sample a combination of $k \in \{4,5,6\}$ of our verifiable constraints that are non-conflicting and use an LLM to generate the input parameters required for the set of selected constraints.
    Finally, we use the resulting input kwargs and the new base prompts to generate the final prompts that integrate the constraints in natural language.
    }
    \label{fig:prompt-synthesis}
\end{figure}

\subsection{Prompt Generation}
\label{subsection:prompt-generation}
We build a pipeline that generates synthetic prompts for instruction-following, incorporating combinations of different verifiable constraints to create a set of challenging prompts that stress-test models' capabilities of following complex instructions.
Our pipeline is similar to \texttt{Instruct-SkillMix}~\citep{DBLP:journals/corr/abs-2408-14774} in terms of mixing skills, except that we have a preexisting set of verifiable constraints functioning as the ``skills'' and incorporate additional layers for generating the keyword arguments before generating the final instructions.
In terms of scalability, our pipeline does not require human supervision and generates synthetic prompts only using \texttt{llama-3.1-70b-instruct}~\citep{DBLP:journals/corr/abs-2407-21783}.

Figure~\ref{fig:prompt-synthesis} provides a visual summary of our synthetic prompt generation pipeline.
Our data generation pipeline can be decomposed into five main processes.
First, we take a set of seed instruction-following prompts (e.g., from \texttt{IFEval}) and remove the additional constraints originally associated with the dataset.
We few-shot prompt \texttt{llama-3.1-70b-instruct} with examples of additional constraints being removed and obtain a set of prompts from the seed dataset without constraints.
Second, we take the base prompts and use \texttt{llama-3.1-70b-instruct} to generate new prompts in their base forms, taking an approach similar to \texttt{Self-Instruct}~\citep{DBLP:conf/acl/WangKMLSKH23} by using the existing base prompts as few-shot examples and prompting the model to generate 20 new prompts at a time.
Third, we remove the newly generated prompts that are semantically duplicates either with any of the seed set of base prompts or any of the other new prompts.
We use \texttt{all-mpnet-base-v2}~\citep{DBLP:conf/nips/Song0QLL20}, a lightweight sentence transformer, to compute semantic embeddings and use the dot product to compute similarity scores.
Fourth, we randomly sample a combination of $k$ constraints which do not contradict each other, and for each constraint in this mixture, we randomly sample or generate the associated keyword arguments.
We perform random sampling for keyword arguments that can be randomly chosen without considering the prompt such as the \texttt{relation} argument in the \texttt{num\_words\_per\_sentence} constraint or the \texttt{num\_alliteration\_words} argument in the \texttt{alliteration} constraint.
Meanwhile, we use \texttt{llama-3.1-70b-instruct} to generate the keyword arguments that require more contextual understanding of the prompt, such as the \texttt{sentence} argument for the \texttt{required\_sentence} constraint.
Fifth, we take the set of constraints along with the keyword arguments chosen for each prompt and use \texttt{llama-3.1-70b-instruct} to rewrite the base prompts into their final forms which integrate the constraints and their keyword arguments in natural language.

We use our data generation pipeline to generate synthetic prompts with mixtures of verifiable constraints such that we can score the quality of any arbitrary response to the given prompt by using the constraints and their keyword arguments along with a deterministic evaluation code.
In contrast to \texttt{IFEval} which integrates at most $k=3$ constraints into the instruction, our pipeline allows us to integrate any number of constraints to create synthetic prompts, resulting in more challenging prompts and leaving room for a more diverse array of quality of responses in our preference datasets.
We use $k \in \{4,5,6\}$ in our experiments.
Our choice of using relatively higher values of $k$ is to make our synthetic training prompts maximally distinct from \texttt{IFEval}~\citep{DBLP:journals/corr/abs-2311-07911} for more reliable evaluation, and to perform evaluations on challenging, out-of-distribution test prompts with equally high values of $k$.
Refer to Table~\ref{tab:prompt_examples} in the appendix for examples of our synthetic prompts.

\subsection{Prompt Information}
\label{subsection:prompt-info}
We generate instruction-following prompts for combinations of $k\in\{4, 5, 6\}$ constraints.
Table~\ref{tab:prompt_statistics} shows the statistics that describe the prompts that we generate.
For each value of $k$, we generate about 16K synthetic prompts, resulting in a total of about 48K prompts being used in our experiments.
The average length of the prompts increases as the number of constraints increases, as the complexity of the prompts increases with larger numbers of constraints.

Table~\ref{tab:prompt_examples} displays examples of synthetic prompts generated using our pipeline for a given combination of verifiable constraints.
Each prompt contains a general-purpose base instruction such as writing a short story or a speech, and is accompanied by a combination of verifiable constraints defined in our ontology with keyword arguments that satisfy the context of the instruction -- for example, the first example uses the word "shouting" for the \texttt{nth\_sent\_first\_word} constraint, which is appropriate for the context of the base instruction involving a boy lost in a shopping mall.
For any arbitrary response to a given prompt, we can use the associated constraints and keyword arguments to automatically assign a score indicating whether the response follows each constraint and aggregate the scores to assign an overall correctness score to the response.

\begin{table}
    \centering
    \resizebox{\textwidth}{!}{
    \begin{tabular}{ccc|ccc|ccc}
    \toprule
        \multicolumn{3}{c}{$k=4$} & \multicolumn{3}{c}{$k=5$} & \multicolumn{3}{c}{$k=6$} \\\midrule
        {\scriptsize \texttt{num\_prompts}} & {\scriptsize \texttt{mean\_words}} & {\scriptsize \texttt{std\_words}} & {\scriptsize \texttt{num\_prompts}} & {\scriptsize \texttt{mean\_words}} & {\scriptsize \texttt{std\_words}} & {\scriptsize \texttt{num\_prompts}} & {\scriptsize \texttt{mean\_words}} & {\scriptsize \texttt{std\_words}} \\\midrule
        \small 15,900 & \small 70.62 & \small 14.37 & \small 15,739 & \small 84.17 & \small 15.61 & \small 15,559 & \small 97.95 & \small 16.91 \\
    \bottomrule
    \end{tabular}
    }
    \caption{
    Statistics of synthetic prompts generated by our pipeline.
    We generate about 16K prompts for each value of $k$, resulting in 48K prompts total across all our experiments.
    Note that \texttt{num\_prompts} refers to the number of prompts, \texttt{mean\_words} refers to the average number of words in each prompt, and \texttt{std\_words} refers to the standard deviation of the number of words in each prompt.
    The number of words in each prompt increases with the number of constraints.
    }
    \label{tab:prompt_statistics}
\end{table}

\begin{table}
    \centering
    \small
    \begin{tabularx}{\textwidth}{p{5.1cm}X}
        \toprule
        constraints & prompt \\\midrule
        \texttt{ascending\_num\_words}, \texttt{freq\_long\_words}, \texttt{max\_word\_length}, \texttt{nth\_sent\_first\_word}, \texttt{start\_checker} & Write a short story about a boy who gets lost in a shopping mall. Include at least 7 words that are at least 12 characters long, and ensure that the sentences have an increasing number of words, i.e. each sentence should contain more words than its previous one. Also, only include words that are at most 12 characters long. Make sure that the fifth sentence starts with the word "shouting", and begin your response with the sentence "As the sounds of loud chatter and clinking of dishes filled the food court, little Tommy suddenly discovered that his parents were nowhere to be seen.". \\
        \\
        \texttt{nth\_sent\_first\_word}, \texttt{num\_bold\_words}, \texttt{num\_exclamations}, \texttt{tldr\_summary}, \ \texttt{vowel\_capitalization} & Write a motivational speech for a high school graduation ceremony. Capitalize the vowels in your response, and include seven words that are bolded in HTML format (e.g., <b>word</b>). Also, ensure that the sixth sentence starts with the word "today". Make sure that the response contains exactly three exclamation marks, and finish the response with the final line including "TL;DR" followed by a one-sentence summary of your response. \\
        \bottomrule
    \end{tabularx}
    \caption{
    Examples of synthetic prompts generated for $k=5$, which contains a combination of five verifiable constraints.
    }
    \label{tab:prompt_examples}
\end{table}

\section{Preference Data Curation}
\label{section:preference_curation}
Using the prompts obtained in Section~\ref{section:prompt_synthesis}, we generate responses and extract preference data using the aggregated correctness scores of the responses.
Here, we assign high-scoring responses as the chosen responses and low-scoring responses as the rejected responses.
We employ two methods for preference data curation: rejection sampling (RS) and Monte Carlo Tree Search (MCTS).
Rejection sampling presents a straightforward and efficient way to obtain preference data but the (chosen, rejected) responses do not share any relationship.
On the other hand, MCTS is more expensive and slower to run, but returns (chosen, rejected) responses that share a prefix with more nuanced contrast.
We use two contrasting approaches to generate diverse types of preference data and additionally investigate the effect of having common prefixes in preference pairs.

\textbf{Rejection Sampling.}
Refer to Figure~\ref{fig:rs-mcts} (left) for a visual overview. 
We first set a filtering criteria, where the chosen response must achieve a score of $\mathbf{c}$ according to our verifier and the rejected response must achieve a score of $\mathbf{r}$.
Then, we generate $N$ different responses independently with the policy model and score each response with our verifier.
Given a prompt $x$, its associated set of verifiable constraints $\mathcal{C}$, the response $r$ and our verifier $\mathcal{V}$ which verifies whether the response satisfies any given constraint $c$, we compute the score as
$$R(r|x, \mathcal{C}) = \frac{1}{\|\mathcal{C}\|}\sum_{c\in\mathcal{C}}\mathcal{V}(r|x, c)$$
We extract all preference pairs such that 1) the chosen score equals $\mathbf{c}$, 2) the rejected score equals $\mathbf{r}$, and 3) there are no overlapping responses between the extracted preference pairs.
Using rejection sampling offers a simple method for automatically curating preference pairs, but it returns pairs that are independently sampled with no common structures between the chosen and rejected responses.

\textbf{Monte Carlo Tree Search.}
Refer to Figure~\ref{fig:rs-mcts} (right) for a visual overview.
We conduct Monte Carlo Tree Search (MCTS) with a granularity level of token sequences -- for a given prompt $x$ and the MCTS tree $T$, each node $s_i$ in $T$ represents a partial response generated for $x$, and each edge $(s_i, s_j)$ in $T$, also known as an \textit{action}, represents a sequence of tokens generated from $s_i \rightarrow s_j$.
In this setup, we use an LLM $\Pi$ as the policy and our verifier as the outcome reward model via three major steps: selection, expansion and backpropagation.
% Figure~\ref{} provides a visualization of each of the four steps.

\paragraph{Selection.}
Given the current node $s_t$ and $K$ different actions $(a_1, ..., a_K)$ generated by the policy from $s_t$, we balance exploitation and exploration to select the next node for tree search.
Each action is a sequence of tokens under a pre-specified maximum length.

Our selection depends on $Q(s_t, a)$ and $N(s_t, a)$, the Q-value and visit count of each subsequent node reached by taking action $a$ from $s_t$, respectively.
We use the Predictor+Upper Confidence bounds applied to Trees (PUCT) and select the next node $s_{t+1}$ according to the following formula:
$$s_{t+1}^* = \argmax_{(s_{t+1} = s_t\rightarrow a_i)}\left[Q(s_t,a_i)+c_{\text{puct}}\cdot \Pi(a_i|s_t)\frac{\sqrt{N(s_t)}}{1 + N(s_t, a_i)}\right]$$
Refer to Appendix~\ref{sec:mcts_details} for more information on how to compute the policy score $\Pi(a_i|s_t)$.
Using PUCT, we prioritize exploration during the initial stages of tree building when the visit counts have low values, and prioritize exploitation during the later stages of tree building as the visit counts increase and effectively weigh the Q-values more for scoring.
This results in a balanced trade-off during our tree search.

\paragraph{Expansion.}
We perform expansion from the current node $s_t$ and generate $K$ new actions with the policy $\Pi$.
For each new action, we perform $M$ separate rollouts and score each rollout using a linear combination of the score $\mathcal{V}$ assigns and the self-evaluation score assigned by $\Pi$.
We use self-evaluation, denoted as $\Pi_{\text{self-eval}}(s_t)$, in addition to the verifier scores, to provide step-level feedback during the tree search. Again, refer to Appendix~\ref{sec:mcts_details} for more information on how to compute $\Pi_{\text{self-eval}}(s_t)$.
$$R(s_t) = (1 - \lambda)\cdot \left[\frac{1}{M\|\mathcal{C}\|}\sum_{i\in [1...M]}\sum_{c\in\mathcal{C}}\mathcal{V}(\Pi_{\text{rollout}}(s_t)|x, c)\right] + \lambda \cdot \Pi_{\text{self-eval}}(s_t)$$
After assigning the reward scores to each rollout, we average the scores across the rollouts for each new action $a_i$ for $i \in [1,\ldots,K]$ and add the node $s_{t+1} = (s_t\rightarrow a_i)$, with the averaged reward score, to the tree.

\paragraph{Backpropagation.}
After computing the reward scores for the rollouts and adding new nodes, we backpropagate the reward scores through the parent nodes.
We increment the visit counts and directly update the Q-values of a given (state, action) pair based on the Q-values and the visit counts of the children nodes of $s_{t+1} = (s_t, a)$, as the following:
$$N(s_t) = N(s_t) + 1$$
$$Q(s_{t}, a) = \frac{\sum_{i=1}^KQ(s_{t+1}, a_i)\cdot N(s_{t+1}, a_i) + R(s_{t+1})}{\sum_{i=1}^KN(s_{t+1}, a_i) + 1}$$

We repeat the threefold process over multiple iterations for each root node, and we traverse down the tree while switching the root node until we reach a terminal node.

Once the trees are constructed, we curate preference data from pairs of sibling nodes, i.e. nodes that share the same parent node.
For pairs of non-leaf sibling nodes that satisfy the correctness criteria in the tree, we sample their rollouts to obtain complete responses.
We use the same criteria for our data curation as that of rejection sampling -- we set a filtering criteria in which the chosen response must score $\mathbf{c}$ and the rejected response must score $\mathbf{r}$, and sample all pairs with no overlapping responses.
Note that we use the scores assigned by the verifier $\mathcal{V}$ only, and not the policy's self-evaluation, in order to ensure that the correctness of the (chosen, rejected) responses stay consistent across the RS- and MCTS-based curation methods.
% As the pairs share the same parent node, the (chosen, rejected) responses share a common prefix leading up to the parent node and the difference occurs in the suffixes that follow.

\begin{figure}
    \centering
    \includegraphics[scale=0.72, clip, trim=0.1cm 8.2cm 4.5cm 0.1cm]{img/preference_pair_stats.pdf}
    \caption{
    Number of preference pairs for different correctness filtering criteria at $k=5$.
    The light blue color indicates preference pairs obtained via rejection sampling (RS), and the dark blue color indicates preference pairs obtained via Monte Carlo Tree Search (MCTS).
    The left subfigure shows the number of unique prompts with (chosen, rejected) responses associated with each filtering criteria, and the right subfigure shows the total number of preference pairs with (chosen, rejected) responses associated with each filtering criteria.
    }
    \label{fig:preference-pair-stats}
\end{figure}

\textbf{Preference Pair Statistics.}
We curate preference pairs using both RS- and MCTS-based methods for our synthetic prompts with $k \in \{4,5,6\}$.
Figure~\ref{fig:preference-pair-stats} shows the number of preference pairs obtained via both methods for $k=5$ when the correctness criteria for (chosen, rejected) pairs is set as (5 correct, 0 correct), (5 correct, 1 correct), (5 correct, 2 correct), (5 correct, 3 correct) and (5 correct, 4 correct) pairs.
The left subfigure depicts the number of unique prompts corresponding to the preference pairs extracted for each filtering criteria.
Using rejection sampling (RS) returns a higher yield of preference pairs with high contrast between the (chosen, rejected) responses, while using Monte Carlo Tree Search (MCTS) returns a higher yield of preference pairs with low contrast between the responses.
The right subfigure depicts the total number of preference pairs extracted for each criteria -- the same observation can be made about the relative yield, with MCTS yielding a large number of preference pairs due to its tree structure.
We use preference pairs collected for $k \in \{4,5,6\}$ using both methods over different filtering criteria to perform our experiments.

\section{Experiments and Results}
\label{section:experiments}
\section{experiments}
\subsection{Experimental Settings}
\subsubsection{Dataset}
We conduct extensive experiments on three real-world mobility datasets from Shanghai, Senegal, and Xinjiang. Each dataset includes both trajectory and flow data. The details of datasets are summarized in Appendix~\ref{sec::datasets_info}. The experiment section only presents the results for the Shanghai and Senegal datasets. Detailed results for the Xinjiang dataset can be found in Appendix~\ref{sec::Results}.


\subsubsection{Baselines}

We compare the performance of our model with state-of-the-art baselines. Previous methods could only accomplish one type of mobility data prediction task, so the baseline methods are divided into trajectory and flow prediction. 
For \textit{Flow Prediction}, we compared our model with six SOTA baselines (\textbf{HA}~\cite{sun2020predicting}, \textbf{VAR}~\cite{lu2016integrating}, \textbf{ST-ResNet}~\cite{zhang2017deep}, \textbf{MSDR}~\cite{liu2022msdr}, \textbf{STID}~\cite{shao2022spatial}, and \textbf{PriSTI}~\cite{liu2023pristi}.
For \textit{Trajectory Prediction}, we compared our model with seven SOTA baselines (\textbf{Markov}~\cite{gambs2012next}, \textbf{LSTM}~\cite{Kong2018HST}, \textbf{DeepMove}~\cite{feng2018deepmove}, \textbf{STAN}~\cite{luo2021stan}, \textbf{SNPM}~\cite{yin2023next},
\textbf{TrajGDM}~\cite{chu2024simulating}, and \textbf{GETNext}~\cite{yang2022getnext}.
We provide the details of baselines in Appendix~\ref{sec::baselines}.



\subsubsection{Metrics}
For trajectory prediction, we use \textit{Accuracy@k} to sort candidate locations by model-predict probabilities and check if the true position falls within the top k predictions.
For flow prediction, we choose mean absolute errors (\textit{MAE}), Mean Absolute Percentage Error (\textit{MAPE}), and root mean squared errors (\textit{RMSE}) as the evaluation metrics. \textit{MAE} is the mean absolute error between predicted and ground truth values. \textit{MAPE} is the mean absolute percentage error between the predicted and ground truth values. \textit{RMSE} is the square root of the mean squared error between the predicted and ground truth values. %Combining these three metrics provides a comprehensive evaluation of the model's performance.



\begin{table*}[t]
\centering
\caption{Overall Performance on Shanghai and Senegal datasets.}
\vspace{-0.3cm}
\scalebox{0.78}{ % Scale the table to fit within the page width
\begin{tabular}{lcccccccccccc}
\toprule
& \multicolumn{6}{c}{\textbf{Shanghai Dataset}} & \multicolumn{6}{c}{\textbf{Senegal Dataset}} \\
\cmidrule(lr){2-7} \cmidrule(lr){8-13}
& \multicolumn{3}{c}{\textbf{Flow Prediction}} & \multicolumn{3}{c}{\textbf{Trajectory Prediction}} & \multicolumn{3}{c}{\textbf{Flow Prediction}} & \multicolumn{3}{c}{\textbf{Trajectory Prediction}}\\
\cmidrule(lr){2-4} \cmidrule(lr){5-7} \cmidrule(lr){8-10} \cmidrule(lr){11-13}
\textbf{} & \textbf{MAE} & \textbf{MAPE(\%)} & \textbf{RMSE} & \textbf{Acc@1} & \textbf{Acc@3} & \textbf{Acc@5} & \textbf{MAE} & \textbf{MAPE(\%)} & \textbf{RMSE} & \textbf{Acc@1} & \textbf{Acc@3} & \textbf{Acc@5} \\
\midrule
HA & 35.19 & 29.76 & 42.72 & - & - & - & 20.75 & 19.32 & 31.17 & - & - & - \\
VAR & 28.25 & 25.61 & 40.14 & - & - & - & 17.20 & 15.95 & 28.43 & - & - & - \\
ST-ResNet & 21.54 & 19.02 & 34.18 & - & - & - & 15.95 & 13.84 & 26.37 & - & - & - \\
MSDR & 20.01 & 17.84 & 32.63 & - & - & - & 14.08 & 13.26 & 25.04 & - & - & - \\
STID & 18.72 & 15.17 & 30.40 & - & - & - & 13.52 & 12.31 & 23.19 & - & - & - \\
PriSTI & \underline{18.40} & \underline{14.59} & \underline{29.71} & - & - & - & \underline{13.28} & \underline{12.15} & \underline{22.80} & - & - & - \\
Markov & - & - & - & 0.2825 & 0.3986 & 0.5012 & - & - & - & 0.3894 & 0.4418 & 0.5828 \\
LSTM & - & - & - & 0.3401 & 0.4298 & 0.5737 & - & - & - & 0.4573 & 0.5185 & 0.6509 \\
DeepMove & - & - & - & 0.3813 & 0.4672 & 0.6191 & - & - & - & 0.4980 & 0.5764 & 0.7125 \\
STAN & - & - & - & 0.3975 & 0.4746 & 0.6303 & - & - & - & 0.5105 & 0.6042 & 0.7303 \\
SNPM & - & - & - & 0.4012 & 0.4797 & 0.6378 & - & - & - & 0.5236 & 0.6260 & 0.7591 \\
GETNext  & - & - & - & 0.4063 & 0.4836 & 0.6415 & - & - & - & 0.5251 & 0.6287 & 0.7638 \\
TrajGDM & - & - & - & \underline{0.4103} & \underline{0.4875} & \underline{0.6434} & - & - & - & \underline{0.5295} & \underline{0.6302} & \underline{0.7674} \\
UniMob-v1 & 17.93 & 14.01 & 28.65 & 0.4205 & 0.5024 & 0.6570 & 12.70 & 11.65 & 21.94 & 0.5403 & 0.6412 & 0.7889 \\
UniMob-v2 & 17.89 & 13.98 & 28.60 & 0.4228 & 0.5057 & 0.6615 & 12.52 & 11.50 & 21.57 & 0.5439 & 0.6450 & 0.7924 \\
UniMob-v3 & 17.90 & 13.96 & 28.63 & 0.4213 & 0.5040 & 0.6593 & 12.61 & 11.59 & 21.73 & 0.5415 & 0.6436 & 0.7907 \\
UniMob-v4 & \textbf{17.76} & \textbf{13.93} & \textbf{28.50} & \textbf{0.4267} & \textbf{0.5091} & \textbf{0.6653} & \textbf{12.08} & \textbf{11.12} & \textbf{21.03} & \textbf{0.5486} & \textbf{0.6515} & \textbf{0.7993} \\
%Improvement & 3.48\% & 4.52\% & 4.07\% & 4.00\% & 4.43\% & 2.81\% & 9.04\% & 8.48\% & 7.76\% & 3.61\% & 3.38\% & 4.16\% \\
\bottomrule
\end{tabular}
}
\label{tab:two datasets}
\end{table*}


\begin{table*}[t]
\small
\centering
\caption{Ablation study on Shanghai datasets.}
\vspace{-0.3cm}
\scalebox{1.}{
\begin{tabular}{lcccccc}
\toprule
& \multicolumn{3}{c}{\textbf{Trajectory Prediction}} & \multicolumn{3}{c}{\textbf{Flow Prediction}} \\
\cmidrule(lr){2-4} \cmidrule(lr){5-7}
& \textbf{Acc@1} & \textbf{Acc@3} & \textbf{Acc@5}
& \textbf{MAE} & \textbf{MAPE(\%)} & \textbf{RMSE}\\
\midrule
Ours & 0.4205 & 0.5024 &  0.6570 & 17.93 & 14.01 & 28.65 \\
w/o I2C loss & 0.4165 (-0.95\%) & 0.4906 (-2.35\%) &  0.6487 (-1.26\%) & 18.48 (-2.98\%) & 14.76 (-5.35\%) & 29.91 (-4.21\%) \\
w/o C2I loss & 0.4053 (-3.61\%) & 0.4849 (-3.48\%) &  0.6442 (-1.95\%)
 & 18.27 (-1.86\%) & 14.41 (-2.78\%) & 29.25 (-2.05\%)\\
w/o shared transformer & 0.4115 (-2.14\%) & 0.4882 (-2.83\%) & 0.6461 (-1.66\%) & 18.40 (-2.55\%)
 & 14.60 (-4.21\%) & 29.67 (-3.44\%) \\
w/o flow data & 0.4036(-4.02\%) & 0.4840(-3.66\%) & 0.6421(-2.27\%) & - & - & - \\
w/o trajectory data & - & - & - & 18.56(-3.51\%) & 14.86(-6.07\%) & 30.02(-4.78\%) \\
\bottomrule
\end{tabular}
}
%\vspace{-0.3cm}
\label{tab:Ablation1}
\end{table*}



\begin{table*}[t]
\small
\centering
\caption{Ablation study on Senegal datasets.}
\vspace{-0.3cm}
\scalebox{1.}{
\begin{tabular}{lcccccc}
\toprule
& \multicolumn{3}{c}{\textbf{Trajectory Prediction}} & \multicolumn{3}{c}{\textbf{Flow Prediction}} \\
\cmidrule(lr){2-4} \cmidrule(lr){5-7}
& \textbf{Acc@1} & \textbf{Acc@3} & \textbf{Acc@5}
& \textbf{MAE} & \textbf{MAPE(\%)} & \textbf{RMSE}\\
\midrule
Ours & 0.5403 & 0.6412 &  0.7889 & 12.70 & 11.65 & 21.94 \\
w/o I2C loss & 0.5371 (-0.59\%)
 & 0.6356 (-0.87\%)
 & 0.7620 (-3.41\%)
 & 13.32 (-4.88\%)
 & 12.12 (-4.03\%)
 & 22.91 (-4.42\%) \\
w/o C2I loss & 0.5285(-2.18\%)
 & 0.6327 (-1.33\%)
 & 0.7476 (-5.24\%)
 & 12.89 (-1.50\%)
 & 11.73 (-0.69\%)
 & 22.10 (-0.73\%)
 \\
w/o shared transformer & 0.5314 (-1.65\%)
 & 0.6331 (-1.26\%)
 & 0.7538 (-4.45\%)
 & 13.20 (-3.94\%)
 & 11.97 (-2.75\%)
 & 22.62 (-3.10\%)
 \\
w/o flow data & 0.5262(-2.61\%) & 0.6297(-1.80\%)
 & 0.7548(-4.32\%) & - & - & - \\
w/o trajectory data & - & - & - & 13.40(-5.51\%)
 & 12.18(-4.55\%) & 22.98(-4.74\%) \\
\bottomrule
\end{tabular}
}
%\vspace{-0.3cm}
\label{tab:Ablation3}
\end{table*}


\subsection{Overall Performance}
As shown in Tables~\ref{tab:two datasets}, our method demonstrates similar or better performance than the state-of-the-art baselines for all tasks on Shanghai and Senegal datasets (Please refer to Table~\ref{tab:Xinjiang} in Appendix~\ref{sec::Overall Performance} for Xingjiang dataset). We conducted multiple experiments and reported the average performance.
In flow and trajectory prediction tasks conducted on multiple real-world datasets, our UniMob model demonstrated the best performance across all evaluation metrics. Specifically, it achieved a performance improvement of over 6\% in flow prediction and 3.73\% increase in trajectory prediction.
Additionally, compared to other baseline methods, only our model can simultaneously perform flow and trajectory predictions, demonstrating that our model design effectively achieves unified human mobility prediction.
Furthermore, we used four model variants for each task. Each variant outperformed other baseline methods, maintaining flexibility to handle different scenarios while demonstrating excellent performance.
The above conclusions fully demonstrate the feasibility of a unified model in human mobility prediction. Our UniMob model can handle various types of mobility data, showcasing exceptional scalability and robustness. As the first attempt to propose a universal model paradigm for mobility prediction, we have successfully expanded the boundaries of this field.


Notably, mobility prediction models based on diffusion models, such as PriSTI and TrajGDM, demonstrate superior performance compared to other baselines. This underscores the powerful modeling capability of diffusion models in capturing the spatiotemporal correlations of mobility data. Diffusion models effectively handle dynamics and uncertainties in mobility data through an iterative denoising process, significantly enhancing prediction performance. Therefore, our UniMob model leverages diffusion models to accurately capture spatiotemporal dependencies in mobility data accurately, proving its effectiveness.



\subsection{Ablation Study}
To evaluate the impact of each module in UniMob, we conducted ablation experiments, divided into ablations of model design and data usage.
\textbf{Model Design:}
(1) w/o I2C loss: This variant keeps the model structure unchanged but removes the I2C loss.
(2) w/o C2I loss: Similar to the previous one, this variant only removes the C2I loss.
(3) w/o shared transformer: In this variant, the flow and trajectory losses no longer share a transformer; instead, each has its independent transformer.
\textbf{Data Usage:}
(4) w/o flow data: The model is trained using only trajectory data.
(5) w/o trajectory data: The model is trained using only flow data.

The results of the ablation experiments conducted on the Shanghai and Senegal datasets are shown in Tables~\ref{tab:Ablation1} and ~\ref{tab:Ablation3} (see Table~\ref{tab:Ablation2} in Appendix~\ref{sec::ablation} for the Xinjiang dataset). For the ablation experiments on model design, it is evident that the shared transformer offers limited benefits for interacting with different mobility data types. The most significant performance improvements come from task-specific loss functions. For instance, the I2C loss enhances flow prediction by using aggregated trajectory and flow data for spatiotemporal alignment. Similarly, the C2I loss uses contrastive learning to construct positive samples of flow and trajectory with similar spatiotemporal patterns, thereby aligning macro and micro mobility distribution. These experiments highlight the effectiveness of our approach in aligning trajectory and flow data.

We removed different data types for the ablation experiments on data usage and trained the model using only a single type of mobility data. The results showed a significant performance decline. This demonstrates the effectiveness and importance of our model in utilizing different types of mobility data. By combining multiple data types, UniMob can more comprehensively understand and predict human mobility behavior, thereby significantly enhancing the model's overall performance.



%\vspace{-0.2cm}
\subsection{Noise Perturbation}
In real life, mobility data often contains noise. This noise can arise from various sources, such as errors produced by sensors during the collection process or intentionally added by data operators to protect user privacy.  To assess our UniMob model's robustness, we added noise to the data and evaluated its performance.

For flow data, we introduced varying noise levels to simulate different degrees of data quality. Figure~\ref{fig:noisy_flow} shows that our model's improvement over the best baseline is relatively small without noise. When the noise level reaches 0.3, our model demonstrates a relative improvement of more than 10\%. This indicates that compared to other baseline models, UniMob exhibits better robustness in handling noisy data, making it more capable of adapting to flow data with noise for prediction. Moreover, we experimented with adding different noise levels to trajectories. Figure~\ref{fig:noisy_trajectory} shows that as the noise ratio increases, the improvement of our model relative to the best baseline also increases, achieving a maximum gain of up to 17.82\%. Because UniMob integrates two types of mobility data, allowing one type of data to provide the same spatiotemporal dynamics as a supplement when the other type of data is noisy, thereby enhancing the model's robustness. The synergistic effects between different data types can still provide reliable predictions even in noisy data.

% Flow
\begin{figure}[t]
\centering
\subfigure[Shanghai]{\includegraphics[width=.23\textwidth]{figure/shanghai_noisy_flow.pdf}}
\vspace{-0.3cm}
%\subfigure[Xinjiang]{\includegraphics[width=.30\textwidth]{figure/xinjiang_noisy_flow.pdf}}
\subfigure[Senegal]{\includegraphics[width=.23\textwidth]{figure/sainei_noisy_flow.pdf}}
\caption{Flow prediction with noisy data on Shanghai and Senegal datasets.} 
\vspace{-0.3cm}
\label{fig:noisy_flow}
\end{figure}

% Trajectory
\begin{figure}[t]
\centering
\subfigure[Shanghai]{\includegraphics[width=.23\textwidth]{figure/shanghai_noisy_trajectory.pdf}}
\vspace{-0.3cm}
%\subfigure[Xinjiang]{\includegraphics[width=.30\textwidth]{figure/xinjiang_noisy_trajectory.pdf}}
\subfigure[Senegal]{\includegraphics[width=.23\textwidth]{figure/sainei_noisy_trajectory.pdf}}
\caption{Trajectory prediction with noisy data on Shanghai and Senegal datasets.} 
\vspace{-0.3cm}
\label{fig:noisy_trajectory}
\end{figure}


\subsection{Few-shot Performance}
Similarly, the amount of mobility data may be limited in real-world scenarios due to privacy concerns, data collection challenges, or other constraints. To simulate this situation, we reduce the amount of flow and trajectory data through different operations.

As shown in Figure~\ref{fig:low_flow}, we constructed scenarios with varying proportions of locations having missing flow records. As the proportion of regions with missing flow data increased, our model still demonstrated a significant performance improvement compared to the best baseline. For instance, in the Shanghai dataset, UniMob achieves an improvement of up to 14\% when 75\% of the region is missing. UniMob's robustness is evident in its ability to maintain high performance despite the absence of a substantial amount of flow data. This is due to its ability to leverage the available trajectory data, compensating for the missing flow information through its joint modeling approach. 
As shown in Figure~\ref{fig:low_trajectory}, we used datasets of different sizes for trajectory data to explore the performance of trajectory prediction with limited data. When the amount of trajectory data is very limited (e.g., only 25\% of the dataset), our model shows a 25\% improvement in the Shanghai dataset compared to the best baseline. This indicates that when trajectory data is scarce, the flow data provides more diverse mobility patterns, effectively compensating for the lack of trajectory data.

By effectively utilizing the spatiotemporal correlations between different types of mobility data, UniMob can provide accurate predictions even in data-scarce environments. UniMob's ability to deliver reliable predictions with limited data highlights its robustness and practical applicability in various scenarios, ensuring dependable performance regardless of data constraints.

% Flow
\begin{figure}[t]
\centering
\subfigure[Shanghai]{\includegraphics[width=.23\textwidth]{figure/shanghai_low_flow.pdf}}
\vspace{-0.3cm}
%\subfigure[Xinjiang]{\includegraphics[width=.30\textwidth]{figure/xinjiang_low_flow.pdf}}
\subfigure[Senegal]{\includegraphics[width=.23\textwidth]{figure/sainei_low_flow.pdf}}
\caption{Flow prediction with scarce data on Shanghai and Senegal datasets.} 
\vspace{-0.3cm}
\label{fig:low_flow}
\end{figure}

% Trajectory
\begin{figure}[t]
\centering
\subfigure[Shanghai]{\includegraphics[width=.23\textwidth]{figure/shanghai_low_trajectory.pdf}}
\vspace{-0.3cm}
%\subfigure[Xinjiang]{\includegraphics[width=.30\textwidth]{figure/xinjiang_low_trajectory.pdf}}
\subfigure[Senegal]{\includegraphics[width=.23\textwidth]{figure/sainei_low_trajectory.pdf}}
\caption{Trajectory prediction with scarce data on Shanghai and Senegal datasets.} 
\vspace{-0.3cm}
\label{fig:low_trajectory}
\end{figure}

\section{Conclusion}
\label{section:conclusion}
We systematically investigate the effects of various attributes of preference datasets on model capabilities from the perspective of instruction-following.
To this end, we first build a data generation pipeline that combines general-purpose prompts with mixtures of verifiable constraints to synthesize challenging instruction-following prompts.
We then automatically curate preference pairs using two popular methods: rejection sampling (RS) and Monte Carlo Tree Search (MCTS).
Using the preference pairs, we examine the effects of (1) the existence of shared prefixes between the chosen and rejected responses, (2) the contrast and quality of the responses, and (3) the complexity of the training prompts.
Our results indicate that having a common prefix in the preference pairs offers marginal yet consistent improvements, high-contrast preference pairs outperform low-contrast pairs but a mixture is sometimes better than both, and training on moderately difficult prompts is more helpful than training on extremely difficult prompts.
Our work provides a systematic framework for curating different types of preference datasets and sets the groundwork for future studies that extend the scope beyond verifiable instruction-following constraints to more general constraints.

% \section{References and citations}

% Take a look at~\cref{table:demo}, appearing on~\cref{section:intro}.
% %
% Some citation of previous work~\citep{goodman}.

\clearpage
\newpage
\bibliographystyle{assets/plainnat}
\bibliography{paper}

\clearpage
\newpage
\beginappendix
\label{section:appendix}
\section{APPENDIX FOR REPRODUCIBILITY}
\subsection{Related Work}
\subsubsection{Diffusion Models}\label{sec::diffusion}
The diffusion model is a probabilistic generative model first introduced by Sohl-Dickstein et al.~\cite{sohl2015deep} and further improved by Ho et al. ~\cite{ho2020denoising} and Song et al. ~\cite{song2020score}. As a novel generative model, diffusion models have rapidly advanced in time series and spatio-temporal modeling. 
Research on time series modeling based on diffusion models is widely applied, such as time series imputation ~\cite{alcaraz2022diffusion,liu2023pristi}, time series generation~\cite{lim2023regular,lin2023diffusion}, and time series forecasting~\cite{li2022generative,bilovs2022modeling}. 
DiffSTG~\cite{wen2023diffstg} is the first attempt to generalize the widespread denoising diffusion probabilistic models to spatiotemporal graphs (STGs), leading to a novel non-autoregressive framework. 
KSTDiff~\cite{zhou2023towards} designed a knowledge-enhanced denoising network to capture the spatiotemporal dependencies of urban flows and the influence of the urban environment in the denoising process.
DiffTraj~\cite{zhou2023towards} is a spatiotemporal diffusion probabilistic model for trajectory generation. This model effectively combines the generative capabilities of diffusion models with spatiotemporal features derived from real trajectories.
In this work, we introduce the diffusion model for unified mobility prediction adapted to different data types.

\subsection{Datasets Details}\label{sec::datasets_info}
We conducted extensive experiments on three real-world mobility datasets: Shanghai, Senegal, and Xinjiang. The details of datasets are summarized in Table~\ref{table:datasets}. We preprocess the trajectory data for three datasets, filtering out users with fewer than five records per day. For location preprocessing, we map GPS points to predefined grid IDs of a specific granularity. For temporal preprocessing, we organize the time data into fixed intervals, such as hourly or half-hourly segments. Finally, we divide the data into training, validation, and testing sets in a 7:1:2 ratio in chronological order. 

% dataset table
\begin{table}[h]
\setlength{\abovecaptionskip}{0.cm}
\setlength{\belowcaptionskip}{-0.cm}
\caption{Basic statistics of mobility datasets.}
\label{table:datasets}
\begin{center}
\scalebox{0.9}{
\begin{tabular}{ >{\centering\arraybackslash}m{1cm} 
>{\centering\arraybackslash}m{1cm} 
>{\centering\arraybackslash}m{1.2cm} 
>{\centering\arraybackslash}m{1cm}
>{\centering\arraybackslash}m{1cm}
>{\centering\arraybackslash}m{1cm}}
 \hline
City  & Duration & Users & Location \\ 
 \hline
Shanghai & 7 days & 700000 & 4096 \\ 
Senegal & 14 days & 8000 & 1666 \\ 
Xinjiang & 28 days & 1200000 & 4096\\ 
\hline
\end{tabular}}
\end{center}
\vspace{-0.3cm}
%\vspace{-20px}
\end{table}

\subsection{Baselines}\label{sec::baselines}
To evaluate the performance of our proposed model, we compared it with state-of-the-art models. Previous methods could only accomplish one type of mobility data prediction task, so the baseline methods are divided into trajectory and flow prediction. 

\paragraph{Flow Prediction} The baselines for flow prediction are as follows:
\begin{itemize}[leftmargin=*]
\item \textbf{HA}~\cite{sun2020predicting}: It considers the inflow and outflow to be seasonal processes and employs the average of the previous seasons as the prediction for a week-long period. 
\item \textbf{VAR}~\cite{lu2016integrating}: This method is vector autoregressive single-step predictor.
\item \textbf{ST-ResNet}~\cite{zhang2017deep}: ST-ResNet employs the residual neural network framework to model the temporal closeness, period, and trend properties of crowd flow.
\item \textbf{MSDR}~\cite{liu2022msdr}: Multi-Step Dependency Relationship (MSDR) is a brand new variant of recurrent neural networks. Instead of only looking at the hidden state from the latest time step, MSDR explicitly takes those from multiple historical time steps as the input of each time unit.
\item \textbf{STID}~\cite{shao2022spatial}: A simple multi-layer perceptron addresses the indistinguishability of time series samples in spatial and temporal dimensions.
\item \textbf{PriSTI}~\cite{liu2023pristi}: This method extracts coarse but effective spatiotemporal dependencies from conditional information using a diffusion model, serving as a global context prior.
\end{itemize}



\paragraph{Trajectory Prediction} The baselines for trajectory prediction are as follows:
\begin{itemize}[leftmargin=*]
\item \textbf{Markov Model}~\cite{gambs2012next}: The Markov model is a statistical model used to describe the change of states over time. It uses historical trajectory data for location prediction by calculating the transition probabilities between these locations.
\item \textbf{LSTM}~\cite{Kong2018HST}: The LSTM network is good at handling sequential data and has the advantage of encoding long-term dependencies, which can naturally be applied to location prediction.
\item \textbf{DeepMove}~\cite{feng2018deepmove}: The method designs a multimodal embedding recurrent neural network to capture complex sequential transitions by jointly embedding multiple factors that control human mobility.
\item \textbf{STAN}~\cite{luo2021stan}: This model associates non-contiguous but functionally similar visited points that are not adjacent to each other to predict the next location.
\item \textbf{SNPM}~\cite{yin2023next}: The method constructs a Sequence-based, Dynamic Neighbor Graph (SDNG) to find the similarity neighborhood and develop a Multi-Step Dependency Prediction model.
\item \textbf{TrajGDM}~\cite{chu2024simulating}: The method utilizes diffusion models to capture the universal mobility pattern in a trajectory dataset for trajectory prediction.
\item \textbf{GETNext}~\cite{yang2022getnext}: The method employs a global trajectory flow map and a novel Graph Enhanced Transformer model to leverage collaborative signals for more accurate trajectory prediction.
\end{itemize}

\begin{table}[h]
\small
\centering
\caption{Overall Performance on Xinjiang datasets.}
\vspace{-0.3cm}
\scalebox{0.9}{
\begin{tabular}{lcccccc}
\toprule
& \multicolumn{3}{c}{\textbf{Flow Prediction}} & \multicolumn{3}{c}{\textbf{Trajectory Prediction}} \\
\cmidrule(lr){2-4} \cmidrule(lr){5-7}
& \textbf{MAE} & \textbf{MAPE(\%)} & \textbf{RMSE}
& \textbf{Acc@1} & \textbf{Acc@3} & \textbf{Acc@5}\\
\midrule
HA & 33.16 & 30.54 & 44.28 & - & - & - \\
VAR & 23.90 & 22.15 & 36.63 & - & - & - \\
ST-ResNet & 19.72 & 17.36 & 31.56 & - & - & - \\
MSDR & 17.95 & 16.53 & 29.60 & - & - & - \\
STID & 17.01 & 15.70 & 27.36 & - & - & - \\
PriSTI & \underline{16.80} & \underline{15.37} & \underline{26.47} & - & - & - \\
Markov & - & - & - & 0.3156 & 0.3924 & 0.4571 \\
LSTM & - & - & - & 0.3847 & 0.4519 & 0.5450 \\
DeepMove & - & - & - & 0.4261 & 0.5143 & 0.6318 \\
STAN & - & - & - & 0.4432 & 0.5307 & 0.6609 \\
SNPM & - & - & - & 0.4618 & 0.5574 & 0.6926 \\
GETNext & - & - & - & 0.4650 & 0.5598 & 0.6975 \\
TrajGDM & - & - & - & \underline{0.4673} & \underline{0.5632} & \underline{0.7054} \\
UniMob-v1 & 16.31 & 14.91 & 25.98 & 0.4768 & 0.5795 & 0.7217 \\
UniMob-v2 & 15.96 & 14.72 & 25.54 & 0.4815 & 0.5853 & 0.7286 \\
UniMob-v3 & 16.12 & 14.84 & 25.70 & 0.4791 & 0.5830 & 0.7253 \\
UniMob-v4 & \bf{15.87} & \bf{14.50} & \bf{25.19} & \bf{0.4841} & \bf{0.5897} & \bf{0.7336} \\
%Improvement & 5.54\%  &	5.66\%	& 4.84\% &	3.60\% &	4.71\%	& 4.00\% \\
\bottomrule
\end{tabular}
}
%\vspace{-0.3cm}
\label{tab:Xinjiang}
\end{table}


\begin{table*}[t]
\small
\centering
\caption{Ablation study on Xinjiang datasets.}
\vspace{-0.3cm}
\scalebox{1.}{
\begin{tabular}{lcccccc}
\toprule
& \multicolumn{3}{c}{\textbf{Trajectory Prediction}} & \multicolumn{3}{c}{\textbf{Flow Prediction}} \\
\cmidrule(lr){2-4} \cmidrule(lr){5-7}
& \textbf{Acc@1} & \textbf{Acc@3} & \textbf{Acc@5}
& \textbf{MAE} & \textbf{MAPE(\%)} & \textbf{RMSE}\\
\midrule
Ours & 0.4768 & 0.5795 &  0.7217 & 16.31 & 14.91 & 25.98 \\
w/o I2C loss & 0.4736 (-0.67\%) & 0.5730 (-1.12\%) &  0.7125 (-1.28\%)
 & 16.78 (-2.88\%) & 15.43 (-3.49\%) & 27.02 (-4.00\%) \\
w/o C2I loss & 0.4689 (-1.66\%) & 0.5671 (-2.14\%) &  0.7064 (-2.12\%)
 & 16.46 (-0.92\%) & 15.08 (-1.14\%) & 26.90 (-3.54\%) \\
w/o shared transformer & 0.4702 (-1.39\%) & 0.5693 (-1.76\%)
 & 0.7091 (-1.75\%) & 16.67 (-2.21\%) & 15.29 (-2.55\%)
 & 26.97 (-3.81\%)
 \\
w/o flow data & 0.4639(-2.71\%) & 0.5620(-3.02\%)
 & 0.6998(-3.03\%) & - & - & - \\
w/o trajectory data & - & - & - & 16.87(-3.43\%)
 & 15.56(-4.36\%) & 27.20(-4.70\%) \\
\bottomrule
\end{tabular}
}
%\vspace{-0.3cm}
\label{tab:Ablation2}
\end{table*}



\section{Experimental Performance}\label{sec::Results}
\subsection{Overall Performance}\label{sec::Overall Performance}
Table~\ref{tab:Xinjiang} shows the performance of our universal mobility prediction model on the Xinjiang dataset. UniMob not only accomplishes both trajectory and flow predictions simultaneously but also surpasses current advanced baseline models in all evaluation metrics. Specifically, it achieves 5.34\% performance improvement in flow prediction and more than 4\% enhancement in trajectory prediction. These results fully demonstrate the generality and reliability of our model.





\subsection{Ablation study}\label{sec::ablation}
We conducted ablation experiments on two aspects: model design and data utilization. By sequentially removing components of the model design, we identified three design elements that align with different data formats and distributions, each impacting performance, thus validating their effectiveness. Regarding data utilization, by replacing multi-type data with single-type data, we visually demonstrated the performance enhancement brought by using multi-type mobility data in human mobility prediction through our universal model.






\subsection{Noise Perturbation}\label{sec::noise}

\begin{figure}[t]
\centering
\subfigure[Flow prediction]{\includegraphics[width=.23\textwidth]{figure/xinjiang_noisy_flow.pdf}}
\vspace{-0.5cm}
\subfigure[Trajectory prediction]{\includegraphics[width=.23\textwidth]{figure/xinjiang_noisy_trajectory.pdf}}
\caption{Flow and trajectory prediction with noisy data on Xinjiang dataset.} 
%\vspace{-0.3cm}
\label{fig:noisy}
\end{figure}

Due to biases from sensor collection and artificial noise added for privacy protection, the data used for mobility prediction often contains noise. To verify whether our model can still maintain good predictive capabilities in noisy conditions, we added noise to both the flow and trajectory data. Figure~\ref{fig:noisy} shows that as noise levels increase, our model continues to outperform the best baseline model, and our performance advantage becomes even more pronounced relative to the baseline with increasing noise. This effectively demonstrates the high robustness of our UniMob model.



\subsection{Few-shot Performance}\label{sec::few-shot}
Similarly, due to data collection and privacy protection limitations, the amount of mobility data we acquire is often limited. Therefore, we tested the few-shot learning capabilities of our UniMob model. As shown in Figure~\ref{fig:low}, our model still performs excellently even in a data-constrained environment.

\begin{figure}[t]
\centering
\subfigure[Flow prediction]{\includegraphics[width=.23\textwidth]{figure/xinjiang_low_flow.pdf}}
\vspace{-0.5cm}
\subfigure[Trajectory prediction]{\includegraphics[width=.23\textwidth]{figure/xinjiang_low_trajectory.pdf}}
\caption{Flow and trajectory prediction with scarce data on Xinjiang dataset.} 
%\vspace{-0.3cm}
\label{fig:low}
\end{figure}

\end{document}