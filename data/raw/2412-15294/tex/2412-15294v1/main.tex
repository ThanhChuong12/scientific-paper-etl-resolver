%%
%% This is file `sample-authordraft.tex',
%% generated with the docstrip utility.
%%
%% The original source files were:
%%
%% samples.dtx  (with options: `authordraft')
%% 
%% IMPORTANT NOTICE:
%% 
%% For the copyright see the source file.
%% 
%% Any modified versions of this file must be renamed
%% with new filenames distinct from sample-authordraft.tex.
%% 
%% For distribution of the original source see the terms
%% for copying and modification in the file samples.dtx.
%% 
%% This generated file may be distributed as long as the
%% original source files, as listed above, are part of the
%% same distribution. (The sources need not necessarily be
%% in the same archive or directory.)
%%
%% Commands for TeXCount
%TC:macro \cite [option:text,text]
%TC:macro \citep [option:text,text]
%TC:macro \citet [option:text,text]
%TC:envir table 0 1
%TC:envir table* 0 1
%TC:envir tabular [ignore] word
%TC:envir displaymath 0 word
%TC:envir math 0 word
%TC:envir comment 0 0
%%
%%
%% The first command in your LaTeX source must be the \documentclass command.
\documentclass[sigconf]{acmart}
\settopmatter{printacmref=false,  printccs=false,  printfolios=true}
\renewcommand\footnotetextcopyrightpermission[1]{}

%%% make acmart stop complaining about missing country in affiliation %%%
\makeatletter
\def\@ACM@checkaffil{% Only warnings
    \if@ACM@instpresent\else
    \ClassWarningNoLine{\@classname}{No institution present for an affiliation}%
    \fi
    \if@ACM@citypresent\else
    \ClassWarningNoLine{\@classname}{No city present for an affiliation}%
    \fi
    \if@ACM@countrypresent\else
        \ClassWarningNoLine{\@classname}{No country present for an affiliation}%
    \fi
}
\makeatother

\let\Bbbk\relax
\usepackage{hyperref}
\usepackage{amssymb}
\usepackage{amsfonts}
\usepackage{subfigure}
\usepackage{algorithmic}
\usepackage{graphicx}
\usepackage{xcolor}
\usepackage{makecell}
\usepackage{booktabs}
\usepackage{amsmath}
\usepackage{accents}
\usepackage{statex}
\usepackage[normalem]{ulem}
\usepackage{enumitem}
\usepackage{multirow}
\usepackage[nomargin,inline,marginclue,draft]{fixme}
\usepackage{balance}
\usepackage{changepage}
\usepackage{bm}
\usepackage{setspace}
\usepackage{mathrsfs}
\usepackage{ulem}
\usepackage{verbatim}
\usepackage{diagbox}
\usepackage{pdftexcmds}
\usepackage{catchfile}
\usepackage{ifluatex}
\usepackage{ifplatform}
\usepackage{threeparttable}

\usepackage{comment}

\usepackage[T1]{fontenc}
\usepackage{aecompl}
\usepackage[utf8]{inputenc}
\usepackage{bm} 

% Define colors for checkmarks and crosses
\newcommand{\cmark}{\textcolor{blue}{\textbf{$\bm{\checkmark}$}}} % Green checkmark
\newcommand{\xmark}{\textcolor{red}{\textbf{$\bm{\times}$}}} % Red cross

%% end 

%\pagestyle{plain} % removes running headers

%% NOTE that a single column version may required for 
%% submission and peer review. This can be done by changing
%% the \doucmentclass[...]{acmart} in this template to 
%% \documentclass[manuscript,screen]{acmart}
%% 
%% To ensure 100% compatibility, please check the white list of
%% approved LaTeX packages to be used with the Master Article Template at
%% https://www.acm.org/publications/taps/whitelist-of-latex-packages 
%% before creating your document. The white list page provides 
%% information on how to submit additional LaTeX packages for 
%% review and adoption.
%% Fonts used in the template cannot be substituted; margin 
%% adjustments are not allowed.

%%
%% \BibTeX command to typeset BibTeX logo in the docs
\AtBeginDocument{%
  \providecommand\BibTeX{{%
    \normalfont B\kern-0.5em{\scshape i\kern-0.25em b}\kern-0.8em\TeX}}}

%% Rights management information.  This information is sent to you
%% when you complete the rights form.  These commands have SAMPLE
%% values in them; it is your responsibility as an author to replace
%% the commands and values with those provided to you when you

%%
%% Submission ID.
%% Use this when submitting an article to a sponsored event. You'll
%% receive a unique submission ID from the organizers
%% of the event, and this ID should be used as the parameter to this command.
%%\acmSubmissionID{123-A56-BU3}

%%
%% For managing citations, it is recommended to use bibliography
%% files in BibTeX format.
%%
%% You can then either use BibTeX with the ACM-Reference-Format style,
%% or BibLaTeX with the acmnumeric or acmauthoryear sytles, that include
%% support for advanced citation of software artefact from the
%% biblatex-software package, also separately available on CTAN.
%%
%% Look at the sample-*-biblatex.tex files for templates showcasing
%% the biblatex styles.
%%

%%
%% For managing citations, it is recommended to use bibliography
%% files in BibTeX format.
%%
%% You can then either use BibTeX with the ACM-Reference-Format style,
%% or BibLaTeX with the acmnumeric or acmauthoryear sytles, that include
%% support for advanced citation of software artefact from the
%% biblatex-software package, also separately available on CTAN.
%%
%% Look at the sample-*-biblatex.tex files for templates showcasing
%% the biblatex styles.
%%

%%
%% The majority of ACM publications use numbered citations and
%% references.  The command \citestyle{authoryear} switches to the
%% "author year" style.
%%
%% If you are preparing content for an event
%% sponsored by ACM SIGGRAPH, you must use the "author year" style of
%% citations and references.
%% Uncommenting
%% the next command will enable that style.
%%\citestyle{acmauthoryear}

%%
%% end of the preamble, start of the body of the document source.


\newcommand{\para}[1]{{\vspace{3pt} \bf \noindent #1 \hspace{0pt}}}

%%
%% The "author" command and its associated commands are used to define
%% the authors and their affiliations.
%% Of note is the shared affiliation of the first two authors, and the
%% "authornote" and "authornotemark" commands
%% used to denote shared contribution to the research.
% \author{Ben Trovato}
% \authornote{Both authors contributed equally to this research.}
% \email{trovato@corporation.com}
% \orcid{1234-5678-9012}
% \author{G.K.M. Tobin}
% \authornotemark[1]
% \email{webmaster@marysville-ohio.com}
% \affiliation{%
%   \institution{Institute for Clarity in Documentation}
%   \streetaddress{P.O. Box 1212}
%   \city{Dublin}
%   \state{Ohio}
%   \country{USA}
%   \postcode{43017-6221}
% }

% \author{Lars Th{\o}rv{\"a}ld}
% \affiliation{%
%   \institution{The Th{\o}rv{\"a}ld Group}
%   \streetaddress{1 Th{\o}rv{\"a}ld Circle}
%   \city{Hekla}
%   \country{Iceland}}
% \email{larst@affiliation.org}

% \author{Valerie B\'eranger}
% \affiliation{%
%   \institution{Inria Paris-Rocquencourt}
%   \city{Rocquencourt}
%   \country{France}
% }

%\author{Anonymous author(s)}
%\author{Qingyue~Long, Yong~Li}
%\affiliation{
%  \institution{Department of Electronic Engineering, BNRist, Tsinghua University, Beijing, China}

\author{Qingyue~Long}
\affiliation{
  \institution{Department of Electronic Engineering\\ BNRist, Tsinghua University}
  \state{Beijing}
  \country{China}
}

\author{Yuan~Yuan}
\affiliation{
  \institution{Department of Electronic Engineering\\ BNRist, Tsinghua University}
  \state{Beijing}
  \country{China}
}

\author{Yong~Li}
\affiliation{
  \institution{Department of Electronic Engineering\\ BNRist, Tsinghua University}
  \state{Beijing}
  \country{China}
}




%%
%% By default, the full list of authors will be used in the page
%% headers. Often, this list is too long, and will overlap
%% other information printed in the page headers. This command allows
%% the author to define a more concise list
%% of authors' names for this purpose.


%\thanks{
%$\dagger$Huandong~Wang is the corresponding author (wanghuandong@tsinghua.edu.cn).}


\copyrightyear{2024}
\acmYear{2024}
\setcopyright{acmlicensed}\acmConference[KDD '25]{Proceedings of the 31th
ACM SIGKDD Conference on Knowledge Discovery and Data Mining}{August
25--29, 2025}{Barcelona, Spain}
\acmBooktitle{Proceedings of the 30th ACM SIGKDD Conference on Knowledge
Discovery and Data Mining (KDD '25), August 25--29, 2025, Barcelona, Spain}
%\acmPrice{15.00}
%\acmDOI{10.1145/3580305.3599888}
%\acmISBN{979-8-4007-0103-0/23/08}

%\settopmatter{printacmref=true}

\begin{document}
%\renewcommand{\shortauthors}{Qingyue Long et al.}
\title{A Universal Model for Human Mobility Prediction}
%%
%% The abstract is a short summary of the work to be presented in the
%% article.
\begin{abstract}
%无论是细粒度的个体移动性预测还是粗粒度的群体移动性预测,它们都依托于共同的底层时空模式,例如时间周期性和空间邻近性。现有的人类移动性预测方法通常是针对特定任务的,它们的适用场景十分局限。因此,我们试图统一人类移动预测模型,旨在突破任务专用模型的局限性,实现可扩展性和泛化性。

Predicting human mobility is crucial for urban planning, traffic control, and emergency response. Mobility behaviors can be categorized into individual and collective, and these behaviors are recorded by diverse mobility data, such as individual trajectory and crowd flow. As different modalities of mobility data, individual trajectory and crowd flow have a close coupling relationship. Crowd flows originate from the bottom-up aggregation of individual trajectories, while the constraints imposed by crowd flows shape these individual trajectories.
Existing mobility prediction methods are limited to single tasks due to modal gaps between individual trajectory and crowd flow.
In this work, we aim to unify mobility prediction to break through the limitations of task-specific models. 
We propose a universal human mobility prediction model (named \textbf{UniMob}), which can be applied to both individual trajectory and crowd flow.
UniMob leverages a multi-view mobility tokenizer that transforms both trajectory and flow data into spatiotemporal tokens, facilitating unified sequential modeling through a diffusion transformer architecture. To bridge the gap between the different characteristics of these two data modalities, we implement a novel bidirectional individual and collective alignment mechanism. This mechanism enables learning common spatiotemporal patterns from different mobility data, facilitating mutual enhancement of both trajectory and flow predictions.
Extensive experiments on real-world datasets validate the superiority of our model over state-of-the-art baselines in trajectory and flow prediction. Especially in noisy and scarce data scenarios, our model achieves the highest performance improvement of more than 14\% and 25\% in MAPE and Accuracy@5. 
\end{abstract}



%%
%% The code below is generated by the tool at http://dl.acm.org/ccs.cfm.
%% Please copy and paste the code instead of the example below.
%%
\iffalse
\begin{CCSXML}
<ccs2012>
   <concept>
       <concept_id>10002951.10003227</concept_id>
       <concept_desc>Information systems~Information systems applications</concept_desc>
       <concept_significance>500</concept_significance>
       </concept>
 </ccs2012>
\end{CCSXML}

\ccsdesc[500]{Information systems~Information systems applications}
\fi

\begin{CCSXML}
<ccs2012>
<concept>
<concept_id>10002951.10003227.10003236</concept_id>
<concept_desc>Information systems~Spatial-temporal systems</concept_desc>
<concept_significance>500</concept_significance>
</concept>
<concept>
<concept_id>10003033.10003079.10003081</concept_id>
<concept_desc>Networks~Network simulations</concept_desc>
<concept_significance>300</concept_significance>
</concept>
<concept>
<concept_id>10010147.10010341.10010366.10010369</concept_id>
<concept_desc>Computing methodologies~Simulation tools</concept_desc>
<concept_significance>300</concept_significance>
</concept>
</ccs2012>
\end{CCSXML}

\ccsdesc[500]{Information systems~Spatial-temporal systems}
\ccsdesc[300]{Networks~Network simulations}
\ccsdesc[300]{Computing methodologies~Simulation tools}

%%
%% Keywords. The author(s) should pick words that accurately describe
%% the work being presented. Separate the keywords with commas.
\keywords{Universal model, trajectory prediction, flow prediction}



%%
%% This command processes the author and affiliation and title
%% information and builds the first part of the formatted document.
\maketitle

\section{Introduction}
We have seen considerable interest in machine learning models based on transformer architecture~\cite{vaswani2023attention}. They are trained across modalities: Models trained on textual data have given rise to large language models~(LLMs)~\cite{devlin2018bert, llm_gpt2, llm_gpt3, llm_gpt4, llm_gemini}, which are now widely used for interaction with computers through chatbots. Similarly, models that are trained on images~\cite{rombach2022high} can now generate photorealistic visuals.  Models that can generate music are also trained on analog information like acoustic data~\cite{huang2018musictransformer}. Nonetheless, we find that a common thread across these models is the scaling of the training data size. This follows the observation that larger training datasets result in models that can provide more accurate responses while demonstrating general-purpose capabilities~\cite{emergent_abilities, scaling_laws, agi_gpt4}.

At a high level, a machine-learning model is defined by its parameters—weights that the model learns during training. The larger the model, the more parameters it has, and the more data it requires to effectively learn from training~\cite{scaling_laws, compute_optimal_LLM}. State-of-the-art (SoTA) models now reach hundreds of billions of parameters~\cite{llm_llama3, llm_gpt3}, posing significant challenges due to elevated computational demands. 

As parameter size grows, so do the memory and processing requirements for training and inference, limiting the feasibility of training and using these models on commodity computing systems. Typically, a model of tens to hundreds of billions of parameters requires clusters of expensive graphical processing units~(GPU) running for a prolonged period~\cite{carbon_emissions}, making this task highly challenging and infeasible for most people and organizations. For example, the Llama 3.1 70B variant required approximately 7 million GPU hours on Nvidia H100-80GB hardware, while the 405B variant required over 31 million GPU hours~\cite{llm_llama3}. Using 16,000 GPUs for pre-training, this translates to around 20 days of training for the 70B model and 78 days for the 405B model.


\begin{figure}[t!]
      \includegraphics[width=1\columnwidth]{figures/overview_v1.pdf}
  \vspace{-4mm}
  \caption{\emph{An embedded application often involves sensors that collect environmental data, which is then communicated to an edge device. \system\space provides a framework for training foundational models tailored for edge deployment, enabling these models to support a variety of tasks. This work explores training custom foundational models to enhance sensor data analysis. Our approach demonstrates a significantly smaller parameter-sized model than state-of-the-art language models, facilitating high-accuracy sensor data analysis while enabling rapid, local inference on even a constrained edge platform.}}
  \vspace{-4mm}
  \label{fig:overview}
\end{figure}

Beyond training, a larger parameter size model also negatively impacts the inference process. This is a step where the weights are loaded into memory and then used to answer queries through prompts provided by the user. As parameter size grows, so do the memory and processing requirements for inference, limiting the feasibility of using these models on commodity systems~\cite{llm_inferencing,memory_constraint, energy_llms}. For instance, loading a 70B parameter model in half-precision (FP16) would require at least 140GB of memory, exceeding the capacity of most GPUs. Today, even high-specification workstations struggle with memory bandwidth limitations, leading to the rise of alternative strategies to tackle model scaling~\cite{llm_llama3, squeeze_llm, smoothquant}. Consequently, the dominant mode of accessing models from mobile and edge devices has become function calls over a network to remotely hosted models. However, this approach introduces several challenges, including latency issues, unpredictable network conditions, and privacy concerns related to sharing sensitive information.



% Even with sufficient resources to load a model, inference time—the time needed for the model to generate a response—becomes a bottleneck. Today, even high-specification workstations face difficulties performing inference efficiently on SoTA models.

\begin{figure}[t!]
  \includegraphics[width=\linewidth]{figures/training_v2.pdf}
  \vspace{-4mm}
  \caption{\emph{\system\space trains a custom foundational model for deployment at the edge device following a series of steps. It begins by appending a curated dataset with general conversational data. After pre-processing, the dataset is tokenized to pre-train a small model~(30-120M parameter). The pre-trained model undergoes fine-tuning with the custom dataset before deployment on the edge device to support embedded applications.}}
    \vspace{-4mm}
  \label{fig:training}
\end{figure}

A promising approach to tackle this challenge is explicitly trading off parameter size. A smaller parameter-sized model requires proportionally smaller memory and computing resources and can also perform inference faster, even on devices with constrained processing capabilities such as edge computers. \system\space framework builds on this approach.

%However, even these smaller models remain too large for embedded sensing platforms, as they often require a few gigabytes of RAM and consist of billions of parameters. This includes recent state-of-the-art models like Llama 3.2-1B and others. Today, a high-end platform in embedded sensing applications may only support tens to hundreds of megabytes of memory, such as single-board computers like Raspberry Pi.

\fakepar{\system\space Framework Overview}  We systematically study various trade-offs, models, and architectures and design a framework to pre-train foundational models from scratch. This framework is tailored to deploy such models at the edge. We prototype it for a challenging case related to embedded sensing, finding that much smaller models, with only tens of millions of parameters—orders of magnitude smaller than SoTA models—are sufficient for sensor data inference. We consolidate these insights into a framework called \system, meaning “more” in Swedish and “own” in Hindi. Pre-trained on carefully curated data, these smaller models offer significant benefits for embedded sensing applications. They can run locally on constrained edge platforms and perform rapid inference on modestly configured edge and mobile devices.

\fakepar{\system\space Design} \system\space simplifies the process for end-users to train custom foundational models for deployment at the edge. Users only need to provide a suitable training or fine-tuning dataset, and the framework manages the remaining steps to create a tailored foundational model. Performing this process involves following steps, as illustrated in Figure~\ref{fig:training}.

The first step involves preparing the dataset as the foundation for pre-training a model. Users can provide their dataset; however, even for smaller custom models, large amounts of data are typically required. It may be necessary to augment this data with relevant datasets related to language and conversation. \system\space facilitates this process by preparing the dataset for pre-training and offering a pre-curated collection that users can use to complement their datasets, ensuring an effective pre-training process.

The next step in the framework involves processing the data, which is crucial due to the diverse application scenarios requiring custom foundational models. For instance, in embedded sensing applications, even a simple sensor like an accelerometer may record motion across different axes. However, variations in resolution, sampling modes (analog vs. digital), and other intricacies can introduce inconsistencies. This step ensures data consistency by separating individual sensor readings by timestamp and organizing them into rows and columns. Additionally, it performs basic preprocessing tasks, such as removing unnecessary characters or spaces to fit the data within the model’s context window. Tokenization is the final step in the processing of the data.

Next, the framework involves training the foundational models. The framework adopts an architecture similar to existing models like GPT-2, which has proven effective for creating smaller models. Our results show that this approach achieves high accuracy for sensor data analysis. Additionally, the architecture allows flexibility in configuring parameter sizes as low as 30 million. While training these smaller foundational models still requires a GPU, the overall computing resources are minimal. For instance, we completed training on a single  Nvidia H100 in just a few hours.

After pre-training the model, we found that through extensive experiments, despite careful curation of the dataset, smaller models may still struggle to achieve high accuracy for specific applications. Therefore, fine-tuning becomes crucial to enhance their performance. The framework efficiently manages this step, requiring only a small set of examples for fine-tuning. Specifically, we employed the LoRa method for fine-tuning, significantly reducing the number of samples needed compared to traditional machine learning methods. For instance, in hand gesture sensing one of the use cases presented in the work. We only required 440 examples for the model to achieve high accuracy in detecting future events.

%Using a dataset involving YYY, we successfully fine-tuned the custom model with only YYY examples, ensuring optimal performance for the target application.

Once the custom model is trained and fine-tuned, it can be deployed on an edge device to support embedded sensing applications. While smaller models are generally expected to struggle with tasks involving mathematical operations and reasoning—key elements in many sensor data analysis tasks—we intentionally used these as a challenging test case for our system. Surprisingly, we found that smaller models can be highly effective for embedded sensing analysis and, in some cases, even outperform much larger models with significantly greater parameter sizes.

\fakepar{Summary of Results} The key results are:

\begin{itemize}
  \item We present a framework that supports two primary tasks: (1) training smaller models for edge deployment on user-defined datasets and (2) fine-tuning these models or off-the-shelf LLMs on domain-specific datasets. We demonstrate this by training five smaller models, ranging from 30M to 124M parameters, following the GPT-2 architecture. We also fine-tuned other models, such as Phi 2, Phi 3, and Llama 2, Llama 3.
  \item We compare the accuracy of smaller custom models trained through our framework, with fewer than 125M parameters, against larger models with billions of parameters across different IoT sensor datasets, including our collected and external datasets. Our results demonstrate that these smaller models perform comparably to larger ones while requiring significantly fewer GPU resources and less training time. 
  
  \item We investigate the suitability for deployment of smaller models on resource-constrained edge platforms and demonstrate that they lead to significantly faster inference or token generation rates.
\end{itemize}


%One solution that this work focuses on is using smaller models that can run locally within the limited constraints of edge devices.  trading parameter size for reduced accuracy, these models can enable local inference to edge devices at a reasonable token generation rate. Today, models with a few billion parameters already demonstrate capabilities to perform specialized tasks and can run locally on mobile devices. However, these models typically train on general data with little control over the source, remaining general-purpose for tasks. Consequently, they may not suit numerous applications without appropriate fine-tuning, and even then, they may be too large or insufficiently capable for specific applications.







%Today, we find that even the highest configuration workstation str

%Even high-end workstations struggle to perform inference on some of the largest language models, like Llama 3-405B. 

%Consequently, due to the computational requirements, today, applications relying on these models often make calls to a LLM hosted remotely. 









%Indeed, moderately sized LLMs find it challenging to perform inference in a reasonable time on edge devices.







\section{Related Work}
\paragraph{\textbf{Mobility Prediction}}
% 各自完成任务
% flow增强traj: GETNext
% traj增强flow: TrGNN
Mobility prediction can be divided into individual and collective categories. Individual prediction focuses on personal preferences~\cite{qingyue2024privacy}. For example, Qiao et al.~\cite{gao2019predicting} and Wang et al.~\cite{wang2021attentional} developed a Markov-based model by considering the spatiotemporal characteristics of individual mobility. Collective flow prediction emphasizes modeling collective mobility trends. For example, DeepSTN+~\cite{feng2021context} uses a context-aware spatiotemporal neural network for flow prediction. CrowdNet utilizes graph convolutional networks to achieve flow prediction adapted to various spatial and temporal granularities~\cite{cardia2022enhancing}. 
Researchers have integrated individual and collective mobility data better to understand human mobility~\cite{chen2023multi, bontorin2024mixing,li2023learning}. TrGNN~\cite{li2021traffic} uses vehicle trajectories to infer short-term traffic flow, predicting unseen and non-recurring traffic patterns. GETNext~\cite{yang2022getnext} constructs a global flow graph to integrate transition patterns into trajectory prediction. With the emergence of large language models (LLMs), researchers have begun exploring their potential in mobility prediction, such as LLM-Mob~\cite{wang2023would},  AgentMove~\cite{feng2024agentmove}, TrajAgent~\cite{du2024trajagent} and CoPB~\cite{shao2024beyond}.
However, there is still a gap in using LLM to understand and reason about human behavior. Thus, it is necessary to develop foundational models from scratch, specifically trained on pure mobility data. Table~\ref{tab:model_comparison} compares the advantages of our model with existing solutions. 
In this work, we build a universal model using different types of mobility data, which can effectively predict both trajectories and flows, demonstrating exceptional robustness.


\paragraph{\textbf{Diffusion Models and Foundation Models}}
We focus on human mobility studies based on diffusion models. 
DiffTraj~\cite{zhou2023towards} and ControlTraj~\cite{zhu2024controltraj} combine the generative capabilities of diffusion models with spatiotemporal features derived from trajectories. PriSTI~\cite{liu2023pristi} uses a conditional diffusion framework for spatiotemporal imputation with enhanced prior modeling. TrajGDM~\cite{chu2024simulating} utilizes diffusion models to capture the universal mobility pattern in a trajectory dataset.
Foundation models have revolutionized natural language processing~\cite{achiam2023gpt} and computer vision~\cite{liu2024sora} through their ability to generalize across different applications. 
% For example, GPT~\cite{achiam2023gpt}, as a foundation model in natural language processing, learns general language features by pre-training on text datasets and then adapts better to specific tasks through fine-tuning on task-specific data.
% In other fields, such as computer vision, most foundation models are based on diffusion models. For example, Sora~\cite{liu2024sora} works by breaking down videos into visual patches and utilizing the diffusion transformer to create cohesive and detailed video content.
% DiffusionSat~\cite{khanna2023diffusionsat} addresses the issue that existing diffusion models cannot support remote sensing data and build a foundation model for satellite imagery.
Building on the success of foundation models in these fields, extending them to the spatiotemporal domain is a natural next step. Although models like UniST~\cite{yuan2024unist} and GPD~\cite{yuan2024spatio}, have made some progress in predicting collective dynamics, there remains a significant gap in universal models capable of adapting to various types of mobility data. 
In this work, we present the UniMob model, marking the first attempt to apply a universal mobility prediction model based on a diffusion transformer.
We provide more discussions of diffusion models in Appendix~\ref{sec::diffusion}.

\begin{table}[t]
    \centering
    \caption{Comparison of UniMob with other mobility prediction models regarding important properties.}
    \vspace{-0.3cm}
    \label{tab:model_comparison}
    {\fontsize{8}{10}\selectfont
    \begin{tabular}{ccccc}
    \toprule
    \makecell{Model} & \makecell{Trajectory\\ Prediction} & \makecell{Flow\\ Prediction} & \makecell{Data \\  Fusion$^{(1)}$} & \makecell{Robustnes$^{(2)}$} \\ 
    \midrule

    DeepMove~\cite{feng2018deepmove} & \cmark & \xmark & \xmark & \xmark \\ 
    SNPM~\cite{yin2023next} & \cmark & \xmark & \xmark & \xmark \\ 
    TrajGDM~\cite{chu2024simulating} & \cmark & \xmark & \xmark & \cmark \\ 
    ST-ResNet~\cite{zhang2017deep} & \xmark & \cmark & \xmark & \xmark \\ 
    STID~\cite{shao2022spatial} & \xmark & \cmark & \xmark & \xmark \\ 
    PriSTI~\cite{liu2023pristi} & \xmark & \cmark & \xmark & \cmark \\ 
    TrGNN~\cite{li2021traffic} & \xmark & \cmark & \cmark & \xmark \\ 
    GETNext~\cite{yang2022getnext} & \cmark & \xmark & \cmark & \xmark \\ 
    LLM-Mob~\cite{wang2023would} & \cmark & \xmark & \xmark & \cmark \\ \midrule
    UniMob & \cmark & \cmark & \cmark & \cmark \\ 
    \bottomrule
    \end{tabular}

    \vspace{2mm} % Add a little space between tables

    \footnotesize
    \begin{tabular}{ll}
        (1) & Use of multi-source data (trajectory and flow). \\
        (2) & Keep ability with noisy or scarce data. \\
    \end{tabular}}
\vspace{-0.5cm}
\end{table}

\section{PRELIMINARIES}
% framework
\begin{figure*}[t]
\centering
\includegraphics[width=0.95\textwidth]{figure/framework.pdf}
\vspace{-0.3cm}
\caption{The overview architecture of UniMob, which consists of four modules: (1) Multi-view Mobility Tokenizer, (2) Bidirectional Individual-Collective Alignment, (3) Joint Noise Predictor, (3) Mobility Predictor.} \label{fig:framework}
%\vspace{-0.3cm}
\end{figure*} 

\subsection{Problem Definition}
Human mobility data can be divided into two types: individual and collective. Trajectories can describe individual mobility, while flow data can characterize collective mobility.

\para{Definition 1: (Individual Trajectory).} An individual trajectory can be defined as $X^{traj}=\{(l_1, t_1),(l_2, t_2),...,(l_n, t_n)\}$, where each location $l_i$ is represented as the form in latitude and longitude coordinates or a region ID.

\para{Definition 2: (Crowd Flow).} Crowd flow includes inflow and outflow, defined as the number of people entering or leaving a region within a given time interval. The crowd flow for a region $l$ can be represented as $X^{flow}_l \in \mathbb{R}^{N \times T}$, where $T$ is the number of
time intervals, and $N$ is the dimension of the flow, such as $N = 2$ for inflow and outflow. The entire city's flow can be represented as $Y \in \mathbb{R}^{N \times T \times L}$, where $L$ is the number of regions.

\para{Problem Statement: (Mobility Prediction).} Given $p$ historical records of mobility data (which can be trajectory $X^{traj}_{[t-p:t]}$ or flow $X^{flow}_{[t-p:t]}$), our goal is to predict the future $k$ steps $X^{traj}_{[t:t+k]}$ or $X^{flow}_{[t:t+k]}$.


\subsection{Denoising Diffusion Probabilistic Model}
Diffusion models use latent variable models, denoted as $p_\theta(x_0) :=\int p_\theta(x_{0:T}) dx_{1:T}$. The latent variables $x_1, ..., x_T$ have the same dimension as the data $x_0 \sim q(x_0)$. The model uses two Markov chains: a forward chain that perturbs data into noise and a reverse chain that converts noise back into data.
The forward diffusion process:
\begin{equation}\label{equ:DDPM1}
q\left(\mathbf{x}_{1: T} \mid \mathbf{x}_{0}\right):=\prod_{t=1}^{T} q\left(\mathbf{x}_{t} \mid \mathbf{x}_{t-1}\right),
\end{equation}
where $q\left(\mathbf{x}_{t} \mid \mathbf{x}_{t-1}\right):=\mathcal{N}\left(\sqrt{1-\beta_{t}} \mathbf{x}_{t-1}, \beta_{t} \mathbf{I}\right)$. Equivalently, $x_t$ can be expressed as $x_{t}=\sqrt{\alpha_{t}} x_{0}+\left(1-\alpha_{t}\right) \epsilon$ for $\epsilon \sim \mathcal{N}(0, \mathbf{I})$, with $\alpha_{t}=\sum_{i=1}^{t}\left(1-\beta_{i}\right)$.

The reverse process denoises $x_t$ to retrieve $x_0$, where $\mathbf{x}_{T} \sim \mathcal{N}(\mathbf{0}, \mathbf{I})$. Assuming $p_\theta(x_{t-1}|x_t)$ follows a normal distribution:
\begin{equation}\label{equ:DDPM2}
\left\{
\begin{array}{l}
p_{\theta}\left(\mathbf{x}_{0: T}\right) := p\left(\mathbf{x}_{T}\right) \prod_{t=1}^{T} p_{\theta}\left(\mathbf{x}_{t-1} \mid \mathbf{x}_{t}\right), \\
p_{\theta}\left(\mathbf{x}_{t-1} \mid \mathbf{x}_{t}\right) := \mathcal{N}\left(\mathbf{x}_{t-1}; \boldsymbol{\mu}_{\theta}\left(\mathbf{x}_{t}, t\right), \sigma_{\theta}\left(\mathbf{x}_{t}, t\right) \mathbf{I}\right)
\end{array}
\right.
\end{equation}

Ho et al.~\cite{ho2020denoising} introduced denoising diffusion probabilistic models:
\begin{equation}\label{equ:DDPM4}
\begin{cases}
\boldsymbol{\mu}_{\theta}\left(\mathbf{x}_{t}, t\right)=\frac{1}{\alpha_{t}}\left(\mathbf{x}_{t}-\frac{\beta_{t}}{\sqrt{1-\alpha_{t}}} \boldsymbol{\epsilon}_{\theta}\left(\mathbf{x}_{t}, t\right)\right),\\
\sigma_{\theta}\left(\mathbf{x}_{t}, t\right)=\tilde{\beta}_{t}^{1 / 2}, \tilde{\beta}_{t}=\left\{\begin{array}{ll}
\frac{1-\alpha_{t-1}}{1-\alpha_{t}} \beta_{t} & t>1 \\
\beta_{1} & t=1
\end{array}\right.
\end{cases}
\end{equation}
where $\epsilon_\theta$ is a trainable denoising function. The objective for training the reverse process is:
\begin{equation}\label{equ:DDPM5}
\min _{\theta} \mathcal{L}(\theta):=\min _{\theta} \mathbb{E}_{\mathbf{x}_{0} \sim q\left(\mathbf{x}_{0}\right), \boldsymbol{\epsilon} \sim \mathcal{N}(\mathbf{0}, \mathbf{I}), t}\parallel\boldsymbol{\epsilon}-\boldsymbol{\epsilon}_{\theta}\left(\mathbf{x}_{t}, t\right)\parallel_{2}^{2},
\end{equation}
where $\mathbf{x}_{t}=\sqrt{\alpha_{t}} \mathbf{x}_{0}+\left(1-\alpha_{t}\right) \boldsymbol{\epsilon}$. This can be seen as a weighted variational constraint on the negative log-likelihood, reducing the significance of terms at low $t$ when little noise is present.




 % !TEX root = ../main.tex
 

 
 
 
% \subsection{Group Robust Methods with Adaptive Robust Scaling}
% \subsection{Instance-wise Robust Scaling for Improving Group Robustness}
 \subsection{Instance-wise Robust Scaling}
 \label{sec:method}
The optimal scaling factor can be applied adaptively to each test example and the instance-specific scaling has the potential to overcome the trade-off and improve accuracy even further.  
Previous approaches~\cite{seo2021unsupervised, sohoni2020no} have shown the capability to identify hidden spurious attributes via clustering on the feature space for debiased representation learning.
Likewise, we take advantage of feature clustering for adaptive robust scaling; we obtain the optimal class-specific scaling factors based on the cluster membership for each sample.
The overall algorithm of instance-wise robust scaling (IRS) is described as follows.

\begin{enumerate}
  \item Perform clustering with validation data on the feature space and store the cluster centroids.   \item Find the optimal scaling factor for each cluster.
%  \item Assign each test example to the nearest cluster using the stored cluster centroids in step 1.\vspace{-0.05cm}
%  \textbf{4)} Apply the optimal scaling factors obtained in 2) to the samples of each cluster in the test split.
  \item Apply the estimated scaling factor to the test example based on its cluster membership.
\end{enumerate}

In step 1, we use a simple \textit{k}-means clustering algorithm, where the number of clusters $K$ is set to the value that gives the highest robust coverage in the validation set. 
Empirically, numbers larger than 10, \ie, $K > 10$, yield stable and superior results, compared to the original class-specific scaling.








\section{experiments}
\subsection{Experimental Settings}
\subsubsection{Dataset}
We conduct extensive experiments on three real-world mobility datasets from Shanghai, Senegal, and Xinjiang. Each dataset includes both trajectory and flow data. The details of datasets are summarized in Appendix~\ref{sec::datasets_info}. The experiment section only presents the results for the Shanghai and Senegal datasets. Detailed results for the Xinjiang dataset can be found in Appendix~\ref{sec::Results}.


\subsubsection{Baselines}

We compare the performance of our model with state-of-the-art baselines. Previous methods could only accomplish one type of mobility data prediction task, so the baseline methods are divided into trajectory and flow prediction. 
For \textit{Flow Prediction}, we compared our model with six SOTA baselines (\textbf{HA}~\cite{sun2020predicting}, \textbf{VAR}~\cite{lu2016integrating}, \textbf{ST-ResNet}~\cite{zhang2017deep}, \textbf{MSDR}~\cite{liu2022msdr}, \textbf{STID}~\cite{shao2022spatial}, and \textbf{PriSTI}~\cite{liu2023pristi}.
For \textit{Trajectory Prediction}, we compared our model with seven SOTA baselines (\textbf{Markov}~\cite{gambs2012next}, \textbf{LSTM}~\cite{Kong2018HST}, \textbf{DeepMove}~\cite{feng2018deepmove}, \textbf{STAN}~\cite{luo2021stan}, \textbf{SNPM}~\cite{yin2023next},
\textbf{TrajGDM}~\cite{chu2024simulating}, and \textbf{GETNext}~\cite{yang2022getnext}.
We provide the details of baselines in Appendix~\ref{sec::baselines}.



\subsubsection{Metrics}
For trajectory prediction, we use \textit{Accuracy@k} to sort candidate locations by model-predict probabilities and check if the true position falls within the top k predictions.
For flow prediction, we choose mean absolute errors (\textit{MAE}), Mean Absolute Percentage Error (\textit{MAPE}), and root mean squared errors (\textit{RMSE}) as the evaluation metrics. \textit{MAE} is the mean absolute error between predicted and ground truth values. \textit{MAPE} is the mean absolute percentage error between the predicted and ground truth values. \textit{RMSE} is the square root of the mean squared error between the predicted and ground truth values. %Combining these three metrics provides a comprehensive evaluation of the model's performance.



\begin{table*}[t]
\centering
\caption{Overall Performance on Shanghai and Senegal datasets.}
\vspace{-0.3cm}
\scalebox{0.78}{ % Scale the table to fit within the page width
\begin{tabular}{lcccccccccccc}
\toprule
& \multicolumn{6}{c}{\textbf{Shanghai Dataset}} & \multicolumn{6}{c}{\textbf{Senegal Dataset}} \\
\cmidrule(lr){2-7} \cmidrule(lr){8-13}
& \multicolumn{3}{c}{\textbf{Flow Prediction}} & \multicolumn{3}{c}{\textbf{Trajectory Prediction}} & \multicolumn{3}{c}{\textbf{Flow Prediction}} & \multicolumn{3}{c}{\textbf{Trajectory Prediction}}\\
\cmidrule(lr){2-4} \cmidrule(lr){5-7} \cmidrule(lr){8-10} \cmidrule(lr){11-13}
\textbf{} & \textbf{MAE} & \textbf{MAPE(\%)} & \textbf{RMSE} & \textbf{Acc@1} & \textbf{Acc@3} & \textbf{Acc@5} & \textbf{MAE} & \textbf{MAPE(\%)} & \textbf{RMSE} & \textbf{Acc@1} & \textbf{Acc@3} & \textbf{Acc@5} \\
\midrule
HA & 35.19 & 29.76 & 42.72 & - & - & - & 20.75 & 19.32 & 31.17 & - & - & - \\
VAR & 28.25 & 25.61 & 40.14 & - & - & - & 17.20 & 15.95 & 28.43 & - & - & - \\
ST-ResNet & 21.54 & 19.02 & 34.18 & - & - & - & 15.95 & 13.84 & 26.37 & - & - & - \\
MSDR & 20.01 & 17.84 & 32.63 & - & - & - & 14.08 & 13.26 & 25.04 & - & - & - \\
STID & 18.72 & 15.17 & 30.40 & - & - & - & 13.52 & 12.31 & 23.19 & - & - & - \\
PriSTI & \underline{18.40} & \underline{14.59} & \underline{29.71} & - & - & - & \underline{13.28} & \underline{12.15} & \underline{22.80} & - & - & - \\
Markov & - & - & - & 0.2825 & 0.3986 & 0.5012 & - & - & - & 0.3894 & 0.4418 & 0.5828 \\
LSTM & - & - & - & 0.3401 & 0.4298 & 0.5737 & - & - & - & 0.4573 & 0.5185 & 0.6509 \\
DeepMove & - & - & - & 0.3813 & 0.4672 & 0.6191 & - & - & - & 0.4980 & 0.5764 & 0.7125 \\
STAN & - & - & - & 0.3975 & 0.4746 & 0.6303 & - & - & - & 0.5105 & 0.6042 & 0.7303 \\
SNPM & - & - & - & 0.4012 & 0.4797 & 0.6378 & - & - & - & 0.5236 & 0.6260 & 0.7591 \\
GETNext  & - & - & - & 0.4063 & 0.4836 & 0.6415 & - & - & - & 0.5251 & 0.6287 & 0.7638 \\
TrajGDM & - & - & - & \underline{0.4103} & \underline{0.4875} & \underline{0.6434} & - & - & - & \underline{0.5295} & \underline{0.6302} & \underline{0.7674} \\
UniMob-v1 & 17.93 & 14.01 & 28.65 & 0.4205 & 0.5024 & 0.6570 & 12.70 & 11.65 & 21.94 & 0.5403 & 0.6412 & 0.7889 \\
UniMob-v2 & 17.89 & 13.98 & 28.60 & 0.4228 & 0.5057 & 0.6615 & 12.52 & 11.50 & 21.57 & 0.5439 & 0.6450 & 0.7924 \\
UniMob-v3 & 17.90 & 13.96 & 28.63 & 0.4213 & 0.5040 & 0.6593 & 12.61 & 11.59 & 21.73 & 0.5415 & 0.6436 & 0.7907 \\
UniMob-v4 & \textbf{17.76} & \textbf{13.93} & \textbf{28.50} & \textbf{0.4267} & \textbf{0.5091} & \textbf{0.6653} & \textbf{12.08} & \textbf{11.12} & \textbf{21.03} & \textbf{0.5486} & \textbf{0.6515} & \textbf{0.7993} \\
%Improvement & 3.48\% & 4.52\% & 4.07\% & 4.00\% & 4.43\% & 2.81\% & 9.04\% & 8.48\% & 7.76\% & 3.61\% & 3.38\% & 4.16\% \\
\bottomrule
\end{tabular}
}
\label{tab:two datasets}
\end{table*}


\begin{table*}[t]
\small
\centering
\caption{Ablation study on Shanghai datasets.}
\vspace{-0.3cm}
\scalebox{1.}{
\begin{tabular}{lcccccc}
\toprule
& \multicolumn{3}{c}{\textbf{Trajectory Prediction}} & \multicolumn{3}{c}{\textbf{Flow Prediction}} \\
\cmidrule(lr){2-4} \cmidrule(lr){5-7}
& \textbf{Acc@1} & \textbf{Acc@3} & \textbf{Acc@5}
& \textbf{MAE} & \textbf{MAPE(\%)} & \textbf{RMSE}\\
\midrule
Ours & 0.4205 & 0.5024 &  0.6570 & 17.93 & 14.01 & 28.65 \\
w/o I2C loss & 0.4165 (-0.95\%) & 0.4906 (-2.35\%) &  0.6487 (-1.26\%) & 18.48 (-2.98\%) & 14.76 (-5.35\%) & 29.91 (-4.21\%) \\
w/o C2I loss & 0.4053 (-3.61\%) & 0.4849 (-3.48\%) &  0.6442 (-1.95\%)
 & 18.27 (-1.86\%) & 14.41 (-2.78\%) & 29.25 (-2.05\%)\\
w/o shared transformer & 0.4115 (-2.14\%) & 0.4882 (-2.83\%) & 0.6461 (-1.66\%) & 18.40 (-2.55\%)
 & 14.60 (-4.21\%) & 29.67 (-3.44\%) \\
w/o flow data & 0.4036(-4.02\%) & 0.4840(-3.66\%) & 0.6421(-2.27\%) & - & - & - \\
w/o trajectory data & - & - & - & 18.56(-3.51\%) & 14.86(-6.07\%) & 30.02(-4.78\%) \\
\bottomrule
\end{tabular}
}
%\vspace{-0.3cm}
\label{tab:Ablation1}
\end{table*}



\begin{table*}[t]
\small
\centering
\caption{Ablation study on Senegal datasets.}
\vspace{-0.3cm}
\scalebox{1.}{
\begin{tabular}{lcccccc}
\toprule
& \multicolumn{3}{c}{\textbf{Trajectory Prediction}} & \multicolumn{3}{c}{\textbf{Flow Prediction}} \\
\cmidrule(lr){2-4} \cmidrule(lr){5-7}
& \textbf{Acc@1} & \textbf{Acc@3} & \textbf{Acc@5}
& \textbf{MAE} & \textbf{MAPE(\%)} & \textbf{RMSE}\\
\midrule
Ours & 0.5403 & 0.6412 &  0.7889 & 12.70 & 11.65 & 21.94 \\
w/o I2C loss & 0.5371 (-0.59\%)
 & 0.6356 (-0.87\%)
 & 0.7620 (-3.41\%)
 & 13.32 (-4.88\%)
 & 12.12 (-4.03\%)
 & 22.91 (-4.42\%) \\
w/o C2I loss & 0.5285(-2.18\%)
 & 0.6327 (-1.33\%)
 & 0.7476 (-5.24\%)
 & 12.89 (-1.50\%)
 & 11.73 (-0.69\%)
 & 22.10 (-0.73\%)
 \\
w/o shared transformer & 0.5314 (-1.65\%)
 & 0.6331 (-1.26\%)
 & 0.7538 (-4.45\%)
 & 13.20 (-3.94\%)
 & 11.97 (-2.75\%)
 & 22.62 (-3.10\%)
 \\
w/o flow data & 0.5262(-2.61\%) & 0.6297(-1.80\%)
 & 0.7548(-4.32\%) & - & - & - \\
w/o trajectory data & - & - & - & 13.40(-5.51\%)
 & 12.18(-4.55\%) & 22.98(-4.74\%) \\
\bottomrule
\end{tabular}
}
%\vspace{-0.3cm}
\label{tab:Ablation3}
\end{table*}


\subsection{Overall Performance}
As shown in Tables~\ref{tab:two datasets}, our method demonstrates similar or better performance than the state-of-the-art baselines for all tasks on Shanghai and Senegal datasets (Please refer to Table~\ref{tab:Xinjiang} in Appendix~\ref{sec::Overall Performance} for Xingjiang dataset). We conducted multiple experiments and reported the average performance.
In flow and trajectory prediction tasks conducted on multiple real-world datasets, our UniMob model demonstrated the best performance across all evaluation metrics. Specifically, it achieved a performance improvement of over 6\% in flow prediction and 3.73\% increase in trajectory prediction.
Additionally, compared to other baseline methods, only our model can simultaneously perform flow and trajectory predictions, demonstrating that our model design effectively achieves unified human mobility prediction.
Furthermore, we used four model variants for each task. Each variant outperformed other baseline methods, maintaining flexibility to handle different scenarios while demonstrating excellent performance.
The above conclusions fully demonstrate the feasibility of a unified model in human mobility prediction. Our UniMob model can handle various types of mobility data, showcasing exceptional scalability and robustness. As the first attempt to propose a universal model paradigm for mobility prediction, we have successfully expanded the boundaries of this field.


Notably, mobility prediction models based on diffusion models, such as PriSTI and TrajGDM, demonstrate superior performance compared to other baselines. This underscores the powerful modeling capability of diffusion models in capturing the spatiotemporal correlations of mobility data. Diffusion models effectively handle dynamics and uncertainties in mobility data through an iterative denoising process, significantly enhancing prediction performance. Therefore, our UniMob model leverages diffusion models to accurately capture spatiotemporal dependencies in mobility data accurately, proving its effectiveness.



\subsection{Ablation Study}
To evaluate the impact of each module in UniMob, we conducted ablation experiments, divided into ablations of model design and data usage.
\textbf{Model Design:}
(1) w/o I2C loss: This variant keeps the model structure unchanged but removes the I2C loss.
(2) w/o C2I loss: Similar to the previous one, this variant only removes the C2I loss.
(3) w/o shared transformer: In this variant, the flow and trajectory losses no longer share a transformer; instead, each has its independent transformer.
\textbf{Data Usage:}
(4) w/o flow data: The model is trained using only trajectory data.
(5) w/o trajectory data: The model is trained using only flow data.

The results of the ablation experiments conducted on the Shanghai and Senegal datasets are shown in Tables~\ref{tab:Ablation1} and ~\ref{tab:Ablation3} (see Table~\ref{tab:Ablation2} in Appendix~\ref{sec::ablation} for the Xinjiang dataset). For the ablation experiments on model design, it is evident that the shared transformer offers limited benefits for interacting with different mobility data types. The most significant performance improvements come from task-specific loss functions. For instance, the I2C loss enhances flow prediction by using aggregated trajectory and flow data for spatiotemporal alignment. Similarly, the C2I loss uses contrastive learning to construct positive samples of flow and trajectory with similar spatiotemporal patterns, thereby aligning macro and micro mobility distribution. These experiments highlight the effectiveness of our approach in aligning trajectory and flow data.

We removed different data types for the ablation experiments on data usage and trained the model using only a single type of mobility data. The results showed a significant performance decline. This demonstrates the effectiveness and importance of our model in utilizing different types of mobility data. By combining multiple data types, UniMob can more comprehensively understand and predict human mobility behavior, thereby significantly enhancing the model's overall performance.



%\vspace{-0.2cm}
\subsection{Noise Perturbation}
In real life, mobility data often contains noise. This noise can arise from various sources, such as errors produced by sensors during the collection process or intentionally added by data operators to protect user privacy.  To assess our UniMob model's robustness, we added noise to the data and evaluated its performance.

For flow data, we introduced varying noise levels to simulate different degrees of data quality. Figure~\ref{fig:noisy_flow} shows that our model's improvement over the best baseline is relatively small without noise. When the noise level reaches 0.3, our model demonstrates a relative improvement of more than 10\%. This indicates that compared to other baseline models, UniMob exhibits better robustness in handling noisy data, making it more capable of adapting to flow data with noise for prediction. Moreover, we experimented with adding different noise levels to trajectories. Figure~\ref{fig:noisy_trajectory} shows that as the noise ratio increases, the improvement of our model relative to the best baseline also increases, achieving a maximum gain of up to 17.82\%. Because UniMob integrates two types of mobility data, allowing one type of data to provide the same spatiotemporal dynamics as a supplement when the other type of data is noisy, thereby enhancing the model's robustness. The synergistic effects between different data types can still provide reliable predictions even in noisy data.

% Flow
\begin{figure}[t]
\centering
\subfigure[Shanghai]{\includegraphics[width=.23\textwidth]{figure/shanghai_noisy_flow.pdf}}
\vspace{-0.3cm}
%\subfigure[Xinjiang]{\includegraphics[width=.30\textwidth]{figure/xinjiang_noisy_flow.pdf}}
\subfigure[Senegal]{\includegraphics[width=.23\textwidth]{figure/sainei_noisy_flow.pdf}}
\caption{Flow prediction with noisy data on Shanghai and Senegal datasets.} 
\vspace{-0.3cm}
\label{fig:noisy_flow}
\end{figure}

% Trajectory
\begin{figure}[t]
\centering
\subfigure[Shanghai]{\includegraphics[width=.23\textwidth]{figure/shanghai_noisy_trajectory.pdf}}
\vspace{-0.3cm}
%\subfigure[Xinjiang]{\includegraphics[width=.30\textwidth]{figure/xinjiang_noisy_trajectory.pdf}}
\subfigure[Senegal]{\includegraphics[width=.23\textwidth]{figure/sainei_noisy_trajectory.pdf}}
\caption{Trajectory prediction with noisy data on Shanghai and Senegal datasets.} 
\vspace{-0.3cm}
\label{fig:noisy_trajectory}
\end{figure}


\subsection{Few-shot Performance}
Similarly, the amount of mobility data may be limited in real-world scenarios due to privacy concerns, data collection challenges, or other constraints. To simulate this situation, we reduce the amount of flow and trajectory data through different operations.

As shown in Figure~\ref{fig:low_flow}, we constructed scenarios with varying proportions of locations having missing flow records. As the proportion of regions with missing flow data increased, our model still demonstrated a significant performance improvement compared to the best baseline. For instance, in the Shanghai dataset, UniMob achieves an improvement of up to 14\% when 75\% of the region is missing. UniMob's robustness is evident in its ability to maintain high performance despite the absence of a substantial amount of flow data. This is due to its ability to leverage the available trajectory data, compensating for the missing flow information through its joint modeling approach. 
As shown in Figure~\ref{fig:low_trajectory}, we used datasets of different sizes for trajectory data to explore the performance of trajectory prediction with limited data. When the amount of trajectory data is very limited (e.g., only 25\% of the dataset), our model shows a 25\% improvement in the Shanghai dataset compared to the best baseline. This indicates that when trajectory data is scarce, the flow data provides more diverse mobility patterns, effectively compensating for the lack of trajectory data.

By effectively utilizing the spatiotemporal correlations between different types of mobility data, UniMob can provide accurate predictions even in data-scarce environments. UniMob's ability to deliver reliable predictions with limited data highlights its robustness and practical applicability in various scenarios, ensuring dependable performance regardless of data constraints.

% Flow
\begin{figure}[t]
\centering
\subfigure[Shanghai]{\includegraphics[width=.23\textwidth]{figure/shanghai_low_flow.pdf}}
\vspace{-0.3cm}
%\subfigure[Xinjiang]{\includegraphics[width=.30\textwidth]{figure/xinjiang_low_flow.pdf}}
\subfigure[Senegal]{\includegraphics[width=.23\textwidth]{figure/sainei_low_flow.pdf}}
\caption{Flow prediction with scarce data on Shanghai and Senegal datasets.} 
\vspace{-0.3cm}
\label{fig:low_flow}
\end{figure}

% Trajectory
\begin{figure}[t]
\centering
\subfigure[Shanghai]{\includegraphics[width=.23\textwidth]{figure/shanghai_low_trajectory.pdf}}
\vspace{-0.3cm}
%\subfigure[Xinjiang]{\includegraphics[width=.30\textwidth]{figure/xinjiang_low_trajectory.pdf}}
\subfigure[Senegal]{\includegraphics[width=.23\textwidth]{figure/sainei_low_trajectory.pdf}}
\caption{Trajectory prediction with scarce data on Shanghai and Senegal datasets.} 
\vspace{-0.3cm}
\label{fig:low_trajectory}
\end{figure}






\section{Conclusion}
In this paper, we address an important problem of unified human mobility prediction.
We propose a universal human mobility prediction model named UniMob, achieving broad adaptability to various data formats and characteristics. 
UniMob successfully captures the spatiotemporal dynamics inherent in different modalities of mobility data by unified tokenization and bidirectional alignment between them, enabling unified modeling.
Extensive experiments on real-world datasets demonstrate that UniMob outperforms in trajectory and flow prediction. Moreover, UniMob can flexibly adapt to noisy and scarce data scenarios, showcasing its robustness.
In the future, we will explore integrating additional urban data modalities, such as weather data, social network data, and GIS data. These factors influence human mobility, we can better predict human mobility patterns by combining them with mobility data.



%\section*{Acknowledgment}

 
\clearpage
\bibliographystyle{ACM-Reference-Format}
\balance
\bibliography{sample-base}
\clearpage

\appendix
\balance

\section{APPENDIX FOR REPRODUCIBILITY}
\subsection{Related Work}
\subsubsection{Diffusion Models}\label{sec::diffusion}
The diffusion model is a probabilistic generative model first introduced by Sohl-Dickstein et al.~\cite{sohl2015deep} and further improved by Ho et al. ~\cite{ho2020denoising} and Song et al. ~\cite{song2020score}. As a novel generative model, diffusion models have rapidly advanced in time series and spatio-temporal modeling. 
Research on time series modeling based on diffusion models is widely applied, such as time series imputation ~\cite{alcaraz2022diffusion,liu2023pristi}, time series generation~\cite{lim2023regular,lin2023diffusion}, and time series forecasting~\cite{li2022generative,bilovs2022modeling}. 
DiffSTG~\cite{wen2023diffstg} is the first attempt to generalize the widespread denoising diffusion probabilistic models to spatiotemporal graphs (STGs), leading to a novel non-autoregressive framework. 
KSTDiff~\cite{zhou2023towards} designed a knowledge-enhanced denoising network to capture the spatiotemporal dependencies of urban flows and the influence of the urban environment in the denoising process.
DiffTraj~\cite{zhou2023towards} is a spatiotemporal diffusion probabilistic model for trajectory generation. This model effectively combines the generative capabilities of diffusion models with spatiotemporal features derived from real trajectories.
In this work, we introduce the diffusion model for unified mobility prediction adapted to different data types.

\subsection{Datasets Details}\label{sec::datasets_info}
We conducted extensive experiments on three real-world mobility datasets: Shanghai, Senegal, and Xinjiang. The details of datasets are summarized in Table~\ref{table:datasets}. We preprocess the trajectory data for three datasets, filtering out users with fewer than five records per day. For location preprocessing, we map GPS points to predefined grid IDs of a specific granularity. For temporal preprocessing, we organize the time data into fixed intervals, such as hourly or half-hourly segments. Finally, we divide the data into training, validation, and testing sets in a 7:1:2 ratio in chronological order. 

% dataset table
\begin{table}[h]
\setlength{\abovecaptionskip}{0.cm}
\setlength{\belowcaptionskip}{-0.cm}
\caption{Basic statistics of mobility datasets.}
\label{table:datasets}
\begin{center}
\scalebox{0.9}{
\begin{tabular}{ >{\centering\arraybackslash}m{1cm} 
>{\centering\arraybackslash}m{1cm} 
>{\centering\arraybackslash}m{1.2cm} 
>{\centering\arraybackslash}m{1cm}
>{\centering\arraybackslash}m{1cm}
>{\centering\arraybackslash}m{1cm}}
 \hline
City  & Duration & Users & Location \\ 
 \hline
Shanghai & 7 days & 700000 & 4096 \\ 
Senegal & 14 days & 8000 & 1666 \\ 
Xinjiang & 28 days & 1200000 & 4096\\ 
\hline
\end{tabular}}
\end{center}
\vspace{-0.3cm}
%\vspace{-20px}
\end{table}

\subsection{Baselines}\label{sec::baselines}
To evaluate the performance of our proposed model, we compared it with state-of-the-art models. Previous methods could only accomplish one type of mobility data prediction task, so the baseline methods are divided into trajectory and flow prediction. 

\paragraph{Flow Prediction} The baselines for flow prediction are as follows:
\begin{itemize}[leftmargin=*]
\item \textbf{HA}~\cite{sun2020predicting}: It considers the inflow and outflow to be seasonal processes and employs the average of the previous seasons as the prediction for a week-long period. 
\item \textbf{VAR}~\cite{lu2016integrating}: This method is vector autoregressive single-step predictor.
\item \textbf{ST-ResNet}~\cite{zhang2017deep}: ST-ResNet employs the residual neural network framework to model the temporal closeness, period, and trend properties of crowd flow.
\item \textbf{MSDR}~\cite{liu2022msdr}: Multi-Step Dependency Relationship (MSDR) is a brand new variant of recurrent neural networks. Instead of only looking at the hidden state from the latest time step, MSDR explicitly takes those from multiple historical time steps as the input of each time unit.
\item \textbf{STID}~\cite{shao2022spatial}: A simple multi-layer perceptron addresses the indistinguishability of time series samples in spatial and temporal dimensions.
\item \textbf{PriSTI}~\cite{liu2023pristi}: This method extracts coarse but effective spatiotemporal dependencies from conditional information using a diffusion model, serving as a global context prior.
\end{itemize}



\paragraph{Trajectory Prediction} The baselines for trajectory prediction are as follows:
\begin{itemize}[leftmargin=*]
\item \textbf{Markov Model}~\cite{gambs2012next}: The Markov model is a statistical model used to describe the change of states over time. It uses historical trajectory data for location prediction by calculating the transition probabilities between these locations.
\item \textbf{LSTM}~\cite{Kong2018HST}: The LSTM network is good at handling sequential data and has the advantage of encoding long-term dependencies, which can naturally be applied to location prediction.
\item \textbf{DeepMove}~\cite{feng2018deepmove}: The method designs a multimodal embedding recurrent neural network to capture complex sequential transitions by jointly embedding multiple factors that control human mobility.
\item \textbf{STAN}~\cite{luo2021stan}: This model associates non-contiguous but functionally similar visited points that are not adjacent to each other to predict the next location.
\item \textbf{SNPM}~\cite{yin2023next}: The method constructs a Sequence-based, Dynamic Neighbor Graph (SDNG) to find the similarity neighborhood and develop a Multi-Step Dependency Prediction model.
\item \textbf{TrajGDM}~\cite{chu2024simulating}: The method utilizes diffusion models to capture the universal mobility pattern in a trajectory dataset for trajectory prediction.
\item \textbf{GETNext}~\cite{yang2022getnext}: The method employs a global trajectory flow map and a novel Graph Enhanced Transformer model to leverage collaborative signals for more accurate trajectory prediction.
\end{itemize}

\begin{table}[h]
\small
\centering
\caption{Overall Performance on Xinjiang datasets.}
\vspace{-0.3cm}
\scalebox{0.9}{
\begin{tabular}{lcccccc}
\toprule
& \multicolumn{3}{c}{\textbf{Flow Prediction}} & \multicolumn{3}{c}{\textbf{Trajectory Prediction}} \\
\cmidrule(lr){2-4} \cmidrule(lr){5-7}
& \textbf{MAE} & \textbf{MAPE(\%)} & \textbf{RMSE}
& \textbf{Acc@1} & \textbf{Acc@3} & \textbf{Acc@5}\\
\midrule
HA & 33.16 & 30.54 & 44.28 & - & - & - \\
VAR & 23.90 & 22.15 & 36.63 & - & - & - \\
ST-ResNet & 19.72 & 17.36 & 31.56 & - & - & - \\
MSDR & 17.95 & 16.53 & 29.60 & - & - & - \\
STID & 17.01 & 15.70 & 27.36 & - & - & - \\
PriSTI & \underline{16.80} & \underline{15.37} & \underline{26.47} & - & - & - \\
Markov & - & - & - & 0.3156 & 0.3924 & 0.4571 \\
LSTM & - & - & - & 0.3847 & 0.4519 & 0.5450 \\
DeepMove & - & - & - & 0.4261 & 0.5143 & 0.6318 \\
STAN & - & - & - & 0.4432 & 0.5307 & 0.6609 \\
SNPM & - & - & - & 0.4618 & 0.5574 & 0.6926 \\
GETNext & - & - & - & 0.4650 & 0.5598 & 0.6975 \\
TrajGDM & - & - & - & \underline{0.4673} & \underline{0.5632} & \underline{0.7054} \\
UniMob-v1 & 16.31 & 14.91 & 25.98 & 0.4768 & 0.5795 & 0.7217 \\
UniMob-v2 & 15.96 & 14.72 & 25.54 & 0.4815 & 0.5853 & 0.7286 \\
UniMob-v3 & 16.12 & 14.84 & 25.70 & 0.4791 & 0.5830 & 0.7253 \\
UniMob-v4 & \bf{15.87} & \bf{14.50} & \bf{25.19} & \bf{0.4841} & \bf{0.5897} & \bf{0.7336} \\
%Improvement & 5.54\%  &	5.66\%	& 4.84\% &	3.60\% &	4.71\%	& 4.00\% \\
\bottomrule
\end{tabular}
}
%\vspace{-0.3cm}
\label{tab:Xinjiang}
\end{table}


\begin{table*}[t]
\small
\centering
\caption{Ablation study on Xinjiang datasets.}
\vspace{-0.3cm}
\scalebox{1.}{
\begin{tabular}{lcccccc}
\toprule
& \multicolumn{3}{c}{\textbf{Trajectory Prediction}} & \multicolumn{3}{c}{\textbf{Flow Prediction}} \\
\cmidrule(lr){2-4} \cmidrule(lr){5-7}
& \textbf{Acc@1} & \textbf{Acc@3} & \textbf{Acc@5}
& \textbf{MAE} & \textbf{MAPE(\%)} & \textbf{RMSE}\\
\midrule
Ours & 0.4768 & 0.5795 &  0.7217 & 16.31 & 14.91 & 25.98 \\
w/o I2C loss & 0.4736 (-0.67\%) & 0.5730 (-1.12\%) &  0.7125 (-1.28\%)
 & 16.78 (-2.88\%) & 15.43 (-3.49\%) & 27.02 (-4.00\%) \\
w/o C2I loss & 0.4689 (-1.66\%) & 0.5671 (-2.14\%) &  0.7064 (-2.12\%)
 & 16.46 (-0.92\%) & 15.08 (-1.14\%) & 26.90 (-3.54\%) \\
w/o shared transformer & 0.4702 (-1.39\%) & 0.5693 (-1.76\%)
 & 0.7091 (-1.75\%) & 16.67 (-2.21\%) & 15.29 (-2.55\%)
 & 26.97 (-3.81\%)
 \\
w/o flow data & 0.4639(-2.71\%) & 0.5620(-3.02\%)
 & 0.6998(-3.03\%) & - & - & - \\
w/o trajectory data & - & - & - & 16.87(-3.43\%)
 & 15.56(-4.36\%) & 27.20(-4.70\%) \\
\bottomrule
\end{tabular}
}
%\vspace{-0.3cm}
\label{tab:Ablation2}
\end{table*}



\section{Experimental Performance}\label{sec::Results}
\subsection{Overall Performance}\label{sec::Overall Performance}
Table~\ref{tab:Xinjiang} shows the performance of our universal mobility prediction model on the Xinjiang dataset. UniMob not only accomplishes both trajectory and flow predictions simultaneously but also surpasses current advanced baseline models in all evaluation metrics. Specifically, it achieves 5.34\% performance improvement in flow prediction and more than 4\% enhancement in trajectory prediction. These results fully demonstrate the generality and reliability of our model.





\subsection{Ablation study}\label{sec::ablation}
We conducted ablation experiments on two aspects: model design and data utilization. By sequentially removing components of the model design, we identified three design elements that align with different data formats and distributions, each impacting performance, thus validating their effectiveness. Regarding data utilization, by replacing multi-type data with single-type data, we visually demonstrated the performance enhancement brought by using multi-type mobility data in human mobility prediction through our universal model.






\subsection{Noise Perturbation}\label{sec::noise}

\begin{figure}[t]
\centering
\subfigure[Flow prediction]{\includegraphics[width=.23\textwidth]{figure/xinjiang_noisy_flow.pdf}}
\vspace{-0.5cm}
\subfigure[Trajectory prediction]{\includegraphics[width=.23\textwidth]{figure/xinjiang_noisy_trajectory.pdf}}
\caption{Flow and trajectory prediction with noisy data on Xinjiang dataset.} 
%\vspace{-0.3cm}
\label{fig:noisy}
\end{figure}

Due to biases from sensor collection and artificial noise added for privacy protection, the data used for mobility prediction often contains noise. To verify whether our model can still maintain good predictive capabilities in noisy conditions, we added noise to both the flow and trajectory data. Figure~\ref{fig:noisy} shows that as noise levels increase, our model continues to outperform the best baseline model, and our performance advantage becomes even more pronounced relative to the baseline with increasing noise. This effectively demonstrates the high robustness of our UniMob model.



\subsection{Few-shot Performance}\label{sec::few-shot}
Similarly, due to data collection and privacy protection limitations, the amount of mobility data we acquire is often limited. Therefore, we tested the few-shot learning capabilities of our UniMob model. As shown in Figure~\ref{fig:low}, our model still performs excellently even in a data-constrained environment.

\begin{figure}[t]
\centering
\subfigure[Flow prediction]{\includegraphics[width=.23\textwidth]{figure/xinjiang_low_flow.pdf}}
\vspace{-0.5cm}
\subfigure[Trajectory prediction]{\includegraphics[width=.23\textwidth]{figure/xinjiang_low_trajectory.pdf}}
\caption{Flow and trajectory prediction with scarce data on Xinjiang dataset.} 
%\vspace{-0.3cm}
\label{fig:low}
\end{figure}


\end{document}

