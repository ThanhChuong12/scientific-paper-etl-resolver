%File: anonymous-submission-latex-2025.tex
\documentclass[letterpaper]{article} % DO NOT CHANGE THIS
%\usepackage[submission]{aaai25}  % DO NOT CHANGE THIS
\usepackage{aaai25}
\usepackage{times}  % DO NOT CHANGE THIS
\usepackage{helvet}  % DO NOT CHANGE THIS
\usepackage{courier}  % DO NOT CHANGE THIS
\usepackage[hyphens]{url}  % DO NOT CHANGE THIS
\usepackage{graphicx} % DO NOT CHANGE THIS
\urlstyle{rm} % DO NOT CHANGE THIS
\def\UrlFont{\rm}  % DO NOT CHANGE THIS
\usepackage{natbib}  % DO NOT CHANGE THIS AND DO NOT ADD ANY OPTIONS TO IT
\usepackage{caption} % DO NOT CHANGE THIS AND DO NOT ADD ANY OPTIONS TO IT
\frenchspacing  % DO NOT CHANGE THIS
\setlength{\pdfpagewidth}{8.5in} % DO NOT CHANGE THIS
\setlength{\pdfpageheight}{11in} % DO NOT CHANGE THIS
%
% These are recommended to typeset algorithms but not required. See the subsubsection on algorithms. Remove them if you don't have algorithms in your paper.
\usepackage{algorithm}
\usepackage{algorithmic}
\usepackage{booktabs}
\usepackage{times}
\usepackage{amsmath}
\usepackage{inconsolata}
\usepackage{amssymb}
\usepackage{graphicx}
\usepackage{array}   
\usepackage{multirow}
\usepackage{makecell}
\usepackage{rotating}
\usepackage{caption}
\usepackage{subcaption}
\usepackage{enumitem}
\usepackage{bm}
\usepackage[bottom,hang,flushmargin]{footmisc}

%
% These are are recommended to typeset listings but not required. See the subsubsection on listing. Remove this block if you don't have listings in your paper.
\usepackage{newfloat}
\usepackage{listings}
\DeclareCaptionStyle{ruled}{labelfont=normalfont,labelsep=colon,strut=off} % DO NOT CHANGE THIS
\lstset{%
	basicstyle={\footnotesize\ttfamily},% footnotesize acceptable for monospace
	numbers=left,numberstyle=\footnotesize,xleftmargin=2em,% show line numbers, remove this entire line if you don't want the numbers.
	aboveskip=0pt,belowskip=0pt,%
	showstringspaces=false,tabsize=2,breaklines=true}
\floatstyle{ruled}
\newfloat{listing}{tb}{lst}{}
\floatname{listing}{Listing}
%
% Keep the \pdfinfo as shown here. There's no need
% for you to add the /Title and /Author tags.
\pdfinfo{
/TemplateVersion (2025.1)
}

% DISALLOWED PACKAGES
% \usepackage{authblk} -- This package is specifically forbidden
% \usepackage{balance} -- This package is specifically forbidden
% \usepackage{color (if used in text)
% \usepackage{CJK} -- This package is specifically forbidden
% \usepackage{float} -- This package is specifically forbidden
% \usepackage{flushend} -- This package is specifically forbidden
% \usepackage{fontenc} -- This package is specifically forbidden
% \usepackage{fullpage} -- This package is specifically forbidden
% \usepackage{geometry} -- This package is specifically forbidden
% \usepackage{grffile} -- This package is specifically forbidden
% \usepackage{hyperref} -- This package is specifically forbidden
% \usepackage{navigator} -- This package is specifically forbidden
% (or any other package that embeds links such as navigator or hyperref)
% \indentfirst} -- This package is specifically forbidden
% \layout} -- This package is specifically forbidden
% \multicol} -- This package is specifically forbidden
% \nameref} -- This package is specifically forbidden
% \usepackage{savetrees} -- This package is specifically forbidden
% \usepackage{setspace} -- This package is specifically forbidden
% \usepackage{stfloats} -- This package is specifically forbidden
% \usepackage{tabu} -- This package is specifically forbidden
% \usepackage{titlesec} -- This package is specifically forbidden
% \usepackage{tocbibind} -- This package is specifically forbidden
% \usepackage{ulem} -- This package is specifically forbidden
% \usepackage{wrapfig}  -- This package is specifically forbidden
% DISALLOWED COMMANDS
% \nocopyright -- Your paper will not be published if you use this command
% \addtolength -- This command may not be used
% \balance -- This command may not be used
% \baselinestretch -- Your paper will not be published if you use this command
% \clearpage -- No page breaks of any kind may be used for the final version of your paper
% \columnsep -- This command may not be used
% \newpage -- No page breaks of any kind may be used for the final version of your paper
% \pagebreak -- No page breaks of any kind may be used for the final version of your paperr
% \pagestyle -- This command may not be used
% \tiny -- This is not an acceptable font size.
% \vspace{- -- No negative value may be used in proximity of a caption, figure, table, section, subsection, subsubsection, or reference
% \vskip{- -- No negative value may be used to alter spacing above or below a caption, figure, table, section, subsection, subsubsection, or reference

\setcounter{secnumdepth}{0} %May be changed to 1 or 2 if section numbers are desired.

% The file aaai25.sty is the style file for AAAI Press
% proceedings, working notes, and technical reports.
%

% Title

% Your title must be in mixed case, not sentence case.
% That means all verbs (including short verbs like be, is, using,and go),
% nouns, adverbs, adjectives should be capitalized, including both words in hyphenated terms, while
% articles, conjunctions, and prepositions are lower case unless they
% directly follow a colon or long dash
\title{Channel Merging: Preserving Specialization for Merged Experts}
\author{
    %Authors
    % All authors must be in the same font size and format.
    Mingyang Zhang\textsuperscript{\rm 1},
    Jing Liu\textsuperscript{\rm 2}, 
    Ganggui Ding\textsuperscript{\rm 1}, 
    Linlin Ou\textsuperscript{\rm 3}, 
    Xinyi Yu\textsuperscript{\rm 3}, \\
    Bohan Zhuang\textsuperscript{\rm 1}\thanks{BZ is the corresponding author.} 
}
\affiliations{
    %Afiliations
    \textsuperscript{\rm 1} Zhejiang University\\
    \textsuperscript{\rm 2} Monash University \\
    \textsuperscript{\rm 3} Zhejiang University of Technology
    % If you have multiple authors and multiple affiliations
    % use superscripts in text and roman font to identify them.
    % For example,

    % Sunil Issar\textsuperscript{\rm 2},
    % J. Scott Penberthy\textsuperscript{\rm 3},
    % George Ferguson\textsuperscript{\rm 4},
    % Hans Guesgen\textsuperscript{\rm 5}
    % Note that the comma should be placed after the superscript
%
% See more examples next
}

%Example, Single Author, ->> remove \iffalse,\fi and place them surrounding AAAI title to use it
% \iffalse
% \title{My Publication Title --- Single Author}
% \author {
%     Author Name
% }
% \affiliations{
%     Affiliation\\
%     Affiliation Line 2\\
%     name@example.com
% }
% \fi

% \iffalse
% %Example, Multiple Authors, ->> remove \iffalse,\fi and place them surrounding AAAI title to use it
% \title{My Publication Title --- Multiple Authors}
% \author {
%     % Authors
%     First Author Name\textsuperscript{\rm 1},
%     Second Author Name\textsuperscript{\rm 2},
%     Third Author Name\textsuperscript{\rm 1}
% }
% \affiliations {
%     % Affiliations
%     \textsuperscript{\rm 1}Affiliation 1\\
%     \textsuperscript{\rm 2}Affiliation 2\\
%     firstAuthor@affiliation1.com, secondAuthor@affilation2.com, thirdAuthor@affiliation1.com
% }
% \fi


% REMOVE THIS: bibentry
% This is only needed to show inline citations in the guidelines document. You should not need it and can safely delete it.
\usepackage{bibentry}
% END REMOVE bibentry

\begin{document}

\maketitle

\begin{abstract}
Lately, the practice of utilizing task-specific fine-tuning has been implemented to improve the performance of large language models (LLM) in subsequent tasks.
Through the integration of diverse LLMs, the overall competency of LLMs is significantly boosted. Nevertheless, traditional ensemble methods are notably memory-intensive, necessitating the simultaneous loading of all specialized models into GPU memory.
To address the inefficiency, model merging strategies have emerged, merging all LLMs into one model to reduce the memory footprint during inference. Despite these advances, model merging often leads to parameter conflicts and performance decline as the number of experts increases. Previous methods to mitigate these conflicts include post-pruning and partial merging. However, both approaches have limitations, particularly in terms of performance and storage efficiency when merged experts increase. To address these challenges, we introduce Channel Merging, a novel strategy designed to minimize parameter conflicts while enhancing storage efficiency. This method clusters and merges channel parameters based on their similarity to form several groups offline. By ensuring that only highly similar parameters are merged within each group, it significantly reduces parameter conflicts. During inference, we can instantly look up the expert parameters from the merged groups, preserving specialized knowledge. 
Our experiments demonstrate that Channel Merging consistently delivers high performance, matching unmerged models in tasks like English and Chinese reasoning, mathematical reasoning, and code generation. Moreover, it obtains results comparable to model ensemble with just 53\% parameters when used with a task-specific router.
\end{abstract}

% Uncomment the following to link to your code, datasets, an extended version or similar.
%
% \begin{links}
%     \link{Code}{https://aaai.org/example/code}
%     \link{Datasets}{https://aaai.org/example/datasets}
%     \link{Extended version}{https://aaai.org/example/extended-version}
% \end{links}


\begin{figure*}[htbp]
    \centering
    \includegraphics[width=1\linewidth]{imgs/overview.pdf}
    \vspace{-2em}
    \caption{ \label{fig:overview} This diagram contrasts various methods of handling multiple experts in LLMs. Panel (a) illustrates the conventional model ensemble approach, which requires loading all expert models into memory, leading to significant storage inefficiency. Panel (b) depicts the model merging strategy that simplifies the memory load but results in performance degradation due to parameter conflicts. Panel (c) presents our proposed Channel Merging method, which clusters and merges channel parameters, retaining each expert's unique features and ensuring efficient and effective performance.}
\end{figure*}

\section{Introduction}
Recent advancements in large language models (LLMs) such as LLaMA \cite{touvron2023llama} and Mistral \cite{jiang2023mistral} have significantly pushed the boundaries of artificial intelligence, achieving near-human performance across various general tasks. Despite these achievements, a notable performance gap remains in specialized domains such as coding and mathematics. Addressing this gap, many LLMs undergo task-specific fine-tuning to enhance their capabilities within these targeted areas \cite{Chinese-Mistral,yu2023metamath,wu2023pmc}. In multi-task %environments, 
scenarios,
it is common to ensemble LLMs specialized in different tasks to optimize performance. However, as shown in Figure \ref{fig:overview} (a), traditional ensembling methods \cite{tang2024merging,lu2023routing} require loading all specialized models into GPU memory, which is highly storage-intensive.
%For instance, managing the half-precision weights of four 70B-parameter LLM experts necessitates substantial hardware resources, including eight A100 (80G) GPUs, even though the actual active parameters would only require two GPUs. This significant disparity between memory usage and computational needs underscores the inefficiencies of current approaches. 

\begin{table}[htbp]
    \centering
    \scalebox{0.85}{
    \begin{tabular}{c|c|c}
    \toprule
    \toprule
         Method& Performance & Efficiency \\
        \midrule
         One-size-fit-all \cite{yu2023language}&$\times$&$\checkmark$\\ 
         Partial Merging \cite{jiang2023effective}&$\checkmark$&$\times$\\
         Channel Merging (Ours)&$\checkmark$&$\checkmark$\\
    \bottomrule
    \bottomrule
    \end{tabular}
    }
    \vspace{-0.5em}
    \caption{This table compares the scalability of different merging approaches in terms of maintaining performance and efficiency as the number of experts increases.}
    \label{table:comparison_pd_ed}
\end{table}

\noindent To combat the inefficiency of the model ensemble, model merging strategies \cite{yadav2024ties,yang2023adamerging} have been introduced, where delta parameters \cite{ilharco2022editing} from each expert are merged into the pre-trained parameters, allowing the system to load weights equivalent to a single expert during inference. Nevertheless, as shown in Figure \ref{fig:overview} (b), this one-size-fits-all merging can exacerbate parameter conflicts as the number of experts increases, often leading to a decline in downstream performance.
To mitigate parameter conflicts, previous approaches can be divided into two main strategies: (1) Post pruning, which prunes delta parameters—the alteration of the model parameters before and after fine-tuning \cite{yu2023language,yadav2024ties}. However, performance degradation still occurs as the number of merged experts increases. (2) Implementing partial merging \cite{jiang2023effective}, which merges task-agnostic parameters and maintains task-specific ones. 
%thereby maintaining distinct parameters.
Nonetheless, this strategy becomes less storage-efficient with the escalation in the number of experts, as it necessitates the retention of more separate parameters.

\noindent To effectively mitigate parameter conflicts while enhancing storage efficiency, we specifically analyze the layer-by-layer channel similarities between several experts and highlight that merging with finer granularity can further reduce parameter conflicts. Based on our analysis, we introduce a novel strategy called Channel Merging. As illustrated in Figure \ref{fig:overview}(c), this approach first clusters channel parameters that are across various experts into several groups based on their similarities, merging only those within identical groups to minimize conflicts. Subsequently, during inference, Channel Merging adaptively looks up the task-specific parameters required for each expert. Compared with model merging and model ensemble methods, Channel Merging mitigates performance degradation by retaining the unique knowledge of each expert model while reducing the total parameters loaded to GPU memory.
%As shown in Table \ref{table:comparison_pd_ed}, Channel Merging mitigates performance degradation compared to one-size-fits-all methods by retaining the unique knowledge of each expert model. Additionally, unlike partial merging, Channel Merging maintains a consistent number of parameter groups regardless of the number of experts, ensuring sustained storage efficiency.
%As shown in Table \ref{table:comparison_pd_ed}, \textbf{Channel Merging effectively maintains performance without significant degradation while also preserving storage efficiency, as the number of merged experts increases.}
\begin{figure*}[t]
    \centering
    \includegraphics[scale=0.35]{imgs/similarity.pdf}
    \caption{\label{fig:similarity} Layer-wise proportion of channel similarity between Instruction expert and other experts in (a) Mistral-7B model and (b) CodeLLaMA-7B families. The blue portions represent the proportion of channels in the Instruction expert that are more similar to the Code expert, while the red portions indicate the proportion of channels more similar to the Math expert.
    }
\end{figure*}

\noindent Our experiments, including specialized tasks like English reasoning, mathematic reasoning, code generation, and Chinese reasoning demonstrate that Channel Merging achieves performance on par with unmerged models. Additionally, we utilize a task-specific router to optimize expert selection for each query, enhancing the model's overall effectiveness. Our findings from general tasks show that Channel Merging, combined with the router, not only outperforms the Chinese-Mistral-7B-Instruct-v0.1 by 1.26\% on the AGIEval benchmark and achieves a 2.13\% improvement on the combined MMLU+CMMLU benchmarks but also requires only 53\% of the parameter load compared to traditional ensemble methods, clearly demonstrating its efficacy and versatility in various applications. Our contributions are mainly three-fold:
\begin{itemize}[leftmargin=*]
\item Through a detailed analysis of layer-by-layer channel similarities between different experts, we highlight the limitations of traditional model-level merging methods and demonstrate the necessity of finer-grained merging.

\item Based on our analysis, we introduce Channel Merging, a novel strategy that merges parameters at the channel level. This method mitigates parameter interference and maintains parameter efficiency as the number of experts increases.

\item The effectiveness of Channel Merging is demonstrated through extensive experimental results. For example, it shows minimal performance loss compared to one-size-fits-all and partial merging methods across various downstream tasks, such as English reasoning, mathematical reasoning, code generation, and Chinese reasoning. Additionally, when combined with a task-specific router, it achieves performance comparable to the model ensemble method on general tasks while requiring only 53\% of the parameters.
\end{itemize} 

\section{Related Work}
\textbf{Model merging.} 
As large pre-trained models are a repository of extensive knowledge, fine-tuning them for new tasks has become a prevalent method \cite{dodge2020fine}. Model merging, involving merging task-specific models fine-tuned from the same pre-trained model, has been increasingly recognized as an effective strategy to enhance generalization and multi-task capabilities in LLMs \cite{yadav2024ties,yang2023adamerging,yu2023language,matena2022merging,ilharco2022editing,ainsworth2022git,entezari2021role}. 
Although model merging provides enhanced flexibility and utility \cite{daheim2023model,NEURIPS2022_70c26937}, straightforward techniques like model averaging \cite{wortsman2022model} often lead to substantial performance degradation across multiple tasks due to parameter conflicts. To address these conflicts, strategies such as TIES Merging \cite{yadav2024ties} and DARE \cite{yu2023language} have been proposed, which involve pruning some delta parameters before the merging process. Nonetheless, parameter conflicts can still be stringent as the number of merged models increases. Another mitigation strategy is partial merging \cite{stoica2023zipit,kim2023solar,jiang2023effective}, which involves merging only a part of the parameters while preserving others independently. For example, ZipIT \cite{stoica2023zipit} selectively unmerges certain layers, effectively creating a multi-head model. Passthrough \cite{kim2023solar} concatenates layers from different LLMs, producing a deeper model. BYOM \cite{jiang2023effective} preserves some task-specific parameters according to magnitude. However, these partial merging methods become storage-inefficient as the number of merged experts increases. In contrast, our Channel Merging approach merges experts into fixed groups, thereby retaining storage efficiency even as the number of merged experts grows. To mitigate parameter conflicts, Channel Merging operates at the channel level and considers the similarity between channels of different experts. As shown in Table \ref{table:comparison_pd_ed}, Channel Merging preserves both performance and efficiency as the number of merged experts increases.

\noindent\textbf{Large language model ensemble.} 
LLM ensembling aims to combine off-the-shelf large language models (LLMs) to consistently improve performance across a variety of downstream tasks. LLM-BLENDER \cite{jiang2023llm} infers outputs from all candidate models and then ranks them using a reward function, which introduces significant computational overhead. To mitigate this, FrugalGPT \cite{chen2023frugalgpt} adopts a sequential inference process that stops as soon as it generates a response of sufficient quality, thereby reducing the need to infer from all models. Additionally, to cut down on computation further, several router-based methods \cite{tang2024merging,lu2023routing,shnitzer2023large} have been developed, which employs a trained routing function that accurately directs each query to the LLM best suited for it. Consequently, only one expert LLM is activated during each inference cycle. For instance, \citep{shnitzer2023large} shows the utility and limitations of learning model routers from various benchmark datasets. Zooter \citep{lu2023routing} distills a reward model to a task-specific router, assigning queries to experts more accurately. However, these methods often require preserving all expert models' weights in GPU memory, which can lead to memory inefficiencies. In contrast, our approach leverages the channel similarities between experts by merging multiple expert parameters into a few clusters, significantly reducing the parameter storage requirements during inference. Moreover, our method dynamically activates and reconstructs different experts based on the incoming query, maintaining expert diversity while minimizing the total parameters. 

\section{Preliminaries}
\textbf{Formulation of model merging.} 
Assuming a pretrained model, let $\boldsymbol{P} \in \mathbb{R}^{O\times I}$ represent the parameters of a specific layer, where $O$ and $I$ correspond to the output and input channel number, respectively.
From this model, we derive a set of $N$ task-specific models with parameters $\boldsymbol{\theta} = \{\boldsymbol{\theta}^{t_1}, \boldsymbol{\theta}^{t_2}, ..., \boldsymbol{\theta}^{t_N}\} \in \mathbb{R}^{N\times O\times I}$, each fine-tuned for a distinct task. Model merging is the process of integrating the modifications of all task-specific models back into a single model. This is achieved by first calculating the delta parameters for each task-specific model, which represent the changes made during the fine-tuning process relative to the pretrained model. These delta parameters are defined as $\boldsymbol{\delta}^{t_n} = \boldsymbol{\theta}^{t_n} - \boldsymbol{P}$ for each task $n$. Using the task arithmetic method \cite{ilharco2022editing}, the merged parameters, $\boldsymbol{\theta^{*}} \in \mathbb{R}^{O\times I}$, are computed as follows:
\begin{equation}
    \boldsymbol{\theta^{*}} = \boldsymbol{P} + \lambda \sum_{t_n} \boldsymbol{\delta}^{t_n},
\end{equation}
where $\lambda$ is a scaling factor that adjusts the influence of the delta parameters on the merged model. 

\noindent\textbf{Task-specific routing.}
To address computational efficiency concerns within LLM ensembles, task-specific routing is employed. This technique strategically selects a single LLM expert, denoted as $m^{*}$, that is best suited to respond to a given query $q$. This selection is determined through a scoring function that evaluates the appropriateness of each expert for the query, formalized as follows:
\begin{equation}
    m^{*} = \underset{m\in M}{\mathrm{argmax}} \ \mathcal{Z}(q, m),
\end{equation}
where $M$ represents the set of all available LLM experts, and $\mathcal{Z}(\cdot)$ is a function that scores each expert $m$ based on its predicted effectiveness in responding to query $q$. The expert with the highest score is chosen for the task, optimizing the ensemble’s computational resources by activating only the most relevant model.

\begin{figure*}[htbp]
    \centering
    \includegraphics[width=1\linewidth]{imgs/method_detail.pdf}
    \caption{\label{fig:method}An illustration of our Channel Merging method. The process involves two core parts: Channel-wise Merging and Instant Lookup. In the channel-wise Merging stage, for each channel, delta parameters from each expert $\mathbf{\delta}_{i}^{t_n}$ are clustered into different groups, and parameters within the same group are merged into $\mathbf{\Theta}_{i}^{k}$. $S^{t_n}$ records the group index that each expert’s parameters have been merged into (shown in ForestGreen). During inference, the activated expert can instantly look up its parameters from the groups based on $S^{t_n}$.}
\end{figure*}


\section{Method}
In this section, we begin by analyzing the inconsistency in channel similarity across different expert models, illustrating how simplistic merging strategies—such as merging models at the model level—can be sub-optimal. Subsequently, we introduce Channel Merging, a channel-wise merging strategy tailored to optimize both parameter compression and model performance. As illustrated in Figure \ref{fig:overview} (c), Channel Merging groups parameters from different experts in the model family according to their channel similarities. During the inference, the parameters of the activated expert are instantly looked up from these offline merged groups, ensuring efficient and effective performance.

\subsection{Similarity Inconsistency}
Before we delve into Channel Merging, it’s essential to address a fundamental question: Why merge at the channel level instead of the model level, as was common in previous methods? To answer this, we specifically analyze the layer-by-layer channel similarities between the Instruction expert and the other two experts—Code and Math. This analysis involves quantifying how many channels in the Instruction expert are more similar to the Code expert compared to the Math expert based on their cosine similarities. We employ cosine similarity to measure the similarity between parameters since high cosine similarity between neural network parameters correlates with similar activations produced by the layers \cite{mason2024makes,klabunde2023towards}. The details of LLM candidates used for comparison can be found in \textbf{Appendix}.

\noindent In the Mistral-7B model family (Figure \ref{fig:similarity}(a)), overall, the Instruction expert channels show closer similarity to the Code expert channels, with the proportion of more similar channels typically being higher. However, even in the layers where the similarity proportion peaks, more than 20\% of Instruction channels are closer to the Math expert, highlighting significant similarity inconsistency between those experts. Moreover, the CodeLLaMA-7B model family (Figure \ref{fig:similarity}(b)) exhibits more random variations in similarity. Approximately half of the channels in all layers show greater similarity to the Code expert, while the other half are more aligned with the Math expert. These experiments highlight a crucial insight: \textbf{Finer granularity, such as channel-level merging, can further reduce parameter conflicts arising from insufficiently similar parameters.}

\subsection{Merging with Channel Similarity}
To effectively merge multiple expert LLMs while accommodating significant variations in channel similarities, we introduce a novel method termed Channel Merging. As illustrated in Figure \ref{fig:method}, our method unfolds in two parts: Channel-wise Merging and Instant Lookup.

\begin{table*}[htbp]
\renewcommand{\arraystretch}{1.3}
% \vspace{-0.1in}
\caption{Comparison on downstream tasks. `Baseline' refers to the performance metrics of experts when unmerged. TIES-CM and DARE-CM represent TIES and DARE methods combined with Channel Merging, respectively.}
\vspace{-0.5em}
\centering
\scalebox{0.75}
{
\begin{tabular}{c|ccc|ccc|ccc|cccc}
\toprule
\toprule
\multirow{2}{*}{Method} & \multicolumn{3}{c}{Instruction Expert (\%) $\uparrow$} & \multicolumn{3}{c}{Math Expert (\%) $\uparrow$} & \multicolumn{3}{c}{Code Expert (\%) $\uparrow$} & \multicolumn{3}{c}{Chinese Expert (\%) $\uparrow$} \\
\cmidrule(l){2-4} \cmidrule(l){5-7} \cmidrule(l){8-10} \cmidrule(l){11-13}
% \cmidrule{l}{2-12}
& CommonSenseQA & TriviaQA & Avg. & GSM8K & Math & Avg. & HumanEval & MBPP & Avg.& CMMLU & CEVAL & Avg. \\
\midrule
Baseline& 75.86& 58.39 & 67.12&73.92&20.62&47.27&45.36&43.20&44.28&47.52&47.50&47.51 \\
\midrule
BYOM&75.67&63.86&69.77&65.08&16.95&41.02&43.62&40.23&41.93&48.15&47.83&48.00\\
TIES &70.62&50.06&60.34&63.51&10.84&37.17&30.26&35.80&33.03&43.05&45.80&44.42\\
DARE&73.85&51.93&62.89&63.02&10.38&36.70&31.58&36.26&33.92&44.32&45.96&45.14\\
TIES-CM (Ours) &73.61&60.85&67.23&68.14&18.26&43.2&43.28&41.81&42.55&47.93&47.38&47.66\\
DARE-CM (Ours)&\textbf{75.27}&\textbf{64.49}&\textbf{69.88}&\textbf{70.05}&\textbf{19.89}&\textbf{44.95}&\textbf{45.13}&\textbf{43.00}&\textbf{44.06}&\textbf{48.41}&\textbf{47.69}&\textbf{48.05}\\
DARE-CM + router (Ours)&75.27&64.49&69.88&70.01&19.85&44.93&45.12&43.00&44.06&48.41&47.69&48.05\\
\bottomrule
\bottomrule
\end{tabular}
}
\label{table:results_downstream_tasks}
\end{table*}

\noindent\textbf{Channel-wise merging.} In this stage, we cluster the delta parameters for each output channel from different experts based on cosine similarity. Specifically, for the $i$-th output channel, delta parameters across all experts $\boldsymbol{\delta}_{i} = \{\boldsymbol{\delta}_{i}^{t_1}, \boldsymbol{\delta}_{i}^{t_2}, ..., \boldsymbol{\delta}_{i}^{t_N}\} \in \mathbb{R}^{N \times I}$ are grouped using the K-Means clustering algorithm into $K$ clusters $\boldsymbol{C}_{i} = \{\boldsymbol{C}_{i}^{1}, \boldsymbol{C}_{i}^{2}, ..., \boldsymbol{C}_{i}^{K}\}$ ($K < N$).
%\bohan{please double check the shape}
% which can be represented as 
% \begin{equation}
%     \boldsymbol{C}_{i} = \mathrm{KMeans}(\boldsymbol{\delta}_{i}, K).
%     \label{eq:cluster}
% \end{equation}
%where $\boldsymbol{S}_{i,l}=\{\boldsymbol{S}_{i,l}^{t_1}, ..., \boldsymbol{S}_{i,l}^{t_N}\} \in \mathbb{R}^{N\times 1}$ denotes the group index that each expert parameter are clustered into which group. 
This clustering ensures that each group contains parameters with high similarity, thus reducing conflicts during merging. These similar parameters within each cluster $k$ are then merged with the task arithmetic \cite{ilharco2022editing}:
\begin{equation}
    \boldsymbol{\Theta}_{i}^k = \boldsymbol{P}_{i} + \lambda \sum_{\boldsymbol{\delta}_{i}^{t_n} \in \boldsymbol{C}_{i}^k} {\boldsymbol{\delta}_{i}^{t_n}}.
    \label{eq:merge}
\end{equation}
Repeating the clustering and merging for all channels in the layer, we can obtain $K$ groups new parameters $\mathbf{\Theta} \in \mathbb{R}^{K\times O\times I}$, where the $k$-th group parameter $\boldsymbol{\Theta}^k$ can be represented as:
%Following clustering, the parameters for all channels in layer $l$ are merged according to the clusters, forming a new parameter set:
\begin{equation}
    \boldsymbol{\Theta}^k \in \mathbb{R}^{O\times I} = \{\boldsymbol{\Theta}_{1}^k, \boldsymbol{\Theta}_{2}^k,...,\boldsymbol{\Theta}_{O}^k\}.
\end{equation}
Additionally, an index set $S^{t_n}=\{S_{1}^{t_n}, ...,S_{O}^{t_n}\} \in \mathbb{R}^{1\times O}$, where $S_{i}^{t_n} \in \{1, ..., K\}$, is maintained for each expert $t_n$, indicating which group their channel parameters are merged into. It is worth noting that the Channel-wise Merging stage is executed offline, thus imposing no additional computational overhead during inference.

\noindent\textbf{Instant lookup.} Channel Merging merges different channels from each expert into distinct groups, thereby preserving the unique characteristics of each expert compared to one-size-fits-all merging. During inference, we can selectively activate the expert that is the most suitable to respond to the input query. Since the merged parameter groups and expert index are constructed offline in the Channel-wise Merging stage, the parameter of the activated expert can be instantly looked up and concatenated from the corresponding groups according to the index stored in $S^{t_n}$. Specifically, the layer-level parameters of activated experts $\boldsymbol{\hat{\theta}}^{t_n} \in \mathbb{R}^{O\times I}$ can be formally written as:
\begin{equation}
    \boldsymbol{\hat{\theta}}^{t_n} = \underset{S_{i}^{t_n} \in S^{t_n}}{\bigoplus} \boldsymbol{\Theta}_{i}^{S_{i}^{t_n}},
\end{equation}
where $\bigoplus$ denotes the concatenation operation, which aligns the channel parameters from selected groups to reconstruct the full parameter set for each layer of the $t_n$-th expert. This concatenation ensures that the structural and functional characteristics of each expert's layers are maintained, while only the parameters of the relevant expert are activated.

\noindent\textbf{Model size reduction analysis.} Assuming each expert in a model zoo has $\Psi$ parameters, managing $N$ distinct experts would require storing $N\Psi$ parameters separately. In contrast, our Channel Merging method organizes each expert’s parameters by clustering them along the output channel dimension into $K$ distinct categories, each containing a full set of $\Psi$ parameters. Consequently, the total storage required for parameters is reduced to $K\Psi$. Although each expert maintains an index $S^{t_n}$, the size of this index is considerably smaller than $\Psi$—equivalent to the total number of channels—therefore, it can be considered negligible in the overall parameter count. With the implementation of Channel Merging, the total number of parameters necessary is effectively diminished to $\frac{K\Psi}{N\Psi} = \frac{K}{N}$ ($K < N$). Since $K$ is a constant, unlike partial merging, Channel Merging can maintain storage efficiency even as the number of merged experts increases.
\subsection{Task-specific Routing}
As shown in Figure \ref{fig:overview}(c), following the paradigm of previous LLM ensemble methods \cite{liu2024meswitch}, we employ a task-specific router to determine which expert to activate and reconstruct for a given query. To train this router efficiently, we operate under the assumption that an expert will perform optimally on queries that originate from its fine-tuning dataset. To implement this, we sample a set of queries $Q$ from the datasets of various tasks, using the originating task classes (e.g., code, math, instruction, Chinese) as the label $Y$. The optimization process for training the router is then defined as follows:
\begin{equation}
    \mathcal{Z}^* = \underset{\mathcal{Z}}{\mathrm{argmin}} \sum_{(q, y)\in (Q, Y)} -y\cdot log(\mathcal{Z}(q, m))
\end{equation}
In this equation, $\mathcal{Z}(q,m)$ calculates the probability that expert $m$ is the most suitable for handling query $q$, based on the learned task-specific affinities. Once this model is trained, the arrival of a new query triggers the task-specific router, which employs the optimized function $\mathcal{Z}^{*}$ to determine which expert should be activated. This process ensures that each query is handled by the expert most likely to achieve the best performance, thereby enhancing overall efficiency and effectiveness.

\section{Experiments}
\subsection{Experimental Setting}
%\noindent\textbf{Candidate LLMs.}
%In our experiments, we apply our method to the Mistral-7B-v0.1 model family, including several specialized LLMs: Dolphin-2.2.1-Mistral-7B \cite{dolphin-2.2.1-mistral-7b} as the instruction expert, Speechless-Code-Mistral-7B \cite{speechless-code-mistral-7b-v1.0} as the code expert, MetaMath-Mistral-7B \cite{yu2023metamath} as the math expert, and Chinese-Mistral-7B-Instruct-v0.1 \cite{Chinese-Mistral-7B-Instruct-v0.1} as the Chinese expert. These models are all fine-tuned derivatives of the foundational pretrained model Mistral-7B-v0.1 \cite{jiang2023mistral}. To efficiently assign queries to the optimal expert, We utilize the bert-base-multilingual-cased \cite{bert-base-multilingual-cased} that is a tiny model with only 110M parameters as the task-specific router. 
\noindent\textbf{Test datasets.}
% \noindent\textbf{Training and test datasets.} We collect a diverse set of instruction samples from various open-source datasets and randomly select 50k samples to train the router, including Dolphin \cite{Dolphin} for the instruction domain, MetaMathQA \cite{MetaMath-Mistral-7B} for the mathematics domain, WizardLM-evol-instruct-V2-196k \cite{WizardLM-evol-instruct-V2-196k} for the code domain, and Wizard-LM-Chinese-instruct-evol \cite{Wizard-LM-Chinese-instruct-evol} for the Chinese domain. 
To evaluate the performance of merging,
%performance degradation after merging, 
we report accuracy on several benchmarks across different domains: CommonSenseQA \cite{talmor2018commonsenseqa} and TriviaQA \cite{joshi2017triviaqa} for the instruction, GSM8K \cite{cobbe2021training} and MATH \cite{hendrycks2021measuring} for the mathematics, HumanEval \cite{chen2021evaluating} and MBPP \cite{austin2021program} for the code, and CEval \cite{huang2024c} and CMMLU \cite{li2023cmmlu} for the Chinese. Besides, we evaluate the merged model with the task-specific router on several general task benchmarks: MMLU \cite{hendrycks2020measuring}, CMMLU, and AGIEval \cite{zhong2023agieval}. We use the OpenCompass toolbox \cite{contributors2023opencompass} to evaluate all datasets.

\begin{table*}[htbp]
\renewcommand{\arraystretch}{1.3}
\caption{Comparison of general tasks across multiple domains. "Total Param." and "Activate Param." refer to the total number and the activated number of parameters, respectively.}
\vspace{-0.5em}
\centering
\scalebox{0.75}
{
\begin{tabular}{ccc|cccc|cccccc}
\toprule
\toprule
\multirow{2}{*}{Model}& \multirow{2}{*}{Total} & \multirow{2}{*}{Activate} &\multicolumn{4}{c}{AGIEval (\%) $\uparrow$} & \multicolumn{5}{c}{MMLU+CMMLU (\%) $\uparrow$} \\
\cmidrule(l){4-7} \cmidrule(l){8-13}
% \cmidrule{l}{2-12}
&Param. (B)& Param. (B)& Chinese & English & Gaokao & Avg. & Humanities & Social & Stem&Other & Chinese & Avg. \\
\midrule
Dolphin-2.2.1-Mistral-7B& 6.7 & 6.7&32.30&39.21&34.84&35.45&56.26&59.61&44.63&57.76&41.80&52.01\\
MetaMath-Mistral-7B&6.7&6.7&31.65&37.63&34.53&34.60&54.91&	58.615&	43.675&	57.35&	40.3&	50.97\\
Speechless-Code-Mistral-7b-V1.0&6.7&6.7&32.53&39.16&35.89&35.86&55.67&	59.67&	44.36&	57.89&	40.30&	51.58\\
Chinese-Mistral-7B-Instruct-v0.1&6.7&6.7&36.10&37.09&37.30&36.75&56.30&	49.95&	51.65&	56.75&	\textbf{46.80}&	52.29\\
\midrule
BYOM + router &14.7&6.7&35.51&37.93&39.41&37.62&57.86&61.83&50.03&55.97&46.08&54.35\\
Model Ensemble + router &26.8&6.7&\textbf{36.10}&\textbf{39.21}&39.44&\textbf{38.25}&\textbf{58.30}&60.32&\textbf{51.72}&57.04&44.80&\textbf{54.43}\\
DARE-CM + router (Ours) &\textbf{14.3}&6.7&35.99&38.49&\textbf{39.57}&38.01&58.29&\textbf{62.51}&46.14&\textbf{59.49}&45.69&54.42\\
\bottomrule
\bottomrule
\end{tabular}
}
\label{table:results_general_tasks}
\end{table*}
\begin{table*}[h]
\renewcommand{\arraystretch}{1.3}
\caption{Experimental results on different merging granularities.}
\centering
\scalebox{0.77}
{
\begin{tabular}{c|ccc|ccc|ccc|cccc}
\toprule
\toprule
\multirow{2}{*}{Granularity} & \multicolumn{3}{c}{Instruction Expert (\%) $\uparrow$} & \multicolumn{3}{c}{Math Expert (\%) $\uparrow$} & \multicolumn{3}{c}{Code Expert (\%) $\uparrow$} & \multicolumn{3}{c}{Chinese Expert (\%) $\uparrow$} \\
\cmidrule(l){2-4} \cmidrule(l){5-7} \cmidrule(l){8-10} \cmidrule(l){11-13}
% \cmidrule{l}{2-12}
& CommonSenseQA & TriviaQA & Avg. & GSM8K & Math & Avg. & HumanEval & MBPP & Avg.& CMMLU & CEVAL & Avg. \\
\midrule
Channel&75.27&\textbf{64.49}&\textbf{69.88}&\textbf{70.05}&\textbf{19.89}&\textbf{44.95}&\textbf{45.13}&\textbf{43.00}&\textbf{44.01}&48.41&\textbf{47.69}&\textbf{48.05} \\
Layer&74.85&63.80&69.33&67.70&18.26&42.98&43.28&41.33&42.31&\textbf{48.45}&47.53&47.99 \\
Model&\textbf{75.61}&61.20&68.41&66.20&15.20&40.7&40.25&40.20&40.23&47.38&47.20&47.29\\
\bottomrule
\bottomrule
\end{tabular}
}

\label{table:ablation_granularities}
\end{table*}

\noindent\textbf{Implementation details.} For model merging, we cluster the expert weights into several groups. Subsequently, we use the commonly used model merging algorithms from MergeKit \cite{goddard2024arcee} to merge the parameters in the same group: (1) \textbf{DARE-CM}, we randomly prune 30\% of the delta parameters for each expert before merging. (2) \textbf{TIES-CM},  we prune 30\% of the delta parameters based on their magnitude and sign for each expert before merging. $\lambda$ in Eq. (\ref{eq:merge}) is set to 0.5.  Unless otherwise specified, we define the number of groups as two. The detail of model candidates can be found in \textbf{Appendix}. The merging experiments can be done on only a single A100 GPU. To fairly assess the performance loss due to model merging, we only activate the corresponding expert for each task during downstream task testing. For general tasks, we use a task-specific router to activate different experts based on the task requirements. We show the detail of the task-specific router in \textbf{Appendix}.

\noindent\textbf{Contenders.} 
For the one-size-fit-all merging strategy, we compare with DARE \cite{yu2023language} and TIES \cite{yadav2024ties} which use the same post-pruning strategy as we mentioned in \textbf{implementation details} and merge all experts into one model. For the partial merging strategy, we compare our method with BYOM \cite{jiang2023effective}, which retains the top 30\% of parameters by magnitude for each expert and merges the remaining parameters into one group.

\begin{table*}[htbp]
\renewcommand{\arraystretch}{1.3}
\caption{Experimental results on different cluster methods.}
\centering
\scalebox{0.77}
{
\begin{tabular}{c|ccc|ccc|ccc|cccc}
\toprule
\toprule
\multirow{2}{*}{Cluster Method} & \multicolumn{3}{c}{Instruction Expert (\%) $\uparrow$} & \multicolumn{3}{c}{Math Expert (\%) $\uparrow$} & \multicolumn{3}{c}{Code Expert (\%) $\uparrow$} & \multicolumn{3}{c}{Chinese Expert (\%) $\uparrow$} \\
\cmidrule(l){2-4} \cmidrule(l){5-7} \cmidrule(l){8-10} \cmidrule(l){11-13}
% \cmidrule{l}{2-12}
& CommonSenseQA & TriviaQA & Avg. & GSM8K & Math & Avg. & HumanEval & MBPP & Avg.& CMMLU & CEVAL & Avg. \\
\midrule
KMeans&75.27&64.49&69.88&\textbf{70.05}&\textbf{19.89}&\textbf{44.95}&\textbf{45.13}&\textbf{43.00}&\textbf{44.01}&\textbf{48.41}&47.69&\textbf{48.05} \\
Random&75.01&63.45&69.23&68.84&19.07&43.96&44.21&41.00&42.61&47.82&46.05&46.94 \\
Sign&\textbf{75.93}&\textbf{64.96}&\textbf{70.45}&68.69&19.73&44.21&45.01&42.1&43.56&48.09&\textbf{47.93}&48.01\\
\bottomrule
\bottomrule
\end{tabular}
}
\label{table:ablation_cluster}
\end{table*}
\begin{figure*}[h]
    \centering
    \includegraphics[scale=0.35]{imgs/number_experts.pdf}
\caption{\label{fig:ablation_num_experts} Experimental results on four different task categories as the number of experts varies from one to six. 'Channel' and 'Model' represent the accuracy achieved with channel-level and model-level merging, respectively.}
\vspace{-0.5em}
\end{figure*}
\subsection{Main Results}
\noindent\textbf{Downstream tasks.} 
To assess the necessity of channel-level merging, we conduct experiments across various downstream tasks to compare different merging methods, with and without the incorporation of channel merging. We use unmerged-downstream models as our baseline to establish a clear performance benchmark. The results, as shown in Table \ref{table:results_downstream_tasks}, highlight the effectiveness of channel-level merging. For example, in the Instruction Expert tasks, DARE-CM improves performance significantly, achieving an average score of 69.88\%, compared to 62.89\% with the model-level DARE. This represents a substantial 6.99\% increase in performance, underscoring the reduced performance degradation offered by our channel-level approach. Similarly, in the Code Expert tasks, DARE-CM scored 44.01\% on average, outperforming DARE by 7.31\%. When comparing DARE-CM to BYOM, we observe that DARE-CM consistently outperforms BYOM across various metrics. For instance, in the Math Expert tasks, DARE-CM achieves an average score of 44.95\%, compared to BYOM's 41.02\%. Notably, our DARE-CM yields the most optimal results, maintaining performance levels close to or even surpassing the baseline in specialized tasks such as Instruction and Chinese tasks. These findings demonstrate that our method not only mitigates the performance loss but can also enhance the model’s effectiveness in handling domain-specific queries. We also compare the performance metrics with and without the use of the trained router on DARE-CM. The results in Table \ref{table:results_downstream_tasks} clearly demonstrate that the router's deployment maintains the effectiveness of our model.  

\noindent\textbf{General tasks.} Given that the goal of model merging is to achieve a more versatile model, we integrate our merged models DARE-CM with a task-specific router and compare their performance on general tasks against the unmerged baseline models. As indicated in Table \ref{table:results_general_tasks}, these benchmarks encompass a variety of tasks including those in Chinese, English, mathematics, and coding. Consequently, Channel Merging combined with the router consistently outperforms the separate unmerged models. For example, Channel Merging exhibited a 1.26\% higher average accuracy over the Chinese-Mistral-7B-Instruct-v0.1 in the AGIEval benchmark, and a 2.13\% improvement in the combined MMLU+CMMLU benchmarks. Moreover, we find that DARE-CM can surpass BYOM in several benchmarks. For instance, in the AGIEval and MMLU+CMMLU benchmarks, DARE-CM + router achieves an average score of 38.01\% and 54.42\%, notably higher than the 37.62\% and 37.62\% of BYOM + router, respectively. When compared to the traditional model ensemble with the router, Channel Merging achieves comparable outcomes across all benchmarks, requiring only 53\% of the parameter used by traditional methods.
%. Specifically, it scored only 0.24\% lower on the AGIEval and 0.01\% lower on the MMLU+CMMLU benchmarks. However, it is important to note that Channel Merging required only 53\% of the parameter volume used by traditional methods. This substantial reduction in parameter count without a significant drop in performance highlights the effectiveness of Channel Merging in creating compact yet powerful generalized models.

\subsection{Ablation Studies}
\noindent\textbf{Merging granularities.} A key distinction of our Channel Merging approach compared to other merging methods lies in the granularity at which the merge is executed—specifically, at the channel level. To assess the impact of different merging granularities on model performance, we conducted an ablation study. This study involved merging operations at three different levels: Channel, Layer, and the entire Model, with subsequent performance evaluation on downstream tasks. As shown in Table \ref{table:ablation_granularities}, channel-level merging outperformed both layer-level and model-level merging across various tasks, achieving the highest average performance metrics. This suggests that finer granularity in merging helps to reduce parameter conflicts, thereby preserving more of the downstream task performance.

\noindent\textbf{The sensitivity to the merged experts number.} To explore how the number of experts influences the effectiveness of Channel Merging, we carry out experiments that assessed the performance across various downstream tasks when integrating varying numbers of expert models. We expand our set of candidate models to include Hercules-2.5-Mistral-7B and CollectiveCognition-v1.1-Mistral-7B, allowing for the integration of up to six experts. Additionally, we compare channel-level merging with results from model-level merging.
%, where all models were consolidated into a single model. 
The experimental results, as depicted in the figures, show that, in contrast to model-level merging, channel-level merging exhibits a markedly slower rate of performance degradation as the number of merged experts increases, particularly in tasks involving mathematic reasoning and code generation. 

\noindent\textbf{The effect of clustering methods.} To validate the appropriateness of using the KMeans method for clustering channel parameters, we compare its impact on experimental results with two alternative clustering strategies: (1) Random, where channels are grouped randomly, and (2) Sign \cite{yadav2024ties}, where parameters are grouped based on having the same sign. The results, as shown in Table \ref{table:ablation_cluster}, reveal that both KMeans and Sign clustering significantly outperform the random grouping method. This indicates that logically group parameters (either by minimizing intra-cluster variance in KMeans or aligning parameter signs) lead to better performance than arbitrary grouping.

\section{Conclusion}
In this paper, we introduced Channel Merging, a novel strategy designed to enhance the efficiency and performance of merging LLMs specialized in various tasks. By clustering and merging channel parameters based on their similarities, Channel Merging mitigates the parameter conflicts associated with traditional one-size-fits-all merging methods. Through extensive experiments, we have demonstrated that Channel Merging achieves comparable performance to unmerged experts in tasks such as English reasoning, mathematical reasoning, code generation, and Chinese reasoning. Additionally, when integrated with a task-specific router, Channel Merging outperforms traditional ensemble methods in general tasks while requiring only 53\% of the parameters, showcasing significant improvements in both performance and storage efficiency.

\noindent\textbf{Limitation and future work.}
Channel Merging requires that the experts to be merged are fine-tuned from the same pretrained model. Additionally, compared to one-size-fits-all approaches, Channel Merging may increase the parameters of the merged model. In future work, we plan to explore further compression of the merged model's parameter size by setting different groups for each layer.

\bibliography{aaai25}

%\section{Paper Checklist}
\begin{enumerate}
\item Includes a conceptual outline and/or pseudocode description of AI methods introduced: Yes
\item Clearly delineates statements that are opinions, hypothesis, and speculation from objective facts and results: Yes
\item Provides well marked pedagogical references for less-familiare readers to gain background necessary to replicate the paper: Yes
\item Does this paper make theoretical contributions? No
\item Does this paper rely on one or more datasets? Yes
\item A motivation is given for why the experiments are conducted on the selected datasets: yes
\item All novel datasets introduced in this paper are included in a data appendix. NA
\item All novel datasets introduced in this paper will be made publicly available upon publication of the paper with a license that allows free usage for research purposes. NA
\item All datasets drawn from the existing literature (potentially including authors’ own previously published work) are accompanied by appropriate citations. Yes
\item All datasets drawn from the existing literature (potentially including authors’ own previously published work) are publicly available. Yes
\item All datasets that are not publicly available are described in detail, with explanation why publicly available alternatives are not scientifically satisficing. NA
\item Does this paper include computational experiments? Yes
\item Any code required for pre-processing data is included in the appendix. NA
\item All source code required for conducting and analyzing the experiments will be made publicly available upon publication of the paper with a license that allows free usage for research purposes. Yes
\item All source code implementing new methods have comments detailing the implementation, with references to the paper where each step comes from: Yes
\item If an algorithm depends on randomness, then the method used for setting seeds is described in a way sufficient to allow replication of results: Yes
\item This paper specifies the computing infrastructure used for running experiments (hardware and software), including GPU/CPU models; amount of memory; operating system; names and versions of relevant software libraries and frameworks: Yes
\item This paper formally describes evaluation metrics used and explains the motivation for choosing these metrics: Yes
\item This paper states the number of algorithm runs used to compute each reported result: Yes
\item Analysis of experiments goes beyond single-dimensional summaries of performance (e.g., average; median) to include measures of variation, confidence, or other distributional information: Yes
\item The significance of any improvement or decrease in performance is judged using appropriate statistical tests (e.g., Wilcoxon signed-rank): No
\item This paper lists all final (hyper-)parameters used for each model/algorithm in the paper’s experiments: NA
\item This paper states the number and range of values tried per (hyper-) parameter during development of the paper, along with the criterion used for selecting the final parameter setting: NA
\end{enumerate}


\newpage

\appendix

\label{sec:appendix}
\section{Bosonization} \label{App:Sec:Bosonization}
\subsection{Fermionic continuum fields for \gls{obc}}

We consider fermionic fields defined on the interval $[0, \tilde{L}]$ with \gls{obc} following Ref.~\cite{Bosonization_OBC}, which obey the boundary conditions
\begin{align}
\psi_\gamma(0) = \psi_\gamma(\tilde{L}) = 0, \quad \gamma = \mathrm{a, b}. \label{EqnApp:def_OBC}    
\end{align}
The fields can be expanded in Fourier modes
\begin{align}
\psi_\gamma(x) = \sqrt{\frac{2}{\tilde{L}}} \sum_{n = 1}^\infty \sin(k_n x) c_{\gamma, n} \label{EqnApp:Mode_expansion_OBC}
\end{align}
where $c_n$ annihilates a particle with momentum $k_n = n \pi / \tilde{L}$. Note that there are only positive momenta and accordingly only a single Fermi point at some $k_\mathrm{F} > 0$. Now, slowly varying chiral fields can be defined 
\begin{align}
\psi_\mathrm{\gamma, R}(x) = -\frac{i}{\sqrt{2 \tilde{L}}} \sum_{n = 1}^\infty e^{i (k_n - k_F) x}c_{\gamma,n}, \quad \psi_\mathrm{\gamma, L}(x) = \frac{i}{\sqrt{2 \tilde{L}}} \sum_{n = 1}^\infty e^{-i (k_n - k_F)  x}c_{\gamma, n}, \label{EqnApp:Mode_expansion_Psi_RL_OBC}    
\end{align}
Later on, the approximation of letting the sums run to $- \infty$ will be made. This is the usual approximation taken in bosonization schemes, justified by the assumption that all the relevant physics take place close to the Fermi surface where $n \approx \tilde{L} k_\mathrm{F} / \pi$. The L/R fields are composed of the same set of momentum operators and related by 
\begin{align}
\psi_\mathrm{\gamma, L}(x) = - \psi_\mathrm{\gamma, R}(-x). \label{EqnApp:relation_Psi_RL_OBC}    
\end{align}
The fermionic field can be written in terms of L/R fields as 
\begin{align}
\psi_\gamma(x) = e^{i k_\mathrm{F} x} \psi_\mathrm{\gamma, R}(x)  + e^{-i k_\mathrm{F} x} \psi_\mathrm{\gamma, L}(x) = e^{i k_\mathrm{F} x} \psi_\mathrm{\gamma, R}(x)  - e^{-i k_\mathrm{F} x} \psi_\mathrm{\gamma, R}(-x).  \label{EqnApp:LR_decomposition}    
\end{align}

\subsection{Bosonization identity and commutators}
The R fields have periodicity $\tilde{L}' = 2 \tilde{L}$ and are therefore bosonized by a constructive bosonization approach for periodic fermion fields following Schönhammer \cite{Bosonization_Schoenhammer}
\begin{align}
\psi_\mathrm{a, R}(x) &= \frac{1}{\sqrt{2 \pi \alpha}} e^{i \hat k_\mathrm{a}} e^{i \frac{2\pi}{\tilde{L}'} \Delta N_\mathrm{a} x} e^{i \vartheta_\mathrm{a}(x)} = \frac{1}{\sqrt{2 \pi \alpha}} e^{i \hat k_\mathrm{a}} e^{i \frac{\pi}{\tilde{L}} \Delta N_\mathrm{a} x} e^{i \vartheta_\mathrm{a}(x)} = \frac{1}{\sqrt{2 \pi \alpha}} e^{i \frac{\pi}{\tilde{L}} [\Delta N_\mathrm{a} +1]x} e^{i [\hat k_\mathrm{a} + \vartheta_\mathrm{a}(x)]}, \nonumber \\
\psi_\mathrm{b, R}(x) &= \frac{(-1)^{N_\mathrm{a}}}{\sqrt{2 \pi \alpha}} e^{i \hat k_\mathrm{b}} e^{i \frac{2\pi}{\tilde{L}'} \Delta  N_\mathrm{b}x} e^{i \vartheta_\mathrm{b}(x)} = \frac{(-1)^{N_\mathrm{a}}}{\sqrt{2 \pi \alpha}} e^{i \hat k_\mathrm{b}} e^{i \frac{\pi}{\tilde{L}} \Delta  N_\mathrm{b}x} e^{i \vartheta_\mathrm{b}(x)} = \frac{(-1)^{N_\mathrm{a}}}{\sqrt{2 \pi \alpha}} e^{i \frac{\pi}{\tilde{L}} [\Delta  N_\mathrm{b}+ 1]x} e^{i[\hat k_\mathrm{b} +  \vartheta_\mathrm{b}(x)]}, \label{EqnApp:Bosonization_ID_2LL}
\end{align}
for which we formally extend the summation in \eq{EqnApp:Mode_expansion_Psi_RL_OBC} to $- \infty$. In the above equation, $\alpha$ plays the role of a regularization parameter, $\Delta  N_\gamma = N_\gamma - \tilde{L} k_\mathrm{F} / \pi$ counts the number of $\gamma = \mathrm{a, b}$ particles relative to the Fermi surface, and the Hermitian operators $\hat k_\gamma$ are conjugate to the particle number in the sense that 
\begin{align}
[N_\gamma, e^{\pm i \hat k_{\gamma'}}] = \mp e^{\pm i \hat k_\gamma}\delta_{\gamma, \gamma'} \quad \Leftrightarrow \quad (N_\gamma \pm \delta_{\gamma, \gamma'}) e^{\pm i \hat k_\gamma'} = e^{\pm i \hat k_\gamma'}  N_\gamma, \label{EqnApp:N_k_relation_2}
\end{align}
while they commute among themselves and with the fields $\vartheta_\gamma$. The operator $e^{\pm i \hat k_\gamma}$ represents the particle-number changing property of $\psi_\mathrm{\gamma, R}(x)$ and anticommutes with the associated parity $(-1)^{N_\gamma}$. This ensures the anticommutation of different fermion species in \eq{EqnApp:Bosonization_ID_2LL} and provides an explicit construction of Klein factors. As a side remark, some parts of the literature claim the stronger relation $[N_\gamma, \hat k_{\gamma'}] = i \delta_{\gamma, \gamma'}$, however, corrections arise on the level of constructive bosonization that only permit the weaker statement (see again \cite{Bosonization_Schoenhammer}). All results presented here are derived using the correct commutator.

To construct the phase fields, bosonic operators are defined from the Fourier components of the electron density as 
\begin{align}
b_{\gamma, n} = \frac{-i}{\sqrt{|n|}} \sum_{m} c^\dagger_{\gamma, n} c_{\gamma, n+m}, \quad b_{\gamma, n}^\dagger = \frac{i}{\sqrt{|n|}} \sum_{m} c^\dagger_{\gamma, n +m} c_{\gamma, n}, \quad \text{for } n > 0. \label{EqnApp:def_b_n}
\end{align}
whose bosonic commutation relations 
\begin{align}
[b_{\gamma, n},b_{\gamma', n'}] = [b_{\gamma, n}^\dagger,b_{\gamma', n'}^\dagger] = 0, \quad [b_{\gamma, n},b_{\gamma', n'}^\dagger] = \delta_{\gamma, \gamma'} \delta_{n, n'} \label{EqnApp:b_com}
\end{align}
follow immediately from the properties of the fermionic operators $c_{\gamma, n}$. The fields from \eq{EqnApp:Bosonization_ID_2LL} are then 
\begin{align}
\vartheta_\gamma(x) = \sum_{n > 0} \frac{e^{- \alpha q_n / 2}}{\sqrt{n}} \left[e^{i q_n x} b_{\gamma, n} +  e^{- i q_n x}b_{\gamma, n}^\dagger \right ] \label{EqnApp:def_vartheta_ab}
\end{align}
and the commutator has been derived in \cite{Bosonization_Delft} 
\begin{align}
[\vartheta_\gamma(x), \vartheta_{\gamma'}(y)] =&\; \; \delta_{\gamma,\gamma'} i [2\arctan[(x-y)/\alpha] - \pi (x-y) /  \tilde{L}], \nonumber \\
\stackrel{\mathclap{\alpha \to 0}}{=} & \;\; \delta_{\gamma,\gamma'} i \pi [\text{sign}(x-y) -(x-y) /  \tilde{L}] \quad \text{for } x,y \in [- \tilde{L}, \tilde{L}]. \label{EqnApp:com_vartheta_ab}
\end{align}
It is convenient to define the fields
\begin{align}
\theta_\gamma(x) &= \frac{\vartheta_\gamma(x) + \vartheta_\gamma(-x)}{2} = \sum_{n > 0} \frac{e^{ - \alpha q_n / 2} \cos(q_n x)}{\sqrt{n}} \left [b_{\gamma, n} + b_{\gamma, n}^\dagger \right], \nonumber \\ 
\phi_\gamma(x) &= \frac{\vartheta_\gamma(x) - \vartheta_\gamma(-x)}{2} = \sum_{n > 0} \frac{e^{ - \alpha q_n / 2} i\sin(q_n x)}{\sqrt{n}} \left [b_{\gamma, n} - b_{\gamma, n}^\dagger \right]. \label{EqnApp:def_theta_phi}
\end{align}
which are used to express the bosonization identity in the main text. The above equation makes clear that these fields have a periodicity of $2 \tilde{L}$ as well as the properties $\theta_\gamma(-x) = \theta_\gamma(x)$, $\phi_\gamma(-x) = -\phi_\gamma(x)$, and crucially $\phi_\gamma(0) = \phi_\gamma(\tilde{L}) = 0$. Their commutators are readily derived from \eq{EqnApp:b_com} and \eq{EqnApp:com_vartheta_ab}:

\begin{align}
[\theta_{\gamma}(x), \theta_{\gamma'}(y)] = [\phi_{\gamma}(x), \phi_{\gamma'}(y)]  = 0
\end{align}
and 
\begin{align}
[\theta_{\gamma}(x), \phi_{\gamma'}(y)] &= \delta_{\gamma,\gamma'} \frac{i}{2} \big \{ 2\arctan[(x-y)/\alpha] - 2\arctan[(x+y)/\alpha] + 2 \pi y / \tilde{L} \big\} ] \quad \text{for } x,y \in [0, \tilde{L}]. 
\end{align}
Assuming that $x$ and $y$ differ by a sufficiently large amount, we can approximate $2\arctan[(x+y)/\alpha] \approx \pi$ and write
\begin{align}
[\theta_{\gamma}(x), \phi_{\gamma'}(y)] &\approx \delta_{\gamma, \gamma'} \frac{i \pi}{2} \left\{ [\text{sign}(x-y) - 1] + \frac{2 }{\tilde{L}} y \right \} \quad \text{for } x,y \in [0, \tilde{L}]. 
\end{align}

\subsubsection{Symmetric and antisymmetric fields}
Later on, we will find it useful to work with the symmetric / antisymmetric superpositions of the phase fields
\begin{align}
\vartheta_{\pm}(x) = \frac{1}{\sqrt{2}} [\vartheta_\mathrm{a}(x) \pm \vartheta_\mathrm{b}(x)] \label{EqnApp:def_vartheta_pm}
\end{align}
Using \eq{EqnApp:com_vartheta_ab}, it is straightforward to show that they satisfy the similar relations to $\vartheta_\mathrm{a/b}$, i.e.,
\begin{align}
[\vartheta_{s}(x), \vartheta_{s'}(y)] =& \;\; \delta_{s,s'} i [2\arctan[(x-y)/\alpha] - \pi (x-y) / \tilde{L}] \nonumber \\
\stackrel{\mathclap{\alpha \to 0}}{=} & \;\; \delta_{s,s'} i \pi [\text{sign}(x-y) - (x-y) / \tilde{L}] \quad \text{for } x,y \in [-\tilde{L}, \tilde{L}], \label{EqnApp:com_vartheta_pm}
\end{align}
where $s, s' = \pm$. We also introduce symmetric and antisymmetric combinations of the fields $\theta_\gamma$, $\phi_\gamma$
\begin{align}
\hat \theta_{\pm}(x) &= \frac{1}{\sqrt{2}} [\hat k_\mathrm{a} \pm  \hat k_\mathrm{b} + \theta_\mathrm{a}(x) \pm \theta_\mathrm{b}(x)], \nonumber \\
\phi_{\pm}(x) &= \frac{1}{\sqrt{2}} [\phi_\mathrm{a}(x) \pm \phi_\mathrm{b}(x)],  \label{EqnApp:theta_phi_pm}
\end{align}
where the operators $\hat k_\gamma$ are absorbed by the fields $\hat \theta_{\pm}$. The final bosonized version of our theory will be expressed in terms of these fields. Importantly, the fields from the $+$ and $-$ sector commute as is evident from the previously discussed relations.

\subsection{Bosonization of the Majorana Hamiltonian}
\subsubsection{Free part}
The free part is given by the lattice model
\begin{align}
H_0 =& -t\sum_{\gamma = \mathrm{a,b}}\sum_{j = 1}^{L-1} \left[(c_{\gamma,j}^\dagger c_{\gamma,j+1} + c_{\gamma,j+1}^\dagger c_{\gamma,j}) \right],
\end{align}
which exhibits a dispersion $2 t \cos(k a_0)$ for both species $\gamma = \mathrm{a,b}$ of fermions, where we introduced the lattice constant $a_0$. Assuming a filling fraction $\nu$ such that $k_\mathrm{F} = \frac{\nu  \pi}{a_0}$, this can be seen as a lattice approximation to the continuum theory 
\begin{align}
H_0 &\sim v_\mathrm{F} \sum_\mathrm{\gamma = a. b} \int_0^{\tilde{L}} :\left[\psi_\mathrm{\gamma,R}^\dagger(x) (-i \partial_x) \psi_\mathrm{\gamma,R}(x) + \psi_\mathrm{\gamma,L}^\dagger(x)  (i \partial_x) \psi_\mathrm{\gamma,L}(x) \right]: \mathrm d x \nonumber\\
&= v_\mathrm{F} \sum_\mathrm{\gamma  = a. b} \int_{-\tilde{L}}^{\tilde{L}}  :\left[\psi_\mathrm{\gamma,R}^\dagger(x) (-i \partial_x) \psi_\mathrm{\gamma,R}(x) \right]: \mathrm d x,
\end{align}
where $:...:$ denotes normal-ordering w.r.t. the Fermi surface and $v_\mathrm{F} = 2 t a_0 \sin(k_\mathrm{F} a_0)$. We assume that the continuum fields obey OBC in the sense of \eq{EqnApp:def_OBC} and set $\tilde{L} = (L + 1) a_0$, which can be thought of as adding an additional site at each end of the lattice where the wave functions are zero. \gls{obc} justify the second line as an immediate consequence of \eq{EqnApp:relation_Psi_RL_OBC}. Applying the standard bosonization procedure \cite{Bosonization_Delft} to the $2 \tilde{L}$ periodic fields $\psi_{\gamma,R}(x)$ yields 
\begin{align}
H_0 &\sim v_\mathrm{F}  \sum_\mathrm{\gamma = a. b} \left[\int_{-\tilde{L}}^{\tilde{L}} \frac{1}{2} :(\partial_x \vartheta_\gamma(x) )^2:  \frac{\mathrm d x}{2\pi}  + \frac{\pi}{\tilde{L}} \frac{1}{2} \Delta  N_\gamma (\Delta N_\gamma + 1) \right] \nonumber \\
&= v_\mathrm{F}  \sum_\mathrm{\gamma = a. b} \left[ \sum_{n>0} k_n b_{\gamma,n}^\dagger b_{\gamma,n} + \frac{\pi}{\tilde{L}} \frac{1}{2} \Delta  N_\gamma (\Delta N_\gamma + 1) \right],
\end{align}
where $k_n = n \frac{\pi}{\tilde{L}}$. The finite-size terms do not affect the physics in any relevant way and will vanish in the limit $\tilde{L} \to \infty$, so we neglect them in the following. The remaining part can be expressed through the fields from \eq{EqnApp:def_theta_phi} or \eq{EqnApp:theta_phi_pm} as 
\begin{align}
H_0 \;\; \stackrel{\mathclap{ \tilde{L} \to \infty}}{\sim}& \;\;\; \frac{v_\mathrm{F}}{2 \pi} \sum_\mathrm{\gamma = a, b} \int_{0}^{ \tilde{L}} :\left[ (\partial_x \theta_\gamma(x) )^2 + (\partial_x \phi_\gamma(x) )^2 \right] : \mathrm d x \nonumber \\
=& \;\;\; \frac{v_\mathrm{F}}{2 \pi} \sum_{s = {\pm}} \int_{0}^{\tilde{L}} :\left \{ [\partial_x \hat \theta_s(x)]^2 + [\partial_x \phi_s(x)]^2 \right\} : \mathrm d x.
\end{align}

\subsubsection{Pair hopping}
The pair hopping Hamiltonian is 
\begin{align}
H_W &= W \sum_{j = 1}^{L-1} \left[c_{\mathrm{a},j}^\dagger c_{\mathrm{a},j+1}^\dagger c_{\mathrm{b},j} c_{\mathrm{b},j+1} + \hc \right]
\end{align}
and we start by deriving the associated fermionic continuum theory. To this end, we take the operators on the lattice to be continuum fields evaluated at discrete positions: $c_{\gamma, j} = \sqrt{a_0} \psi_\gamma(j a_0)$. The resulting expression is

\begin{align}
H_W &= W a_0^2  \sum_{j = 1}^{L-1}\left[\psi_\mathrm{a}^\dagger[j a_0] \psi_\mathrm{a}^\dagger[(j+1) a_0] \psi_\mathrm{b}[j a_0] \psi_\mathrm{b}[(j+1) a_0] + \hc \right], \label{EqnApp:pair_hopping}
\end{align}
which we may write in terms of the L/R fields by using \eq{EqnApp:LR_decomposition}. Then, terms with oscillating prefactors $\propto e^{\pm2 i k_\mathrm{F} j a_0}$ and $e^{\pm4i k_\mathrm{F} j a_0}$ appear, whose contributions will integrate out from the long-wavelength effective theory, allowing us to neglect them \cite{Giamarchi}. At filling fraction $\nu = \frac{1}{2}$ corresponding to $k_\mathrm{F} = \frac{\pi}{2 a_0}$, terms with the prefactor $e^{\pm4i k_\mathrm{F} j a_0} = 1$ such as $\psi_\mathrm{a, R}^\dagger [j a_0] \psi_\mathrm{b, L}[j a_0] \psi_\mathrm{a, R}^\dagger[(j+1) a_0] \psi_\mathrm{b, L}[(j+1) a_0]$ should be retained. Based on the bosonization analysis presented later on, we expect these interchain backscattering terms to prevent the formation of a Majorana phase at $\nu = \frac{1}{2}$ in consistency with the numerical results of \cite{Zoller_model}.

Keeping this in mind, we find 
\begin{align}
&\psi_\mathrm{a}^\dagger[j a_0] \psi_\mathrm{a}^\dagger[(j+1) a_0] \psi_\mathrm{b}[j a_0] \psi_\mathrm{b}[(j+1) a_0] \nonumber \\
\sim & \left[\psi_\mathrm{a, R}^\dagger[j a_0] \psi_\mathrm{a, R}^\dagger[(j+1) a_0] \psi_\mathrm{b, R}[j a_0] \psi_\mathrm{b, R}[(j+1) a_0] \right . \nonumber \\
& + \psi_\mathrm{a, R}^\dagger[j a_0] \psi_\mathrm{a, L}^\dagger[(j+1) a_0] \psi_\mathrm{b, R}[j a_0] \psi_\mathrm{b, L}[(j+1) a_0] \nonumber \\
& +  e^{2 i k_\mathrm{F} a_0} \psi_\mathrm{a, R}^\dagger[j a_0] \psi_\mathrm{a, L}^\dagger[(j+1) a_0] \psi_\mathrm{b, L}[j a_0] \psi_\mathrm{b, R}[(j+1) a_0] \nonumber \\
& +  e^{-2 i k_\mathrm{F} a_0} \psi_\mathrm{a, L}^\dagger[j a_0] \psi_\mathrm{a, R}^\dagger[(j+1) a_0] \psi_\mathrm{b, R}[j a_0] \psi_\mathrm{b, L}[(j+1) a_0] \nonumber \\
& + \psi_\mathrm{a, L}^\dagger[j a_0] \psi_\mathrm{a, R}^\dagger[(j+1) a_0] \psi_\mathrm{b, L}[j a_0]\psi_\mathrm{b, R}[(j+1) a_0] \nonumber \\
& \left. + \psi_\mathrm{a, L}^\dagger[j a_0] \psi_\mathrm{a, L}^\dagger [(j+1) a_0]  \psi_\mathrm{b, L}[j a_0]\psi_\mathrm{b, L}[(j+1) a_0] \right].
\end{align}	
This expression contains hopping terms such as $\psi_\mathrm{a, R}^\dagger[j a_0] \psi_\mathrm{a, R}^\dagger[(j+1) a_0] \psi_\mathrm{b, R}[j a_0] \psi_\mathrm{b, R}[(j+1) a_0]$, which will be suppressed by the Pauli principle in the continuum limit. We thus neglect the first and last term of the previous expression. Moving forward, we note that all the fields appearing in the remaining terms such as $\psi_\mathrm{a, R}^\dagger[j a_0] \psi_\mathrm{a, L}^\dagger[(j+1) a_0] \psi_\mathrm{b, R}[j a_0] \psi_\mathrm{b, L}[(j+1) a_0]$ simply anticommute with one another, so the whole expression is already normal-ordered and there is no need for regularization / point splitting later on. Hence, we may neglect the differences of $a_0$ in the spatial arguments of the fields and arrive at the fermionic continuum theory corresponding to \eq{EqnApp:pair_hopping}:

\begin{align}
H_W \sim 2 W a_0 \left[1 - \cos(k_\mathrm{F} a_0) \right] \int_{0}^{\tilde{L}} \left [\psi_\mathrm{a, R}^\dagger(x) \psi_\mathrm{a, L}^\dagger(x) \psi_\mathrm{b, R}(x) \psi_\mathrm{b, L}(x) + \hc\right ] \mathrm d x.
\end{align}

Using \eq{EqnApp:relation_Psi_RL_OBC} and the bosonization identity \eq{EqnApp:Bosonization_ID_2LL} yields
\begin{align}
&\quad\psi_\mathrm{a, R}^\dagger(x) \psi_\mathrm{a, L}^\dagger(x) \psi_\mathrm{b, R}(x) \psi_\mathrm{b, L}(x)  = - \psi_\mathrm{a, R}^\dagger(x) \psi_\mathrm{b, R}(x) \psi_\mathrm{a, R}^\dagger(-x) \psi_\mathrm{b, R}(-x) \nonumber\\
&= - \frac{1}{(2 \pi \alpha)^2} e^{-i \vartheta_\mathrm{a}(x)} e^{-i \frac{\pi}{\tilde{L}} \Delta  N_\mathrm{a} x} e^{-i \hat k_\mathrm{a}}   (-1)^{N_\mathrm{a}}   e^{i \hat k_\mathrm{b}} e^{i \frac{\pi}{\tilde{L}} \Delta  N_\mathrm{b}x} e^{i \vartheta_\mathrm{b}(x)} \nonumber \\
&\quad \times e^{-i \vartheta_\mathrm{a}(-x)} e^{i \frac{\pi}{\tilde{L}} \Delta  N_\mathrm{a}  x} e^{-i \hat k_\mathrm{a}}    (-1)^{N_\mathrm{a}}   e^{i \hat k_\mathrm{b}} e^{-i \frac{\pi}{\tilde{L}} \Delta  N_\mathrm{b}x} e^{i \vartheta_\mathrm{b}(-x)} \nonumber \\
&=e^{-i \frac{\pi}{\tilde{L}} \Delta  N_\mathrm{a} x} e^{i \frac{\pi}{\tilde{L}} (\Delta  N_\mathrm{b}+1) x}   e^{i \frac{\pi}{\tilde{L}} (\Delta  N_\mathrm{a} -1) x}  e^{-i \frac{\pi}{\tilde{L}} (\Delta  N_\mathrm{b}+2) x}  \frac{1}{(2 \pi \alpha)^2} e^{-i \vartheta_\mathrm{a}(x)}  e^{-i \hat k_\mathrm{a}}    e^{i \hat k_\mathrm{b}}  e^{i \vartheta_\mathrm{b}(x)} \nonumber \\
&\quad \times e^{-i \vartheta_\mathrm{a}(-x)} e^{-i \hat k_\mathrm{a}}     e^{i \hat k_\mathrm{b}}  e^{i \vartheta_\mathrm{b}(-x)} \nonumber \\
&= e^{-i 2 \pi x/\tilde{L}} \frac{1}{(2 \pi \alpha)^2} e^{-i [\hat k_\mathrm{a} + \vartheta_\mathrm{a}(x) - \hat k_\mathrm{b}- \vartheta_\mathrm{b}(x)]} e^{-i [\hat k_\mathrm{a} + \vartheta_\mathrm{a}(-x) - \hat k_\mathrm{b}- \vartheta_\mathrm{b}(-x)]} \nonumber \\
&= e^{-i 2 \pi x/\tilde{L}} \frac{1}{(2 \pi \alpha)^2} e^{-i [\hat k_\mathrm{a} - \hat k_\mathrm{b}+ \sqrt{2} \vartheta_- (x)]} e^{-i [\hat k_\mathrm{a} - \hat k_\mathrm{b}+ \sqrt{2} \vartheta_-(-x) ]}.
\end{align}
Remember that the operators $\hat k_\gamma$ commute with the fields $\vartheta$. If $[X, [X,Y]] = 0$ and $[Y, [X,Y]] = 0$, the \gls{bch} identity $e^{X} e^{Y} = e^{X+Y} e^{[X,Y]/2}$ holds, and we may use the commutator \eq{EqnApp:com_vartheta_pm} to write
\begin{align}
e^{-i \sqrt{2} \vartheta_- (x)} e^{-i \sqrt{2} \vartheta_-(-x)} &= e^{-i \sqrt{2} [\vartheta_- (x) + \vartheta_- (-x)]} e^{-[\vartheta_- (x), \vartheta_- (-x)]} \nonumber \\
&= e^{-i \sqrt{2} [\vartheta_- (x) + \vartheta_- (-x)]} e^{-i\pi[ 1 - 2x/\tilde{L}} = -e^{-i \sqrt{2} [\vartheta_- (x) + \vartheta_- (-x)]} e^{i 2 \pi x/\tilde{L}}
\end{align}
The phase factor $e^{i 2 \pi x/\tilde{L}}$ precisely cancels the one from the penultimate equation. We arrive at
\begin{align}
\psi_\mathrm{a, R}^\dagger(x) \psi_\mathrm{a, L}^\dagger(x) \psi_\mathrm{b, R}(x) \psi_\mathrm{b, L}(x)  = -\frac{1}{(2 \pi \alpha)^2} e^{-i \sqrt{2} [\sqrt{2} (\hat k_\mathrm{a} - \hat k_\mathrm{b}) + \vartheta_- (x) + \vartheta_- (-x)]}
\end{align}
Using \eq{EqnApp:def_theta_phi}, \eq{EqnApp:def_vartheta_pm}, and \eq{EqnApp:theta_phi_pm} leads to the final expression 
\begin{align}
H_W \sim \frac{4[\cos(2 k_\mathrm{F} a_0) - 1] W a_0}{(2 \pi \alpha)^2} \int_0^{\tilde{L}} \cos(\sqrt{8} \hat \theta_-) \mathrm d x. 
\end{align}

\subsection{Bosonization of flux hopping}
The lattice Hamiltonian 
\begin{align}
H_\phi &= r \sum_{j = 1}^{L}  \left[ e^{2 \pi i \phi j}c_{\mathrm{a},j}^\dagger c_{\mathrm{b},j} + e^{-2 \pi i \phi j}c_{\mathrm{b},j}^\dagger c_{\mathrm{a},j} \right], \label{EqnApp:H_phi}
\end{align}
is mapped to a continuum fermionic theory in the same way as before:
\begin{align}
H_\phi &\sim r \int_{0}^{\tilde{L}} \left\{ e^{2 \pi i \phi x/a_0} \left[ \psi_\mathrm{R, a}^\dagger(x) \psi_\mathrm{R, b}(x) + \psi_\mathrm{L, a}^\dagger(x) \psi_\mathrm{L, b}(x) + e^{-2i k_\mathrm{F} x} \psi_\mathrm{R, a}^\dagger(x) \psi_\mathrm{L, b} [x]+  e^{2 i k_\mathrm{F} x} \psi_\mathrm{L, a}^\dagger(x) \psi_\mathrm{R, b}(x)\right] + \hc \right \}\mathrm d x. \label{EqnApp:H_phi_continuum} 
\end{align}
In general, the oscillating prefactors will suppress this term from the effective low-energy theory. At $\phi = 0$, the \gls{fs} terms survive, while at $\phi = \pm \nu$, one of the \gls{bs} terms will make an impact because $2 \pi \phi$ exactly cancels $2k_\mathrm{F} = \frac{2 \pi \nu }{a_0}$.

\subsubsection{Bosonization for $\phi = 0$}
At $\phi = 0$, \eq{EqnApp:H_phi_continuum} reduces to regular \gls{fs}. We apply the bosonization identity \eq{EqnApp:Bosonization_ID_2LL} to find 
\begin{align}
\left[\psi_\mathrm{R, a}^\dagger(x) \psi_\mathrm{R, b}(x) + \hc \right] &= \frac{1}{2 \pi \alpha} \left [e^{-i \vartheta_\mathrm{a}(x)} e^{-i \frac{\pi}{\tilde{L}} \Delta  N_\mathrm{a} x} e^{-i \hat k_\mathrm{a}}   (-1)^{N_\mathrm{a}}  e^{i \frac{\pi}{\tilde{L}} (\Delta  N_\mathrm{b} + 1) x} e^{i \hat k_\mathrm{b}} e^{i \vartheta_\mathrm{b}(x)} \right . \nonumber \\
&\quad+ \left .  e^{- i \vartheta_\mathrm{b}(x)} e^{-i \frac{\pi}{\tilde{L}} \Delta N_\mathrm{b} x} e^{- i \hat k_\mathrm{b}}(-1)^{N_\mathrm{a}} e^{i \frac{\pi}{\tilde{L}} (\Delta  N_\mathrm{a} + 1) x} e^{i \hat k_\mathrm{a}} e^{i \vartheta_\mathrm{a}(x)} \right] \nonumber \\
&= \frac{1}{2 \pi \alpha} (-1)^{N_\mathrm{a}} \left[e^{i \frac{\pi}{\tilde{L}} (N_\mathrm{b}  - N_\mathrm{a} + 1) x} e^{i[\hat k_\mathrm{a} - \hat k_\mathrm{b} + \sqrt{2} \vartheta_-(x)]} - \hc\right]
\end{align}
and similarly 
\begin{align}
\left[\psi_\mathrm{L, a}^\dagger(x) \psi_\mathrm{L, b}(x) + \hc \right] = \left[\psi_\mathrm{R, a}^\dagger(-x) \psi_\mathrm{R, b}(-x) + \hc \right]  = \frac{1}{2 \pi \alpha} (-1)^{N_\mathrm{a}} \left[e^{-i \frac{\pi}{\tilde{L}} (N_\mathrm{b}  - N_\mathrm{a} + 1) x} e^{i[\hat k_\mathrm{a} - \hat k_\mathrm{b} + \sqrt{2} \vartheta_-(-x)]} - \hc\right].
\end{align}
Given the relations between the various fields, it is readily seen that 
\begin{align}
\hat k_\mathrm{a} - \hat k_\mathrm{b} +  \sqrt{2} \vartheta_-(\pm x) = \sqrt{2} \hat \theta_-(x) \pm \sqrt{2} \phi_-(x),
\end{align}
which leads to the bosonized expression stated in the main text
\begin{align}
H_{\phi = 0} &\sim r \frac{(-1)^{N_\mathrm{a}}}{2 \pi \alpha} \int_{0}^{\tilde{L}}  \left[e^{i \frac{\pi}{\tilde{L}} (N_\mathrm{b}  - N_\mathrm{a} + 1) x} e^{i\sqrt{2}  [\hat \theta_-(x) + \phi_-(x)]} +  e^{- i \frac{\pi}{\tilde{L}} (N_\mathrm{b}  -  N_\mathrm{a} + 1) x} e^{i\sqrt{2} [\hat \theta_-(x) - \phi_-(x)]} - \hc \right] \mathrm d x.
\end{align}

In the main text, we argue that the relative minus sign arising from the finite size term $\pi (N_\mathrm{b}  - N_\mathrm{a} + 1) x / (\tilde{L})$ will cancel contributions to the \gls{gs} splitting from the left and right end. While this is more of a qualitative argument, the cancellation can be shown on an exact level by considering that in addition to parity symmetry, the Hamiltonian $H_0 + H_W$ is also invariant under the action of the unitary inversion symmetry $U_\mathrm{I}$ defined by
\begin{align}
U_\mathrm{I} c_{\gamma, j} U_\mathrm{I}^\dagger = c_{\gamma, N - j + 1}, \quad U_\mathrm{I} c^\dagger_{\gamma, j} U_\mathrm{I}^\dagger = c^\dagger _{\gamma, N - j + 1}. \label{EqnApp:U_I}
\end{align}
This is nothing but a mirroring at the center of the chain. $U_\mathrm{I}$ squares to one, which restricts the possible eigenvalues to $\pm 1$,  and commutes with $P_\mathrm{a}$, so the two \gls{gs} $\ket{P_\mathrm{a} = \pm 1}$ are also eigenstates of $U_\mathrm{I}$. 
To determine the relative contribution of fluxless hoppings to the \gls{gs} splitting from the left and right end, we consider a collection of hoppings located on the left side of the chain and denote it by $H_{\phi = 0}^\mathrm{L}$. Inversion symmetry maps this to the mirrored set of hoppings on the opposite side: $U_\mathrm{I} H_{\phi = 0}^\mathrm{L} U_\mathrm{I}^\dagger = H_{\phi = 0}^\mathrm{R}$. At small enough hopping amplitude $r$, it is sufficient to take into account the matrix elements between the two \gls{gs} to determine the splitting as the antisymmetric sector exhibits a large excitation gap. Since the operators will change the parity, we only need to look at the matrix element

\begin{align}
\bra{P_\mathrm{a} = 1} H_{\phi = 0}^\mathrm{R} \ket{P_\mathrm{a} = -1} = \bra{P_\mathrm{a} = 1} U_\mathrm{I} H_{\phi = 0}^\mathrm{L} U_\mathrm{I}^\dagger \ket{P_\mathrm{a} = -1} = u_\mathrm{I,+} u_\mathrm{I,-}\bra{P_\mathrm{a} = 1} H_{\phi = 0}^\mathrm{L} \ket{P_\mathrm{a} = -1}.
\end{align}
Here, $u_\mathrm{I,\pm}$ denotes the eigenvalue of $U_\mathrm{I}$ associated with the positive or negative parity eigenstate $U_\mathrm{I} \ket{P_\mathrm{a} = \pm 1} =  u_\mathrm{I,\pm} \ket{P_\mathrm{a} = \pm 1}$. Depending on whether these eigenvalues have the same or opposite signs, the contributions from the left and right end will either amplify or cancel exactly. The relative sign that we derive from the field-theoretical analysis suggests that $u_\mathrm{I,\pm}$ will have opposite sign for even $N_\mathrm{tot}$ and same sign for odd $N_\mathrm{tot}$ in consistency with numerical data. 

\subsubsection{Bosonization for $\phi = \nu$}
We derive the bosonization of \eq{EqnApp:H_phi} for $\phi = \nu$ here, the case of $\phi = -\nu$ can be treated on similar footing. Applying \eq{EqnApp:Bosonization_ID_2LL} to the \gls{bs} term appearing in \eq{EqnApp:H_phi_continuum} yields
\begin{align}
&\quad[\psi_\mathrm{R, a}^\dagger(x) \psi_\mathrm{L, b}(x) + \psi_\mathrm{L, b}^\dagger(x) \psi_\mathrm{R, a}(x) + \hc] = -[\psi_\mathrm{R, a}^\dagger(x) \psi_\mathrm{R, b}(-x) + \psi_\mathrm{R, b}^\dagger(-x) \psi_\mathrm{R, a}(x) + \hc] \nonumber \\
&= - \left[ e^{-i \vartheta_\mathrm{a}(x)} e^{-i \frac{\pi}{\tilde{L}} \Delta  N_\mathrm{a} x} e^{-i \hat k_\mathrm{a}} \frac{(-1)^{N_\mathrm{a}}}{2 \pi \alpha} e^{-i \frac{\pi}{\tilde{L}} (\Delta  N_\mathrm{b} + 1)x} e^{i \hat k_\mathrm{b}}  e^{i \vartheta_\mathrm{b}(-x)} + \hc \right ] \nonumber \\
&=\frac{(-1)^{N_\mathrm{a}}}{2 \pi \alpha} \left[ e^{-i \frac{\pi}{\tilde{L}} (\Delta  N_\mathrm{a} + \Delta  N_\mathrm{b} + 1) x}  e^{-i[\hat k_\mathrm{a} - \hat k_\mathrm{b} +  \vartheta_\mathrm{a}(x) -  \vartheta_\mathrm{b}(-x)]} - \hc \right ] 
\end{align}
We have $\vartheta_\mathrm{a}(x) - \vartheta_\mathrm{b}(-x) = [\phi_\mathrm{a}(x) + \phi_\mathrm{b}(x) + \theta_\mathrm{a}(x) - \theta_\mathrm{b}(x)]$. Keeping in mind that the fields from the symmetric and antisymmetric sector commute and that all operators without hat commute with the particle numbers, we write 
\begin{align}
[\psi_\mathrm{R, a}^\dagger(x) \psi_\mathrm{L, b}(x) + \psi_\mathrm{L, b}^\dagger(x) \psi_\mathrm{R, a}(x) + \hc] =\frac{(-1)^{N_\mathrm{a}}}{2 \pi \alpha} \left[ e^{-i [\frac{\pi}{\tilde{L}} (\Delta  N_\mathrm{a} + \Delta  N_\mathrm{b} + 1) x + \sqrt{2} \phi_+(x)]}  e^{-i \sqrt{2} \hat \theta_-(x)} - \hc \right ]. 
\end{align}
Because $e^{\pm i (k_\mathrm{a} - k_\mathrm{b})}$ does not change the total particle number, $\Delta  N_\mathrm{a} + \Delta  N_\mathrm{b}$ commutes with $e^{\pm i \sqrt{2} \hat \theta_-(x)}$, allowing us to write
\begin{align}
H_{\phi = \nu} &\sim \frac{-i r (-1)^{N_\mathrm{a}}}{\pi \alpha}  \int_{0}^{\tilde{L}} \left \{ \sin \left [\frac{\pi}{\tilde{L}} (\Delta  N_\mathrm{a} + \Delta  N_\mathrm{b} + 1) x + \sqrt{2} \phi_+(x) \right] \cos \left[ \sqrt{2} \hat \theta_-(x) \right]  \right. \nonumber \\
&\quad + \left . \cos \left [\frac{\pi}{\tilde{L}} (\Delta  N_\mathrm{a} + \Delta  N_\mathrm{b} + 1) x + \sqrt{2} \phi_+(x) \right] \sin \left[ \sqrt{2} \hat \theta_-(x) \right]  \right \} \mathrm d x. 
\end{align}
We have $k_\mathrm{F} = \frac{\nu  \pi}{a_0}$ and $\tilde{L} = (L+1) a_0$, so the finite-size term is $\Delta N_\mathrm{a} + \Delta N_\mathrm{b} = N_\mathrm{a} + N_\mathrm{b} - 2 k_F \tilde{L} / \pi = N_\mathrm{tot} - 2 \nu (L + 1)$, which yields the expression stated in the main text. 
	
To conclude this section, we argue why $H_{\phi = \nu}$ will always lift the finite-size gap based on a mean-field treatment. The bosonized version of the base model decouples into a symmetric and an antisymmetric sector, i.e., $H_0 + H_W \sim H_+ \otimes \mathbb I_- +  \mathbb I_+\otimes H_- $, such that the eigenstates can be thought of as tensor products $\ket{\psi}_+ \otimes \ket{\psi'}_-$ of $H_+$ and $H_-$ eigenstates. In the antisymmetric sector, there are two degenerate \gls{gs}  $\ket{\theta_1}_-$ and $ \ket{\theta_2}_-$ separated from the rest by a large gap, allowing us to restrict the antisymmetric sector to these two states in the spirit of degenerate perturbation theory. At negative $W$, the (Hermitian!) term $i (-1)^{N_\mathrm{a}} \cos \left[ \sqrt{2} \hat \theta_-(x) \right] $ is zero in this restricted subspace, while $i (-1)^{N_\mathrm{a}} \sin \left[ \sqrt{2} \hat \theta_-(x) \right] $ has some non-trivial action ($\sin \left[ \sqrt{2} \hat \theta_-(x) \right]$ yields $\pm 1$ when applied to the states $\ket{\theta_1}_-$, $\ket{\theta_2}_-$, while $(-1)^{N_\mathrm{a}}$ exchanges them). After diagonalizing this operator in the two-state space, the Hilbert space further decouples into states of the form $\ket{\psi}_+ \otimes \ket{\Gamma_+}_-$ and $\ket{\psi}_+ \otimes \ket{\Gamma_-}_-$, where $\Gamma_{\pm}$ denotes the eigenvalue of $i (-1)^{N_\mathrm{a}} \cos \left[ \sqrt{2} \hat \theta_-(x) \right] $.
In these two subspaces, we are left with the Sine-Gordon theory

\begin{align}
\frac{v_\mathrm{F}}{2 \pi} \int_{0}^{\tilde{L}} \left \{ [\partial_x \hat \theta_+(x)]^2 + [\partial_x \phi_+(x)]^2 \right\}  \mathrm d x -  \frac{r \Gamma_{\pm} }{\pi \alpha}  \int_{0}^{\tilde{L}} \cos \left [ \sqrt{2} \phi_+(x) \right] \mathrm d x,
\end{align}
for which standard RG-flow equations indicate the formation of a massive phase for any value of $r$ \cite{Giamarchi}, in consistency with numerical data. The same line of reasoning applies to the case $W> 0$.

\section{Critical state preparation with a spatially inhomogeneous ramp} \label{App:Sec:IH_ramp}
Spatially inhomogeneous ramps of the mass term have also been proposed in the literature as a way to achieve optimal (i.e., $\propto 1/L$) scaling of the preparation time $t_\mathrm{tot}$ \cite{crit_state_prep_3, crit_state_prep_4}. However, in the present case, we find that the homogeneous ramp with a power $p$ adjusted to system size significantly outperforms the inhomogeneous ramp approach. Concretely, we implement a procedure oriented on Ref.~\cite{crit_state_prep_4}, but with the generalization to multiple critical fronts that propagate in space with a velocity $v_\mathrm{r}$. For this, we introduce the ramp function
\begin{align}
\epsilon(u) = \mathbbm{1}_{(-\infty, \pi/2]}(u) + 0.5 [1 - \sin(u)] \mathbbm{1}_{(-\pi/2, \pi/2]}(u), \label{EqnApp:def_epsilon}
\end{align}
and the auxiliary function 
\begin{align}
u(x, t) = \min_{l = 1, ..., n_\mathrm{r}}\alpha\left[v  t  - |x - x_l| - d/2 \right], \label{EqnApp:def_epsilon}
\end{align}
where $x_l = 1 + \Delta_{n_\mathrm{r}} (2l - 1) $ is the starting position of the $l$th front, the spacing is $\Delta_{n_\mathrm{r}} = (L - 1) / (2 n_r)$, and the offset is $d = \pi / \alpha$. The composition $\epsilon(u(x, t))$ describes $n_\mathrm{r}$ fronts starting at evenly spaced points on the interval $[1, L]$ and propagating at velocity $v$ through the system. The parameter $\alpha$ controls the smoothness of the ramp by smearing out each front over the distance $d$. 

We then consider the flux-hopping Hamiltonian with spatially varying and time-dependent amplitudes  
\begin{align}
H_\phi(t) &=  \sum_{j = 1}^L  r_j(t) \left[ e^{2 \pi i \phi j}c_{\mathrm{a},j}^\dagger c_{\mathrm{b},j} + e^{-2 \pi i \phi j}c_{\mathrm{b},j}^\dagger c_{\mathrm{a},j} \right] \label{EqnApp:H_phi_r_j}
\end{align}
and set their time-dependence to $r_j(t) = r(j, t)$ by introducing the function
\begin{align}
r(x,t) = r_0 \epsilon(u(x, t)). \label{EqnApp:def_r_x_t}
\end{align}
Other than that, we still work at an exact filling fraction of $\nu = 1/3$ and set $\phi = \nu = 1/3$. At $t = 0$, all couplings are set to $r_j  = r_0$, which we put to $r_0 = 0.1$, thereby starting at the beginning of stage three of \eq{Eqn:time_dependence_p1}. We illustrate the ramp function at two different times in Fig.~\ref{Fig:fidelity_data_appendix_a} for $n_\mathrm{r} = 2$ fronts, velocity $v_{n_\mathrm{r} = 2} = 0.1$, smoothness parameter $\alpha = 1/4$, and a system size of $L = 48$. The relation between the time $t_\mathrm{tot}$ to complete the protocol in the sense of arriving at $r_j(t_\mathrm{tot} = 0)\; \forall j$ and the ramp velocity $v_\mathrm{r}$ is
\begin{align}
t_\mathrm{tot} = \left[d + \frac{L-1}{2 n_\mathrm{r}} \right] \frac{1}{v_{n_\mathrm{r}}}\; \Leftrightarrow \; v_{n_\mathrm{r}} = \left[d + \frac{L - 1}{2 n_\mathrm{r}} \right] \frac{1}{t_\mathrm{tot}}, \label{EqnApp:t_tot_v_r}
\end{align}
where the constant offset is due to the finite width $d = \pi / \alpha$ of the critical front. 

We conduct \gls{mpste} simulations to compare this approach against the strategy of a global ramp with a power law $p(L)$ adjusted to system size that we present in the main text. In general, we find that a smoothness $\alpha \lesssim 1/4$ is sufficient for adiabatic preparation. However, contrary to the claims of \cite{crit_state_prep_3}, the ramp speed $v_\mathrm{r}$ has to be adapted to system size to keep the time evolution adiabatic when starting only a single front, corresponding to $n_\mathrm{r} = 1$. While we find that this can be countered to some degree by starting multiple fronts for larger systems, the strategies presented in the main text outperform this procedure in either case for the investigated system sizes, with a clear trend of the advantage to increase with system size. Concretely, we present data for system sizes $L = 24$, $L = 48$, and $L = 72$ in Fig.~\ref{Fig:fidelity_data_appendix_b} to Fig.~\ref{Fig:fidelity_data_appendix_d} for the case of $n_\mathrm{r} = 1$ (similar to the study in \cite{crit_state_prep_3}) and an increasing value of $n_\mathrm{r}$.



\begin{figure}[htp!]	 
{
        \vbox to 0pt {
                \raggedright
                \textcolor{white}{
                    \subfloatlabel[1][Fig:fidelity_data_appendix_a]
                    \subfloatlabel[2][Fig:fidelity_data_appendix_b]
                    \subfloatlabel[3][Fig:fidelity_data_appendix_c]
                    \subfloatlabel[4][Fig:fidelity_data_appendix_d]
                }
            }
    }
    \includegraphics[width=\textwidth]{plots/Fidelities_appendix/Fidelities_appendix.png}
\caption{(a) Illustration of the ramp function $r(x,t)$ as per \eq{EqnApp:def_r_x_t} for a system size of $L = 48$, with $r_0 = 0.1$, $v_{n_\mathrm{r} = 2} = 0.1$, $\alpha  =1/4$, and $n_\mathrm{r} = 2$ at times $t= 50$ and $t = 100$. (b) \gls{gs} fidelity $F$ after adiabatic evolution with the inhomogeneous ramp for $n_\mathrm{r} = 1$, $\alpha = 1/4$ as a function of preparation time $t_\mathrm{tot}$ for a system of size $L = 24$ at filling  $\nu = 1/3$ with the time axis rescaled proportional to system size by $12 / (5 L)$. Other Parameters are $t = 1$, $W = -1.8$, and $\phi = 1/3$. Additionally, the reciprocal $[v_{n_\mathrm{r} = 1}]^{-1}$ of the corresponding ramp velocity as per \eq{EqnApp:t_tot_v_r} is indicated on the upper axis. For comparison, the fidelity curves of the global ramp protocol discussed in the main text (cf. Fig.~(\ref{Fig:fidelity_data})) are also shown in black. (c) Similar data for a system size of $L = 48$ at filling $\nu = 1/3$. We present data for a single front $n_\mathrm{r} = 1$ (red line) and two fronts $n_\mathrm{r} = 2$ (blue line) in comparison to the data from the global ramp. The reciprocal velocity associated to the $n_\mathrm{r} = 1$ case is again indicated on the upper axis. (d) Similar to (c), but for $L = 72$ at filling $\nu = 1/3$ and with the blue line representing the fidelity for $n_\mathrm{r} = 3$ critical fronts.}
\end{figure}

\end{document}
