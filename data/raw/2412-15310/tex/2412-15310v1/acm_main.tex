%\documentclass[sigconf,review,anonymous]{acmart}
\documentclass[11pt]{article}

% Remove the "review" option to generate the final version.
% \usepackage[review]{ACL2023}
\usepackage[]{ACL2023}
\usepackage{times}
\usepackage{latexsym}

% For proper rendering and hyphenation of words containing Latin characters (including in bib files)
\usepackage[T1]{fontenc}
% For Vietnamese characters
% \usepackage[T5]{fontenc}
% See https://www.latex-project.org/help/documentation/encguide.pdf for other character sets

% This assumes your files are encoded as UTF8
\usepackage[utf8]{inputenc}

% This is not strictly necessary, and may be commented out.
% However, it will improve the layout of the manuscript,
% and will typically save some space.
\usepackage{microtype}
\usepackage{pdflscape}
\usepackage{rotating}
\usepackage{amssymb}
% This is also not strictly necessary, and may be commented out.
% However, it will improve the aesthetics of text in
% the typewriter font.
\usepackage{inconsolata}



\AtBeginDocument{%
  \providecommand\BibTeX{{%
    Bib\TeX}}}


% Added
\usepackage{amsmath,amsfonts}
\usepackage{graphicx}
\usepackage{textcomp}
\usepackage{xcolor}
\usepackage{caption}
\usepackage{lipsum}
\usepackage{makecell}
% \usepackage{subfigure}
\usepackage{xspace}
\usepackage{float}
\usepackage{tipa}

\usepackage{arydshln}
\usepackage{enumitem}
\usepackage{multirow}
\usepackage{tcolorbox}
\usepackage{colortbl}
\usepackage{adjustbox} % for \sbox
\usepackage[T1]{fontenc}

\usepackage{threeparttable}
% \usepackage{easyReview}
\usepackage{listings}
\usepackage{xcolor}
\usepackage{booktabs}
\usepackage{subcaption}
\usepackage{algorithm}
\usepackage{algorithmic}
\let\comment\undefined
\usepackage{changes}
\usepackage{tabularx}
\usepackage{amssymb} % For \checkmarkuse
\usepackage{bbding}
\usepackage{enumitem}
% Define \xmark if not already defined
\newcommand{\etal}{{\em et al.}\xspace}
\newcommand{\ie}{{\em i.e.},\xspace}
\newcommand{\eg}{{\em e.g.},\xspace}
\newcommand{\taskname}{MRWeb\xspace}
% \newcommand{\webname}{MRWeb\xspace}
\definecolor{lightblue}{rgb}{0.2, 0.2, 0.8}
\definecolor{lightgreen}{rgb}{0.2, 0.8, 0.2}

\newcommand{\wcz}[1]{\textcolor{green}{{#1}}}

\newsavebox{\arrangebox}
\newlength{\arrangeht}

\lstset{
    backgroundcolor=\color{gray!10},
    basicstyle=\ttfamily\footnotesize,
    breaklines=true,
    frame=single,
}

%%
%% end of the preamble, start of the body of the document source.
\begin{document}

%%
%% The "title" command has an optional parameter,
%% allowing the author to define a "short title" to be used in page headers.
\title{MRWeb: An Exploration of Generating Multi-Page Resource-Aware Web Code from UI Designs}



\author{
    {\bf Yuxuan Wan}$^{1}$,
    {\bf Yi Dong}$^{1}$,
    {\bf Jingyu Xiao}$^{1}$, 
    {\bf Yintong Huo}$^{1}$, \\
    {\bf Wenxuan Wang}$^{1}$\thanks{\ \ Wenxuan Wang is the corresponding author.},
    {\bf Michael R. Lyu}$^{1}$ \\
    $^1$The Chinese University of Hong Kong, Hong Kong, China \\
    \texttt{\{yxwan9, ythuo, wxwang, lyu\}@cse.cuhk.edu.hk},\\ \texttt{dy13367795706@gmail.com, whalexiao99@gmail.com}
}



\maketitle

\begin{abstract}
Multi-page websites dominate modern web development. However, existing design-to-code methods rely on simplified assumptions, limiting to single-page, self-contained webpages without external resource connection. To address this gap, we introduce the Multi-Page Resource-Aware Webpage (MRWeb) generation task, which transforms UI designs into multi-page, functional web UIs with internal/external navigation, image loading, and backend routing. We propose a novel resource list data structure to track resources, links, and design components. Our study applies existing methods to the MRWeb problem using a newly curated dataset of 500 websites (300 synthetic, 200 real-world).
Specifically, we identify the best metric to evaluate the similarity of the web UI, assess the impact of the resource list on MRWeb generation, analyze MLLM limitations, and evaluate the effectiveness of the MRWeb tool in real-world workflows. The results show that resource lists boost navigation functionality from 0\% to 66\%-80\% while facilitating visual similarity. Our proposed metrics and evaluation framework provide new insights into MLLM performance on MRWeb tasks. We release the MRWeb tool, dataset, and evaluation framework to promote further research\footnote{\url{https://github.com/WebPAI/MRWeb}}.

\end{abstract}


\section{Introduction}

\begin{figure}
    \centering
    \includegraphics[width=\linewidth]{Sections/figs/pipeline.pdf}
    \vspace{-10pt}
    \caption{Comparison between self-contained webpage and multi-page resource-aware webpage (MRWeb). MRWeb supports multi-page navigation, real-image loading and backend routing.}
    \label{fig:pipeline}
    \vspace{-15pt}
\end{figure}

%%% Importance of website
Websites are important in today's digital landscape, serving as essential platforms for diverse applications in our daily lives~\cite{website_statistics_2024}.
%%%% the prevalence and advantages of multi-page

They can be categorized into two types: single-page and multi-page. Single-page websites update content dynamically on a single page without reloading, while multi-page websites consist of multiple interconnected pages. In reality, multi-page websites are dominant due to their scalability and structured navigation functionality~\cite{waytogrow2024spa, web2024onepage}. To examine this, we conducted a preliminary study by sampling the top 300 most visited websites ranked by the Tranco list~\footnote{https://tranco-list.eu}. For each website, we analyzed its structure by recursively visiting all internal links to identify distinct internal pages, as well as counting the number of external links, internal links, and images present on each webpage. The results showed that 271 (90.3\%) of the sampled websites are multi-page, highlighting their widespread usage and the complex structures they often exhibit. Appendix~\ref{sec:motivating-study} describes the study in detail.

Practical solutions for generating multi-page web user interfaces (UIs) from designs remain underexplored. Existing work focuses on the simpler task of design-to-code, which generates self-contained web pages (Figure~\ref{fig:pipeline}). This leaves a gap in addressing the complexities of multi-page, resource-aware web UI development, where \textbf{(1) webpages link to internal pages with unlimited navigation paths, (2) webpages link to external resources like websites and images, and (3) webpages route to backends for data exchange}. We term this task \underline{\taskname} to highlight the shift from design-to-code to \underline{M}ulti-Page \underline{R}esource-Aware \underline{Web}page generation. Figure~\ref{fig:pipeline} contrasts self-contained webpages with MRWebs, which \textbf{enable code-free development from UI designs to resource-aware, navigable websites, democratizing web development}.


% \begin{figure*}[ht]
%     \centering
%     \includegraphics[width=0.8\linewidth]{Sections/figs/comparison.pdf}
%     \vspace{-5pt}
%     \caption{Comparison between design-to-code webpage and our multipages resource-aware webpages (MRWebs). Design-to-code webpages contain placeholder images and empty links, whereas MRWebs contain real images and links.}
%     \label{fig:data-compare}
%     \vspace{-10pt}
% \end{figure*}

%%% technical challenge
Unfortunately, constructing such a framework poses several challenges. First, there is no established data structure that integrates visual design elements with internal or external resources and tracks their correspondences. Unlike the design-to-code task that generates self-contained code from design images, \taskname should incorporate resources, navigation paths, and their links to visual design elements, necessitating a new data structure. 
Second, there is a lack of high-quality datasets for \taskname since the previousn-to-code task did not contain real images or internal/external links. 
Third, there are no standardized metrics for evaluating the performance of the \taskname task.  This makes it difficult to measure the accuracy of generated links, images, and their correspondence with visual design elements. As a result, there is insufficient understanding of how effectively MLLMs produce \taskname code from designs.

To bridge this gap, we propose the first evaluation framework for \taskname generation. Specifically, we define a novel data structure, the \textbf{resource list}, which uses a dictionary-like format to store internal/external resources, such as links and images, and their correspondence with visual designs (e.g., screenshots). Next, we collect the first \taskname dataset, comprising 500 websites (300 synthetic, 200 real-world), along with their associated resource lists, screenshots, and ground truth code. The generation framework accepts the resource lists and screenshots as inputs and calls MLLMs to generate functional \taskname code directly. To assess the generated code, we introduce a suite of metrics designed to evaluate and analyze both the visual and functional performance of MLLMs on the \taskname task. Additionally, we implement this framework as a user-friendly tool for \taskname generation, releasing all code and data to encourage future research.

\begin{table*}[ht]
    \centering
    \caption{Comparison and statistics of benchmarks and datasets. All statistics are in the ``average $\pm$ standard deviation'' format.}
    \label{tab:comparison}
    \vspace{-5pt}
    \begin{adjustbox}{width=\textwidth}
    \begin{tabular}{|l|c|c|c|c|c|c|}
        \toprule
        & \textbf{WebSight} & \textbf{VISION2UI} & \textbf{Design2Code} & \textbf{DWCG} & \textbf{Interaction2Code} & \textbf{MRWeb } \\
        & \cite{laurençon2024unlocking} & \cite{Gui2024VISION2UIAR} & \cite{Si2024Design2CodeHF} & \cite{Yun2024Web2CodeAL} & \cite{xiao2024interaction2code} & (Ours) \\
        \midrule
        \textbf{Scope} & Self-contained & Self-contained & Self-contained & Self-contained & Self-contained & \taskname \\
        \textbf{Int. navigation} & \XSolidBrush & \XSolidBrush & \XSolidBrush & \XSolidBrush& \XSolidBrush & \Checkmark \\
        \textbf{Ext. navigation } & \XSolidBrush & \XSolidBrush & \XSolidBrush & \XSolidBrush & \XSolidBrush & \Checkmark \\
        \textbf{Backend routing} & \XSolidBrush & \XSolidBrush & \XSolidBrush & \XSolidBrush & \XSolidBrush & \Checkmark \\
        \textbf{Real img insertion} & \XSolidBrush & \XSolidBrush & \XSolidBrush & \XSolidBrush & \XSolidBrush & \Checkmark \\
        \midrule
        \textbf{Source} & Synthetic & Real-World & Real-World & Synthetic & Real-World & Synthetic / Real-World \\
        \textbf{Size} & 823K & 20k & 484 & 60K & 97 & 500 (300 / 200) \\
        \textbf{Avg. Len (tokens)} & $647\pm216$ & $8460\pm7120$ & $31216\pm23902$ & $471.8\pm162.3$ & $141084\pm160438$ & $692\pm227$ / $113724\pm139761$ \\
        \textbf{Avg. Tags} & $19\pm8$ & $175\pm94$ & $158\pm100$ & $28.1\pm10.6$ & $1291\pm1574$ & $18\pm7$ / $543\pm768$ \\
        \textbf{Avg. DOM Depth} & $5\pm1$ & $15\pm5$ & $13\pm5$ & $5.3\pm1.0$ & $18\pm6$ & $5\pm1$ / $15\pm6$ \\
        \textbf{Avg. Unique Tags} & $10\pm3$ & $21\pm5$ & $22\pm6$ & $13.6\pm2.7$ & $31\pm9$ & $10\pm3$ / $22\pm8$ \\
        \textbf{Avg. Resource List Len.} & - & - & - & - & - & $3\pm2$ / $92\pm130$ \\
        \bottomrule
    \end{tabular}
    \end{adjustbox}
\vspace{-15pt}
\end{table*}

% To demonstrate the effectiveness of the framework, we conduct extensive experiments and human evaluations on three SOTA MLLMs, formulating the following research questions:
% \paragraph{RQ1: What is the most suitable metric for Web UI similarity? (Section~\ref{sec:rq1})} 
% A critical challenge in the \taskname task is accurately evaluating web UI similarity. To address this, we initiated an image quality assessment (IQA) in the Web UI domain, where we compare various image similarity methods and discuss their alignment with human preferences.

% \textbf{Results:} Pixel-based metrics outperform semantic and structural metrics. Specifically, Mean Absolute Error (MAE) achieves the best overall alignment with human perception, with a rank-order correlation coefficient of 0.542, followed by Normalized Earth Mover’s Distance (NEMD), which performs well in both low and medium similarity cases. In high-similarity cases, semantic and structural metrics like SSIM, CLIP, and LPSIS perform better but exhibit weaker alignment in low-similarity scenarios. 


% \paragraph{RQ2: Is the resource list useful? (Section~\ref{sec:rq2})} 
% Central to our framework is a novel data structure, the resource list, which encodes links, external resources, and their correspondences with design elements. To assess its impact, we evaluate the ability of MLLMs to generate multi-page web UI code using various prompting strategies. 

% \textbf{Results:} The resource list significantly enhances both visual and functional similarity. With resource lists, functional similarity scores improve from 0\% to 66\%-80\%, as MLLMs become capable of generating pages with valid internal/external links and backend routing. Moreover, visual similarity also benefits from resource lists, as they provide MLLMs with explicit cues about navigation paths, images, and resource links.

% \paragraph{RQ3: What are the challenges for MLLMs in the \taskname task? (Section~\ref{sec:rq3})}
% To understand the limitations of current MLLMs in generating multi-page resource-aware web UIs, we explore various production challenges, including layout generation, design complexity, and external resource integration. 

% \textbf{Results:} One of the major challenges for MLLMs is the visual grounding problem, where MLLMs struggle to accurately position and size elements within a webpage. Quantitatively, MLLMs exhibit a 15\%-25\% positional shift and a 30\%-40\% size difference in the placement of elements compared to the original design. Additionally, as design complexity increases, the performance of MLLMs deteriorates, particularly in maintaining alignment between visual components and functional elements like links and images.

% \paragraph{RQ4: Can the \taskname tool help real-world development workflows? (Section~\ref{sec:rq4})} Finally, we assess the real-world applicability of our \taskname tool in practical development scenarios. We create a user-friendly UI-to-\taskname tool that allows developers to convert visual design assets into functional multi-page web UI code. This tool supports the automatic generation of internal/external navigation paths, valid link structures, and backend routing, streamlining the integration of frontend design with backend functionality.

% \textbf{Results:} Our real-world testing confirms that the tool bridges the gap between design assets and production-ready web UI code, providing direct support for development workflows.

To sum up, the contributions of this study are:
\begin{itemize}[leftmargin=*]
    \item Define the \taskname problem, introduce the first \taskname benchmark with an innovative resource list data structure, and construct the first \taskname dataset consisting of \textit{500} websites.
    \item Propose a suite of metrics, conduct the first comprehensive image quality assessment (IQA) in the web UI domain to determine the best evaluation metric for web UI code generation, and collect an annotated IQA dataset for future studies.
    \item Conduct a comprehensive study to evaluate the performance of SOTA MLLMs on \taskname generation and highlighting some challenges faced by MLLMs.
    \item Develop a user-friendly tool for \taskname tasks and release all code and data to foster further research and development in this emerging area.
\end{itemize}





\section{Related Work}
Existing UI code generation methods operate on three unrealistic assumptions: (1) websites have a single or limited number of pages, (2) webpages lack external links, and (3) webpages do not interact with backends. Our study in Appendix~\ref{sec:motivating-study} highlights the misalignment of these assumptions with real-world web UI needs. Table~\ref{tab:comparison} provides a comparison of our benchmark with existing works, contextualizing our contribution.  A detailed discussion of related works is in Appendix~\ref{appendix:related-work}.


\section{Task Formulation} 

\paragraph{Resource List} The \textit{resource list} serves as a structured representation of a webpage's navigational and visual elements, such as hyperlinks, images, and backend routing. Each entry in the resource list includes attributes like position, type, and URL, for instance, \texttt{\{position: $bonding\_box$, type: image, link:/dog.png\}}. This structure is crucial for enabling MRWeb generation to replicate navigational features and image sources accurately. Without the resource list, MLLMs would generate static replicas lacking interactivity and navigation. In our experimental setup, resource lists are extracted automatically using Python Selenium\footnote{\url{https://selenium-python.readthedocs.io/}}. For real-world applications, we developed an intuitive user interface that allows users to highlight actionable elements by drawing bounding boxes and inputting the corresponding resource information, as illustrated in Appendix~\ref{appendix:demo}.

\paragraph{Task Definition} Let the ground-truth webpage's HTML+CSS code be $C_0$, screenshot be $I_0$, and resource list be $R_0$, the \textit{MRWeb generation} task uses an MLLM $M$ to produce HTML+CSS code $C_g = M(I_0, R_0)$ that approximates $C_0$. The quality of $C_g$ is assessed by comparing the generated resource list $R_g$ with $R_0$ and the screenshot $I_g$ rendered from $C_g$ with $I_0$, ensuring both functional and visual alignment.


\section{Dataset Collection}
We collect two types of data: synthetic and real-world. Synthetic data enables controlled, diverse examples, including rare edge cases, but lacks the variety of real-world content. Real-world data captures authentic webpage diversity, supporting model robustness across HTML structures and styles. Combining both data types provides a comprehensive benchmark.

This section outlines our collection of code-screenshot pairs for synthetic and real-world data, the extraction of resource lists, and the statistics of the sampled data.

\subsection{Synthetic Data Collection} To create the synthetic UI-to-MRWeb dataset, we adopt and modify the WebSight dataset~\cite{laurençon2024unlocking}. The WebSight dataset contains 2 million HTML samples and their corresponding screenshots, covering a broad spectrum of website concepts. However, it cannot be used directly for an MRWeb dataset because 1) its websites lack valid internal or external links, making navigation to other pages impossible, and 2) images on the sites are randomly loaded via the Unsplash\footnote{\url{https://source.unsplash.com/}} API. This random loading causes visual inconsistencies, as identical code can result in different visuals, complicating benchmarking. To address these issues, we enhance the WebSight dataset through link insertion and image replacement.

\paragraph{Link insertion.} Using all website links from the C4~\cite{Raffel2019ExploringTL} validation set, we create a URL list. For each HTML document, we parse its content and iterate over all hyperlink tags, assigning a randomly chosen URL from our list to each hyperlink attribute. This modification ensures that every website includes valid external links to other sites.

\paragraph{Image replacement.} To ensure consistency and diversity in visual representation, we replace random images in the WebSight dataset with static, unique images for each webpage. Using the Unsplash API, we fetch images with specific keywords, dimensions, and properties to guarantee that the pictures remain consistent yet unique. 


\subsection{Real-world Data Collection} We collect real-world data by capturing and simplifying HTML content from live websites. We first collect 500 URLs of real-world websites from the C4~\cite{Raffel2019ExploringTL} validation set as our data source. However, HTML files on the web often contain non-visible noise—such as comments, scripts, and hidden content—that makes them excessively lengthy and can exceed the token limits of most models. To create the real-world UI-to-MRWeb dataset, we develop a pipeline that collects and processes HTML code and screenshots from live websites. This pipeline ensures that each webpage captures authentic and static content while maintaining a simplified HTML structure compatible with our UI-to-MRWeb benchmark. The primary steps in this pipeline include saving HTML files, filtering HTML, simplifying HTML, and capturing screenshots, as outlined below:
\begin{enumerate}[leftmargin=*]
    \item Saving HTML files: The HTML+CSS content from each URL is saved into a single HTML file, ensuring all components are intact.
    \item Filtering HTML: We discard websites that are blank or erroneous (e.g., page not found).
    \item HTML Simplification: We simplify the HTML by removing all non-visible elements, comments, and non-functional Javascripts.
    \item Final Screenshot: A final screenshot is taken after simplification, completing the real-world data pipeline.
\end{enumerate}

\subsection{Resource List Extraction} 
Resource lists capture navigational and visual elements such as links, images, and backgrounds, structured to preserve the functionality and layout of each webpage. For each webpage:
\begin{itemize}[leftmargin=*]
    \item Links (\texttt{<a>} tags): We extract each hyperlink’s position, type, and target URL.
    \item Images and Background Images: For images, including both image tags and CSS background images, we record their position and source URL.
\end{itemize}
The resource list is automatically constructed by iterating through the webpage’s elements using Python Selenium, collecting attributes for each, and verifying their visibility and functionality. 

\subsection{Automation \& Dataset Statistics}
We emphasize that the data collection pipeline is fully automated, enabling the on-demand generation of large-scale MRWeb training data. In principle, the synthetic dataset could match the full size of the WebSight dataset (two million), and the real-world data could encompass any website accessible on the internet. To support future research, however, we sampled 300 synthetic and 200 real-world instances. The statistics, quantitative metrics of the sampled dataset, and comparison with other datasets are provided in Table~\ref{tab:comparison}. To get a sense of the range of domains covered in our benchmark, we manually categorize what type of webpages they are based on their functions.  We present the pie chart of the most frequent domains in Figure~\ref{fig:domain-distribution}. The most prominent genres are companies' or organizations' websites and blogs. 


\begin{figure}
    \centering
    \includegraphics[width=0.8\linewidth]{Sections/figs/categories_pie_chart.pdf}
    \caption{Topic distribution of real-world web data in \taskname dataset.}
    \label{fig:domain-distribution}
    \vspace{-15pt}
\end{figure}



\section{Study Setup}
\subsection{Evaluated Models}
\label{sec:setup}
We employ three state-of-the-art (SOTA) MLLMs: Gemini 1.5 \cite{google_gemini_api}, GPT-4o \cite{openai_gpt4o} and Claude-3.5 \cite{anthropic_claude} to evaluate their performance on \taskname. the specific model numbers are 20240806 for GPT-4o, 20240620 for Claude-3.5-Sonnet, and Gemini-1.5-Pro accessed during November 2024. For MLLM model configurations, we set the temperature to 0, the random seed to 42, and the $max\_tokens$ parameter to 4096 for each model. Other parameters are maintained at their default settings as specified in the corresponding API documentation~\cite{google_gemini_api_docs, openai_vision_guide, anthropic_vision_docs}.

\subsection{Prompting Strategies}
\label{subsubsec:prompt} 
We use four types of prompting methods: self-contained, zero-shot, CoT, and self-refine. Self-contained prompting is adapted from Si et al.~\cite{Si2024Design2CodeHF} to let the model directly generate code from screenshots without resource lists. This method serves as a baseline for other methods that adopt input resource lists. Zero-shot prompting directly lets the model generate HTML code from screenshots and resource lists.  Chain-of-Thought (CoT) prompting~\cite{Wei2022ChainOT} generates a chain of thought for each question and then generates the corresponding code. For CoT, we use the "let’s think step by step" instruction from Chae et al. ~\cite{Chae2024LanguageMA}. Self-refine prompting~\cite{Chen2023TeachingLL} let the model refine its own generated code via multi-turn conversation. We adopt the self-refine prompting and direct promoting method from Si et al.~\cite{Si2024Design2CodeHF}. We list the exact prompts used in our experiments in Appendix~\ref{appendix:prompts}.

\subsection{Metrics}
\label{sec:metrics}
\subsubsection{High-level Metrics}
For high-level performance, we evaluate visual similarity and functional similarity. For visual similarity, we explore three levels of image similarity metrics commonly applied in design-to-code or other computer vision (CV) tasks~\cite{Wang2004ImageQA}: pixel, structural, and semantic. The detailed background and calculation of these metrics are in Appendix~\ref{appendix:similarity}.

\noindent\textbf{Visual: Pixel-level metrics} 
\begin{itemize}[leftmargin=*]
    \item Mean Absolute Error (MAE)~\cite{Nguyen2015ReverseEM, Moran2018MachineLP}: Measures the average absolute difference in pixel intensities.
    \item Peak Signal-to-Noise Ratio (PSNR)~\cite{Lim2017EnhancedDR, Wang2019EDVRVR}: Based on Mean Squared Error (MSE), with higher values indicating greater similarity.
    \item Normalized Earth Mover's Distance (NEMD)~\cite{Arjovsky2017WassersteinG, Rubner2000TheEM}: Captures spatial differences between images, normalized to be size-independent. Higher NEMD values indicate greater similarity.
\end{itemize}

\noindent\textbf{Visual: Structure-level metrics} 
\begin{itemize}[leftmargin=*]
    \item Structural Similarity Index Measure (SSIM)~\cite{Zhou2024BridgingDA, Wang2004ImageQA}: Measures luminance, contrast, and structural changes.
\end{itemize}

\noindent\textbf{Visual: Semantic-level metrics} 
\begin{itemize}[leftmargin=*]
    \item CLIP Score~\cite{Radford2021LearningTV, Si2024Design2CodeHF}: Aligns image embeddings with language representations to capture high-level conceptual similarity.
    \item Learned Perceptual Image Patch Similarity (LPIPS)~\cite{Zhang2018TheUE, Simonyan2014VeryDC}: Assesses perceptual similarity using deep features from VGG~\cite{Simonyan2014VeryDC}.
\end{itemize}

\noindent\textbf{Functional Metric:}
\begin{itemize}[leftmargin=*]
    \item Resource Existence Ratio (RER)$_\uparrow$: The proportion of resources in the reference resource list that exist (i.e., are successfully matched to corresponding resources) in the generated list is calculated as $\text{RER} = \frac{\text{\# Matched Resources in G}}{\text{\# Total Resources in R}}$. Matching is determined based on relevant attributes of resources, such as whether navigational elements direct to the same link or whether images share the same source. 
\end{itemize}


All image pairs, except for those used with CLIP Score, are padded with random noise to ensure consistent image sizes for comparison.

\subsubsection{Fine-Grained Metrics}
Beyond assessing visual and functional similarity, we employ a suite of fine-grained metrics to evaluate the specific capabilities of MLLMs, including visual grounding, color recognition, and text extraction. For each pair of matched resources in RER, we calculate:
\begin{itemize}[leftmargin=*]
    \item Position offset$_\downarrow$: Measures the position shift between the generated and the original element with respect to the size of the original web page. For each pair of matched resources $(r_p, g_q)$, the positional alignment is evaluated by comparing the normalized offset of their corresponding web elements' center points: $\text{Position Offset} = \max\left(\frac{|x_p - x_q|}{W}, \frac{|y_p - y_q|}{H}\right)$. $(x_p, y_p)$ and $(x_q, y_q)$ are the center positions of the bounding boxes enclosing the elements; $W$ and $H$ represent the width and height of the original webpage.  

    \item Area Difference$_\downarrow$: Measures the differences in size between corresponding actionable elements with respect to the original area of the element: $\text{Area Difference} = \frac{|A_p - A_q|}{A_p}$, where $A_p$ and $A_q$ are the areas occupied by the reference and generated actionable elements. 
    
    \item Color Difference$_\downarrow$: We use the CIEDE2000 color difference formula~\cite{Luo2001TheDO} to assess the perceptual difference between the colors of element $r_p$ and $g_q$.

    \item Text Difference$_\downarrow$: For resources that involve text, such as buttons, their text similarity $\text{\textit{Text Sim}}(r_p, g_q)$ is calculated by normalizing the number of matching characters by the total length of the text. We calculate the text difference by 1 - $\text{\textit{Text Sim}}(r_p, g_q)$.
\end{itemize}
\begin{table*}[ht]
\centering
\footnotesize
\caption{Overall correlation with human scores for web UI similarity metrics. CC: Correlation coefficient (absolute value); OR: Outlier ratio; MAE: Mean absolute error; RMS: Root mean square error; SROCC: Spearman's rank-order correlation coefficient (absolute value). We mark the \textbf{best results} with bold font and the \underline{second best} with underline. Metrics are sorted by their SROCC.}
\label{tab:IQA}
\vspace{-5pt}
\begin{tabular}{@{}lcccccccccc@{}}
\toprule
 & \multicolumn{4}{c}{Variance-Weighted Regression} & \multicolumn{4}{c}{Non-Linear Regression} & \multicolumn{1}{c}{Direct} \\ 
\cmidrule(lr){2-5} \cmidrule(lr){6-9} \cmidrule(lr){10-10}
 & CC $\uparrow$ & MAE $\downarrow$ & RMS $\downarrow$ & OR $\downarrow$ & CC $\uparrow$ & MAE $\downarrow$ & RMS $\downarrow$ & OR $\downarrow$ & SROCC $\uparrow$ \\ 
\midrule
MAE   & \textbf{0.547} & \underline{4.10} & \textbf{1.95} & 0.049 & \underline{0.515} & \underline{0.646} & \underline{0.765} & 0.013 & \textbf{0.542} \\
NEMD  & \underline{0.469} & \textbf{3.98} & \textbf{1.95} & 0.052 & \textbf{0.532} & \textbf{0.628} & \textbf{0.752} & 0.023 & \underline{0.508} \\
PSNR  & 0.323 & 5.69 & 2.46 & \textbf{0.000} & 0.434 & 0.679 & 0.800 & 0.016 & 0.451 \\

CLIP  & 0.314 & 5.37 & \underline{2.41} & 0.013 & 0.426 & 0.681 & 0.800 & \underline{0.010} & 0.340 \\

SSIM  & 0.305 & 5.47 & 2.42 & \underline{0.010} & 0.381 & 0.699 & 0.817 & \textbf{0.000} & 0.218 \\

LPIPS & 0.221 & 5.70 & 2.49 & \underline{0.010} & 0.290 & 0.726 & 0.847 & 0.013 & 0.168 \\

\midrule
Human  &       &      &      &       &       &       &       &       & 0.640 \\
\bottomrule
\end{tabular}
\vspace{-5pt}
\end{table*}


\begin{table*}[h!]
    \centering
    \footnotesize
    \caption{Visual metrics of models across methods. SC: Self-contained; ZS: Zero-shot; CoT: Chain-of-thought; SR: Self-refine. The best result per model is highlighted in bold.}
    \label{tab:visual-metrics}
    \vspace{-5pt}
    \begin{tabular}{@{}lcccccccccccc@{}}
        \toprule
        Model & \multicolumn{4}{c}{Gemini-Pro} & \multicolumn{4}{c}{GPT-4o} & \multicolumn{4}{c}{Claude-3.5} \\
        \cmidrule(lr){2-5} \cmidrule(lr){6-9} \cmidrule(lr){10-13}
        Method & SC  & ZS  & CoT & SR & SC  & ZS  & CoT & SR & SC  & ZS  & CoT & SR \\
        \midrule
        MAE\textsubscript{\(\downarrow\)} & 68.3 & 66.7 & 66.5 & \textbf{63.1} & 69.3 & 67.7 & 67.9 & \textbf{65.1} & 71.6 & 69.1 & 69.3 & \textbf{69.4} \\
        NEMD\textsubscript{\(\uparrow\)} & 0.732 & 0.744 & 0.743 & \textbf{0.757} & 0.730 & 0.734 & 0.735 & \textbf{0.750} & 0.717 & \textbf{0.732} & 0.729 & 0.725 \\
        CLIP\textsubscript{\(\uparrow\)} & 0.771 & 0.785 & 0.789 & \textbf{0.797} & 0.753 & 0.766 & 0.766 & \textbf{0.774} & 0.767 & 0.788 & 0.784 & \textbf{0.792} \\
        \bottomrule
    \end{tabular}%
\vspace{-10pt}
\end{table*}

\begin{table}[h!]
    \centering
    \footnotesize
    \caption{RER of models across methods. SC: Self-contained; ZS: Zero-shot; CoT: Chain-of-thought; SR: Self-refine. The best result per model is highlighted in bold.}
    \label{tab:function-metrics}
    \vspace{-5pt}
    \begin{tabular}{@{}lcccc@{}}
        \toprule
         Act. Exist. Ratio$_\uparrow$ & SC    & ZS     & CoT    & SR \\
        \midrule
        Gemini-Pro   & 0.008 & 0.822 & 0.819 & \textbf{0.832} \\
        GPT-4o       & 0.006 & 0.779 & 0.774 & \textbf{0.803} \\
        Claude-3.5   & 0.007 & 0.640 & 0.615 & \textbf{0.668} \\
        \bottomrule
    \end{tabular}%
    \vspace{-10pt}
\end{table}

\begin{table*}[ht]
    \centering
    \footnotesize
    \caption{Comparison of metrics across different models and prompting strategies. ZS: Zero-shot; CoT: Chain-of-thought; SR: Self-refine. The best result per dimension is highlighted in bold.}
    \vspace{-5pt}
    \label{tab:finegrain}
    \begin{tabular}{@{}lccccccccc|c@{}}
        \toprule
        & \multicolumn{3}{c}{Gemini-Pro} & \multicolumn{3}{c}{GPT-4o} & \multicolumn{3}{c}{Claude-3.5} & \multirow{2}{*}{\textbf{Avg.}} \\
        \cmidrule(lr){2-4} \cmidrule(lr){5-7} \cmidrule(lr){8-10}
        \textbf{Difference}$_\downarrow$ & \textbf{ZS} & \textbf{CoT} & \textbf{SR} & \textbf{ZS} & \textbf{CoT} & \textbf{SR} & \textbf{ZS} & \textbf{CoT} & \textbf{SR} & \\
        \midrule
        \textbf{Pos. Shift}   & 0.239 & 0.260 & 0.236 & 0.217 & 0.267 & 0.209 & \textbf{0.155} & 0.161 & 0.193 & 0.215 \\
        \textbf{Area Diff}    & 0.376 & 0.389 & 0.403 & 0.360 & 0.375 & 0.360 & \textbf{0.279} & 0.326 & 0.334 & 0.356 \\
        \textbf{Color Diff}   & 0.128 & 0.144 & 0.110 & 0.043 & 0.066 & 0.055 & 0.057 & \textbf{0.024} & 0.038 & 0.074 \\
        \textbf{Text Diff}    & 0.014 & 0.011 & 0.009 & 0.013 & \textbf{0.004} & 0.005 & 0.020 & 0.011 & 0.005 & 0.010 \\
        \bottomrule
    \end{tabular}%
    \vspace{-10pt}
\end{table*}


\section{Experiment Results}
\subsection{The Best Web UI Similarity Metric}
\label{sec:rq1}
A critical challenge in the \taskname task is accurately evaluating web UI similarity. To verify the effectiveness of the evaluation metrics and determine the most suitable one, we initiated a human evaluation in the web UI domain, where we compared various image similarity methods (Section \ref{sec:metrics}) and discussed their alignment with human preferences. We sample 600 pairs of original and generated screenshots and recruit 14 college students with varying levels of familiarity with web applications to rate their perceived similarity on a Likert scale~\cite{joshi2015likert} within five categories: ``Highly Dissimilar'', ``Dissimilar'', ``Moderately Similar'', ``Similar'', and ``Highly Similar''. This setup follows standard image quality assessment (IQA) procedure~\cite{Wang2004ImageQA, VQEG2000}.

We analyzed the alignment between human judgments and objective similarity scores following established evaluation protocols~\cite{VQEG2000, Wang2004ImageQA}:

\begin{itemize}[leftmargin=*]
    \item Spearman Rank-Order Correlation Coefficient (SROCC): Measure the rank correlation between human and objective scores by capturing the consistency of the rankings.
    \item Variance-weighted Regression Analysis: Evaluate how well objective metrics predict human subjective scores using four key metrics: \textit{absolute correlation coefficient (CC)}, \textit{outlier ratio (OR)}, \textit{weighted mean absolute error (MAE)}, and \textit{weighted root mean square error (RMSE)}. This analysis identifies how well objective scores align with human judgments.
    \item Nonlinear Regression Analysis: Use logistic functions to model the relationship between objective and subjective scores when the relationship is monotonic but nonlinear. We compute \textit{CC}, \textit{OR}, \textit{MAE}, and \textit{RMSE} to measure the strength and quality of the alignment.
    \item Human Evaluators' Consistency: we calculated the SROCC for each pair of evaluators and averaged the results. This inter-rater reliability serves as a benchmark to evaluate the performance of objective metrics relative to human consensus.
\end{itemize}

Please refer to Appendix~\ref{appendix:IQA} for a detailed illustration of the metrics and evaluation practices.

\textbf{Pixel-based methods generally perform better. MAE and NEMD perform the best across all approaches} (Table~\ref{tab:IQA}). The result is initially counter-intuitive as most previous works adopt semantic-level and structure-level metrics such as CLIP and SSIM~\cite{Si2024Design2CodeHF, xiao2024interaction2code, zhou2024bridging}. However, this result aligns with the practice in frontend development, where developers use various tools~\cite{pixelparallel, perfectpixel} to overlay the design image and the result web page to compare their pixel-level differences to refine their UI code~\cite{pixelperfectmedium, pixelperfectsteps}. Variance-weighted regression results showed that MAE and NEMD achieved the highest correlation coefficients (CC: 0.547, 0.469) and lowest RMS (1.95), demonstrating strong predictive accuracy. In non-linear regression, NEMD excelled with the highest CC (0.532) and lowest MAE (0.628) and RMS (0.752), effectively capturing non-linear relationships. 

\textbf{Inter-rater reliability (SROCC) among human evaluators is 0.640, indicating a relatively high agreement and consistency in subjective assessments}. Among the metrics, MAE (0.542) and NEMD (0.508) demonstrated the best consistency with human rankings. 

\textbf{Despite their lower overall alignment scores, SSIM, CLIP, and LPIPS excelled in minimizing outliers (low OR)}, demonstrating their ability to reduce significant mismatches.

To further understand the characteristics of these metrics, we divide the image pairs into three equal-sized groups according to their human ratings (i.e., low, medium, high) and analyze their correspondence with human scores (details in Appendix~\ref{appendix:sim-analysis}). Analysis reveals pixel-based metrics, particularly MAE and NEMD, excel in low and medium-similarity cases. MAE is the most robust metric, maintaining strong performance across all similarity levels. While semantic and structural metrics, such as SSIM and LPIPS, perform better in high-similarity cases, they have near-zero performance in low and medium-similarity groups. This indicates that \textbf{pixel-level features are more effective at distinguishing dissimilar images, whereas semantic and structural information better captures fine-grained similarities.}





\subsection{Effectiveness of the Resource List}
\label{sec:rq2}
Central to our framework is a novel data structure, the resource list. To assess its impact, we employ MLLMs to generate web page code under various prompting strategies (Section~\ref{subsubsec:prompt}), using the two best-performing metrics (MAE and NEMD) and the best-performing high-level metrics (CLIP) for visual similarity and RER to measure the function similarity. We use the result of the self-contained (SC) prompt as a baseline. 

\textbf{Adding resource lists can improve the visual similarity of a generated webpage across different MLLMs and metrics}. Table~\ref{tab:visual-metrics} shows the comparison of visual metrics. We observe that SC consistently results in the lowest visual scores. This is because resource lists enable MLLMs to include the exact images displayed on the webpage, thus enhancing the overall similarity. Without resource lists, MLLMs can only use placeholder images in the generated web code. Some examples of such cases are in Appendix~\ref{appendix:visual-compare}, Fig.~\ref{fig:data-compare}. This highlights the practical value of resource lists in real-world web development compared to self-contained methods.

\textbf{Resource lists enable MLLMs to generate webpages with valid resources, significantly boosting RER from 0\% to 66\%-80\%}. Table~\ref{tab:function-metrics} shows that under SC prompting, MLLMs exhibit near-zero functional similarity due to the lack of guidance for generating valid links. However, some links are inferred through common sense (e.g., ``facebook.com'' for Facebook). Among prompting strategies, self-refine (SR) consistently achieves the highest scores across models, making it the most effective for both visual and functional metrics. Genimi-Pro emerges as the top-performing MLLM in reproducing functionality.

An interesting observation is that CoT prompting slightly decreases performance despite its reasoning capability. We discuss this phenomenon in Appendix~\ref{appendix:CoT}.



\subsection{Limitations of MLLMs in MRWeb}
\label{sec:rq3}
For all matched actionable element pairs, we calculate their fine-grained metrics to have a deeper understanding of MLLMs' strengths and weaknesses.


\textbf{The main challenge in \taskname generation is the visual grounding problem, where MLLMs struggle to replicate the position and size of elements} (Table~\ref{tab:finegrain}). This is reflected by the Positional Shift and Area Difference metrics. MLLMs generate elements with an average 21.5\% positional shift relative to the entire webpage and an average 35.6\% size difference compared to the original elements.



\textbf{Despite these issues, MLLMs perform well in recognizing color and text}, with Color Difference and Text Similarity metrics showing much smaller errors. Among the models, Claude-3.5 demonstrates the best positional accuracy and size recognition, with the lowest Positional Shift (15.5\%) and Area Difference (27.9\%) in the Zero-shot strategy. GPT-4o excels in Text Similarity (0.004), showing strong semantic accuracy. 

\begin{figure}[ht]
    \centering
        % \hspace{5px}
        \begin{subfigure}[b]{0.49\linewidth}
            \includegraphics[width=\linewidth]{Sections/figs/gpt4o_MAE.pdf}
            \vspace{-20pt}
            \caption{GPT-4o MAE$_\downarrow$}
            \label{fig:gpt4o-mae}
        \end{subfigure}
         \begin{subfigure}[b]{0.49\linewidth}
            \includegraphics[width=\linewidth]{Sections/figs/gpt4o_match_ratio.pdf}
            \vspace{-20pt}
            \caption{GPT-4o RER$_\uparrow$}
            \label{fig:gpt4o-aer}
        \end{subfigure}
        \vspace{-20pt}
        \caption{GPT-4o performance decreases with input image size (million pixels) and resource list length.}
        \vspace{-10pt}
\end{figure}


\textbf{The performance of MLLMs degrades with increasing input complexity, suggesting further optimization in handling large-scale and complex inputs.} To evaluate the robustness of MLLMs in the \taskname task, we analyzed their visual (MAE) and functional (RER) performances across varying image sizes and resource list lengths. As shown in Fig. \ref{fig:gpt4o-mae} and \ref{fig:gpt4o-aer}, GPT-4o's performance lowers with increasing input complexity. While it performs well on simpler websites with smaller image sizes and shorter resource lists, its accuracy declines significantly for intricate websites. Similar patterns are observed in other MLLMs, as detailed in Appendix~\ref{appendix:complexity}.




\subsection{\taskname’s Practical Capabilities.}
\label{sec:rq4}
We developed a user-friendly tool within the \taskname framework to convert visual designs into multi-page, realistic web UI code. The tool's interface is shown in Appendix~\ref{appendix:demo}. We conducted a case study using AI-generated design images\footnote{\url{https://openai.com/index/dall-e-3/}} to build a personal website with three pages: home, project, and contact (Appendix~\ref{appendix:demo}). We introduced various resources to each page to test the tool’s capabilities. These resources include internal links, external links, images, and backend routing (Table~\ref{tab:practical}). The tool successfully addressed all the challenges, achieving a 100\% success rate. The demonstration video of the entire development procedure is available online\footnote{\url{https://github.com/WebPAI/MRWeb}}. Specifically, the generated home page and project page included internal and external links and embedded images with pixel-perfect alignment. The contact page demonstrated the tool’s ability to integrate backend routing seamlessly, implying its full-stack capabilities. 

\begin{table}[ht]
    \centering
    \footnotesize
    \caption{Total number of challenges in the home, project, and contact page versus tool success rate.}
    \vspace{-5pt}
    \label{tab:practical}
    \begin{tabular}{lcc}
        \toprule
        Resource & Count & Success \\
        \midrule
        Internal links &  7 & 100\% \\
        External links &  9 & 100\% \\
        Images         &  13 & 100\% \\
        Backend routing & 1 & 100\% \\
        \bottomrule
    \end{tabular}
    \vspace{-5pt}
\end{table}


\section{Discussions}

This section highlights key implications of our work for future research.

\paragraph{Visual metrics for UI quality (RQ1)} While prior studies emphasize structural and semantic metrics, our findings show that pixel-based metrics better align with human judgment, especially in low-to-medium similarity cases. This suggests hybrid approaches that combine these metrics could provide more robust evaluations. The limited performance of learning-based methods in these cases indicates a need for targeted fine-tuning.

\paragraph{Enhancing website generation with resource lists (RQ2)} Incorporating resource lists significantly boosts both visual and functional metrics, underscoring their potential for advancing automated full-stack development.

\paragraph{Improving MLLM visual grounding (RQ3)} Metrics on positional shift and area difference highlight MLLMs’ limitations in precise positioning and sizing. Addressing this may involve improving visual grounding or developing layout-aware prompts for better layout reproduction.

\paragraph{Advancing MRWeb generation (RQ4)} MRWeb generation connects design and functionality, supporting links, images, and routing. However, non-link-based functionalities remain underexplored, presenting opportunities for more comprehensive full-stack development.




\section{Conclusion}
In this paper, we introduce the MRWeb generation task, addressing the limitations of single-page design-to-code methods. Our contributions include defining the MRWeb problem, creating a benchmark dataset, conducting a comprehensive IQA for web UIs, analyzing MLLM performance, and developing a dedicated MRWeb generation tool. We release the tool, dataset, and evaluation framework to facilitate future research.
\newpage

\section*{Limitations}

\textit{Limited Support for Non-Link Functionalities.}
While MRWeb effectively handles links, images, and routing, it does not currently support non-link-based functionalities, as these require distinct formulations and evaluation metrics. Addressing this limitation is a key focus for future work, with the goal of enabling full-stack development capabilities.

\noindent\textit{Context Length Constraints.}
MLLMs have limited context windows (e.g., 128K tokens for GPT-4o), which can be a challenge for websites with extensive token requirements. However, our experiments show that all prompts remain within these limits, highlighting the approach's feasibility for most practical scenarios.

\noindent\textit{Backbone Model Selection.}
We validate \taskname using three popular multimodal LLMs, but smaller models struggle with complex prompts. To improve adaptability and generalization, future work will explore the potential of emerging models and investigate strategies to handle more complex input scenarios.

%%
%% The next two lines define the bibliography style to be used, and
%% the bibliography file.
\bibliographystyle{acl_natbib}
\bibliography{reference}

\appendix

\section{APPENDIX FOR REPRODUCIBILITY}
\subsection{Related Work}
\subsubsection{Diffusion Models}\label{sec::diffusion}
The diffusion model is a probabilistic generative model first introduced by Sohl-Dickstein et al.~\cite{sohl2015deep} and further improved by Ho et al. ~\cite{ho2020denoising} and Song et al. ~\cite{song2020score}. As a novel generative model, diffusion models have rapidly advanced in time series and spatio-temporal modeling. 
Research on time series modeling based on diffusion models is widely applied, such as time series imputation ~\cite{alcaraz2022diffusion,liu2023pristi}, time series generation~\cite{lim2023regular,lin2023diffusion}, and time series forecasting~\cite{li2022generative,bilovs2022modeling}. 
DiffSTG~\cite{wen2023diffstg} is the first attempt to generalize the widespread denoising diffusion probabilistic models to spatiotemporal graphs (STGs), leading to a novel non-autoregressive framework. 
KSTDiff~\cite{zhou2023towards} designed a knowledge-enhanced denoising network to capture the spatiotemporal dependencies of urban flows and the influence of the urban environment in the denoising process.
DiffTraj~\cite{zhou2023towards} is a spatiotemporal diffusion probabilistic model for trajectory generation. This model effectively combines the generative capabilities of diffusion models with spatiotemporal features derived from real trajectories.
In this work, we introduce the diffusion model for unified mobility prediction adapted to different data types.

\subsection{Datasets Details}\label{sec::datasets_info}
We conducted extensive experiments on three real-world mobility datasets: Shanghai, Senegal, and Xinjiang. The details of datasets are summarized in Table~\ref{table:datasets}. We preprocess the trajectory data for three datasets, filtering out users with fewer than five records per day. For location preprocessing, we map GPS points to predefined grid IDs of a specific granularity. For temporal preprocessing, we organize the time data into fixed intervals, such as hourly or half-hourly segments. Finally, we divide the data into training, validation, and testing sets in a 7:1:2 ratio in chronological order. 

% dataset table
\begin{table}[h]
\setlength{\abovecaptionskip}{0.cm}
\setlength{\belowcaptionskip}{-0.cm}
\caption{Basic statistics of mobility datasets.}
\label{table:datasets}
\begin{center}
\scalebox{0.9}{
\begin{tabular}{ >{\centering\arraybackslash}m{1cm} 
>{\centering\arraybackslash}m{1cm} 
>{\centering\arraybackslash}m{1.2cm} 
>{\centering\arraybackslash}m{1cm}
>{\centering\arraybackslash}m{1cm}
>{\centering\arraybackslash}m{1cm}}
 \hline
City  & Duration & Users & Location \\ 
 \hline
Shanghai & 7 days & 700000 & 4096 \\ 
Senegal & 14 days & 8000 & 1666 \\ 
Xinjiang & 28 days & 1200000 & 4096\\ 
\hline
\end{tabular}}
\end{center}
\vspace{-0.3cm}
%\vspace{-20px}
\end{table}

\subsection{Baselines}\label{sec::baselines}
To evaluate the performance of our proposed model, we compared it with state-of-the-art models. Previous methods could only accomplish one type of mobility data prediction task, so the baseline methods are divided into trajectory and flow prediction. 

\paragraph{Flow Prediction} The baselines for flow prediction are as follows:
\begin{itemize}[leftmargin=*]
\item \textbf{HA}~\cite{sun2020predicting}: It considers the inflow and outflow to be seasonal processes and employs the average of the previous seasons as the prediction for a week-long period. 
\item \textbf{VAR}~\cite{lu2016integrating}: This method is vector autoregressive single-step predictor.
\item \textbf{ST-ResNet}~\cite{zhang2017deep}: ST-ResNet employs the residual neural network framework to model the temporal closeness, period, and trend properties of crowd flow.
\item \textbf{MSDR}~\cite{liu2022msdr}: Multi-Step Dependency Relationship (MSDR) is a brand new variant of recurrent neural networks. Instead of only looking at the hidden state from the latest time step, MSDR explicitly takes those from multiple historical time steps as the input of each time unit.
\item \textbf{STID}~\cite{shao2022spatial}: A simple multi-layer perceptron addresses the indistinguishability of time series samples in spatial and temporal dimensions.
\item \textbf{PriSTI}~\cite{liu2023pristi}: This method extracts coarse but effective spatiotemporal dependencies from conditional information using a diffusion model, serving as a global context prior.
\end{itemize}



\paragraph{Trajectory Prediction} The baselines for trajectory prediction are as follows:
\begin{itemize}[leftmargin=*]
\item \textbf{Markov Model}~\cite{gambs2012next}: The Markov model is a statistical model used to describe the change of states over time. It uses historical trajectory data for location prediction by calculating the transition probabilities between these locations.
\item \textbf{LSTM}~\cite{Kong2018HST}: The LSTM network is good at handling sequential data and has the advantage of encoding long-term dependencies, which can naturally be applied to location prediction.
\item \textbf{DeepMove}~\cite{feng2018deepmove}: The method designs a multimodal embedding recurrent neural network to capture complex sequential transitions by jointly embedding multiple factors that control human mobility.
\item \textbf{STAN}~\cite{luo2021stan}: This model associates non-contiguous but functionally similar visited points that are not adjacent to each other to predict the next location.
\item \textbf{SNPM}~\cite{yin2023next}: The method constructs a Sequence-based, Dynamic Neighbor Graph (SDNG) to find the similarity neighborhood and develop a Multi-Step Dependency Prediction model.
\item \textbf{TrajGDM}~\cite{chu2024simulating}: The method utilizes diffusion models to capture the universal mobility pattern in a trajectory dataset for trajectory prediction.
\item \textbf{GETNext}~\cite{yang2022getnext}: The method employs a global trajectory flow map and a novel Graph Enhanced Transformer model to leverage collaborative signals for more accurate trajectory prediction.
\end{itemize}

\begin{table}[h]
\small
\centering
\caption{Overall Performance on Xinjiang datasets.}
\vspace{-0.3cm}
\scalebox{0.9}{
\begin{tabular}{lcccccc}
\toprule
& \multicolumn{3}{c}{\textbf{Flow Prediction}} & \multicolumn{3}{c}{\textbf{Trajectory Prediction}} \\
\cmidrule(lr){2-4} \cmidrule(lr){5-7}
& \textbf{MAE} & \textbf{MAPE(\%)} & \textbf{RMSE}
& \textbf{Acc@1} & \textbf{Acc@3} & \textbf{Acc@5}\\
\midrule
HA & 33.16 & 30.54 & 44.28 & - & - & - \\
VAR & 23.90 & 22.15 & 36.63 & - & - & - \\
ST-ResNet & 19.72 & 17.36 & 31.56 & - & - & - \\
MSDR & 17.95 & 16.53 & 29.60 & - & - & - \\
STID & 17.01 & 15.70 & 27.36 & - & - & - \\
PriSTI & \underline{16.80} & \underline{15.37} & \underline{26.47} & - & - & - \\
Markov & - & - & - & 0.3156 & 0.3924 & 0.4571 \\
LSTM & - & - & - & 0.3847 & 0.4519 & 0.5450 \\
DeepMove & - & - & - & 0.4261 & 0.5143 & 0.6318 \\
STAN & - & - & - & 0.4432 & 0.5307 & 0.6609 \\
SNPM & - & - & - & 0.4618 & 0.5574 & 0.6926 \\
GETNext & - & - & - & 0.4650 & 0.5598 & 0.6975 \\
TrajGDM & - & - & - & \underline{0.4673} & \underline{0.5632} & \underline{0.7054} \\
UniMob-v1 & 16.31 & 14.91 & 25.98 & 0.4768 & 0.5795 & 0.7217 \\
UniMob-v2 & 15.96 & 14.72 & 25.54 & 0.4815 & 0.5853 & 0.7286 \\
UniMob-v3 & 16.12 & 14.84 & 25.70 & 0.4791 & 0.5830 & 0.7253 \\
UniMob-v4 & \bf{15.87} & \bf{14.50} & \bf{25.19} & \bf{0.4841} & \bf{0.5897} & \bf{0.7336} \\
%Improvement & 5.54\%  &	5.66\%	& 4.84\% &	3.60\% &	4.71\%	& 4.00\% \\
\bottomrule
\end{tabular}
}
%\vspace{-0.3cm}
\label{tab:Xinjiang}
\end{table}


\begin{table*}[t]
\small
\centering
\caption{Ablation study on Xinjiang datasets.}
\vspace{-0.3cm}
\scalebox{1.}{
\begin{tabular}{lcccccc}
\toprule
& \multicolumn{3}{c}{\textbf{Trajectory Prediction}} & \multicolumn{3}{c}{\textbf{Flow Prediction}} \\
\cmidrule(lr){2-4} \cmidrule(lr){5-7}
& \textbf{Acc@1} & \textbf{Acc@3} & \textbf{Acc@5}
& \textbf{MAE} & \textbf{MAPE(\%)} & \textbf{RMSE}\\
\midrule
Ours & 0.4768 & 0.5795 &  0.7217 & 16.31 & 14.91 & 25.98 \\
w/o I2C loss & 0.4736 (-0.67\%) & 0.5730 (-1.12\%) &  0.7125 (-1.28\%)
 & 16.78 (-2.88\%) & 15.43 (-3.49\%) & 27.02 (-4.00\%) \\
w/o C2I loss & 0.4689 (-1.66\%) & 0.5671 (-2.14\%) &  0.7064 (-2.12\%)
 & 16.46 (-0.92\%) & 15.08 (-1.14\%) & 26.90 (-3.54\%) \\
w/o shared transformer & 0.4702 (-1.39\%) & 0.5693 (-1.76\%)
 & 0.7091 (-1.75\%) & 16.67 (-2.21\%) & 15.29 (-2.55\%)
 & 26.97 (-3.81\%)
 \\
w/o flow data & 0.4639(-2.71\%) & 0.5620(-3.02\%)
 & 0.6998(-3.03\%) & - & - & - \\
w/o trajectory data & - & - & - & 16.87(-3.43\%)
 & 15.56(-4.36\%) & 27.20(-4.70\%) \\
\bottomrule
\end{tabular}
}
%\vspace{-0.3cm}
\label{tab:Ablation2}
\end{table*}



\section{Experimental Performance}\label{sec::Results}
\subsection{Overall Performance}\label{sec::Overall Performance}
Table~\ref{tab:Xinjiang} shows the performance of our universal mobility prediction model on the Xinjiang dataset. UniMob not only accomplishes both trajectory and flow predictions simultaneously but also surpasses current advanced baseline models in all evaluation metrics. Specifically, it achieves 5.34\% performance improvement in flow prediction and more than 4\% enhancement in trajectory prediction. These results fully demonstrate the generality and reliability of our model.





\subsection{Ablation study}\label{sec::ablation}
We conducted ablation experiments on two aspects: model design and data utilization. By sequentially removing components of the model design, we identified three design elements that align with different data formats and distributions, each impacting performance, thus validating their effectiveness. Regarding data utilization, by replacing multi-type data with single-type data, we visually demonstrated the performance enhancement brought by using multi-type mobility data in human mobility prediction through our universal model.






\subsection{Noise Perturbation}\label{sec::noise}

\begin{figure}[t]
\centering
\subfigure[Flow prediction]{\includegraphics[width=.23\textwidth]{figure/xinjiang_noisy_flow.pdf}}
\vspace{-0.5cm}
\subfigure[Trajectory prediction]{\includegraphics[width=.23\textwidth]{figure/xinjiang_noisy_trajectory.pdf}}
\caption{Flow and trajectory prediction with noisy data on Xinjiang dataset.} 
%\vspace{-0.3cm}
\label{fig:noisy}
\end{figure}

Due to biases from sensor collection and artificial noise added for privacy protection, the data used for mobility prediction often contains noise. To verify whether our model can still maintain good predictive capabilities in noisy conditions, we added noise to both the flow and trajectory data. Figure~\ref{fig:noisy} shows that as noise levels increase, our model continues to outperform the best baseline model, and our performance advantage becomes even more pronounced relative to the baseline with increasing noise. This effectively demonstrates the high robustness of our UniMob model.



\subsection{Few-shot Performance}\label{sec::few-shot}
Similarly, due to data collection and privacy protection limitations, the amount of mobility data we acquire is often limited. Therefore, we tested the few-shot learning capabilities of our UniMob model. As shown in Figure~\ref{fig:low}, our model still performs excellently even in a data-constrained environment.

\begin{figure}[t]
\centering
\subfigure[Flow prediction]{\includegraphics[width=.23\textwidth]{figure/xinjiang_low_flow.pdf}}
\vspace{-0.5cm}
\subfigure[Trajectory prediction]{\includegraphics[width=.23\textwidth]{figure/xinjiang_low_trajectory.pdf}}
\caption{Flow and trajectory prediction with scarce data on Xinjiang dataset.} 
%\vspace{-0.3cm}
\label{fig:low}
\end{figure}

\end{document}
\endinput
%%
%% End of file `sample-authordraft.tex'.
