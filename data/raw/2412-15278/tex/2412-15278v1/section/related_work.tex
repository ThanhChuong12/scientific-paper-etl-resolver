\section{Related Work}

\subsection{Text-to-3D Content Generation}

One category of text-to-3D generation starts from DreamField \cite{jain2022zero}, which trains NeRF with CLIP \cite{radford2021clip} guidance to achieve text-to-3D distillation. However, the generated content is unsatisfactory due to the weak supervision from CLIP loss. Hence, our work will not consider watermarking CLIP-guided 3D content generation. Another category starts from Dreamfusion \cite{poole2022dreamfusion}, which pioneerly introduces Score Distillation Sampling (SDS) to optimize NeRF by distilling a pre-trained text-to-image diffusion model. This motivates a great number of following works to propose critical incremental. These works improve the quality of generation in various ways. For example, Fantasia3D \cite{chen2023fantasia3d}, Magic3D \cite{lin2023magic3d}, Latent-nerf \cite{metzer2023latent}, DreamGaussian \cite{tang2023dreamgaussian} and GaussianDreamer \cite{yi2023gaussiandreamer} improve the visual quality of generated content by changing 3D representations or improving NeRF structure. MVDream \cite{shi2023mvdream} focuses on addressing Janus problems by fine-tuning the pre-trained diffusion model to make it 3D aware. However, SDS guidance still suffers from over-saturation problems, as is shown in Magic3D \cite{lin2023magic3d}, Dreamfusion \cite{poole2022dreamfusion}, and AvatarVerse \cite{zhang2024avatarverse}. The other, like ProlificDreamer \cite{wang2024prolificdreamer} and LucidDreamer \cite{liang2024luciddreamer}, focus on improving SDS itself. For example, LucidDreamer uncovers the reason for the overly-smoothed problem that SDS guides the generation process towards an averaged pseudo-ground-truth and proposes ISM to relieve such a problem. ProlificDreamer proposes VSD guidance instead of SDS guidance and shows that SDS is just a special case of VSD. Although extensive research has been proposed to improve text-to-3D generation, these works still require a much longer training stage, which makes it necessary to protect the copyright of generated content.

\subsection{Digital Watermarking}

Digital watermarking hides watermarks into multimedia for copyright protection or leakage source tracing. Various research works have been proposed to protect traditional multimedia content like 2D images and 3D meshes. Early works watermark images and meshes by embedding a secret message in either the least significant bits \cite{van1994digital} or the most significant bits \cite{tsai2020separable, jiang2017reversible, tsai2022integrating, peng2022semi, peng2021general} of image pixels and vertex coordinates. HiDDeN \cite{zhu2018hidden} and Deep3DMark \cite{zhu2024rethinking} have made substantial improvements using deep learning networks.

Recently, several watermarking methods have emerged in the NeRF domain. StegaNeRF \cite{li2023steganerf} designed a steganography algorithm that hides natural images in 3D scene representation. CopyRNeRF \cite{luo2023copyrnerf} protects the copyright of NeRF by verifying the secret message extracted from images rendered from the protected NeRF. WateNeRF \cite{jang2024waterf} further improves NeRF watermarking by hiding secret messages into the frequency domain of rendered images, increasing the robustness of the watermark. However, CopyRNeRF and WateNeRF are two \textit{post-generation} watermarking methods, \ie they watermark by fine-tuning a pre-trained NeRF. This poses a delay between the NeRF generation and watermarking. A malicious user could obtain the pre-trained NeRF before it is watermarked. Besides, CopyRNeRF requires additional changes in NeRF architecture. We would like the watermarking method to be architecture agnostic due to the fact that some text-to-3D generation methods, like Magic3D, Fantasia3D, and Latent-nerf, require specific NeRF architecture for visual quality improvement. To address these issues, we design an architecture-agnostic method that watermarks NeRF during generation.