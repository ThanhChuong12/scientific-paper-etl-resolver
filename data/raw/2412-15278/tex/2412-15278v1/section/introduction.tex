\section{Introduction}


Digital 3D content has become indispensable in Metaverse and virtual and augmented reality, enabling visualization, comprehension, and interaction with complex scenes that represent our real lives. Recent progress in 3D content generation \cite{poole2022dreamfusion,lin2023magic3d,wang2024prolificdreamer,liang2024luciddreamer} can generate high-quality 3D assets that need a lot of time, computational resources, and skilled expertise. Therefore, protecting the ownership of generated 3D content has become more critical.

We focus on Text-to-3D generation \cite{poole2022dreamfusion,lin2023magic3d,wang2024prolificdreamer,liang2024luciddreamer} and the neural radiance field (NeRF) \cite{mildenhall2021nerf, muller2022instant}, which have emerged into the spotlight in 3D content modeling. Current trending 3D generation algorithms generate 3D representations such as meshes and NeRFs. This paper focuses on NeRF generation since NeRF can represent 3D models more compactly. Given a textual description, recent text-to-3D methods generate NeRFs by distilling pre-trained diffusion models, such as Stable Diffusion \cite{rombach2022high}. This remarkable progress is grounded in the use of Score Distillation Sampling (SDS). With SDS, NeRF training can be conducted without realistic images. Thus, the research question we address in this paper is: \textit{how to protect the score distillation sampling generated neural radiance fields?}

\begin{figure}[t]
    \centering
    \includegraphics[width=\linewidth]{figures/motivation_egs.pdf}
    \caption{Attack scenario. If company-generated content is considered company property, internal staff could steal non-watermarked intermediates in the post-generation pipeline (top row). However, such intermediates do not exist in the during-generation pipeline (bottom row).}
    \label{fig:motivation_egs}
\end{figure}

One way to protect the generated NeRF is to apply post-generation watermarking methods, such as CopyRNeRF \cite{luo2023copyrnerf} and WateNeRF \cite{jang2024waterf}, to watermark NeRF after it is generated. However, these methods exhibit two problems. First, post-generation methods pose a risk of data leakage. As shown in Figure \ref{fig:motivation_egs}, since non-watermarked intermediates are generated in the post-generation pipeline, a malicious user could leak the non-watermarked version of the generated content. Second, CopyRNeRF increases the watermarking expense since it requires an additional message feature field in the NeRF structure. Integrating CopyRNeRF with an arbitrary text-to-NeRF pipeline requires additional modifications to the NeRF structure. Recognizing these limitations of previous work, can we conduct a \textit{during-generation} watermarking without modifying the NeRF structure?

We propose \tool, the first \textit{during-generation} text-to-3D watermarking method, which is gracefully combined with score distillation sampling to generate high-quality and watermarked NeRF. Different from \textit{post-generation} NeRF watermarking method, \tool directly generates watermarked NeRF without changing NeRF architecture, increasing the flexibility for future development on 3D generation. Our method is inspired by black-box model watermarking methods \cite{adi2018turning, zhang2018protecting, jia2021entangled, le2020adversarial, chen2019blackmarks, szyller2021dawn} which watermark a deep neural network by injecting backdoors. To inject backdoors in NeRF during generation, we first generate a trigger view set dependent only on the given secret message. Then, we conduct score distillation sampling in a way that the secret message can be extracted from images rendered from arbitrary trigger viewports. To extract the secret message from the rendered image, we use a pre-trained watermark decoder from HiDDeN \cite{zhu2018hidden}. All NeRF generated by \tool can be verified as watermarked by such a unique decoder.

Two critical evaluation metrics for watermarking algorithms are invisibility and robustness. For robustness, we evaluate bit accuracy under multiple image transformations, such as Gaussian noise, before images rendered from trigger viewports are fed into the watermark decoder. For invisibility, there is no such the ``original NeRF" since we root watermarks during a generation task, so \tool cannot be evaluated by Peak Signal-to-Noise Ratio (PSNR) as is done in \textit{post-generation} methods. However, we can still evaluate the invisibility by evaluating the generation quality as previous 2D watermarking tasks \cite{wen2024tree, yang2024gaussian}, where they use CLIP Score \cite{radford2021clip} to show the generation quality. Extensive experiments show that \tool successfully embeds the watermark in a \textit{during-generation} way and maintains robustness under multiple image transformations without degrading the generation quality. In summary, our contributions are as follows:
\begin{itemize}
    \item To the best of our knowledge, we propose \tool, the first during-generation 3D watermarking method, which eliminates the delay between NeRF generation and watermarking, ensuring that no non-watermarked version of the NeRF is ever produced, thereby preventing NeRF theft.
    
    \item The key novelty of our \tool is that it watermarks NeRF by injecting backdoors during score distillation sampling, such that the secret message can be extracted from images rendered from arbitrary trigger viewport.
  
    \item Extensive experiments show that the embedded watermark achieves 90+\% bit accuracy against multiple image transformations, and the watermarking process does not degrade the generation quality.
\end{itemize}