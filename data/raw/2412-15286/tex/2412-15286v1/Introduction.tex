Quantum simulators, e.g., based on  ultracold atoms in optical lattices \cite{Ultracold_atoms_1, Ultracold_atoms_2, Ultracold_atoms_3, Ultracold_atoms_4, Hall_Ribbons}, offer an impressive flexibility in engineering many-body Hamiltonians, now rendering the preparation of low-entropy states of complex quantum matter a key remaining challenge in the field . In this context, topological states inevitably separated from a trivial product state by a topological quantum phase transition are of primary interest \cite{HasanKane2010, Qi2011, Budich2013, Wen2017}. A conceptually simple example of this type is provided by topological superconductors hosting Majorana bound states \cite{Majorana_review_1, Majorana_review_2, Majorana_review_3, Majorana_review_4, Majorana_TSC_1, Majorana_TSC_2, Majorana_TSC_3, Majorana_TSC_4}. These exotic quasiparticles have been widely discussed for topological quantum computing architectures \cite{TQC_1, TQC_2, TQC_3, TQC_4, TQC_5, TQC_6, TQC_7} and related settings \cite{Majorana_FQH_1, Majorana_FQH_2, Majorana_FQH_3, Majorana_FQH_4, Majorana_FQH_5, Majorana_FQH_6, Kitaev_Honeycomb} due to their non-Abelian braiding statistics \cite{Non_abelian_statistics_1, Non_abelian_statistics_2, Non_abelian_statistics_3, Non_abelian_statistics_4}.

	

\begin{figure}[thp!]	 
  {
        \vbox to 0pt {
                \raggedright
                \textcolor{white}{
                    \subfloatlabel[1][Fig:illustration_a]
                    \subfloatlabel[2][Fig:illustration_b]
                    \subfloatlabel[3][Fig:illustration_c]
                }
            }
    }
\includegraphics{Illustrations/sketch.pdf}
\caption{Illustration of the protocol in  \eq{Eqn:time_dependence_p1}. (a) Starting from a Mott state  stabilized at filling $\nu = 1/3$ by a staggered chemical potential $\mu$, in (b) the pair hopping term $W$ may be adiabatically switched on thanks to an additional interchain hopping $r$ that threads a commensurate flux $\phi = \nu$ through each plaquette. Importantly, this flux-hopping breaks the $\mathbb{Z}_2$ wire-parity symmetry $P_\mathrm{a} = (-1)^{N_\mathrm{a}}$ and induces a bulk gap in combination with either of both $\mu$ and $W$. (c) Finally, the target Hamiltonian hosting the atomic Majorana phase is approached by turning off $r$.  As the final state is critical, this requires a specialized strategy to optimize preparation time (see Sec.~\ref{Sec:crit_state_prep}).}\label{Fig:illustration}
\end{figure}	

The paradigmatic example of a Majorana phase is known as the Kitaev chain \cite{Kitaev_chain, KC_related_models_1, KC_related_models_2, KC_related_models_3, KC_related_models_4}, a \gls{1d} (proximity induced \cite{1D_TSC_from_proximity_1, 1D_TSC_from_proximity_2, 1D_TSC_from_proximity_3, 1D_TSC_from_proximity_4, 1D_TSC_from_proximity_5, 1D_TSC_from_BEC_proximity_1, 1D_TSC_from_BEC_proximity_2}) superconductor with global fermion parity conservation guaranteed by the bulk superconducting gap, and a single zero-energy Majorana bound state at each end. Remarkably, a variant of the Kitaev chain with global particle number conservation has been taylored for the toolbox of ultracold atoms in optical lattices  \cite{Tu_et_al, Fidkowski_et_al, Halperin_Sarma, Number_conserving_Majorana_1, Number_conserving_Majorana_2, Number_conserving_Majorana_3, Number_conserving_Majorana_4, Number_conserving_Majorana_5, Number_conserving_Majorana_6, Number_conserving_Majorana_7, Zoller_model}. There, a \gls{1d} two-leg ladder (or double wire) system (see Fig. \ref{Fig:illustration}), in which inter-chain pair hopping provides the counterpart of proximity induced Cooper pair tunneling, has been shown to also stabilize a single Majorana end state. However, the absence of the bulk superconductor has two important consequences. First, the resulting Majorana phase is symmetry protected by a sub-wire fermion parity $P_\mathrm{a} = (-1)^{N_\mathrm{a}}$ with the particle number $N_\mathrm{a}$ in wire $\mathrm{a}$ \cite{Tu_et_al, Fidkowski_et_al, Halperin_Sarma}. Since $P_\mathrm{a}$ may be broken by simple single particle inter-wire tunneling, the atomic Majorana phase becomes a conventional symmetry protected topological phase. Second, the closed 1D nature of the double wire system limits the counterpart of superconducting order to the emergence of a power-law decaying pair-hopping induced gap \cite{Tu_et_al, Fidkowski_et_al, Halperin_Sarma, Mermin_Wagner, Hohenberg, SC_in_1D}. From a vantage point of state preparation, the first point is good news while the second one represents an extra challenge. More specifically, symmetry protection can be exploited by controlled intermediate symmetry breaking and restoring during the protocol, while the smaller gap of the target state requires an adiabatic time-scale that grows with system size, even in the sophisticated framework of critical state preparation  \cite{crit_state_prep_1, crit_state_prep_2, crit_state_prep_3, crit_state_prep_4, Review_QPT_Dziarmaga}. Yet, the simplest conceivable symmetry breaking term in the form of a single-particle inter-chain hopping with real strength $r$ (cf. $\phi = 0$ case below) has been found not to open a bulk gap \cite{Zoller_model,Tu_et_al, Fidkowski_et_al}. An efficient state preparation protocol for the atomic Majorana phase from a trivial initial state has so far remained elusive.      


Here, by adding a Peierls phase that amounts to a commensurate synthetic flux to the inter-chain hopping, we identify a symmetry breaking mass term opening a constant gap in system size. This allows us to extend the notion of critical state preparation  to a protocol for preparing the number-conserving atomic Majorana phase with an optimal scaling in system size (see Fig. \ref{Fig:illustration} for an illustration). Specifically, we provide clear evidence that the target state can be prepared in a time that asymptotically scales linear in system size, while a protocol using plain finite size effects would require quadratic ramp times \cite{Adiabatic_theorem_Born_Fock, Adiabatic_theorem_Kato}. Our in-depth theoretical analysis of the crucial single-particle flux-hopping term is based on analytical techniques in the framework of constructive Bosonization \cite{Bosonization_OBC, Bosonization_Delft, Bosonization_Schoenhammer, Giamarchi} that take into account finite size terms and resolve effects sensitive to the total fermion parity. The resulting qualitative predictions are then corroborated and quantified by numerical simulations using \gls{mps} methods \cite{Itensor_basic, Itensor_codebase_release}. 

We start by introducing the model Hamiltonian and the considered perturbations in Sec.~\ref{Sec:Model}, before we proceed to describe and investigate the \gls{gs} preparation protocol in detail in Sec.~\ref{Sec:GSP}. Our bosonization analysis is presented in Sec.~\ref{Sec:Bosonization}, and a concluding discussion in \ref{Sec:conclusion}.





