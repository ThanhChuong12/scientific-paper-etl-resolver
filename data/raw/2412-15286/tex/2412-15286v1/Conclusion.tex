We have investigated the preparation of a critical Majorana phase in a two-leg ladder with total particle number conservation, where the wire parity $P_\mathrm{a} = (-1)^{N_\mathrm{a}}$ acts as the protecting $\mathbb Z_2$ symmetry. The key ingredient of our protocol is a symmetry-breaking perturbation in the form of an inter-wire tunneling that threads an artificial magnetic flux commensurate with the filling fraction through each plaquette of the ladder, thereby acting as a ``mass'' term that immediately gaps out the critical phase. This additional term is feasible with existing experimental techniques. We argue how our protocol achieves an asymptotically optimal scaling of preparation time linear with system size, and provide \gls{mpste} data to substantiate our findings. For comparison, we also study a symmetry-respecting preparation approach that is found to require quadratic ramp times.

The theoretical backbone of this work is a constructive bosonization analysis of the Majorana Hamiltonian and the flux hopping term, taking into account particle-number dependent finite-size terms and carefully constructed Klein factors. On this basis, we demonstrate why previously discussed symmetry-breaking terms, e.g., a regular inter-wire tunneling without flux, are not helpful for state preparation, by contrast to the commensurate flux term. Moreover, the bosonized theory is able to resolve subtle effects such as qualitative differences between systems with odd and even total fermion parity. The field-theoretical predictions are fully confirmed through \gls{dmrg} simulations.

Finally, we would like to discuss the relation to some previous work on topological state preparation. First, we note that the dissipative preparation as the steady state of a quantum master equation of a \gls{1d} Majorana phase in a number-conserving setting has also been discussed in Ref.~\cite{Dissipative_Majorana_Preparation}. There, the typical preparation time is expected to scale with system size as $t_\mathrm{tot}\propto L^2$. Second, a detailed proposal for the preparation of an integer Chern insulator state through the augmentation with the inverted topological phase has been put forward in Ref.~\cite{Barbarino_et_al}. As topological superconductors at mean field level, including the Kitaev chain, formally fall into the class of invertible topological phases, one may consider extending these ideas to the state preparation of Majorana phases.  However, we note that our present system operates at a critical point in a strongly interacting regime, which at least challenges any mean field picture of topological superconductivity, such that the results of Ref.~\cite{Barbarino_et_al} cannot be adapted directly. 

%Perhaps the most paradigmatic example along these lines is the integer Chern insulator as a topological state that does not rely on symmetry-protection and thus lies in class A of the \gls{caz} classification scheme. While the phase transition can accordingly not be avoided through a symmetry-breaking perturbation, the phase is invertible and can be prepared by augmenting the system with its inverted copy, thereby keeping the total system trivial at all times \cite{Barbarino_et_al}. As a complementary route, schemes measuring and pairing up anyonic defects to reach the \gls{gs} can be devised for certain systems and have already been implemented experimentally \cite{Veresen_1, Veresen_2, Veresen_3, Veresen_4}...  and therefore  belonging to \gls{caz} class D. From the perspective of \gls{gsp}, the Kitaev chain is similar to the Chern insulator in so far as the global fermion parity is a physical symmetry that cannot be broken and the phase is invertible. 
%Ergo, extending the program of Ref.~\cite{Barbarino_et_al} would be straightforward on a formal level, yet the absence of true \gls{sc} order in one dimension \cite{Mermin_Wagner, Hohenberg, SC_in_1D} requires the proximity of a bulk superconductor \cite{Majorana_TSC_3, 1D_TSC_from_proximity_1, 1D_TSC_from_proximity_2, 1D_TSC_from_proximity_3, 1D_TSC_from_proximity_4, 1D_TSC_from_proximity_5} or other reservoir \cite{1D_TSC_from_BEC_proximity_1, 1D_TSC_from_BEC_proximity_2} for a realistic experimental proposal. This moves the Kitaev chain beyond the range of current quantum simulation techniques.