\section{Related Work}

\fakepar{Sensing and Models}
While it is widely known that LLMs excel at language-based tasks, various attempts are made to test LLMs on different modalities like images, audio, and time-series data. There are significant advancements in audio and image domains~\cite{audio_transformer, vision_transformer}. Time-series as input to LLMs still remains a bigger challenge to be solved, although there are many works aimed that have had decent progress~\cite{times-series-llm}. Advances in this benefit many domains like medical~\cite{health_learners} and sensor data analysis~\cite{penetrative_ai}, which primarily contain time-series data from sensors. Mo et al.~\cite{iot-lm} makes LLMs comprehend sensory data by modifying the LLM's architecture. A new multisensory multi-task adapter layer is introduced, making the model capable of perceiving eight IoT tasks. 

\fakepar{LLMs and Programming} Recent years have seen growing interest in using LLMs in the software development process. They demonstrate an increasing ability to generate relevant code from natural language prompts. LLMs are also used for other coding tasks like completion, syntax correction, and refactoring. These capabilities have led to surprising results: AlphaDev~\cite{AlphaDev2023}, for instance, discovered a faster sorting algorithm that surpasses previously known human benchmarks. Meta's LLM Compiler~\cite{MetaCompiler2024}, designed for compiler optimization, is another breakthrough by enhancing code generation efficiency and aims to optimize code for better performance and resource utilization. Consequently, alongside larger, cloud-based LLMs such as ChatGPT and Claude, smaller LLMs designed specifically for coding tasks have also emerged.
Examples include CodeLlama~\cite{roziere2023code}, StarCoder~\cite{llm_starcoder}, Codestral~\cite{llm_codestral}, DeepSeek Coder~\cite{llm_deepseek}, and CodeBERT~\cite{feng2020codebert}. Many of these LLMs are now part of commercial products, including GitHub CoPilot\footnote{\url{https://github.com/features/copilot}} and OpenAI Codex\footnote{\url{https://openai.com/index/openai-codex/}}. \system\space is complementary to these systems and can utilize LLMs optimized for coding purposes. 