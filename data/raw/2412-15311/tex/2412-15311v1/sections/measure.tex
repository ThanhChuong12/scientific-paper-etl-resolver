 % !TEX root = ../main.tex
 
 
 \subsection{Robust Coverage for Measuring Group Robustness}
 \label{sec:robust_coverage}
 
 In the previous subsection, we have seen that there exists a clear trade-off between robust and average accuracies, and robust scaling can identify the optimal point for the target metric.
 However, it still reflects only a single point of the trade-off curve while ignoring all other possible Pareto optimums.
 For a more comprehensive view, we propose a convenient measurement to have a scalar summary of the robust-average accuracy curve.
%  However, there may still have some artifacts that 
%  For more reliable measurement, we propose the \textit{Robust Coverage} that considers the trade-off. 
 Formally, we define the \textit{Robust Coverage} as
 \begin{align}
     \text{Coverage} :=  \int_{c=0}^1\max_\mathbf{s}\big\{\text{UA}^\mathbf{s}|\text{AA}^\mathbf{s}\geq c\big\}dc 
     \approx  \sum_{d=0}^{D-1}\frac{1}{D}\max_\mathbf{s}\big\{\text{UA}^\mathbf{s}|\text{AA}^\mathbf{s}\geq \frac{d}{D}\big\}
     \label{eq:coverage}
 \end{align}
 where $\text{UA}^\mathbf{s}$ and $\text{AA}^\mathbf{s}$ denote the unbiased and average accuracies of applying (\ref{eq:rs}) with the scaling factor $\mathbf{s}$.
 $D = 1000$ is the number of slices for discretization.
 The robust coverage measures the area under the extended robust-average accuracy curve, where the maximum operation in~(\ref{eq:coverage}) finds the Pareto optimum for each threshold.
 We can adopt either UA or WA as a target metric of the robust coverage, referred to as unbiased or worst-group coverages, respectively, as visualized in Fig.~\ref{fig:worst_curve_all_pareto_celeba} and~\ref{fig:unbias_curve_all_pareto_celeba}, respectively.
%Fig.~\ref{fig:worst_curve_all_pareto_celeba} and~\ref{fig:unbias_curve_all_pareto_celeba} visualizes both robust coverage curves on the CelebA dataset, where UA and WA are adopted as a target metric, respectively.
%  $\text{UA}^\mathbf{s}$ can be replaced by worst-group accuracy, $\text{WA}^\mathbf{s}$.
 Please refer to our supplementary document for more discussions.
%  The robust coverage is useful in the model selection process, where we can select the model which has the highest robust coverage in the validation split.

 \iffalse
 \subsubsection{Robust coverage in test split}
 Robust coverage can be directly calculated in the validation split, but unfortunately, it is not available in the test split.
This is because we need to know the values of $\text{UA}^{\mathbf{s}}$ in advance to conduct max operation in \eqref{eq:coverage}, but we cannot use the information in the test set.
To bypass this issue, we first define the set of optimal scaling factors for each threshold in the validation split $\mathcal{S}_\text{val}$ as
  \begin{equation}
%      \mathcal{S}_\text{val} := \Big\{ \max_\mathbf{s}\big\{\text{UA}^\mathbf{s}|\text{AA}^\mathbf{s}\geq c\big\}~~\forall c \in [0, 1]
%      \Big\},
      \mathcal{S}_\text{val} := \Big\{ \max_\mathbf{s}\big\{\text{UA}_{\text{val}}^\mathbf{s}|\text{AA}_{\text{val}}^\mathbf{s}\geq \frac{d}{D} \big\}~~\text{for}~0 \leq d \leq D-1
      \Big\},
  \end{equation}
then the unbiased coverage of test split is calculated by
\begin{equation}
      \text{Test coverage} := \frac{1}{|\mathcal{S}_\text{val}|} \sum_\mathbf{s \in \mathcal{S}_\text{val}} \text{UA}_{\text{test}}^\mathbf{s}.
  \end{equation}
\fi
 
 
 \iffalse
 \begin{table}[t]
\begin{center}
\caption{The results of robust coverage on the CelebA dataset with the average of three runs (standard deviations in parenthesis).
%Unbiased and worst denote unbiased and worst-group accuracy, respectively.
}
\vspace{0.2cm}
\label{tab:celebA_coverage}
 \scalebox{0.85}{
\setlength\tabcolsep{4pt} 
\begin{tabular}{cl|cc|c}
\toprule
Dataset & Method & Worst-. Cov. & Unbiased Cov. & Worst-group Acc.  \\
\hline
%CelebA & ERM & 31.1(1.7) & 76.4(1.2) & {95.5(0.4)} \\
%CelebA & CR & 70.6(6.0) & 88.7(1.2) & {94.2(0.7)} \\
%CelebA & LfF~\citep{LfF} & 55.6(6.6) & 81.5(2.8) & {92.4(0.8)} \\
%CelebA & JTT~\citep{JTT}  &76.1(3.4) &85.9(0.3) &{89.5(1.2)} \\
%CelebA & Group DRO~\citep{GroupDRO} & 88.4(2.3) & 92.0(0.4) & {93.2(0.8)} \\
%CelebA & GR & 87.9(1.7) & 91.9(0.3) & {93.0(0.9)}\\
CelebA & ERM & 83.0(0.7) & 88.1(0.5) & 31.1(1.7) \\ % & 76.4(1.2) & {95.5(0.4)} \\
CelebA & CR & 82.9(0.5) & 88.2(0.3) & 70.6(6.0) \\ %& 88.7(1.2) & {94.2(0.7)} \\
CelebA & LfF~\citep{LfF} & 73.1(2.2) & 79.5(1.8) & 55.6(6.6) \\%& 81.5(2.8) & {92.4(0.8)} \\
CelebA & JTT~\citep{JTT}  &77.3(0.7) & 81.9(0.7) &76.1(3.4) \\% &85.9(0.3) &{89.5(1.2)} \\
CelebA & Group DRO~\citep{GroupDRO} &87.3(0.2) & 88.4(0.2)  & 88.4(2.3) \\% & 92.0(0.4) & {93.2(0.8)} \\
CelebA & GR &86.9(0.4) & 88.3(0.2) & 87.9(1.7) \\% & 91.9(0.3) & {93.0(0.9)}\\
\bottomrule
\end{tabular}
 }
\end{center}
%  \vspace{-0.1cm}
\end{table}
\fi

% \begin{table}[t]
%\begin{center}
%\caption{The results of robust coverage on the CelebA dataset with the average of three runs (standard deviations in parenthesis).
%%Unbiased and worst denote unbiased and worst-group accuracy, respectively.
%}
%\vspace{0.2cm}
%\label{tab:celebA_coverage}
% \scalebox{0.85}{
%\setlength\tabcolsep{4pt} 
%\begin{tabular}{cl|cc|c}
%\toprule
%Dataset & Method & Worst-group Cover. & Unbiased Cover. & Worst-group Acc.  \\
%\hline
%CelebA & ERM & 83.0(0.7) & 88.1(0.5) & 31.1(1.7) \\ % & 76.4(1.2) & {95.5(0.4)} \\
%CelebA & CR & 82.9(0.5) & 88.2(0.3) & 70.6(6.0) \\ %& 88.7(1.2) & {94.2(0.7)} \\
%CelebA & LfF~\citep{LfF} & 73.1(2.2) & 79.5(1.8) & 55.6(6.6) \\%& 81.5(2.8) & {92.4(0.8)} \\
%CelebA & JTT~\citep{JTT}  &77.3(0.7) & 81.9(0.7) &76.1(3.4) \\% &85.9(0.3) &{89.5(1.2)} \\
%CelebA & Group DRO~\citep{GroupDRO} &87.3(0.2) & 88.4(0.2)  & 88.4(2.3) \\% & 92.0(0.4) & {93.2(0.8)} \\
%CelebA & GR &86.9(0.4) & 88.3(0.2) & 87.9(1.7) \\% & 91.9(0.3) & {93.0(0.9)}\\
%\bottomrule
%\end{tabular}
% }
%\end{center}
%%  \vspace{-0.1cm}
%\end{table}
 
 
 % This is more reliable than using the robust accuracy of a single point.
 %% model selection 결과 필요.
 
 

 
  \iffalse
 We clarify that measuring the exact robust coverage in the test set is not possible, because it contains the maximum operation which requires the knowledge of test set results.
 Nevertheless, we use the robust coverage as an additional metric because it gives a tight upper bound.
 We also observe that the robust coverage of validation and test sets are almost the same.
%  As in Figure~\ref{}, we observe that the optimal points which gives the maximum UA given of the validation set is mostly aligned well to that of the test set. 
 Please refer to the supplementary document for more details about this issue.
 \fi
%  However, we observe that the optimal points of the validation set is mostly aligned well to that of the test set. ...
 
 
 % Figure needed.
 
%  \subsubsection{Results}
%  We evaluated robust coverage of existing approaches and ERM baseline.
 
 