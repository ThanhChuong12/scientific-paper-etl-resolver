 % !TEX root = ../main.tex
 

 
\section{Proposed Approach}
\label{sec:method}

This section first presents our class-specific scaling technique, which captures the trade-off landscape and identifies the optimal performance points for desired objectives along the trade-off curve.
We also propose an instance-wise class-specific scaling approach to overcome the trade-off and further improve the performance.
Based on the proposed scaling strategy, we introduce a novel and intuitive measure for evaluating the group robustness of an algorithm with consideration of the trade-off.
 

\subsection{Problem Setup}
\label{sec:setup}
Consider a triplet $(x, y, a)$ with an input feature $x \in \mathcal{X}$, a target label $y \in \mathcal{Y}$, and an attribute $a \in \mathcal{A}$. 
We define groups based on the pair of a target label and an attribute, such that $g := (y, a) \in \mathcal{Y} \times \mathcal{A} =: \mathcal{G}$.
Suppose that the training set consists of $n$ examples without attribute annotations, \eg, $\{(x_1, y_1), ..., (x_n, y_n)\}$, while the validation set includes {$m$ examples with group annotations, \eg, $\{(x_1, y_1, a_1), ..., (x_m, y_m, a_m)\}$, for selecting scaling parameters.
 % \eg, $\{(x_1, g_1), ..., (x_m, g_m)\}$ = $\{(x_1, y_1, a_1), ..., (x_m, y_m, a_m)\}$.
This assumption is known to be essential for model selection or hyperparameter tuning~\cite{GroupDRO, JTT, LfF, idrissi2022simple} although not desirable for the practicality of algorithms. 
However, we will show that our algorithm works well with only a few examples with attribute annotations in the validation set; considering such marginal labeling cost, our approach is a meaningful step to deal with notorious bias problems in datasets and models.
 
% \begin{align}
%     \text{UA} := \frac{1}{N}\sum_{i=1}^N \mathbf{1}(f_\theta(\mathbf{x}_i) = y_i )
% \end{align}
Our goal is to learn a model $f_\theta(\cdot): \mathcal{X} \rightarrow \mathcal{Y}$ that is robust to group distribution shifts.
%  To evaluate the group robustness, let first define the group-wise accuracy (GA).
To measure the group robustness, we typically refer to the robust accuracy such as unbiased accuracy (UA) and worst-group accuracy (WA).
The definitions of UA and WA require the group-wise accuracy (GA), which is formally given by
%
\begin{align}
    \text{GA}_{(r)} := \frac{\sum_{i} \mathds{1}(f_\theta(\mathbf{x}_i) = y_i, g_i=r)}{\sum_{i} \mathds{1}(g_i=r)},
\end{align}
%
where $\mathds{1}(\cdot)$ denotes an indicator function and $\text{GA}_{(r)}$ is the accuracy of the $r^\text{th}$ group samples.
Then, the robust accuracies are defined by
%
\begin{align}
    \text{UA} := \frac{1}{|\mathcal{G}|}\sum_{r\in\mathcal{G}}{\text{GA}_{(r)}}~~~\text{and}~~~ \text{WA} := \min_{r\in\mathcal{G}} \text{GA}_{(r)}.
\end{align}
%
The goal of the group robust optimization is to ensure robust performance in terms of UA or WA regardless of the group membership of a sample.


\begin{figure}[t]
\centering
    \begin{subfigure}[m]{0.85\linewidth}
    	\includegraphics[width=\linewidth]{figures/worst_curve.png}\vspace{-1mm}
%    	\includegraphics[width=\linewidth]{figures/worst_curve_green.png}
	\subcaption{Worst-group accuracy}
	\label{fig:tsne_noise}
	    \vspace{2mm}
    \end{subfigure}
%    	\hspace{0.5cm}
        \begin{subfigure}[m]{0.85\linewidth}
    	\includegraphics[width=\linewidth]{figures/unbias_curve.png}\vspace{-1mm}
%    	\includegraphics[width=\linewidth]{figures/unbias_curve_green.png}
	\subcaption{Unbiased accuracy}
	\label{fig:tsne_erm}
	\end{subfigure} 
%    \vspace{-2mm}
    \caption{The relation between the robust and average accuracies obtained by varying the class-specific scaling factor $\mathbf{s}$ with ERM on CelebA.
    The black marker denotes the original point, where the uniform scaling is applied. 
%    \bhr{The points, located within the green area, on the trade-off curve represent the Pareto optima.}
%    \vspace{-0.1cm}
    }
    \label{fig:observation_curve}
\end{figure}
%\vspace{0.5cm}

 \subsection{Class-Specific Robust Scaling}
 \label{sec:robust_scaling}
 
As illustrated in Figure~\ref{fig:observation_tradeoff}, all algorithms exhibit a clear trade-off between robust accuracy and average accuracy. 
To analyze this behavior more closely, we propose a simple non-uniform scaling method for adjusting the scores associated with individual classes. 
This approach can influence the final decision of the classifier: by upweighting the prediction scores for minority classes, a sample may be classified into a minority class even if its initial score is low. 
Consequently, this adjustment can enhance the worst-group accuracy at the cost of a slight reduction in average accuracy, yielding a favorable trade-off for achieving group robustness.
Formally, the prediction with the class-specific scaling is given by
%
  \begin{align}
     \argmax_c~(\mathbf{s}\odot \hat{\mathbf{y}})_c,
     \label{eq:rs}
 \end{align}
%
where $\mathbf{\hat{y}} \in \mathbb{R}^C$ is a prediction score vector over $C$ classes, $\mathbf{s} \in \mathbb{R}^C$ is a $C$-dimensional scaling coefficient vector, and $\odot$ denotes the element-wise product operator.

Based on the ERM model, we obtain a set of the average and robust accuracy pairs using a wide range of the class-specific scaling factors and illustrate their relations in Figure~\ref{fig:observation_curve}.
The black markers indicate the point with a uniform scaling, \ie, $\mathbf{s} = (1, \dots, 1) \in \mathbb{R}^C$.
The graphs show that a simple class-specific scaling effectively represents the landscape of the trade-off between the two accuracies.
This validates the ability to identify the desired Pareto optimal points between the robust and average accuracies in the test set by following two simple steps: 1) finding the optimal class-specific scaling factors that maximize the target objective (UA, WA, or AA) in the validation set, and 2) apply the scaling factors to the test set\footnote{Refer to our supplementary document for the coherency of robust scaling in the validation and test sets.}.
We refer to this scaling strategy for robust prediction as \textit{robust scaling}.
 
To identify the optimal scaling factor $\mathbf{s}$, we perform a greedy search, where we first identify the best scaling factor for a class and then determine the optimal factors of the remaining ones sequentially conditioned on the previously estimated scaling factors.
The greedy search is sufficient for finding good scaling factors partly because there are many different near-optimal solutions.
Thanks to the simplicity of the process, the entire procedure takes negligible time even in large-scale datasets with multiple classes.
%
It is worth noting that, as a post-processing method, robust scaling can be seamlessly applied to existing robust optimization techniques without requiring additional training. 
Our approach enables the identification of any desired performance point on the trade-off envelope using a pretrained model.
For example, even when dealing with multiple tasks, our robust scaling approach is flexible enough to handle the situation; we only need to apply a scaling factor optimized for each target objective, leaving the trained model unchanged.
Meanwhile, existing robust optimization methods have limited flexibility and require to training separate models for each target objective.



%%%%%%%%% CELEBA TABLE  %%%%%%%%%
 % !TEX root = ../main.tex
 


 
 
 
 \iffalse
\begin{table*}[t]
\begin{center}
\caption{Experimental results of the robust scaling (RS) and instance-wise robust scaling (IRS) on the CelebA dataset using ResNet-18 with the average of three runs (standard deviations in parentheses), where RS and IRS are applied to maximize each target metric independently.
\textit{Gain} indicates the average and standard deviation of performance improvement.
%CR and GR denote class and group reweighting methods, respectively.
%Group supervision indicates that the method requires training examples with group supervision.
`$\ast$' indicates that the backbone method requires group supervision for training examples.
When combined with existing debiasing approaches, RS maximizes all target metrics consistently and IRS further boosts the performance.
%On top of all existing approaches, RS can maximize the robust accuracy at the expense of average accuracy by finding the optimal scaling factor.
}
%\vspace{-2mm}
\label{tab:celebA}
 \scalebox{0.8}{
 \hspace{-0.3cm}
\setlength\tabcolsep{8pt} 
\begin{tabular}{l|cc|cccccc}
\toprule
%& Group  & \multicolumn{2}{c|}{Robust Coverage}  & \multicolumn{6}{c}{Accuracy (\%)}\\
%Method & Supervision & Worst-group & Unbiased &  Worst-group & (Gain) & Unbiased & (Gain) &  Average  & (Gain) \\
&   \multicolumn{2}{c|}{Robust Coverage}  & \multicolumn{6}{c}{Accuracy (\%)}\\
Method  & Worst-group & Unbiased &  Worst-group & (Gain) & Unbiased & (Gain) &  Average  & (Gain) \\
\hline
ERM & -& -& 34.5 (6.1) & - &77.7 (1.8) &- &{95.5 (0.4)} & - \\
ERM + RS & {83.0 (0.8)} & {88.1 (0.6)} &  {82.8 (3.3)} & +47.7 (7.8)&  {91.2 (0.5)} &+13.3 (2.0) & \textbf{95.8 (0.2)}& +0.4 (0.2) \\
ERM + IRS    &\textbf{83.4 (0.1)} & \textbf{88.4 (0.4)} & \textbf{87.2 (2.0)} & +52.7 (3.3) & \textbf{91.7 (0.2)} &+13.8 (1.6) &\textbf{95.8 (0.1)} & +0.4 (0.3)\\
\hline
CR & - & -&  70.6 (6.0) & & 88.7 (1.2) & & {94.2 (0.7)}  & \\
CR + RS &{82.9 (0.5)}  & {88.2 (0.3)}&  {82.7 (5.2)} &+12.2 (7.5)& {91.0 (1.0)} &+2.2 (1.3) & {95.4 (0.5)} & +1.3 (0.4)\\
CR + IRS & \textbf{83.6 (1.1)} &\textbf{88.6 (0.5)} &\textbf{84.8 (1.5)} & +14.2 (5.2) & \textbf{91.3 (0.4)} & +2.5 (1.4) &\textbf{95.5 (0.1)} & +1.3 (0.3)\\
\hline
%~\cite{idrissi2022simple}
SUBY & - & -&65.7 (3.9) &- &87.5 (0.9) &- &{94.5 (0.7)} &- \\
SUBY + RS & 81.5 (1.0) & 87.4 (0.1) & {80.8 (2.9)} &+15.1 (3.0) & {90.5 (0.8)}  &+3.0 (0.9)  & {95.3 (0.6)} & +0.8 (0.6)\\
SUBY + IRS & \textbf{82.3 (1.1)} &\textbf{87.8 (0.2)} &\textbf{82.3 (2.0)} & +16.5 (4.1) & \textbf{90.8 (0.8)} & +3.3 (1.1) & \textbf{95.5 (0.3)}  & +1.1 (0.4) \\
\hline
%~\cite{LfF} 
LfF &-&- &  55.6 (6.6) &- & 81.5 (2.8) &- & {92.4 (0.8)} &- \\
LfF + RS &{74.1 (3.5)}  &{79.7 (2.6)} & {78.7 (4.1)} &+23.2 (2.5) & {85.4 (2.4)} &+4.0 (0.8) & \textbf{93.4 (0.7)} & +1.0 (0.2) \\
LfF + IRS &\textbf{74.6 (4.1)}  &\textbf{79.8 (3.1)} & \textbf{78.9 (5.3)} &+23.4 (4.1) & \textbf{86.0 (2.2)} &+4.6 (1.5) & {93.1 (1.5)} & +0.7 (0.5) \\
\hline
%~\cite{JTT}
JTT  & - & -& 75.1 (3.6) &- &85.9 (1.4) &- &{89.8 (0.8)} &- \\
JTT + RS  &{77.3 (0.7)} &{81.9 (0.7)}  & {82.9 (2.3)} &+7.8 (3.0)  &{87.6 (0.5)} &+1.7 (0.4)  &{90.3 (1.3)} & +0.6 (0.1) \\
JTT + IRS  &\textbf{78.9 (2.1)} &\textbf{82.1 (1.5)}  & \textbf{84.9 (4.5)} &+9.8 (3.7)  &\textbf{88.5 (0.8)} &+2.5 (0.8)  &\textbf{91.0 (1.8)} & +1.2 (0.5) \\
\hline
%& \multirow{3}{*}{\checkmark} 
GR$^\ast$  & - & -  & 88.6 (1.9) &- &92.0 (0.4) &- &{92.9 (0.8)} &- \\
GR$^\ast$ + RS &{86.9 (0.4)} &{88.4 (0.2)} & {90.0 (1.6)} &+1.4 (1.1) & {92.4 (0.5)} &+0.5 (0.4) & {93.8 (0.4)} & +0.8 (0.5)\\
GR$^\ast$ + IRS & \textbf{87.0 (0.2)} & \textbf{88.6 (0.2)} & \textbf{90.0 (2.3)} & +1.4 (1.8) & \textbf{92.6 (0.6)} & +0.6 (0.4) & \textbf{94.2 (0.3)} & +1.3 (1.0) \\
\hline
%~\cite{idrissi2022simple} 
SUBG$^\ast$ & - & -  - &87.8 (1.2) &- &90.4 (1.2) &- &{91.9 (0.3)} &- \\ 
SUBG$^\ast$ + RS & 83.6 (1.6) & 87.5 (0.7) & {88.3 (0.7)} &+0.5 (0.4) & {90.9 (0.5)} & +0.5 (0.5)  & {93.9 (0.2)} & +1.9 (0.6)\\
SUBG$^\ast$ + IRS & \textbf{84.5 (0.8)} &\textbf{87.9 (0.1)} &\textbf{88.7 (0.6)} & +0.8 (0.7) & \textbf{91.0 (0.3)} & +0.6 (0.9) & \textbf{94.0 (0.2)} & +2.1 (1.0) \\
\hline
%~\cite{GroupDRO}
Group DRO$^\ast$ & -& - &  88.4 (2.3) &- & 92.0 (0.4) &- & {93.2 (0.8)} &-  \\
Group DRO$^\ast$ + RS &{87.3 (0.2)} & {88.3 (0.2)}&  {89.7 (1.2)} &+1.4 (1.0) & {92.3 (0.1)} &+0.4 (0.2) & {93.9 (0.3)} & +0.7 (0.5) \\
Group DRO$^\ast$ + IRS & \textbf{87.5 (0.4)} & \textbf{88.4 (0.2)}& \textbf{90.0 (2.3)} & +2.6 (1.8) & \textbf{92.6 (0.6)} & +0.6 (0.4) & \textbf{94.7 (0.3)} &+1.5 (1.1) \\
\bottomrule
\end{tabular}
 }
%\vspace{-0.2cm}
\end{center}
\end{table*}

\fi



\begin{table*}[t]
\begin{center}
\caption{Experimental results of the robust scaling (RS) and instance-wise robust scaling (IRS) on the CelebA dataset using ResNet-18 with the average of three runs (standard deviations in parenthesis), where RS and IRS are applied to maximize each target metric independently.
\textit{Gain} indicates the average (standard deviations) of performance improvement for each run.
%CR and GR denote class and group reweighting methods, respectively.
%Group supervision indicates that the method requires training examples with group supervision.
On top of all existing approaches, RS can maximize all target metrics consistently and IRS further boosts the performance.
%On top of all existing approaches, RS can maximize the robust accuracy at the expense of average accuracy by finding the optimal scaling factor.
}
%\vspace{0.2cm}
\label{tab:celebA}
 \scalebox{0.8}{
 \hspace{-0.3cm}
\setlength\tabcolsep{6pt} 
\begin{tabular}{lc|cc|cccccc}
\toprule
& Group  & \multicolumn{2}{c|}{Robust Coverage}  & \multicolumn{6}{c}{Accuracy (\%)}\\
Method & Supervision & Worst-group & Unbiased &  Worst-group & (Gain) & Unbiased & (Gain) &  Average  & (Gain) \\
\hline
ERM & & -& -& 34.5 (6.1) & - &77.7 (1.8) &- &{95.5 (0.4)} & - \\
ERM + RS & & {83.0 (0.8)} & {88.1 (0.6)} &  {82.8 (3.3)} & +47.7 (7.8)&  {91.2 (0.5)} &+13.3 (2.0) & \textbf{95.8 (0.2)}& +0.4 (0.2) \\
ERM + IRS  &   &\textbf{83.4 (0.1)} & \textbf{88.4 (0.4)} & \textbf{87.2 (2.0)} & +52.7 (3.3) & \textbf{91.7 (0.2)} &+13.8 (1.6) &\textbf{95.8 (0.1)} & +0.4 (0.3)\\
\hline
CR & & - & -&  70.6 (6.0) & & 88.7 (1.2) & & {94.2 (0.7)}  & \\
CR + RS & &{82.9 (0.5)}  & {88.2 (0.3)}&  {82.7 (5.2)} &+12.2 (7.5)& {91.0 (1.0)} &+2.2 (1.3) & {95.4 (0.5)} & +1.3 (0.4)\\
CR + IRS & & \textbf{83.6 (1.1)} &\textbf{88.6 (0.5)} &\textbf{84.8 (1.5)} & +14.2 (5.2) & \textbf{91.3 (0.4)} & +2.5 (1.4) &\textbf{95.5 (0.1)} & +1.3 (0.3)\\
\hline
SUBY~\cite{idrissi2022simple} & & - & -&65.7 (3.9) &- &87.5 (0.9) &- &{94.5 (0.7)} &- \\
SUBY + RS & & 81.5 (1.0) & 87.4 (0.1) & {80.8 (2.9)} &+15.1 (3.0) & {90.5 (0.8)}  &+3.0 (0.9)  & {95.3 (0.6)} & +0.8 (0.6)\\
SUBY + IRS & & \textbf{82.3 (1.1)} &\textbf{87.8 (0.2)} &\textbf{82.3 (2.0)} & +16.5 (4.1) & \textbf{90.8 (0.8)} & +3.3 (1.1) & \textbf{95.5 (0.3)}  & +1.1 (0.4) \\
\hline
LfF~\cite{LfF} & &-&- &  55.6 (6.6) &- & 81.5 (2.8) &- & {92.4 (0.8)} &- \\
LfF + RS & &{74.1 (3.5)}  &{79.7 (2.6)} & {78.7 (4.1)} &+23.2 (2.5) & {85.4 (2.4)} &+4.0 (0.8) & \textbf{93.4 (0.7)} & +1.0 (0.2) \\
LfF + IRS & &\textbf{74.6 (4.1)}  &\textbf{79.8 (3.1)} & \textbf{78.9 (5.3)} &+23.4 (4.1) & \textbf{86.0 (2.2)} &+4.6 (1.5) & {93.1 (1.5)} & +0.7 (0.5) \\
\hline
JTT~\cite{JTT}  & & - & -& 75.1 (3.6) &- &85.9 (1.4) &- &{89.8 (0.8)} &- \\
JTT + RS  & &{77.3 (0.7)} &{81.9 (0.7)}  & {82.9 (2.3)} &+7.8 (3.0)  &{87.6 (0.5)} &+1.7 (0.4)  &{90.3 (1.3)} & +0.6 (0.1) \\
JTT + IRS  & &\textbf{78.9 (2.1)} &\textbf{82.1 (1.5)}  & \textbf{84.9 (4.5)} &+9.8 (3.7)  &\textbf{88.5 (0.8)} &+2.5 (0.8)  &\textbf{91.0 (1.8)} & +1.2 (0.5) \\
\hline
 GR  & \multirow{3}{*}{\checkmark} & - & -  & 88.6 (1.9) &- &92.0 (0.4) &- &{92.9 (0.8)} &- \\
GR + RS & &{86.9 (0.4)} &{88.4 (0.2)} & {90.0 (1.6)} &+1.4 (1.1) & {92.4 (0.5)} &+0.5 (0.4) & {93.8 (0.4)} & +0.8 (0.5)\\
GR + IRS & & \textbf{87.0 (0.2)} & \textbf{88.6 (0.2)} & \textbf{90.0 (2.3)} & +1.4 (1.8) & \textbf{92.6 (0.6)} & +0.6 (0.4) & \textbf{94.2 (0.3)} & +1.3 (1.0) \\
\hline
SUBG~\cite{idrissi2022simple}  & \multirow{3}{*}{\checkmark} & - & -  - &87.8 (1.2) &- &90.4 (1.2) &- &{91.9 (0.3)} &- \\ 
SUBG + RS & & 83.6 (1.6) & 87.5 (0.7) & {88.3 (0.7)} &+0.5 (0.4) & {90.9 (0.5)} & +0.5 (0.5)  & {93.9 (0.2)} & +1.9 (0.6)\\
SUBG + IRS & & \textbf{84.5 (0.8)} &\textbf{87.9 (0.1)} &\textbf{88.7 (0.6)} & +0.8 (0.7) & \textbf{91.0 (0.3)} & +0.6 (0.9) & \textbf{94.0 (0.2)} & +2.1 (1.0) \\
\hline
Group DRO~\cite{GroupDRO} &\multirow{3}{*}{\checkmark}  & -& - &  88.4 (2.3) &- & 92.0 (0.4) &- & {93.2 (0.8)} &-  \\
Group DRO + RS & &{87.3 (0.2)} & {88.3 (0.2)}&  {89.7 (1.2)} &+1.4 (1.0) & {92.3 (0.1)} &+0.4 (0.2) & {93.9 (0.3)} & +0.7 (0.5) \\
Group DRO + IRS & & \textbf{87.5 (0.4)} & \textbf{88.4 (0.2)}& \textbf{90.0 (2.3)} & +2.6 (1.8) & \textbf{92.6 (0.6)} & +0.6 (0.4) & \textbf{94.7 (0.3)} &+1.5 (1.1) \\
\bottomrule
\end{tabular}
 }
 \vspace{-1mm}
\end{center}
\end{table*}


 
 
%%%%%%%%%%%%%%%%%%%%%%%%%%

% \subsection{Group Robust Methods with Adaptive Robust Scaling}
% \subsection{Instance-wise Robust Scaling for Improving Group Robustness}
 \subsection{Instance-wise Robust Scaling}
 \label{sec:method}
%If we partition the dataset based on feature semantics and reserve different scaling factors for each partition, then it can give more flexible trade-offs. 
%If we can apply instance-wise scaling factor to each test example adaptively, then it will have the potential to improve the trade-off furthermore.  
The optimal scaling factor can be adaptively applied to each test example, enabling instance-specific scaling to potentially overcome the trade-off and further improve accuracy. 
Previous approaches~\cite{seo2021unsupervised, sohoni2020no} have demonstrated the ability to identify hidden spurious attributes by clustering in the feature space for debiased representation learning. 
Similarly, we take advantage of feature clustering for adaptive robust scaling; we obtain the optimal class-specific scaling factors based on the cluster membership of each sample.
The overall algorithm of our instance-wise robust scaling (IRS) is outlined as follows.
%
%\vspace{2mm}
\begin{enumerate}
  \item Perform clustering with the validation dataset on the feature space and store the cluster centroids. \vspace{-1mm}
   \item Find the optimal scaling factor for each cluster. \vspace{-1mm}
%  \item Assign each test example to the nearest cluster using the stored cluster centroids in step 1.
%  \textbf{4)} Apply the optimal scaling factors obtained in 2) to the samples of each cluster in the test split.
  \item Apply the estimated scaling factor to each test example based on its cluster membership.
\end{enumerate}
\vspace{2mm}
%
In step 1, we use a simple \textit{K}-means clustering algorithm.
Empirically, when $K$ is sufficiently large, \ie, $K > 10$, IRS achieves stable and superior results, compared to the original class-specific scaling.







% \subsection{Robust Coverage for Comprehensive Performance Evaluation}
 \subsection{Robust Coverage}
 \label{sec:robust_coverage}
Although robust scaling identifies a desired performance point on the trade-off curve, it captures only a single point, overlooking the other Pareto-optimal solutions. 
To enable a more comprehensive evaluation of an algorithm, we propose a convenient scalar measure that summarizes the robust-average accuracy trade-off.
Formally, we define the \textit{robust coverage} as
%
\begin{align}
     \text{(Robust coverage)} &:=  \int_{c=0}^1\max_\mathbf{s}\big\{\text{RA}^\mathbf{s}|\text{AA}^\mathbf{s}\geq c\big\}dc 
      \nonumber \\
     & \hspace{-5mm}\approx \sum_{d=0}^{D-1}\frac{1}{D}\max_\mathbf{s}\big\{\text{RA}^\mathbf{s}|\text{AA}^\mathbf{s}\geq \frac{d}{D}\big\},
     \label{eq:coverage}
\end{align}
%
where $\text{RA}^\mathbf{s}$ and $\text{AA}^\mathbf{s}$ denote the robust and average accuracies, respectively, and $D = 10^3$ is the number of slices used for discretization.
The robust coverage measures the area under the Pareto frontier of the robust-average accuracy trade-off curve, where the maximum operation in~(\ref{eq:coverage}) identifies the Pareto optimum for each threshold.
Depending on the target objective of robust coverage in~(\ref{eq:rs}), we use either WA or UA as the measure of RA.
%Please refer to Appendix~\ref{sec:coverage_clarification} for more discussions.
 

 
 
 
 
 
 

 
 