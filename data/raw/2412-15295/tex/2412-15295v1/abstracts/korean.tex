\chapter*{국문초록}
\addcontentsline{toc}{chapter}{국문초록}

클러스터링은 머신러닝에서 핵심적인 과제로, $k$-means는 단순성과 효율성 덕분에 널리 사용되는 알고리즘이다. 1차원(1D) 클러스터링은 많은 실제 응용에서 발생하지만, 기존 $k$-means 구현체들은 1D 데이터의 구조를 효과적으로 활용하지 못해 비효율이 존재한다. 본 논문에서는 정렬된 데이터, 누적합 배열, 이진탐색의 특성을 활용하여 1D 클러스터링에 최적화된 $k$-means++ 초기화 및 Lloyd 알고리즘을 제안한다.

본 논문은 다음과 같은 로그 시간 알고리즘을 제시한다: (1) \(k\)-cluster 알고리즘은 greedy $k$-means++ 초기화에서 \(O(l \cdot k^2 \cdot \log n)\) 시간복잡도, Lloyd 알고리즘에서 \(O(i \cdot k \cdot \log n)\) 시간복잡도를 달성한다. 여기서 \(n\)은 데이터셋 크기, \(k\)는 클러스터 개수, \(l\)은 greedy $k$-means++ local trials 수, \(i\)는 Lloyd 알고리즘 반복 횟수를 나타낸다. (2) 2-cluster 알고리즘은 이진탐색을 활용하여 \(O(\log n)\) 시간복잡도로 작동하며, 반복 없이 Lloyd 알고리즘의 국소 최적해에 빠르게 수렴한다.

벤치마크 결과, 제시된 알고리즘은 대규모 데이터셋에서 \texttt{scikit-learn} 대비 4500배 이상의 속도 향상을 달성하면서도 within-cluster sum of squares (WCSS) 품질을 유지한다. 또한, 대규모 언어 모델(LLMs) 양자화와 같은 최신 응용에서도 300배 이상의 속도 향상을 보여준다.

본 연구는 1D $k$-means clustering의 이론과 실제 간 간극을 좁히는 효율적이고 실용적인 알고리즘을 제안한다. 제시된 알고리즘은 JIT 컴파일을 통해 최적화된 오픈소스 Python 라이브러리로 구현되었으며, 실제 응용에 쉽게 통합할 수 있도록 설계되었다.

\vspace*{1cm}
\textbf{주요어: $k$-means 클러스터링, Lloyd 알고리즘, $k$-means++ 초기화, 일차원 클러스터링, 이진탐색, 누적합}
