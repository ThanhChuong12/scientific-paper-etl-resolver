\section{Conclusion and Future Directions}
\vspace{-0.1in}
In this work, we present a novel framework for election prediction using large language models (LLMs) with a focus on multi-step reasoning. 
By leveraging both synthetic personas and real-world datasets, we demonstrated the potential of LLMs to capture individual voting behaviors and state-level election outcomes. 
Our iterative design highlights the importance of integrating temporal information and complex reasoning for accurate predictions. 
The 2020 and 2024 simulations reveal both the strengths and limitations of using LLMs in dynamic political environments. These results highlight the model’s ability to generalize on unseen data, while also underscoring several challenges, including: (1) evaluating and addressing biases embedded within the LLM’s pretrained corpus, (2) mitigating stereotypical and overly simplistic representations in persona simulations, and (3) overcoming static demographic assumptions and the lack of real-time data integration.

Future research can extend this work by incorporating multiple LLMs to better understand their internal political tendencies, enhancing temporal modeling with public opinion data and real-time trends for improved accuracy, and developing stronger multi-step reasoning pipelines through refined Chain of Thought (CoT) designs to further enhance prediction performance and mitigate biases. This study lays the foundation for future applications of LLMs in political forecasting, offering promising directions for further development in both election prediction and the broader study of LLM behavior in social contexts.
\vspace{-0.1in}

\section*{Broader Impact Statement}
\vspace{-0.5ex}
This work explores the application of LLMs to improve election forecasting through enhanced reasoning and data synthesis, which can inform policymakers, researchers, and journalists, aiding them in understanding voter behavior and electoral outcomes. 
By offering a more transparent and adaptable approach to prediction, this research may help demystify complex political processes, reduce reliance on narrow historical data, and guide strategic resource allocation for stakeholders.


\clearpage
\newpage

\section*{Ethics Statement}
\vspace{-0.5ex}
This work uses only public or synthetic data with no personally identifiable information. Consequently, an institutional review board determined the study to be exempt from further review. 
While our framework aims to enhance election forecasting accuracy, no model is entirely free of bias. We encourage stakeholders to interpret results carefully, consider domain expertise, and remain vigilant in identifying and mitigating potential biases.
Also note that we used ChatGPT exclusively to improve minor grammar in the final manuscript.


\section*{Limitations}

\textit{First}, our approach to time-dependent modeling does not account for dynamic factors such as shifts in public opinion, evolving media narratives, or unforeseen events—such as economic disruptions or political scandals—that could significantly influence voter behavior. 
Incorporating these variables would require real-time data integration, which is beyond the scope of this study.

\textit{Second}, while our analysis reveals potential political biases within the LLM’s pretrained corpus, we have not quantitatively measured or compared the intrinsic political tendencies of different LLMs. 
Without a systematic framework to quantify and contrast these inclinations, it remains challenging to understand how model-specific political leanings might influence predictions and outcomes.

It is key to emphasize that this study does not seek to provide definitive election forecasts. 
Rather, it serves as an exploratory effort to demonstrate the potential of LLMs to generalize to unseen data. By evaluating how closely these models align with real-world patterns, this work offers a proof of concept for their predictive capacity while acknowledging the limitations inherent in the current framework. 

