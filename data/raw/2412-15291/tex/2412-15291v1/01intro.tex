\section{Introduction}

Large Language Models (LLMs) have shown strong capabilities across diverse domains, including drug discovery \cite{liu2024drugagent}, legal analysis \cite{chalkidis2022lexglue}, and creative tasks \cite{yang2022survey}. Their strength lies in not only understanding and generating text but also leveraging vast common knowledge \cite{roberts2020much}, simulating diverse personas \cite{hu-collier-2024-quantifying}, and modeling human behavior in complex social contexts \cite{bommasani2021opportunities}. LLMs excel in human-like reasoning \cite{zhou2020evaluating, alkhamissi2022review} and have been applied to simulate decision-making in various settings \cite{zhou2023large, ziems2024can}.

Notably, recent work has extended LLMs to political science, analyzing policy documents, campaign speeches, and public sentiment \cite{xu2022deep, haq2023large, li2024political}. Beyond language processing, their ability to integrate knowledge and apply human-like reasoning positions them as powerful tools for simulating the intricate dynamics of political decision-making \cite{Argyle_Busby_Fulda_Gubler_Rytting_Wingate_2023, Bisbee_Clinton_Dorff_Kenkel_Larson_2024}.



\begin{figure*}[!ht]
    \centering
    \includegraphics[width=\textwidth]{figs/Intro_hook2.pdf}  
    \vspace{-0.3in}
    \caption{Demonstration of three prompt designs in \S \ref{subsec:overview}. V1 is the direct prompt on voter demographic information, while V2 introduces time-dependent information to capture candidates' agenda and V3 also uses multi-step reasoning.
    In this example for 2020 Ohio result prediction, only V3 can accurately predict the results, demonstrating the importance of leveraging both time-dependent information and multi-step reasoning for election result prediction.}
    \vspace{-0.2in}
    \label{fig:comparision}
\end{figure*}


\noindent \textbf{Background}.
Despite LLMs' achievements in more straightforward political science tasks, their capacity to handle more challenging efforts, such as election prediction, is not well understood \cite{lerer2022political}.
% On the one hand, 
% LLMs can access large volumes of historical information and have performed well in some predictive tasks, raising the possibility that they could forecast election outcomes.
% % On the other hand, 
% However, election forecasting poses distinct challenges. 
LLMs’ access to extensive historical data suggests potential for forecasting, but election prediction introduces unique challenges. 
\textit{First}, 
the high cost of acquiring voter-level data complicates both experimentation and model validation. 
\textit{Second}, election forecasting involves modeling not only voter behavior, but also the shifting political context, making it unclear whether text-based data alone can capture the full complexity \cite{graefe2014accuracy}.
\textit{Third}, accurate predictions require reasoning about multiple factors\cite{holbrook2016forecasting}. Whether LLMs can perform this more advanced reasoning remains unknown.
\cite{wei2022chain}.



\np{This Work}  
We present \textbf{a large-scale empirical study} to address two key questions in election prediction with LLMs:
(\textit{\textbf{i}}) how can LLMs be leveraged to achieve \textit{accurate} election predictions, and (\textbf{\textit{ii}}) beyond accuracy, what \textit{additional considerations} are crucial for understanding LLM capabilities in election forecasting tasks?


\noindent
\textbf{Contribution 1: Multi-Step Reasoning Framework for Accurate Election Prediction (\S \ref{sec:pipeline})}.
To handle the scarcity of detailed voter-level data, we employ the \textit{Sync} synthetic data generation framework \cite{li2020sync}, which probabilistically reconstructs individual demographic and behavioral profiles from aggregated public datasets. 
We pair this synthetic data with real-world datasets, notably the American National Election Studies (ANES) Time Series \cite{ANES2020}, ensuring that our model aligns with actual voting behaviors.
Our approach also adapts to evolving political conditions by incorporating time-dependent factors. We aggregate information from presidential campaign data, including candidates' policy agendas and backgrounds, to align the model with changing political environments \cite{holbrook2016forecasting}.
Finally, we introduce a multi-step reasoning framework guided by Chain of Thought prompting \cite{wei2022chain}. 
By breaking down the prediction process into intermediate steps, the model can systematically integrate demographic details, ideological alignment, and time-sensitive information. This approach strengthens the model's accuracy and mitigates biases and overfitting issues observed in simpler models.
As shown in Figs. \ref{fig:comparision} and \ref{fig:example}, we refine our pipeline through iterations. The final version significantly improves predictive accuracy and alignment with real-world data via extensive experiments.


\np{Contribution 2: New Insights and Directions Beyond Accurate Prediction (\S \ref{Beyond})}  
Our experiments uncover several key insights that extend beyond raw predictive accuracy. 
First, while our multi-step reasoning pipeline reduces bias and improves alignment with real-world outcomes, residual skewness persists, highlighting the influence of pretrained corpora on political predictions. 
Second, although LLMs can replicate known demographic voting patterns, they tend to exaggerate these differences, raising concerns about reinforcing stereotypes. 
Finally, our results indicate that integrating richer contextual information and refining prompt strategies can further enhance predictive reliability.
We thus propose meaningful future directions on reducing political skewness in pretrained corpora and mitigating stereotype biases in election prediction.
\vspace{-0.1in}

