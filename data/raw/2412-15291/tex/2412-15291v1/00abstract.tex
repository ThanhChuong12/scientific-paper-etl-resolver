
Can Large Language Models (LLMs) accurately predict election outcomes? 
While LLMs have demonstrated impressive performance in healthcare, legal analysis, and creative applications, their capabilities in election forecasting remain uncertain. 
Notably, election prediction poses unique challenges: limited voter-level data, evolving political contexts, and the complexity of modeling human behavior.
In the \textit{first} part of this paper, we explore and introduce a multi-step reasoning framework for election prediction, which systematically integrates demographic, ideological, and time-sensitive factors. 
Validated on 2016 and 2020 real-world data and extensive synthetic personas, our approach adapts to changing political landscapes, reducing bias and significantly improving predictive accuracy. 
We further apply our pipeline to the 2024 U.S. presidential election, illustrating its ability to generalize beyond observed historical data.
Beyond enhancing accuracy, the \textit{second} part of the paper provides insights into the broader implications of LLM-based election forecasting. 
We identify potential political biases embedded in pretrained corpora, examine how demographic patterns can become exaggerated, 
and suggest strategies for mitigating these issues. 
Together, this project, a large-scale LLM empirical study, advances the accuracy of election predictions and establishes directions for more balanced, transparent, and context-aware modeling in political science research and practice.






