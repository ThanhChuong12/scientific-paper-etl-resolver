\section{Related Work}

\vspace{-2mm}\subsection{LLMs in Political Science}\vspace{-1mm}
Recent studies have explored LLMs in policy analysis, election forecasting, and public opinion simulation \cite{smith2023llms, johnson2024political}, yet capturing the complexity of political behaviors remains a challenge \cite{brown2023challenges}. Accurate political modeling often requires domain-specific fine-tuning \cite{chen2024decoding}. While prior frameworks highlighted LLMs' strengths in generative and predictive tasks, their application to voter behavior modeling has been limited. In contrast, our work introduces a large-scale simulation framework that integrates social science theories to better model political decision-making.

\vspace{-2mm}\subsection{Simulating Human Decision-Making}\vspace{-1mm}
LLM-based simulations of human behavior draw insights from public opinion, economics, and social psychology. \citet{chuang2024simulating} modeled opinion dynamics and polarization, whereas \citet{ross2024llm} applied utility theory to capture economic decision-making patterns. \citet{zhang-etal-2024-exploring} examined LLMs' capability to simulate collaboration and conformity, though \citet{chang2024llmsgeneratestructurallyrealistic} noted that these models tend to overestimate political homophily in social networks. Our approach builds on these findings by integrating established political science theories, such as ideological sorting and partisanship, to enhance the realism of voter behavior simulations. For further details, please see Appx.~\ref{appx:related}.



% LLMs' applications in election forecasting is still in its early stages. 
% For instance, the ``Political Campus'' project by \citet{roberts2023political} created a benchmark dataset for evaluating LLM performance on election-related tasks, highlighting both potential and limitations in political forecasting. Similarly, the work by \citet{kim2024sentiment} using LLMs for policy sentiment analysis underscores the need for careful interpretation of results due to the challenges associated with the inherent complexity of political language. 
 

