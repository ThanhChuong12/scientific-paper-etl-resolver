
%Accurately simulating political decision-making behavior is essential for understanding key societal phenomena, such as voter preferences, ideological shifts, and partisan alignment. 

While LLMs have demonstrated remarkable capabilities in text generation and reasoning, their ability to simulate human decision-making---particularly in political contexts---remains an open question. However, modeling voter behavior presents unique challenges due to limited voter-level data, evolving political landscapes, and the complexity of human reasoning. In this study, we develop a theory-driven, multi-step reasoning framework that integrates demographic, temporal and ideological factors to simulate voter decision-making at scale. 
Using synthetic personas calibrated to real-world voter data, we conduct large-scale simulations of recent U.S. presidential elections. Our method significantly improves simulation accuracy while mitigating model biases. We examine its robustness by comparing performance across different LLMs.
We further investigate the challenges and constraints that arise from LLM-based political simulations. 
Our work provides both a scalable framework for modeling political decision-making behavior and insights into the promise and limitations of using LLMs in political science research.

% Can Large Language Models (LLMs) accurately predict election outcomes? 
% While LLMs have demonstrated impressive performance in healthcare, legal analysis, and creative applications, their capabilities in election forecasting remain uncertain. 
% Election prediction presents unique challenges: limited voter-level data, evolving political landscapes, and the complexity of modeling human behavior.
% In the \textit{first} part of this paper, we introduce a multi-step reasoning framework for election prediction, which systematically integrates demographic, ideological, and temporal factors. 
% With extensive synthetic personas, our approach adapts to shifting political landscapes, mitigates prediction bias and significantly improves predictive accuracy. We validate our model performance on real-world data from the 2016 and 2020 U.S. presidential elections.
% We further apply our framework to the 2024 U.S. presidential election, illustrating its ability to generalize beyond observed historical data.
% Beyond improving accuracy, the \textit{second} part of the paper examines the broader implications of LLM-based election forecasting. 
% We identify potential political biases embedded in pretrained corpora, examine how demographic patterns can be exaggerated, 
% and propose strategies for mitigating these issues. 
% Together, our work advances the accuracy of LLM-based election predictions and provides a foundation for more transparent, balanced, and context-aware modeling in political science research and practice.






