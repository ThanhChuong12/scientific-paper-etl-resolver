\section{Discussion}\label{sec:discussion}
IVUS provides low-resolution grayscale images with vague borders, making diagnosis particularly challenging. For junior doctors, identifying plaque regions based solely on professional experience and knowledge can be difficult. Additionally, data collection and analysis across multi-hospital systems face significant regulatory and privacy challenges, necessitating the development of algorithms with decentralized architectures.

This study introduces a two-stage deep-learning method for automatic plaque segmentation on IVUS images within a federated learning framework. The segmentation process involves converting images from Cartesian to polar coordinates, achieving Dice similarity coefficients of approximately 0.890 for EEM, 0.877 for lumen, and 0.706 for plaque burden areas. Preprocessing includes coordinate conversions, and segmentation of the lumen and EEM areas is performed using parallel U-Net models. Plaque regions are determined by subtracting the segmented locations of the lumen and EEM. This approach significantly enhances spatial information and improves image readability.

The smaller size of the converted polar images reduces the number of model parameters from approximately 933k to 721k, optimizing both parameter efficiency and input dimensions while utilizing fewer computing resources. Although post-processing slightly increases computational time due to coordinate conversion, the polar coordinate method delivers superior segmentation performance.

Additionally, this study employs a federated learning framework to segment 60-MHz IVUS images. The proposed area segmentation model improves treatment effectiveness and efficiency by alleviating the challenges of detecting circular boundaries. Plaque borders and locations within the arteries are automatically highlighted, providing valuable insights for clinical decision-making.

The limitations of this study are briefly outlined. Each local dataset contains a limited volume of data, and some IVUS images lack high resolution, with most plaques appearing unclear. The federated learning framework was tested on only three clients due to the limited number of distributed datasets. Furthermore, all clients had equal amounts and similar distributions of IVUS images, whereas real-world scenarios often involve unbalanced datasets with distinct distributions across clients. The lower resolution of 20MHz or 40MHz IVUS images further complicates patient identification across different clinics. Another significant challenge is the segmentation of images for model training, which requires a substantial number of annotated images, a considerable obstacle for federated learning frameworks.

%%%%%%%%%%%%%%%%%%
%The IVUS provides grayscale images of low resolution with vague borders. As a result, diagnosis becomes more challenging. During diagnostic procedures, it is difficult for doctors to identify plaque regions based on their professional experience and knowledge. In addition, sufficient medical images are required to apply DL techniques in the training phase. However, collecting and analyzing these data in multi-hospital systems is difficult due to several regulatory and privacy concerns. Therefore, it is essential to design algorithms using a decentralized architecture. Based on the federated learning architecture, we propose deep learning models and algorithms distinguished by many critical attributes.

%This paper presents a two-stage deep-learning method for automatically segmenting plaques on IVUS images. For segmentation, images are represented in Cartesian and polar coordinates. Regarding Dice similarity coefficients, both methods segment EEM area, lumen area, and plaque burden area around 0.8897, 0.8769, and 0.7055, respectively. The assessment of some critical quantitative indicators can provide further pathological insight. Using experimental results, measurements are in agreement with domain expert measurements.

%However, the most common IVUS imaging frequencies used in clinics are 20 MHz and 40 MHz. This paper applies a federated learning framework to segment 60-MHz IVUS images. As a result of the proposed area segmentation model, treatment efficiency will be improved by alleviating the difficulty of detecting circular boundaries. It is expected that a 60-MHz IVUS imaging system will be able to provide more accurate and detailed images soon. By using an automatic IVUS segmentation model, plaque borders and locations within arteries are highlighted. The data are preprocessed using coordinate conversions, and lumen and EEM areas are segmented using parallel U-Net models. By subtracting both locations from one another, plaques can be calculated. The model could provide spatial information and improve the readability of images.

%Furthermore, segmenting images for model training is a challenging task that requires many annotated images, which is also challenging for federated learning. The federated learning framework is implemented in this study, which allows models to be trained on distributed datasets. Then, it provides a generalizable model for all participants. Therefore, the framework is used across all medical clinics to acquire more data for the purpose of training. The models are trained using a decentralized architecture in which no centralized or exchanged medical images are used. Additionally, the process ensures that patient privacy is protected. 

%The IVUS provides low-resolution grayscale images with vague borders shown in Figure \ref{fig:RelatedWork Canny Edge Detection.}. Ultrasonography, for example, makes diagnosis more challenging because of its low resolution. Doctors identify the plaque regions on IVUS images for medical diagnosis with their naked eyes. The traditional method requires many experiences and professional knowledge, and there is a possibility of misjudgment. In addition, many medical images or videos are collected during medical imaging examinations; however, lesions can differ significantly among patients and may be irregularly shaped. It is a burdensome task to identify lesions during diagnostic procedures. Hence, it would be helpful to have a biomarker tool to annotate regions of interest in medical images faster and more precisely in this context. The application of deep learning (DL) techniques, such as CNNs, has enabled scientists to engage in a brand-new era of AI-based diagnosis, which can detect lesions at an early stage of development. But, at the training stage, sufficient data volume posed a barrier to improving performance. Data collection in multi-hospital systems could not be easily carried out because of regulatory and privacy concerns. Therefore, it is essential to design a decentralized architecture that does not allow centralized or shared medical images while simultaneously protecting patient privacy.
%Our proposed method is distinguished on a rage of key attributes as discussed below.


% It is expected that a 60-MHz IVUS imaging system will provide more accurate and detailed images. 


% In this study, a federated learning framework is also implemented, which allows models to be trained on distributed datasets. In addition, it provides a mechanism that can be generalized to all participants. This framework is used across all medical clinics to obtain more data for training purposes. Models are well-trained using a decentralized architecture, which does not rely on centralized or exchanged medical images. Furthermore, the process ensures that the privacy of the patient is protected.