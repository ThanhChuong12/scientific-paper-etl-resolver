\section{Conclusion}\label{sec:conclusion}
This study highlights practical methodologies and case studies on the IVUS segmentation system and federated learning framework, demonstrating significant real-world medical applications that are adaptable and scalable within decentralized AI systems: 1) Cardiovascular Disease Diagnosis and Treatment: The system accurately identifies and segments arterial structures, including the external elastic membrane, lumen, and plaque borders. This aids in diagnosing atherosclerosis and assessing plaque stability, which are essential for preventing heart attacks and strokes. 2) Surgical Planning: Precise segmentation provides interventional cardiologists with accurate spatial and volumetric data, supporting procedures such as stent placement and angioplasty. 3) Cross-Hospital Collaboration: Utilizing federated learning, hospitals can collaboratively train models on IVUS datasets without sharing sensitive patient data. This approach facilitates multi-center studies and ensures model generalization across diverse populations. 4) Enhanced Image Analysis: Enhanced image readability and spatial information allow clinicians to interpret IVUS images more accurately, reducing diagnostic errors and improving patient outcomes, even when working with a limited number of imaging datasets. 5) Personalized Medicine: The integration of demographic and patient-specific data enables the creation of personalized cardiovascular disease risk profiles, allowing for tailored treatment plans and preventive strategies. 6) Broader Medical Applications: The federated learning framework is adaptable to other imaging modalities, such as CT, MRI, and ultrasound. This expands its utility to fields like oncology, neurology, and prenatal care, particularly where sensitive and distributed data are prevalent. The focus remains on delivering practical, actionable solutions ready for real-world implementation.
% By using segmentation models to detect circular boundaries in diagnosis, the proposed method improves treatment effectiveness and efficiency.  It can measure quantitative parameters and interpret plaque burden in a federated learning framework. 



%我們已經完成 (提出) …
%This paper proposes and pretests 2D U-Net deep learning models and a federated learning framework for IVUS segmentation. Specifically, it can identify the EEM, lumen area, plaque border, and location in arteries. The system architecture provides spatial information and improves image readability in the clinical environment. %不同於過去文獻或者是問題困難點
%In diagnosis, the proposed method aims to improve treatment efficiency by detecting circular boundaries by area segmentation models. Despite this, the exchange of medical data across hospitals is impossible due to regulatory or privacy concerns. Therefore, the federated learning framework is proposed to address these issues and the problem of cross-hospital data sharing. %實驗結果顯示我們突破了哪些
%Based on the experiment's results, the developed IVUS segmentation system can be used in collaboration with domain experts to measure quantitative parameters and interpret plaque burden. Through the overlay of two-dimensional images, 3D cardiovascular restoration techniques can also automatically estimate the lesion area's volume automatically. By implementing more extensive federated learning methods, performance can be improved. This technology is flexible and can be applied to various fields where sensitive and distributed data is present. 
%未來應用
%To achieve successful outcomes for all hospitals, some demographic indexes or patient characteristics may also be considered in order to incorporate more clients and data volumes. 