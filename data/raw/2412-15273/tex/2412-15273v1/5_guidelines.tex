\section{Guidelines for improving the ecosystem} \label{sec:guidelines}

We note that 
Figure~\ref{fig:composite-comparison} gave a composite view of the leading and lagging US states in content quality, of which, 
 an illustration was shown in Figure~\ref{fig:intro-q}. Digging deeper, we found that leaders do a few things differently (and correctly) which others should follow. They are that leaders have more questions and answers (Table~\ref{tab:stat_table}) with  content that cover more topics (Table~\ref{tab: a_topic_analysis_sum}), that are readable (Tables~\ref{tab:q_readability_table},\ref{tab:a_readability_table},\ref{tab:qa_readability_table}),  and exhibit neutral sentiments (Figures~\ref{fig:senti-distrib},\ref{fig:senti-range}).


Based on these analyses, we provide the following guidelines for all states to improve their voter FAQ content. They are that states should (1) provide a reasonably large number of questions (typically $\geq$ 50)  covering a broad set of topics ($\geq$ five) in simple language,  (2) provide precise and specific answers which are  not too terse, (3) reduce overlap across questions by reducing overlap of topics, and (4) keep sentiment of content neutral. 