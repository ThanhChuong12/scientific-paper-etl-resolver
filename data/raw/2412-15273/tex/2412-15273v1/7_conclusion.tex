\section{Conclusion and Future Work} \label{sec:conclusion}

This paper makes many contributions starting by addressing the challenges faced by voters in finding answers to their election related questions by providing the {\bf first  dataset on Voter FAQs  covering all the
US states}.  Second, we introduce metrics for FAQ information quality score (FIQS)  with respect to  questions, answers, and answers to corresponding questions. Third, we use FIQS to analyze US FAQs to identify  leading, mainstream and lagging content practices and corresponding states. Finally, we identify what states across the spectrum can do to improve FAQ quality and thus,  the overall information ecosystem.

We provide verified, curated voter information to counteract widespread misinformation. This work, although promising, is just the first step. In future, one can work to remove the limitations and also build decision-support tools using the data to make effective tools available to voters. One can also separate the analysis by SECs, the primary, official data providers, and by secondary sources, e.g., non-profits like LWV, to understand how they complement each other.