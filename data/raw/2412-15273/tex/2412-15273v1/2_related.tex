\section{Related Work} \label{sec:related}

Going beyond studying the negative impacts of  AI on elections, researchers have begun shifting attention to its positive impacts.
AI-driven tools can enhance voter education by offering personalized, real-time responses to common election questions, and they can support policymakers by identifying trends and disparities in voter access or participation. By providing actionable insights, AI could play a transformative role in improving the transparency and efficiency of electoral systems.
%in the U.S. 
In this regard, \cite{srivastava2025vision} outlines their {\bf CDC} approach of (i) \underline{{\bf C}}ollating frequently anticipated questions and their authoritative answers, (ii) \underline{{\bf D}}istributing reliably by modeling multi-dimensional opinion networks with official information and discovering strategies to control them \cite{muppasani2024effectiveplanningstrategiesdynamic,infospread-planning-demo-Muppasani_Narayanan_Srivastava_Huhns_2024}, and (iii) helping people with diverse backgrounds \underline{{\bf C}}omprehend official information with personalization and provenance using chatbots \cite{safechat-elections-aimag,muppasani2025electionbot},  - all in the service of reducing information gap for increasing voter participation. But it all starts with authentic data.

In US, questions about state-specific election processes—such as voter registration, polling locations, absentee ballot rules, and early voting policies—are crucial for both voters and policymakers. However, the decentralized nature of U.S. elections means that this information is often fragmented across various state and local jurisdictions, creating barriers to accessibility and analysis.
AI has the potential to address these challenges by aggregating, standardizing, and analyzing election-related data. 


%\biplav{There is a tradition on releasing datasets which bootstraps further NLP research in the area. Then pointer to related dataset about other areas. Add one line improve online information - factify dataset \cite{chakraborty-etal-2023-factify3m}}. our dataset github follows the NLP community's best practices in this regard.

Releasing datasets is a key tradition in advancing NLP research, often catalyzing further work in the field. Related datasets, such as Factify3M \cite{chakraborty-etal-2023-factify3m}, have enhanced online information reliability. Our dataset adheres to the NLP community's best practices.