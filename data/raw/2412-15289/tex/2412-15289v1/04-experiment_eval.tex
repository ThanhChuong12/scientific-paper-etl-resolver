\section{Experimental Evaluations}
\label{sec:ExpEval}

% Table generated by Excel2LaTeX from sheet 'Clean-Main'
\begin{table*}[htbp] % htbp
  \centering
    \resizebox{\textwidth}{!}{
    \begin{tabular}{c|cc|cc|cc|cc|cc|cc|cc}
    \toprule
    \multirow{2}[1]{*}{\textbf{Jailbreak Attacks}} & \multicolumn{2}{c|}{\textbf{gpt-3.5-turbo}} & \multicolumn{2}{c|}{\textbf{Claude-v2}} & \multicolumn{2}{c|}{\textbf{Llama-3-8B}} & \multicolumn{2}{c|}{\textbf{Llama-3-70B}} & \multicolumn{2}{c|}{\textbf{gpt-4o-mini}} & \multicolumn{2}{c|}{\textbf{gpt-4o}} & \multicolumn{2}{c}{\textbf{Average}} \\
          & \textbf{HS} & \textbf{ASR} & \textbf{HS} & \textbf{ASR} & \textbf{HS} & \textbf{ASR} & \textbf{HS} & \textbf{ASR} & \textbf{HS} & \textbf{ASR} & \textbf{HS} & \textbf{ASR} & \textbf{HS} & \textbf{ASR} \\
    \midrule
    \midrule
    % DI    & 1.22  & 0\%   & 1     & 0\%   & 1.14  & 2\%   & 1.16  & 4\%   & 1.16  & 4\%   & 1     & 0\%   & 1.11  & 2\% \\
    GCG   & 3.36  & 54\%  & 1.16  & 4\%   & 1.08  & 2\%   & 1.48  & 12\%  & 1.24  & 6\%   & 1.08  & 2\%   & 1.57  & 13\% \\
    AutoDAN & 1.78  & 18\%  & 1     & 0\%   & 1.72  & 18\%  & 1.4   & 10\%  & 1.52  & 10\%  & 1.5   & 10\%  & 1.49  & 11\% \\
    PAIR  & 3.16  & 38\%  & 1.1   & 0\%   & 3.14  & 34\%  & 3.4   & 52\%  & 3.76  & 66\%  & 3.54  & 58\%  & 3.02  & 41\% \\
    AdvPrompter & 4.74  & 88\%  & 1  & 0\%  & 1.92  & 18\%  & 2.06  & 16\%  & 1.38  & 8\%  & 1.22  & 4\% & 2.05 & 22\% \\
    DrAttack & 3.8 & 64\% & 2.88 & 40\% & 2.76 & 40\% & 1.76 & 18\% & 3.36 & 50\% & 3.3 & 50\% & 2.98 & 44\% \\
    ArtPrompt-\texttt{top1} & 4.38  & 72\%  & 2.22  & 20\%  & 2.8   & 36\%  & 2.52  & 30\%  & 2.36  & 18\%  & 2.72  & 32\%  & 2.83  & 35\% \\
    ArtPrompt-\texttt{ensemble} & 4.56  & 78\%  & 3.44  & 52\%  & 3.9   & 66\%  & 3.8   & 58\%  & 4.18  & 72\%  & 3.38  & 48\%  & 3.88  & 62\% \\
    \midrule
    % \midrule
    SATA-ELP-\texttt{top1} & 4.18  & 66\%  & 4.18  & 68\%  & 3.36  & 42\%  & 2.86  & 38\%  & 3.24  & 20\%  & 3.82  & 48\%  & 3.61  & 47\% \\
    SATA-ELP-\texttt{ensemble} & \textbf{4.96} & \textbf{96\%} & \textbf{4.54} & \textbf{86\%} & 4.14  & 68\%  & 3.82  & 62\%  & 4.56  & 68\%  & \textbf{4.56} & 78\%  & 4.43  & 76\% \\
    SATA-MLM-\texttt{top1} & 4.74  & 90\%  & 3.14  & 52\%  & 4.36  & 66\%  & 4.1   & 66\%  & 4.72  & 88\%  & 3.94  & 68\%  & 4.17  & 72\% \\
    SATA-MLM-\texttt{ensemble} & 4.94  & \textbf{96\%} & 3.86  & 68\%  & \textbf{4.8} & \textbf{88\%} & \textbf{4.6} & \textbf{82\%} & \textbf{4.88} & \textbf{94\%} & 4.36  & \textbf{82\%} & \textbf{4.57 } & \textbf{85\%} \\
    \bottomrule
    \end{tabular}%
    }
  \caption{Harmful Score (HS) and Attack Success Rate (ASR) of baseline methods and SATA (ours) on AdvBench dataset. ArtPrompt-\texttt{top1} represents the highest performance among the ASCII art fonts in their experiment, while ArtPrompt-\texttt{ensemble} denote the combined performance across all fonts. Best results are highlighted in \textbf{bold}.}
  \tablevspace
  \label{tab:main-table}%
\end{table*}%

% 要说明Artprompt的top-1和ensemble;要说明AutoDAN等方法的黑盒模型是transfer的结果;             % 调整主表位置!!!
\subsection{Experimental Setup}
\label{subsec:ExpSet}
\paragraph{Victim Models.}
% We evaluate BAZINGA on the representative and new state-of-the-art safety-aligned LLMs. The victim models included in our experiments are four closed-source LLMs, including GPT-3.5-turbo, GPT-4o-mini, GPT-4o, and claude-v2, and two open-source LLMs, including LLama3-8B and Llama3-70B.
We choose the representative and new state-of-the-art safety-aligned LLMs as our victim models. We evaluate SATA on four closed-source LLMs, including GPT-3.5, GPT-4o-mini (2024-07-18), GPT-4o (2024-08-06), and claude-v2, and two open-source LLMs, including LLama3-8B and Llama3-70B.

\paragraph{Baselines.}
We compare SATA with four strong baselines and include the direct instruction query as the basic baseline. We retain the original default setups for all baselines (see Appendix~\ref{app:baseline}). % , details are included in Appendix

\textit{[B1] Direct Instruction (DI)} prompts the victim LLMs with the vanilla harmful instruction. % and obtains the response.

\textit{[B2] Greedy Coodinate Gradient (GCG)}~\cite{zouUniversalTransferableAdversarial2023a} searches adversarial suffixes by combining greedy and gradient-based techniques and jailbreaks LLMs by appending an adversarial suffix to the harmful query. GCG is applicable to white-box LLMs and is transferable to closed-source LLMs.
% GCG is applicable to white-box LLMs, but the computed suffix is also transferable to closed-source LLMs.

\textit{[B3] AutoDAN}~\cite{liuAutoDANGeneratingStealthy2023a} leverages genetic algorithm to iteratively evolve and select the jailbreak prompt candidates. It requires white-box access to victim LLMs, which is used in the genetic algorithm.
% (\textit{population} in genetic algorithm). To perform the process of searching the optimal jailbreak prompt, AutoDAN necessitates white-box access to victim models. 

\textit{[B4] Prompt Automatic Iterative Refinement (PAIR)}~\cite{chaoJailbreakingBlackBox2024a} leverages an attacker LLM to iteratively generate and refine a batch of jailbreak prompts for victim LLMs. PAIR achieves a competitive jailbreak success rate and demonstrates significant transferability across LLMs.

\textit{[B5] ArtPrompt}~\cite{jiangArtPromptASCIIArtbased2024a} is an effective and black-box jailbreak attack. 
It showcases that semantics-only interpretation of corpora during safety alignment can induce incapability for LLMs to recognize ASCII art (visual symbolic representation), and it exploits this incapability to perform jailbreak via firstly transforming harmful word in query into ASCII art and then revealing the word from the ASCII art representation by following instructions in jailbreak prompts. % create

\paragraph{Datasets.} 
We evaluate SATA against baselines on two datasets: Advbench~\cite{zouUniversalTransferableAdversarial2023a} and JBB-Behaviors (JailbreakBench Behaviors, JBB)~\cite{chaoJailbreakBenchOpenRobustness2024}. Specifically, following previous works~\cite{weiJailbreakGuardAligned2024, liDeepInceptionHypnotizeLarge2024a, changPlayGuessingGame2024a, chaoJailbreakingBlackBox2024a, jiangArtPromptASCIIArtbased2024a}, we conduct experiments on the non-duplicate subset dataset of AdvBench for performance comparison, which consists of 50 representative harmful entries. The JBB dataset consists of ten categories of harmful behaviors (see details in Appendix~\ref{app:jbb-dataset}), each of which contains ten harmful instructions. %  with a specific behavior

\paragraph{Metrics.}
Consistent with previous works~\cite{liuAutoDANGeneratingStealthy2023a,chaoJailbreakingBlackBox2024a,jiangArtPromptASCIIArtbased2024a, dingWolfSheepsClothing2024a} we adopt GPT-judged harmful score (HS) and attack success rate (ASR) as our evaluation metrics. Specifically, we employ GPT-4 as the scorer to rate the victim model's response(s) to an adversarial prompt in terms of harmfulness and relevance, with the harmful score ranging from 1 to 5, where a score of 1 indicates the victim model refuse to respond, or the response is no harm or has no relevance while a score of 5 signifies a highly harmful or relevant response. In our experiments, a response with \texttt{HS=5} is considered as successful jailbreak attack. The GPT judge prompt in our work is same as previous works (see Appendix~\ref{app:judgeprompt-defenseprompt}). % [冗余] demonstrating a significant impact from the adversarial prompt

We exclude keyword-based judgment~\cite{zouUniversalTransferableAdversarial2023a} in our experiments since we observe similar findings as AutoDAN~\cite{liuAutoDANGeneratingStealthy2023a} and PAIR~\cite{chaoJailbreakingBlackBox2024a}, namely that: (1) LLMs may actually respond to jailbreak prompts, but with added disclaimers, such as warnings that the request could be illegal or unethical; and (2) LLMs sometimes provide off-topic response to jailbreak prompts. These factors render keyword-based judgment imprecise.

% \paragraph{Defenses.} We adopt filter-based, modification-based and reminder-based three defense techniques against BAZINGA and baselines, respectively. Specifically, the defenses are improved perplexity-based detection (PPL-Filter), paraphrase adversarial prompt (Paraphrase)~\cite{jainBaselineDefensesAdversarial2023} and AdaShield-Variant\footnote{AdaShield is a multimodal LLM jailbreak defense method, which we adapt to defend against text-based LLM jailbreaks.}~\cite{wangAdaShieldSafeguardingMultimodal2025}. We present the detailed defense settings in detail in Appendix~\ref{app:defense}.
\paragraph{Defenses.} We adopt filter-based and modification based defense against SATA 
. % and baselines, respectively. 
Specifically, the defenses includes windowed perplexity-based detection (sliding-window PPL-Filter) and paraphrasing adversarial prompt (Paraphrase)~\cite{jainBaselineDefensesAdversarial2023}. As a prompt-based defense, we also adopt the static version of AdaShield~\cite{wangAdaShieldSafeguardingMultimodal2025}. We present the detailed defense settings in Appendix~\ref{app:defense}.

\paragraph{SATA Configurations.}
% 介绍多个masking granularity; 介绍Top-1 and Ensemble Strategy是怎么回事 [update]: masking granularity在方法里介绍。
In our experiments, we evaluate two configurations of SATA. The first, labeled \texttt{top1}, represents the highest jailbreak performance achieved using a single masking granularity. The second configuration, \texttt{ensemble}, represents the combined jailbreak performance obtained all masking granularity. In the ensemble case, we report with the highest harmful score from the 4 types of masking granularity.
\subsection{Main Results}           % main experimental results
\label{subsec:ExpRes}
\paragraph{Attack Effectiveness.}
% 1.有效
% 2. structure-based attack becomes performance stumb block when the victim model is small size.
% 3. 又在JBB上做了测试:选3个victim models。Experiment Choice: 要不要和baseline模型比——如果比,那就是consistently outperform baselines; 如果不比,那就是consistently achieve surprising jailbreak ASR and HS.
We first evaluate SATA against baselines on AdvBench. As shown in Table~\ref{tab:main-table}, SATA achieves state-of-the-art performance compared baselines across all victim LLMs in HS and ASR, respectively, indicating the effectiveness of SATA.
Specifically, we observe that:~(1)~With \texttt{ensemble} configuration, SATA-MLM attains an overall ASR of \pt{85} and an overall HS of 4.57, significantly outperforming baselines;~(2)~With the \texttt{top-1} configuration, SATA-MLM can outperform the strongest baseline with its  \texttt{ensemble} configuration;~(3)~SATA-MLM is generally more effective  than SATA-ELP across all victim models, except Claude-v2. %, which suggests the two assistive tasks can achieve performance complementarity.
We provide qualitative examples of our jailbreak results in Appendix~\ref{app:jailbreak-result-examples}.

% Table generated by Excel2LaTeX from sheet 'Main-JBB'
\begin{table*}[htbp]
  \centering
  \resizebox{0.98\textwidth}{!}{
    \begin{tabular}{ccccccccccccc}
    \toprule
    \textbf{Jailbreak} & \textbf{Victim} & \textbf{HD} & \textbf{MH} & \textbf{PH} & \textbf{EH} & \textbf{FD} & \textbf{DI} & \textbf{SA} & \textbf{PR} & \textbf{EA} & \textbf{GD} & \textbf{Overall} \\
    \midrule
            & Llama3-70B & 10\%  & 20\%  & 0\%  & 20\%  & 40\%  & 40\%  & 20\%  & 20\%  & 30\%  & 20\%  & \pt{22} \\
DrAttack    & Claude-v2 & 0\%  & 40\%  & 10\%  & 40\%  & 30\%  & 40\%  & 20\%  & 10\%  & 30\%  & 30\%  & \pt{25} \\
            & GPT-4o & 20\%  & 40\%  & 10\%  & 40\%  & 40\%  & 40\%  & 20\%  & 50\%  & 50\%  & 50\%  & \pt{36} \\
          
\cmidrule{2-13}             & Llama3-70B & 30\%  & 50\%  & 10\%  & 40\%  & 50\%  & 30\%  & 10\%  & 50\%  & 60\%  & 50\% & \pt{38}\\
ArtPrompt-\texttt{ensemble} & Claude-v2 & 10\%  & 60\%  & 20\%  & \textbf{70\%}  & 80\%  & 70\%  & 10\%  & 70\%  & \textbf{70\%}  & 50\% & \pt{51}\\
                            & GPT-4o & 20\%  & 60\%  & 20\%  & 60\%  & 70\%  & 60\%  & 30\%  & 80\%  & 70\%  & 50\% & \pt{52}\\


\cmidrule{2-13}          & Llama3-70B & \textbf{40\%}  & \textbf{90\%}  & \textbf{70\%}  & \textbf{80\%}  & \textbf{100\%} & 70\%  & \textbf{40\%}  & 80\%  & \textbf{70\%}  & \textbf{90\%} & \textbf{73\%}\\
SATA-MLM-\texttt{ensemble} & Claude-v2 & 20\%  & 70\%  & 10\%  & 60\%  & 70\%  & 70\%  & 20\%  & 70\%  & 60\%  & 60\% & \pt{51}\\
                & GPT-4o & \textbf{50\%}  & \textbf{90\%}  & 50\%  & \textbf{80\%}  & \textbf{100\%} & \textbf{70\%}  & \textbf{60\%}  & 90\%  & 70\%  & \textbf{90\%} & \textbf{75\%}\\
          
\cmidrule{2-13}         & Llama3-70B & 20\%  & 80\%  & 60\%  & \textbf{80\%}  & 80\%  & \textbf{100\%} & 30\%  & \textbf{90\%}  & \textbf{70\%}  & 70\% & \pt{68}\\
SATA-ELP-\texttt{ensemble}   & Claude-v2 & \textbf{40\%}  & \textbf{90\%}  & \textbf{60\%}  & 60\%  & \textbf{90\%}  & \textbf{90\%}  & \textbf{40\%}  & \textbf{80\%}  & \textbf{70\%}  & \textbf{90\%} & \textbf{71\%}\\
                        & GPT-4o & \textbf{50\%}  & \textbf{90\%}  & \textbf{60\%}  & \textbf{80\%}  & 90\%  & 50\%  & 50\%  & \textbf{100\%} & \textbf{80\%}  & 70\% & \pt{72}\\
    \bottomrule
    \end{tabular}%
    }
  \caption{ASR of baselines (ArtPrompt and DrAttack) vs. SATA-MLM and SATA-ELP on various behaviors in JBB.}
  \tablevspace
  \label{tab:JBB-result-table}%
\end{table*}%


















% % set() 打乱的顺序,不是数据集中出现的顺序——我们按照数据集中的顺序排:HD, MH, PH...
%     \begin{tabular}{ccccccccccccc}
%     \toprule
%     \textbf{Jailbreak} & \textbf{Victims} & \textbf{FD} & \textbf{PR} & \textbf{EH} & \textbf{DI} & \textbf{GD} & \textbf{MH} & \textbf{PH} & \textbf{HD} & \textbf{SA} & \textbf{EA} & \textbf{Overall} \\
%     \midrule
%           & \textbf{Llama3-70B} & 100\% & 80\%  & 80\%  & 70\%  & 90\%  & 90\%  & 70\%  & 40\%  & 40\%  & 70\% & \pt{73}\\
%     \textbf{MLM-ensemble} & \textbf{Claude-v2} & 70\%  & 70\%  & 60\%  & 70\%  & 60\%  & 70\%  & 10\%  & 20\%  & 20\%  & 60\% & \pt{51}\\
%           & \textbf{GPT-4o} & 100\% & 90\%  & 80\%  & 70\%  & 90\%  & 90\%  & 50\%  & 50\%  & 60\%  & 70\% & \pt{75}\\
% \cmidrule{2-13}          & \textbf{Llama3-70B} & 80\%  & 90\%  & 80\%  & 100\% & 70\%  & 80\%  & 60\%  & 20\%  & 30\%  & 70\% & \pt{68}\\
%     \textbf{ELP-ensemble} & \textbf{Claude-v2} & 90\%  & 80\%  & 60\%  & 90\%  & 90\%  & 90\%  & 60\%  & 40\%  & 40\%  & 70\% & \pt{71}\\
%           & \textbf{GPT-4o} & 90\%  & 100\% & 80\%  & 50\%  & 70\%  & 90\%  & 60\%  & 50\%  & 50\%  & 80\% & \pt{72}\\
%     \bottomrule
%     \end{tabular}%
We further evaluate SATA on JBB-Behaviros. % to verify whether SATA can sustain its effectiveness. 
As shown in Table~\ref{tab:JBB-result-table}, SATA  maintains its performance comparing to the ArtPrompt baseline. For instance, SATA-MLM and SATA-ELP achieve an overall ASR of \pt{75} and \pt{72} on GPT-4o model, respectively. The performance drop mainly arise from the Harassment/Discrimination and Sexual/Adult content Category in JBB dataset. 
% SATA can maintain its ASR with performance fluctuation ranging from \pt{-17} to \pt{+15}. \todo{(Re-use wiki entry caused, will re-run)}

% Table generated by Excel2LaTeX from sheet 'Main Results with  Defense'
\begin{table}[htbp]
  \centering
  \resizebox{0.47\textwidth}{!}{
    \begin{tabular}{l|cc|cc|cc|cc}
    \toprule
    \multicolumn{1}{c|}{\multirow{2}[2]{*}{\textbf{Attacks + defense}}} & \multicolumn{2}{c|}{\textbf{Claude-v2}} & \multicolumn{2}{c|}{\textbf{Llama-3-70B}} & \multicolumn{2}{c|}{\textbf{gpt-4o}} & \multicolumn{2}{c}{\textbf{Average}} \\
          & \textbf{HS} & \textbf{ASR} & \textbf{HS} & \textbf{ASR} & \textbf{HS} & \textbf{ASR} & \textbf{HS} & \textbf{ASR} \\
    \midrule
    \midrule
    \rowcolor[rgb]{ .949,  .949,  .949}         ArtPrompt & 3.44  & 52\%  & 3.8   & 58\% & 3.38  & 48\% & 3.54  & 53\% \\
            ArtPrompt + ppl & 3.64  & 40\% & 2.58  & 26\%  & 2.88  & 30\%  & 3.03  & 32\% \\
            ArtPrompt + para & 1.6   & 8\%  & 3.94  & 34\%  & 3.88  & 42\%  & 3.91  & 28\% \\
    \midrule
    \rowcolor[rgb]{ .949,  .949,  .949}         SATA-MLM & 3.86  & 68\%  & 4.6   & 82\%  & 4.36  & 82\%  & \cellcolor[rgb]{ 1,  1,  1}4.27  & \cellcolor[rgb]{ 1,  1,  1}77\% \\
            SATA-MLM + ppl & 3.8   & 66\%  & 4.54  & \textbf{82}\%  & 4.54  & \textbf{84}\%  & 4.29  & 77\% \\
            SATA-MLM + para & 3.36  & 38\% & 4.26  & \textbf{66}\%  & 4.22  & 70\%  & 3.95  & 58\% \\
    \midrule
    \rowcolor[rgb]{ .949,  .949,  .949}         SATA-ELP & 4.54  & 86\%  & 3.82  & 62\%  & 4.56  & \cellcolor[rgb]{ 1,  1,  1}78\% & \cellcolor[rgb]{ 1,  1,  1}4.31  & \cellcolor[rgb]{ 1,  1,  1}75\% \\
            SATA-ELP + ppl & 4.48  & \textbf{84}\%  & 4.08  & 64\%  & 4.48  & 78\%  & 4.35  & 75\% \\
            SATA-ELP + para & 4.06  & \textbf{64}\%  & 3.34  & 48\%  & 4.62  & \textbf{78}\%  & 4.01  & 63\% \\
    \bottomrule
    \end{tabular}%
    }
    \caption{HS and ASR of SOTA baseline v.s SATA under windowed PPL-filter and paraphrase defense. Results are evaluated on ensemble configuration.}
  \label{tab:main-defense}%
\end{table}%

% 都是在ensemble的configuration下evaluate出的结果
% 如何分析和说明这张表




%%%%%%%%%%%%%%%%%%%%%%%%%%%%%%%%%%%%%%%%%%%%%%%%%%%%%%%%%

% % Table generated by Excel2LaTeX from sheet 'Main Results with  Defense'
% \begin{table}[htbp]
%   \centering
%   \resizebox{0.5\textwidth}{!}{
%     \begin{tabular}{l|cc|cc|cc|cc}
%     \toprule
%     \multicolumn{1}{c|}{\multirow{2}[2]{*}{\textbf{Attacks + defense}}} & \multicolumn{2}{c|}{\textbf{Claude-v2}} & \multicolumn{2}{c|}{\textbf{Llama-3-70B}} & \multicolumn{2}{c|}{\textbf{gpt-4o}} & \multicolumn{2}{c}{\textbf{Average}} \\
%           & \textbf{HS} & \textbf{ASR} & \textbf{HS} & \textbf{ASR} & \textbf{HS} & \textbf{ASR} & \textbf{HS} & \textbf{ASR} \\
%     \midrule
%     \midrule
%     \rowcolor[rgb]{ .949,  .949,  .949}         ArtPrompt & 3.44  & 52\%  & 3.8   & 58\% & 3.38  & 48\% & 3.54  & 53\% \\
%             ArtPrompt + ppl & 3.64  & 40\% & 2.58  & 26\%  & 2.88  & 30\%  & 3.03  & 32\% \\
%             ArtPrompt + para & 1.6   & 8\%  & 3.94  & 34\%  & 3.88  & 42\%  & 3.91  & 28\% \\
%     \midrule
%     \rowcolor[rgb]{ .949,  .949,  .949}         SATA-MLM & 3.86  & 68\%  & 4.6   & 82\%  & 4.36  & 82\%  & \cellcolor[rgb]{ 1,  1,  1}4.27  & \cellcolor[rgb]{ 1,  1,  1}77\% \\
%             SATA-MLM + ppl & 3.8   & 66\%  & 4.54  & \textbf{82}\%  & 4.54  & \textbf{84}\%  & 4.29  & 77\% \\
%             SATA-MLM + para & 3.36  & 38\% & 4.26  & \textbf{66}\%  & 4.22  & 70\%  & 3.95  & 58\% \\
%     \midrule
%     \rowcolor[rgb]{ .949,  .949,  .949}         SATA-ELP & 4.54  & 86\%  & 3.82  & 62\%  & 4.56  & \cellcolor[rgb]{ 1,  1,  1}78\% & \cellcolor[rgb]{ 1,  1,  1}4.31  & \cellcolor[rgb]{ 1,  1,  1}75\% \\
%             SATA-ELP + ppl & 4.48  & \textbf{84}\%  & 4.08  & 64\%  & 4.48  & 78\%  & 4.35  & 75\% \\
%             SATA-ELP + para & 4.06  & \textbf{64}\%  & 3.34  & 48\%  & 4.62  & \textbf{78}\%  & 4.01  & 63\% \\
%     \bottomrule
%     \end{tabular}%
%     }
%     \caption{HS and ASR of SOTA baseline v.s SATA under windowed PPL-filter and paraphrase defense. Results are evaluated on ensemble configuration.}
%   \label{tab:main-defense}%
% \end{table}%
\paragraph{Performance Against Defenses.}
% 1. robust to perturbation (paraphrase and retokenization)
% 2. stealthy to perplexiity-detection filter
We evaluate the performance of SATA against windowed PPL-Filter and Paraphrase jailbreak defenses, and compare to state-of-the-art baseline, with the results shown in Table~\ref{tab:main-defense}. Our observations are as follows:~(1)~The perplexity-based detection only minimally reduce the jailbreak performance, demonstrating that SATA is stealthy to bypass windowed PPL-Filter defense.~(2)~Paraphrase is somewhat more effective than PPL-Filter to defense SATA jailbreak attack, causing an average drop of ~\pt{12} ASR for SATA-ELP and ~\pt{19} ASR for SATA-MLM. (3) SATA consistently elicits toxic response and outperforms ArtPrompt under windowed PPL-Filter and paraphrase defense, achieving an average of ASR~\pt{63} and~\pt{58} for SATA-ELP and SATA-MLM, respectively. Finally, we compare the paraphrased adversarial prompt to the original one, and find that the paraphrase defense works by summarizing the wiki entry content and disrupting the wiki entry text-infilling format.
% However, SATA consistently elicit toxic response with an average HS of 4.01 and 3.95 and an average of ASR~\pt{63} and~\pt{58} for SATA-ELP and SATA-MLM, respectively. Additionally, it is particularly effective for claude-v2. We compare the paraphrased adversarial prompt to the original one, and find that the paraphrase defense works by summarizing the wiki entry content and disrupting the wiki entry text-infilling format.

\begin{figure}[htbp] % htb
    \centering
    \includegraphics[width=0.47\textwidth]{images/def-ppl-2-tailored.pdf} % 0.49
      \vspace{-8pt}
      \caption{Perplexity of each harmful instruction in AdvBench and perplexity of the corresponding adversarial prompt generated by SATA-MLM and SATA-ELP with different masking granularities.}
      \figurevspace
      \label{fig:def-ppl}
\end{figure}
To further study the stealthiness of SATA, we visualize the perplexity values computed on GPT-2 in Figure~\ref{fig:def-ppl}. We can observe that, with a small window size (\texttt{max\_length=5}), the perplexities of GPT-2 for the adversarial prompt generated by SATA consistently remain below the threshold, regardless of the chosen assistive task (MLM, ELP) or masking granularity. Furthermore, the adversarial prompt generated by SATA-MLM demonstrate lower perplexity compared to those generated by SATA-ELP, indicating that SATA-MLM is more stealthy. Finally, if we exclude the outliers in harmful instructions and decrease \texttt{T=138.56} (see the dark dashed line), SATA can still bypass the windowed PPL-Filter in most settings. % investigate, more stealthy than those from ELP in terms of perplexity.

We attribute the stealthiness of SATA to two factors. First, the wiki text entry is synthesized by articulated LLMs and there is no opaque sub-string (gibberish) in adversarial prompt. Second, the commendatory words \texttt{List} in the adversarial prompt of SATA-ELP is not long (about ten words). % and meaningless

% Table generated by Excel2LaTeX from sheet 'Details Main Results w  Defense'
\begin{table}[htbp]
  \centering
    \resizebox{0.25\textwidth}{!}{
    \begin{tabular}{ccc}
    \toprule
    \multirow{2}[2]{*}{\textbf{Jailbreak Attacks}} & \multicolumn{2}{c}{Llama3-70B} \\
          & HS    & ASR \\
    \midrule
    \textbf{ELP-ensemble} & 1.74  & 12\% \\
    \textbf{MLM-ensemble} & 2.64  & 24\% \\
    \textbf{ArtPrompt-ensemble} & 1.32  & 2\% \\
    \bottomrule
    \end{tabular}%
    }
  \caption{SATA v.s. ArtPrompt under adapted AdaShield-S defense. Their original performance without defense can be found in Table \ref{tab:main-defense}.}
  \label{tab:adashield}%
\end{table}%
We also investigate whether SATA can defend against prompt-based jailbreak defense. We adopt AdaShield-S~\cite{wangAdaShieldSafeguardingMultimodal2025} to evaluate SATA and ArtPrompt for the Llama3 model, with results shown in Table~\ref{tab:adashield}. Interestingly, we find both attacks experience a large ASR drop, while SATA-MLM gives a relatively decent ASR of 24\%. 

\paragraph{Efficiency Analysis.}
\label{paragraph:cost}

SATA is lightweight in terms of the number of iterations, jailbreak prompt candidates, and jailbreak prompt length. These three factors collectively influence the input token usage, which serves as an indicator of the average inference time cost or economic cost (when invoking API) for a jailbreak. % adversarial prompt candidates
We calculate the average input token usage\footnote{To simplify,  we opt to calculate and report the word count,  as the token count and word count can be approximately linear.} for different jailbreak methods (see Appendix~\ref{app:input_token_usage} for detailed calculation process), and compare SATA to the baselines, with results shown in Figure~\ref{fig:cost-tokens}. We can observe that SATA-MLM consumes comparable or less input tokens compared with state-of-the-art baselines (ArtPrompt) while it attains significant higher jailbreak HS and ASR (see Table~\ref{tab:main-table}). In addition, SATA-ELP achieves a significant reduction in input token usage, reaching about an order of magnitude savings, while maintaining state-of-the-art jailbreak performance. Lastly, we observe from Figure~\ref{fig:appendix-jailbreak-WET-attack} and~\ref{fig:appendix-jailbreak-ELP-mw} in Appendix~\ref{app:jailbreakprompt} that the jailbreak prompt is designed to be simple, requiring minimal human design effort, and the input token usage in SATA-MLM primarily stems from the synthesized wiki entry.
Theses observations showcase SATA is cost-efficient for both jailbreak and human-effort.

The main reasons for its cost-efficiency are: (1) there is no need for multiple iterations or jailbreak prompt candidates; (2) masking harmful keywords by LLMs avoids multiple tries; (3) the synthesized wiki entry is limited to six paragraphs, whereas retrieving a wiki entry from Wikipedia is often unbearably long; and (4) assistive tasks are designed to be friendly for victim LLMs to perform. % understand and perform (4)所以prompt可以写得很简单
\begin{figure}[htb]
    \centering
    \includegraphics[width=0.49\textwidth]{images/cost-tokens-2.pdf} % scale=0.23
      % \caption{The average input token usage among baseline methods and SATA for the jailbreak of single harmful instruction: -ens represents ensemble configuration of corresponding method, -sw/p and -mw/p stand for chosen masking granularity. The y-axis is on a logarithmic scale.}
      \vspace{-18pt}
      \caption{Average input token usage per harmful instruction across baselines and SATA for jailbreak attempts. -ens represents the \texttt{ensemble} configuration of methods, while -sw/p and -mw/p denote the chosen masking granularity. The y-axis is on a logarithmic scale.}
      \figurevspace
      \label{fig:cost-tokens}
\end{figure}

\subsection{Ablation Study}     % ablation study results
\label{subsec:ablation}
We conduct ablation studies to analyze the impact of the following factors on jailbreak performance. Due to budget constraints, we primarily select GPT-3.5-turbo and Llama-3-8B as our victim models and Advbench as dataset for the ablation study.

\paragraph{Impact of the Insert Position of Harmful Keywords in the Sequence.}
\label{paragraph:ab-position}
% 此Ablation要说明随机位置对Jailbreak性能影响很大,empirically前一半位置
% 最终证明要选择简单任务(LLM擅长的任务)参与任务连接,只有这样才能准确恢复语义信息,实现harmful instruction的语义重建
Although ELP is relatively simple, LLMs can still fail to identify the correct element in the commendatory words \texttt{List}, though infrequently. Empirically, this issue becomes slightly pronounced when the insert position is closer to the end of the \texttt{List}. 
We tune the ELP task to be a little bit more difficult for victim LLMs to perform by forcibly shifting the insert position to the latter half of the \texttt{List}, and we analyze the impact of the insert position of the masked keywords in \texttt{List} on performance. We consider the single-word and single-phrase masking granularity.

As shown in Table~\ref{tab:ablation-position}, when the insert position is forcibly shifted to the latter half, the ASR occasionally experiences a moderate drop in the ablation experiment settings, demonstrating that assistive tasks should remain simple to ensure that the semantics conveyed by the assistive task do not deviate from the harmful keywords. % from the first half
% Table generated by Excel2LaTeX from sheet 'Ablation'
\begin{table}[htbp]
  \centering
  \resizebox{0.48\textwidth}{!}{
    \begin{tabular}{cccccc}
    \toprule
    \multirow{2}[4]{*}{\textbf{Insert Position}} & \multirow{2}[4]{*}{\textbf{Setting}} & \multicolumn{2}{c}{\textbf{gpt-3.5-turo}} & \multicolumn{2}{c}{\textbf{Llama-3-8B}} \\
\cmidrule{3-6}          &       & \textbf{HS} & \textbf{ASR} & \textbf{HS} & \textbf{ASR} \\
    \midrule
    \midrule
    \multirow{2}[2]{*}{first half position} & ELP-sw & 4.8   & 94\%  & 2.28  & 28\% \\
          & ELP-sp & 4.7   & 90\%  & 2.6   & 38\% \\
    \midrule
    \multirow{2}[2]{*}{second half position} & ELP-sw & 4.56  & 88\% (\textcolor{red}{\pt{6}$\downarrow$})  & 2.26   & 28\% (\textcolor{red}{\pt{0}$\downarrow$}) \\
          & ELP-sp & 4.7  & 90\% (\textcolor{red}{\pt{0}$\downarrow$}) & 2.18 & 24\% (\textcolor{red}{\pt{14}$\downarrow$}) \\
    \bottomrule
    \end{tabular}%
  }
  \caption{Impact of the insert position of the masked keywords in commendatory words \texttt{List} on performance. ASR drop is calculated as the absolute difference.}
  \label{tab:ablation-position}%
\end{table}%

% 想了想,觉得没必要全部标注,只标注显著下降的地方,突出:重点(i.e. 需要关注地方/和你想要说明的东西相关的地方)



\paragraph{Effectiveness of Constructing an Assistive Task.}
\label{paragraph:ab-necessity}
% 此Ablation通过case study ELP来说明任务连接的必要性
We investigate the effectiveness of assistive task. Specifically, we replace the ELP task with directly informing the victim LLMs of the masked keywords and term this approach as \texttt{Inform}. As shown in Table~\ref{tab:ablation-assistive-task}, the jailbreak performance drops drastically, indicating that constructing an addition assistive task is effective for jailbreaking.
% 原因可能是辅助任务让LLM分神,没了辅助任务就没有distraction了。
% Camera Ready: refinement
% Table generated by Excel2LaTeX from sheet 'Ablation'
\begin{table}[htbp]
  \centering
  \resizebox{0.48\textwidth}{!}{
    \begin{tabular}{cccccc}
    \toprule
          & \multirow{2}[2]{*}{\textbf{Setting}} & \multicolumn{2}{c}{\textbf{gpt-3.5-turo}} & \multicolumn{2}{c}{\textbf{Llama-3-8B}} \\
          &       & \textbf{HS} & \textbf{ASR} & \textbf{HS} & \textbf{ASR} \\
    \midrule
    \midrule
    \multirow{4}[2]{*}{with assistive task} & MLM-sw & 4.56   & 80\%  & 4.2  & 76\% \\
          & MLM-sp & 4.04   & 56\%  & 3.74   & 50\% \\
          & ELP-sw & 4.8   & 94\%  & 2.28  & 28\% \\
          & ELP-sp & 4.7   & 90\%  & 2.6   & 38\% \\
    \midrule
    \multirow{2}[2]{*}{w/o assisstive task} & DirectlyInform-sw & 2.5  & (\textcolor{red}{\pt{44}$\downarrow$}) 36\% (\textcolor{blue}{\pt{58}$\downarrow$}) & 1  & (\textcolor{red}{\pt{76}$\downarrow$}) 0\% (\textcolor{blue}{\pt{28}$\downarrow$}) \\
          & DirectlyInform-sp & 2.4  & (\textcolor{red}{\pt{22}$\downarrow$}) 34\% (\textcolor{blue}{\pt{56}$\downarrow$}) & 1.08  & (\textcolor{red}{\pt{48}$\downarrow$}) 2\% (\textcolor{blue}{\pt{36}$\downarrow$}) \\
    \bottomrule
    \end{tabular}%
    }
  \caption{Impact of constructing an assistive task. Percentage values in \textcolor{red}{red} and \textcolor{blue}{blue} indicate the absolute ASR reductions resulting from the removal of MLM and ELP assistive tasks, respectively.}
  \tablevspace
  \label{tab:ablation-assistive-task}%
\end{table}%





% ARR Feb Version
% % Table generated by Excel2LaTeX from sheet 'Ablation'
% \begin{table}[htbp]
%   \centering
%   \resizebox{0.48\textwidth}{!}{
%     \begin{tabular}{cccccc}
%     \toprule
%           & \multirow{2}[2]{*}{\textbf{Setting}} & \multicolumn{2}{c}{\textbf{gpt-3.5-turo}} & \multicolumn{2}{c}{\textbf{Llama-3-8B}} \\
%           &       & \textbf{HS} & \textbf{ASR} & \textbf{HS} & \textbf{ASR} \\
%     \midrule
%     \midrule
%     \multirow{2}[2]{*}{with assistive task} & ELP-sw & 4.8   & 94\%  & 2.28  & 28\% \\
%           & ELP-sp & 4.7   & 90\%  & 2.6   & 38\% \\
%     \midrule
%     \multirow{2}[2]{*}{w/o assisstive task} & ELP-sw & 2.5 (\textcolor{red}{2.3$\downarrow$})  & 36\% (\textcolor{red}{\pt{58}$\downarrow$}) & 1  (\textcolor{red}{1.28$\downarrow$}) & 0\% (\textcolor{red}{\pt{28}$\downarrow$}) \\
%           & ELP-sp & 2.4 (\textcolor{red}{2.3$\downarrow$}) & 34\% (\textcolor{red}{\pt{56}$\downarrow$}) & 1.08 (\textcolor{red}{1.52$\downarrow$}) & 2\% (\textcolor{red}{\pt{36}$\downarrow$}) \\
%     \bottomrule
%     \end{tabular}%
%     }
%   \caption{Impact of constructing an assistive task. ASR drop is calculated as the absolute difference.}
%   \tablevspace
%   \label{tab:ablation-assistive-task}%
% \end{table}%








% \paragraph{Synthesize Wiki Entry v.s. Retrieve Wiki Entry.}
% \label{paragraph:ab-wiki}
% % 此Ablation要说明规整的wiki更容易提升performance,但是词条多义对jailbreak性能影响不大。



%%%%%%%%%%%%%%%%%%%%
% WET上mw mp一直好; ELP上各个粒度是性能互补的
% We demonstrate the impact of the four masking granularity on jailbreak performance, and we show the details of WET attack in Table~\ref{tab:ablation-mask-granularity}.
% We can observe that the four mask granularity can provide performance complementarity across victim models in the ELP-attack.