\section{Introduction}

Developing a model for multilingual automatic speech recognition (ASR) is appealing due to its applicability to universal languages, including low-resource or unseen languages.
However, this task presents significant challenges as it requires extensive datasets and involves the complexity of capturing shared characteristics across diverse languages in both phonetic and orthographic domains.

Since the revolution in ASR technologies driven by self-supervised learning (SSL) models~\cite{NEURIPS2020_92d1e1eb, conneau2020unsupervisedcrosslingualrepresentationlearning, hsu2021hubert, chen2022wavlm}, monolingual ASR has achieved superhuman transcription performance, shifting the main focus of recent research towards developing a universal model that spans multiple languages.

There are two primary methods for building a multilingual ASR model.
The first approach involves scaling the size of both the labeled dataset and the model itself, using a single universal model to enhance its capacity and cover a vast number (100+) of languages, thereby achieving multilingual ASR~\cite{radford2023robust}.
Another approach involves incorporating language-specific modules into the universal model to address the performance inconsistencies of previous methods. 
For example, MMS~\cite {JMLR:v25:23-1318} demonstrated the feasibility of scaling multilingual technology to over 1,000 languages by leveraging common features across languages and adding language-specific modules to improve the performance of each language.
Indeed, there have been efforts to integrate both methods~\cite{zhang2023googleusmscalingautomatic}, combining their strengths to build a more robust and versatile model.

Although these works have demonstrated strong performance across various languages, the trade-off between performance and complexity remains a substantial challenge.
The first method, using a single universal pipeline, struggles to achieve consistent performance across languages, and its effectiveness in low-resource languages remains uncertain.
On the other hand, despite achieving state-of-the-art performance and parameter efficiency, the second method cannot be considered a single universal model due to the inclusion of language-specific modules. Moreover, the use of language-specific modules like adapters, heads, and language models (LMs) sometimes complicates the training and inference pipeline, suggesting potential areas for future improvement.
\begin{figure*}[]
    \centering
    \includegraphics[width=\textwidth]{figures/pipeline.png}
    \caption{Illustration of our universal ASR pipeline.}
    \label{fig:pipeline}
\end{figure*}

Simultaneously, large language models (LLMs) have garnered considerable attention for their remarkable capabilities in the natural language processing (NLP) domain.
Following this trend, ASR pipelines have integrated audio SSL models as encoders and LLMs as decoders to enhance transcription quality~\cite{li2023prompting, fathullah2024prompting}.
These approaches involve using projectors~\cite{yu2024connecting} or fine-tuning strategies~\cite{tang2024salmonn, du2024lauragptlistenattendunderstand} to align modalities and improve transcription capabilities across multilingual datasets.
Subsequently, shallow fusion~\cite{chorowski2016betterdecodinglanguagemodel, kannan2018analysis} based scoring methods~\cite{hu2023massively, huang2024multilingual} were attempted to replace conventional LMs with LLMs during the decoding stage.
Despite these efforts resulting in performance gain across various languages, a comprehensive method to fully leverage the diverse emergent abilities of LLMs remains to be developed.

In this paper, we introduce a novel language-agnostic multilingual ASR pipeline that spans over 100 languages, including completely unseen languages. 
As in Fig~\ref{fig:pipeline}, the proposed pipeline consists of two phases: universal transcription generation and language-specific transliteration.
In the universal transcription generation phase, we focused on reducing orthographic complexity by unifying diverse orthographic systems into a consistent format, approximating phonetic features across multiple languages.
In the language-specific transliteration phase, we regard the transformation from universal transcription to language-specific transcription as a transliteration task by leveraging a universal converter.
Our experiments demonstrate notable transcription performance of~\shortname~across over 100 languages while using significantly smaller training data (only 680 hours) compared to other state-of-the-art multilingual ASR models.
Furthermore, our proposed pipeline outperforms previous methods, especially in low-resource languages, and demonstrates proficiency in completely unseen languages, achieving performance comparable to existing language-agnostic ASR methods without relying on any language-specific modules.
Our contributions are summarized as follows:

\begin{itemize}
    \item We propose a novel language-agnostic multilingual ASR pipeline consisting of two phases: universal transcription generation and language-specific transliteration.
    \item We enabled our proposed pipeline to perform multilingual ASR with minimal data by unifying diverse orthographic systems through Romanization.
    \item Our pipeline demonstrates consistent performance across over 100 seen languages and excels with completely unseen languages, all without relying on any language-specific modules or additional fine-tuning.
    
\end{itemize}