\section{Experiments}
\subsection{Dataset}
\subsubsection{FLEURS.} FLEURS~\cite{conneau2022fleursfewshotlearningevaluation} is a multilingual speech corpus encompassing 102 languages. It provides a relatively small amount of data per language (approximately 12 hours) while ensuring an unbiased distribution of data across the languages.
Given our focus on demonstrating effective multilingual ASR with minimal data, we utilize the FLEURS and its official splits for experiments.
\subsubsection{CommonVoice.} CommonVoice~\cite{ardila-etal-2020-common} is a multilingual speech dataset crowdsourced from speakers of various languages. For unseen languages, we leverage the official test split of 25 languages from CommonVoice 17.0, which offers sufficient samples for evaluation.

\subsection{Data Preprocessing}
We initially applied NFKC normalization and lowercase transformation to the text transcriptions. 
Subsequently, we excluded samples containing parentheses or numbers from the dataset for the following reasons:
parentheses and digits in transcriptions introduced ambiguity, as some enclosed phrases were pronounced while others were not, and digits had one-to-many pronunciation mappings across languages (e.g. `1' can be pronounced as `one', `eins', `uno', `yi', etc.).
Finally, we utilized the Python library \textit{Uroman}~\cite{hermjakob-etal-2018-box} to obtain Romanized transcription and \textit{Phonemizer}~\cite{Bernard2021} for IPA transcription.
For Japanese, we employed \textit{Pykakasi}~\cite{takahashi1992kakasi} due to the limitation of \textit{Uroman}, which treats Japanese kanji as Chinese characters.
Following these preprocessing steps, we obtained approximately 6 to 8 hours of speech-transcription paired data per language on average.

\begin{table*}[t]
\centering
\small
\begin{tabular}{l||ccccccc|cccc}
\toprule
\multirow{2}{*}{Model}                                                                    & \multicolumn{7}{c|}{Seen (PER / CER $\downarrow$)}                                   & \multicolumn{4}{c}{Unseen (PER / CER $\downarrow$)}                    \\ %\cline{2-12} 
                                                                                          & de & nl  & fr  & es  & it  & \multicolumn{1}{c|}{pt}  & avg  & ia    & eo    & \multicolumn{1}{c|}{eu}    & avg \\ \midrule \midrule
\multirow{2}{*}{\begin{tabular}[c]{@{}l@{}}wav2vec2.0-phoneme\textsuperscript{\dag}\\ + n-gram LM\textsuperscript{\dag}\end{tabular}} & 23.8 & 38.0 & 31.0 & 28.7 & 33.5 & \multicolumn{1}{c|}{45.0} & 33.0 & 10.7 & - & \multicolumn{1}{c|}{20.8} & 31.4   \\
                                                                                          & 14.8 & 26.0  & 26.4  & 12.3    & 21.7    & \multicolumn{1}{c|}{36.5}    & 22.9    & \textbf{6.1}    & -    & \multicolumn{1}{c|}{\textbf{13.7}}    & \textbf{22.2}   \\ \midrule
IPA generator (\shortname)                                                                      & 10.2  & 10.5 & \textbf{9.6} & \textbf{4.2} & 4.6 & \multicolumn{1}{c|}{10.9} & 14.4 & 29.0 & 32.0 & \multicolumn{1}{c|}{36.6} & 35.1  \\ 
\textbf{Roman generator (\shortname)}                                                                    & \textbf{7.3}  & \textbf{9.6} & 12.9 & 4.4 & \textbf{3.6} & \multicolumn{1}{c|}{\textbf{7.2}} & \textbf{11.3} & 14.0 & \textbf{20.8} & \multicolumn{1}{c|}{30.3} & 32.3  \\ \bottomrule
\end{tabular}%
\caption{
Comparison between two orthography unification methods. We report PER and CER for seen and unseen languages. 
Average values are calculated over all 102 seen and 25 unseen languages, respectively. For a fair comparison, all the model sizes are set to 300 million. $\dagger$ denotes results measured by PER, which does not allow for a strict comparison with other results. The language codes and their corresponding language names are provided in the appendix.}
\label{orthography_comp_tab}
\end{table*}
\begin{table}[]
\centering
\small
\begin{tabular}{c|ccc}
\toprule
Model    & \begin{tabular}[c]{@{}c@{}}Data (h)\end{tabular}& \begin{tabular}[c]{@{}c@{}}Universal\end{tabular}  & \begin{tabular}[c]{@{}c@{}}Zero-Shot\end{tabular} \\ 
\midrule\midrule
Whisper & 680k & O &  X \\
MMS-1162   & 122k    & X & X \\
\textbf{\shortname}  &  \textbf{0.6k} & \textbf{O} & \textbf{O} \\ \bottomrule 
\end{tabular}%
\caption{Comparison of previous multi-lingual ASR models with the proposed pipeline. \textit{Data} denotes the total amount of training dataset, \textit{Universal} indicates that no language-specific module is needed, and \textit{Zero-Shot} denotes whether inference on unseen languages is feasible.}
\label{brief_comp_tab}
\end{table}
\subsection{Training Detail}
We performed fine-tuning on all layers except the feature extractor for 3,000 steps with a CTC loss and a batch size of 128.
We bypassed the two-stage fine-tuning pipeline from prior studies~\cite{xu2021simpleeffectivezeroshotcrosslingual, JMLR:v25:23-1318} because our distinct methodology, which used a smaller dataset, caused the divided fine-tuning approach to result in premature convergence and instability.
For hyperparameters, we employed the default AdamW optimizer~\cite{kingma2017adammethodstochasticoptimization, loshchilov2018decoupled} with a tri-stage learning rate scheduler. The warm-up, hold, and decay phases were configured to 10\%, 60\%, and 30\% of the total training steps, respectively.
We then performed a series of experiments to determine the optimal learning rate schedule within the range of 5e-6 to 5e-4.
Finally, the entire training pipeline was conducted on two RTX-3090 GPUs with 24GB of VRAM each, and we leveraged gradient accumulation techniques to address memory issues.

\subsection{Inference Detail}
\subsubsection{Universal Transcription Generator.} 
We leveraged a beam search decoder from flashlight~\cite{kahn2022flashlight} with a beam size of 100. 
No additional lexicons or LMs were utilized in the decoding pipeline to maintain a universal pipeline without relying on language-specific elements.

\subsubsection{Prompting Strategy.}
For the prompting strategy, we utilized language information and a subset of the training data to construct our hypothesis prompt. The specific format of the prompt employed is detailed in Table~\ref{prompt_example_tab}.
In zero-shot prompting, the universal converter automatically transforms Romanized transcriptions into language-specific ones using only the Romanized transcriptions and language information. We employed zero-shot prompting to evaluate the performance of the LLM with minimal input.
Few-shot prompting~\cite{NEURIPS2020_1457c0d6} involves providing examples to help the model generate responses to subsequent instances. 
We hypothesized that this approach would be particularly effective for low-resource or unseen languages by inducing in-context learning. 
Specifically, we randomly sampled five Romanized transcription and target transcription pairs for each few-shot example.
Zero-shot CoT prompting~\cite{NEURIPS2022_8bb0d291} is a technique that supports complex reasoning by inducing the decomposition of intricate tasks into detailed steps. 
Specifically, we appended the phrase ``Let's think step by step'' to the input prompt to encourage the reasoning of the model.
Prompt chaining employs a sequence of prompts, with each prompt building upon the output of the previous one, to manage complex multi-step tasks.
In this aspect, we concentrated on the decomposable process of converting predicted Romanized transcriptions into language-specific transcriptions through \textit{(i) reverse-Romanization} and \textit{(ii) error correction}.
We considered that errors in Romanized transcriptions could propagate during transliteration to language-specific ones, potentially reducing system performance.

\begin{table}[!t]
\centering
\small
\begin{tabular}{c|ccc}
\toprule
                        & Universal Converter       & CER $\downarrow$  & WER $\downarrow$  \\ \midrule\midrule
\multirow{3}{*}{Seen}   & LLaMA-8B    & 26.6 & 46.7 \\
                        & LLaMA-70B   & 15.5 & 35.3 \\
                        & \textbf{GPT-4o-mini} & \textbf{7.5}  & \textbf{18.1} \\ \midrule
\multirow{3}{*}{Unseen} & LLaMA-8B    & 33.0 & 50.2 \\
                        & LLaMA-70B   & 27.2 & 58.9 \\
                        & \textbf{GPT-4o-mini} & \textbf{15.8} & \textbf{38.3} \\ \bottomrule
\end{tabular}%
\caption{
Upper bound performance of the universal converter. 
This upper bound is assessed by feeding ground truth Romanized transcriptions into the universal converter with zero-shot prompting.}
\label{upper_bound_tab}
\end{table}
\subsubsection{Universal Converter.}
Finally, we required the output of the universal converter to conform to a specific format.
We instruct the model to enclose the output within three backticks (e.g., \texttt{```}), which allows us to isolate and sort only the language-specific transcription from the output of the model.
We set the temperature value to 0.0 for all LLMs to obtain deterministic results.